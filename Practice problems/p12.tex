\documentclass[12pt]{article}
\usepackage[margin=1in]{geometry}
\usepackage{amsthm, hhline, enumitem}
\usepackage{stmaryrd, mathtools}
\usepackage{amsmath, amssymb, graphicx}
\newtheorem{theorem}{Theorem}
\newtheorem{lemma}{Lemma}
\newtheorem*{lemma*}{Lemma}
\newtheorem*{theorem*}{Theorem}
\DeclareMathOperator{\ex}{\mathbb{E}}
\DeclareMathOperator{\pr}{\mathbb{P}}
\DeclareMathOperator{\cov}{cov}
\DeclareMathOperator{\var}{var}
\DeclareMathOperator{\sgn}{sgn}
\begin{document}
\subsection*{Problem 12}

From the previous problem, we have the probability distribution
\begin{equation}\label{Eqn1}
	\mathbb{P}(\ell_1, \dots, \ell_N) = \frac{1}{Z} \prod_{1 \leq i < j \leq N} (\ell_i - \ell_j)^2 \prod_{i = 1}^Nw(\ell_i),
\end{equation}
where $l_i=x_id_2\sqrt{L}+d_1L$, and
\begin{equation}\label{Eqn2}
	w(l_i) = \frac{(A+B-l_i-1)! (l_i+C-N)!}{(A+N-l_i-1)!l_i!} 
\end{equation}
By Sterling's Approximation,
\begin{align*}
	&(A+B-l_i-1)!  \\
	=& \left[(a+b-d_1)L-x_id_2\sqrt{L}-1\right]! \\
	\to& \sqrt{2\pi\left[(a+b-d_1)L-x_id_2\sqrt{L}-1\right]} 
	\exp \left[-(a+b-d_1)L+x_id_2\sqrt{L}+1\right] \\
	& \exp\left[ \left((a+b-d_1)L-x_id_2\sqrt{L}-1\right) \ln{\left((a+b-d_1)L-x_id_2\sqrt{L}-1\right)} \right] 
\end{align*}
Using Taylor expansion,
\begin{align*}
	&\ln{\left((a+b-d_1)L-x_id_2\sqrt{L}-1\right)} \\
	=& \ln(a+b-d_1)L-\frac{x_id_2}{(a+b-d_1)\sqrt{L}}-\frac{x_i^2d_2^2}{2(a+b-d_1)^2L}+O(\frac{1}{L})
\end{align*}
The exponent becomes
\begin{align*}
	&\left((a+b-d_1)L-x_id_2\sqrt{L}-1\right) \ln{\left((a+b-d_1)L-x_id_2\sqrt{L}-1\right)}\\
	=& \left((a+b-d_1)L-x_id_2\sqrt{L}-1\right) 
	\left( \ln(a+b-d_1)L-\frac{x_id_2}{(a+b-d_1)\sqrt{L}}-\frac{x_i^2d_2^2}{2(a+b-d_1)^2L}+O(\frac{1}{L}) \right) \\
	=& (a+b-d_1)L\ln(a+b-d_1)L - x_id_2 \sqrt{L}\ln(a+b-d_1)L-x_id_2\sqrt{L} + \frac{x_i^2d_2^2}{2(a+b-d_1)} + O(\frac{1}{\sqrt{L}})
\end{align*}
Similarly, we can calculate the exponent for other factorials
\begin{align*}
	& \left( (c+d_1)L+x_id_2\sqrt{L}-N \right) \ln\left( (c+d_1)L+x_id_2\sqrt{L}-N \right)\\
	=& (c+d_1)L \ln(c+d_1)L+x_id_2\sqrt{L}\ln(c+d_1)L-N\ln(c+d_1)L+x_id_2\sqrt{L}-N-\frac{x_i^2d_2^2}{2(c+d_1)} + O(\frac{1}{\sqrt{L}})
\\
	& \left( (a-d_1)L-x_id_2\sqrt{L}+N-1 \right) \ln\left( (a-d_1)L-x_id_2\sqrt{L}+N-1 \right)\\
	=& (a-d_1)L \ln(a-d_1)L-x_id_2\sqrt{L}\ln(a-d_1)L+(N-1)\ln(a-d_1)L-x_id_2\sqrt{L}\\
	&\qquad +(N-1)-\frac{x_i^2d_2^2}{2(a-d_1)} + O(\frac{1}{\sqrt{L}})
\\
	& \left( d_1L+x_id_2\sqrt{L} \right)\ln\left( d_1L+x_id_2\sqrt{L} \right) \\
	=& d_1L\ln(d_1L)+x_id_2\sqrt{L}\ln(d_1L)+x_id_2\sqrt{L}+\frac{x_i^2d_2^2}{2d_1} + O(\frac{1}{\sqrt{L}})
\end{align*}
Summing the four exponents yields
\begin{align*}
	&(a+b-d_1)L\ln(a+b-d_1)L + ((c+d_1)L-N) \ln(c+d_1)L- ((a-d_1)L-N+1)\ln(a-d_1)L - d_1L\ln(d_1L) \\
	&- x_id_2 \sqrt{L}\ln(a+b-d_1)L + x_id_2\sqrt{L}\ln(c+d_1)L + x_id_2\sqrt{L}\ln(a-d_1)L - x_id_2\sqrt{L}\ln(d_1L)\\
	&+ \frac{x_i^2d_2^2}{2(a+b-d_1)} -\frac{x_i^2d_2^2}{2(c+d_1)} +\frac{x_i^2d_2^2}{2(a-d_1)} -\frac{x_i^2d_2^2}{2d_1} + O(\frac{1}{\sqrt{L}})
\end{align*}
Substitution into the weight function yields the limit
\begin{align*}
	w(l_i) \to& \ L^{(b+c)L-1}
	\frac{(a+b-d_1)^{(a+b-d_1)L} (c+d_1)^{(c+d_1)L-N}}{(a-d_1)^{(a-d_1)L-N+1} d_1^{d_1L}} \exp\left(2N-(b+c)L\right)\\
	&\qquad\cdot
	\exp\left(\frac{x_i^2d_2^2}{2}(\frac{1}{a+b-d_1}-\frac{1}{c+d_1}+\frac{1}{a-d_1}-\frac{1}{d_1})\right)\\
	&\qquad\cdot
	\exp\left(x_id_2\sqrt{L} \ln\frac{(c+d_1)(a-d_1)}{(a+b-d_1)d_1}\right)
\end{align*}
Solving equations
\begin{align*}
	\frac{(c+d_1)(a-d_1)}{(a+b-d_1)d_1} =& \ 1 \\
	 \frac{x_i^2d_2^2}{2}(\frac{1}{a+b-d_1}-\frac{1}{c+d_1}+\frac{1}{a-d_1}-\frac{1}{d_1}) =& -\frac{x_i^2}{2}
\end{align*}
we obtain
\begin{align*}
	d_1 &= \frac{ac}{b+c} \\
	d_2 &= \sqrt{\frac{abc(a+b+c)}{(b^2-c^2)(2a+b+c)}}
\end{align*}

\end{document}