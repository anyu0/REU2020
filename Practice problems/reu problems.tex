\documentclass[12pt]{article}
\usepackage{ragged2e}
\usepackage[margin=1in]{geometry}
\usepackage{amsthm, hhline, enumitem}
\usepackage{stmaryrd}
\usepackage{amsmath, amssymb, graphicx}
\newtheorem{theorem}{Theorem}
\newtheorem*{theorem*}{Theorem}
\DeclareMathOperator{\ex}{\mathbb{E}}
\DeclareMathOperator{\pr}{\mathbb{P}}
\DeclareMathOperator{\cov}{cov}
\DeclareMathOperator{\var}{var}
\DeclareMathOperator{\sgn}{sgn}
\begin{document}
	\begin{flushright}
		Summer 2020
	\end{flushright}
	
	\begin{center}
		\LARGE\textbf{REU Practice Problems}
	\end{center}


\section{Topology and measurability}

	We let $\Sigma$ denote a set $\llbracket p, q\rrbracket = \{p,p+1,\dots,q-1,q\}$ for $p\in\mathbb{N}$, $q\in\mathbb{N}\cup\{\infty\}$, and let $\Lambda$ denote an interval in $\mathbb{R}$. We write $C(X)$ for the space of continuous real-valued functions on $X$ with the topology of compact convergence and the Borel $\sigma$-algebra $\mathcal{C}$. Recall that this topology is generated by the basis of sets
	\[
	B_K(f,\epsilon) := \big\{g\in C(X):\sup_{x\in K} |f(x)-g(x)|<\epsilon\big\},
	\]
	with $K\subset X$ is compact, $f\in C(X)$, and $\epsilon>0$. When $X=\Sigma\times\Lambda$, we write $(C(\Sigma\times\Lambda),\mathcal{C}_\Sigma)$.

	\subsection*{Problem 1}
	
		We aim to construct a metric $d:C(\Sigma\times\Lambda)\times C(\Sigma\times\Lambda)\to [0,\infty)$ which induces the topology of compact convergence on $C(\Sigma\times\Lambda)$. The idea is to write $\Sigma\times\Lambda$ as a union of compact sets $K_n$, such that every compact subset of $\Sigma\times\Lambda$ is contained in one of these sets $K_n$. We then construct $d$ from the sup-metrics on each of these sets $K_n$. We define the sets
		\[
		K_n := \llbracket p, \min(p+n,q)\rrbracket \times \Lambda_n
		\]
		as follows. If $\Lambda=[a,b]$ is compact, then $\Lambda_n=\Lambda$ for all $n$. If $\Lambda=(a,b)$, then
		\[
		\Lambda_n := \left[a+\frac{1}{n}, b-\frac{1}{n}\right],
		\]
		when this set makes sense, and $\Lambda_n=\varnothing$ otherwise. If $\Lambda$ is half-open, we define $\Lambda_n$ similarly, but only modify the endpoint on the open side. (This might not be necessary, not sure if $\Lambda$ is assume to be closed.) In any case, we see that the sets $K_1\subset K_2\subset\cdots\subset\Sigma\times\Lambda$ are compact, they cover $\Sigma\times\Lambda$, and any compact subset $K$ of $\Sigma\times\Lambda$ is contained in all $K_n$ for sufficiently large $n$.
		
		We now define, for each $n$ and $f,g\in C(\Sigma\times\Lambda)$,
		\[
		d_n(f,g) := \sup_{x\in K_n} |f(x)-g(x)|,\quad d_n'(f,g) := \min\{d_n(f,g), 1\}.
		\]
		Clearly each $d_n$ is nonnegative and satisfies the triangle inequality, and it is then easy to see that the same properties hold for $d_n'$. Furthermore, $d_n'\leq 1$, so we can define
		\[
		d(f,g) := \sum_{n=1}^\infty 2^{-n} d_n'(f,g).
		\]
		We first observe that $d$ is a metric on $C(\Sigma\times\Lambda)$. Indeed, it is nonnegative, and if $f=g$, then each $d_n'(f,g)=0$, so the sum is 0. Conversely, if $f\neq g$, then since the $K_n$ cover $\Sigma\times\Lambda$, we can choose $n$ large enough so that $K_n$ contains an $x$ with $f(x)\neq g(x)$. Then $d_n'(f,g)\neq 0$, and hence $d(f,g)\neq 0$. The triangle inequality holds for $d$ since it holds for each $d_n'$.
		
		Now we prove that the topology $\tau_d$ on $C(\Sigma\times\Lambda)$ induced by $d$ is the same as the topology of compact convergence, which we will denote $\tau_c$. First, choose $\epsilon>0$ and $f\in C(\Sigma\times\Lambda)$. Let $g\in B^d_\epsilon(f)$, i.e., $d(f,g)<\epsilon$. We will find a set $A_g\in\tau_c$ such that $g\in A_g\subset B^d_\epsilon(f)$. Let $\delta := d(f,g)$, and choose $n$ large enough so that $\sum_{k>n} 2^{-k} < \frac{\epsilon-\delta}{2}$. Define $A_g := B_{K_n}(g,\frac{\epsilon-\delta}{n})$, and suppose $h\in A_g$. Then since $K_m\subseteq K_n$ for $m\leq n$, we have
		\begin{align*}
		d(f,h) &\leq d(f,g) + d(g,h)\\
		&\leq \delta + \sum_{k=1}^n 2^{-k}d_n(g,h) + \sum_{k>n} 2^{-k}\\
		&\leq \delta + \frac{\epsilon-\delta}{2} + \frac{\epsilon-\delta}{2} = \epsilon.
		\end{align*}
		Therefore $g\in A_g\subset B^d_\epsilon(f)$. It follows that $B^d_\epsilon(f)\in \tau_c$. Indeed, we can write
		\[
		B^d_\epsilon(f) = \bigcup_{g\in B^d_\epsilon(f)} A_g,
		\]
		a union of elements of $\tau_c$. This proves that $\tau_d\subseteq\tau_c$.
		
		To prove the converse, let $K\subset\Sigma\times\Lambda$ be compact, $f\in C(\Sigma\times\Lambda)$, and $\epsilon>0$. Choose $n$ so that $K\subset K_n$, and let $g\in B_K(f,\epsilon)$ and $\delta:= \sup_{x\in K} |f(x)-g(x)|$. If $d(g,h) < 2^{-n}(\epsilon-\delta)$, then $d_n'(g,h) \leq 2^n d(g,h) < \epsilon-\delta$, hence $d_n(g,h) < \epsilon-\delta$. It follows that
		\begin{align*}
		\sup_{x\in K} |f(x)-h(x)| &\leq \delta + \sup_{x\in K} |g(x)-h(x)| \leq \delta + d_n(g,h)\\
		&\leq \delta + \epsilon-\delta = \epsilon.
		\end{align*}
		Thus $g\in B^d_{2^{-n}(\epsilon-\delta)}(f) \subset B_K(f,\epsilon)$. It follows that $\tau_c\subseteq \tau_d$, and we conclude that $\tau_d = \tau_c$.
		
		Next, we show that $(C(\Sigma\times\Lambda), d)$ is a complete metric space. Let $(f_n)_{n\geq 1}$ be Cauchy with respect to $d$. Then we claim that $(f_n)$ must be Cauchy with respect to $d_n'$, on each $K_n$. Indeed, $d(f_\ell, f_m) \geq 2^{-n}d_n'(f_\ell, f_m)$, so if $(f_n)$ were not Cauchy with respect to $d_n'$, it would not be Cauchy with respect to $d$ either. Thus $(f_n)$ is uniformly Cauchy on each $K_n$, and hence converges uniformly to a limit $f^{K_n}$ on each $K_n$. Since the limit must be unique at each point of $\Sigma\times\Lambda$, we have $f^{K_n}(x) = f^{K_m}(x)$ if $x\in K_n\cap K_m$. Since $\bigcup K_n = \Sigma\times\Lambda$, we obtain a well-defined function $f$ on all of $\Sigma\times\Lambda$ given by $f(x)=f^{K_n}(x)$, where $x\in K_n$. Given any compact $K\subset \Sigma\times\Lambda$, if $n$ is large enough so that $K\subset K_n$, then because $f_n \to f^{K_n} = f|_{K_n}$ uniformly on $K_n$, we have $f_n \to f^{K_n}|_K = f|_K$ uniformly on $K$. That is, for any $K\subset\Sigma\times\Lambda$ compact and $\epsilon>0$, we have $f_n \in B_K(f,\epsilon)$ for all sufficiently large $n$. Therefore $(f_n)$ converges to $f$ in the topology of compact convergence, and equivalently in the metric $d$.
		
		Lastly, we prove separability. We consider the subspace $P_{\mathbb{Q}}$ of $C(\Sigma\times\Lambda)$ consisting of ``polynomials" with rational coefficients. That is, $p\in P_{\mathbb{Q}}$ if $p(n,\cdot)$ is a polynomial on $\Lambda$ with rational coefficients for each $n\in\Sigma$. If $f\in C(\Sigma\times\Lambda)$, then on any compact set $K\subset\Sigma\times\Lambda$ we can find a sequence of polynomials converging uniformly to $f$ by the Stone-Weierstrass theorem. These polynomials in turn can be uniformly approximated by polynomials with rational coefficients, so by diagonalization we obtain a sequence in $P_\mathbb{Q}$ converging uniformly to $f$ on $K$. [Can we patch together the sequences for the sets $K_n$ to get one sequence $(p_n)$ in $P_\mathbb{Q}$ that converges uniformly to $f$ on \textit{all} compact subsets? Maybe use diagonalization again?][The result from McCoy, ``Second Countable and Separable Function Spaces," shows that $C(\Sigma\times\Lambda)$ is second-countable and hence separable because $\Sigma\times\Lambda$ is second-countable. But the proof is a bit difficult.]
		
		
	\section*{Problem 2}
	
		Let $(\Omega,\mathcal{F},\pr)$ be a probability space and $X,Y$ random variables on $(\Omega,\mathcal{F},\pr)$ taking values in $C(\Sigma\times\Lambda)$, where $\Sigma = \llbracket 1, N\rrbracket$ with $N\in\mathbb{N}$ or $N=\infty$. We consider the collection $\mathcal{S}_X$ of sets of the form
		\[
		\{\omega\in\Omega : X(\omega)(i_1,t_1)\leq x_1,\dots,X(\omega)(i_n,t_n)\leq x_n\} = \bigcap_{k=1}^n X(i_k,t_k)^{-1}(-\infty,x_k],
		\] 
		ranging over all $n\in\mathbb{N}$, $(i_1,t_1),\dots,(i_n,t_n)\in \Sigma\times\Lambda$, and $x_1,\dots,x_n\in\mathbb{R}$. We first prove that $\mathcal{S}_X \subset \mathcal{F}$. We can write 
		\[
		\{X(i_k,t_k)\leq x_k\} = X^{-1}(\{f\in C(\Sigma\times\Lambda):f(i_k,t_k)\leq x_k\}).
		\]
		We claim that the set $\{f\in C(\Sigma\times\Lambda):f(i_k,t_k)\leq x_k\}$ is closed in the topology of compact convergence. If $f_n(i_k,t_k)\leq x_k$ for all $n$ and $f_n\to f$ in the (metrizable) topology of compact convergence, then by taking limits on a compact set containing $(i_k,t_k)$, we find $f(i_k,t_k)\leq x_k$ as well. This proves the claim, and it follows from the measurability of $X$ that $\{X(i_k,t_k)\leq x_k\} = X^{-1}(\{f(i_k,t_k)\leq x_k\})\in\mathcal{F}$. The finite intersection is thus also in $\mathcal{F}$, proving that $\mathcal{S}_X \subset \mathcal{F}$. On the other hand, it is clear that $\{\omega\in\Omega:X(\omega)\in A\} = X^{-1}(A)\in\mathcal{F}$ for any $A\in\mathcal{C}_\Sigma$ since $X$ is measurable.
		
		Now we prove that $\mathbb{P}|_{\mathcal{S}_X}$ determines the distribution $\mathbb{P}\circ X^{-1}$. To do so, note that $\mathcal{S}_X = \sigma(\{X^{-1}(A) : A\in\mathcal{S}\})$, where $\mathcal{S}$ is the collection of cylinder sets
		\[
		\{f\in C(\Sigma\times\Lambda) : f(i_1,t_1)\in A_1, \dots, f(i_n,t_n) \in A_n\}, \quad A_1,\dots,A_n\in\mathcal{B}(\mathbb{R}). 
		\]
		This follows from the fact that $\mathcal{B}(\mathbb{R})$ is generated by intervals of the form $(-\infty,x]$. Observe that the intersection of two elements of $\mathcal{S}$ is clearly another element of $\mathcal{S}$, so $\mathcal{S}$ is a $\pi$-system. We now argue that $\mathcal{S}$ generates the Borel sets, i.e., $\sigma(\mathcal{S}) = \mathcal{C}_\Sigma$. By the argument above, $\mathcal{S}\subset \mathcal{C}_\Sigma$. Conversely, we will show that every basis element of the topology of compact convergence on $C(\Sigma\times\Lambda)$ is contained in $\sigma(\mathcal{S})$, and consequently so is every Borel set. More precisely, let $K\subset\Sigma\times\Lambda$ be compact, $f\in C(\Sigma\times\Lambda)$, and $\epsilon>0$, and let $H$ be a countable dense subset of $K$. (Recall that every compact metric space is separable, and $K$ is homeomorphic to a product of finitely many compact sets in $\mathbb{R}$, which are metrizable. So $K$ is separable.) We claim that
		\[
		B_K(f,\epsilon) = \bigcup_{n=1}^\infty\,\bigcap_{(i,t)\in H} \{g\in C(\Sigma\times\Lambda) : g(i,t) \in  (f(i,t)-(1-2^{-n})\epsilon, f(i,t) + (1-2^{n})\epsilon)\}.
		\]
		Indeed, if $g\in B_K(f,\epsilon)$, i.e., $\sup_{(i,t)\in K} |g(i,t)-f(i,t)| < \epsilon$. Then since $1-2^{-m}\nearrow 1$, we can choose $m$ large enough so that 
		\[
		|g(i,t)-f(i,t)| < (1-2^{-n})\epsilon
		\] 
		for all $(i,t)\in K$ (in particular with $(i,t)\in H$). Conversely, suppose $g$ is in the set on the right. Then since $g$ is continuous and $H$ is dense in $K$, we find that for some $n\geq 1$,
		\[
		|g(i,t)-f(i,t)| \leq (1-2^{-n})\epsilon < \epsilon
		\]
		for all $(i,t)\in K$. Hence $g\in B_K(f,\epsilon)$. This proves the claim. Since $H$ is countable, $B_K(f,\epsilon)$ is formed from countably many unions and intersections of sets in $\mathcal{S}$, thus $B_K(f,\epsilon)\in\sigma(\mathcal{S})$.
		
		In summary, we have shown that the collection $\mathcal{S}$ is a $\pi$-system generating $\mathcal{C}_\Sigma$, so the probability measure $\mathbb{P}\circ X^{-1}$ on $\mathcal{C}_\Sigma$ is uniquely determined by its restriction to $\mathcal{S}$. Suppose
		\begin{align*}
		&\mathbb{P}\left(\{\omega\in\Omega : X(\omega)(i_1,t_1)\leq x_1,\dots,X(\omega)(i_n,t_n)\leq x_n\}\right) =\\
		&\qquad\qquad \mathbb{P}\left(\{\omega\in\Omega : Y(\omega)(i_1,t_1)\leq x_1,\dots,Y(\omega)(i_n,t_n)\leq x_n\}\right)
		\end{align*}
		for all $(i_1,t_1), x_1,\dots,x_n$. This says that the two probability measures $\mathbb{P}\circ X^{-1}$ and $\mathbb{P}\circ Y^{-1}$ agree on $\mathcal{S}$. Then they must agree on all of $\mathcal{C}_\Sigma$, i.e.,
		\[
		\mathbb{P}\left(\{\omega\in\Omega : X(\omega)\in A\}\right) = \mathbb{P}\left(\{\omega\in\Omega : Y(\omega)\in A\}\right)
		\]
		for all $A\in\mathcal{C}_\Sigma$. In other words, the law of a line ensemble is determined by its finite dimensional distributions.
		

\section{Algebra}

	\subsection*{Problem 3}
	
	\subsection*{Problem 4}


\section{Weak convergence}

	\subsection*{Problem 5}
	
	\subsection*{Problem 6}
	
	\subsection*{Problem 7}


\section{Tightness}

	\subsection*{Problem 8}
	
	\subsection*{Problem 9}


\section{Lozenge tilings of the hexagon}

	\subsection*{Problem 10}
	
	\subsection*{Problem 11}

\end{document}