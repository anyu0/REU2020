%-------------------------------------------------------------------------------------------------------------------------------------------------------------------------------------------------
%    Section 5
%
%-------------------------------------------------------------------------------------------------------------------------------------------------------------------------------------------------
\section{Bounding the max and min}\label{Section5}

In this section we prove Lemmas \ref{PropSup} and \ref{PropSup2} and we assume the same notation as in the statements of these lemmas. In particular, we assume that $k \in \mathbb{N}$, $k \geq 2$, $p \in (0,1)$, $\alpha, \lambda > 0$ are all fixed and 
\begin{equation*}
	\big\{\mathfrak{L}^N = (L^N_1,L^N_2, \dots, L^N_k)\big\}_{N=1}^{\infty},
\end{equation*}
is an $(\alpha,p,\lambda)$-good sequence of $\llbracket 1, k\rrbracket$-indexed Bernoulli line ensembles as in Definition \ref{Def1} that are all defined on a probability space with measure $\mathbb{P}$. The proof of Lemma \ref{PropSup} is given in Section \ref{Section5.1} and Lemma \ref{PropSup2} is proved in Section \ref{Section5.2}.


%-------------------------------------------------------------------------------------------------------------------------------------------------------------------------------------------------
%    Section 5.1
%
%-------------------------------------------------------------------------------------------------------------------------------------------------------------------------------------------------
\subsection{Proof of Lemma \ref{PropSup}}\label{Section5.1}

Our proof of Lemma \ref{PropSup} is similar to that of \cite[Lemma 5.2]{CD}. For clarity we split the proof into three steps. In the first step we introduce some notation that will be required in the proof of the lemma, which is presented in Steps 2 and 3. \\

{\bf \raggedleft Step 1.} We write $s_4 = \lceil r+4 \rceil N^\alpha$, $s_3 = \lfloor r+3 \rfloor N^\alpha$, so that $s_3 \leq t_3 \leq s_4$, and assume that $N$ is large enough so that $\psi(N)N^{\alpha}$ from Definition \ref{Def1} is at least $s_4$. Notice that such a choice is possible by our assumption that $\mathfrak{L}^N$ is an $(\alpha,p,\lambda)$-good sequence and in particular, we know that $L_i^N$ are defined at $\pm s_4$ for $i \in \llbracket 1, k \rrbracket$. We define events 
$$E(M) = \Big\{\big|L_1^N(-s_4) + ps_4\big| > MN^{\alpha/2}\Big\}, \quad F(M) = \Big\{L_1^N(-s_3) > -ps_3 + MN^{\alpha/2} \Big\},$$
$$G(M) = \Bigg\{\sup_{s\in[0,s_4]} \big[L_1^N(s) - ps \big] \geq (6r+22)(2r+10)^{1/2}(M+1)N^{\alpha/2} \Bigg\}.$$

If $\epsilon > 0$ is as in the statement of the lemma, we note by (\ref{globalParabola}) that we can find $M$ and $\tilde{N}_1$ sufficiently large so that if $N \geq \tilde{N}_1$ we have 
\begin{equation}\label{4.2EFbounds}
	\mathbb{P}(E(M)) < \epsilon/4 \mbox{ and } \mathbb{P}(F(M)) < \epsilon/12.
\end{equation}
In the remainder of this step we show that the event $G(M) \setminus E(M)$ can be written as a {\em countable disjoint union} of certain events, i.e. we show that
\begin{equation}\label{WR3}
	\bigsqcup\limits_{(a,b,s,\ell_{top},\ell_{bot}) \in D(M)}  E(a,b,s,\ell_{top},\ell_{bot}) = G(M) \setminus E(M),
\end{equation}
where the sets $E(a,b,s,\ell_{top},\ell_{bot})$ and $D(M)$ are described below. 

For $a,b,z_1,z_2,z_3 \in\mathbb{Z}$ with $z_1\leq a$, $z_2\leq b$, $s\in\llbracket 0, s_4 \rrbracket$, $\ell_{bot}\in\Omega(-s_4,s,z_1,z_2)$ and $\ell_{top}\in\Omega(s,s_4,b,z_3)$ we define $E(a,b,s,\ell_{top},\ell_{bot})$ to be the event that $L_1^N(-s_4) = a$, $L_1^N(s) = b$, $L_1^N$ agrees with $\ell_{top}$ on $\llbracket s,s_4\rrbracket$, and $L_2^N$ agrees with $\ell_{bot}$ on $\llbracket -s_4,s \rrbracket$.  Let $D(M)$ be the set of tuples $(a,b,s, \ell_{top}, \ell_{bot})$ satisfying
\begin{enumerate}[label=(\arabic*)]
	\item $0\leq s\leq s_4$,
	\item $0\leq b-a \leq s + s_4$, $|a + ps_4| \leq MN^{\alpha/2}$, and $b-ps \geq (6r+22)(2r+10)^{1/2}(M+1)N^{\alpha/2}$,
	\item $z_1\leq a$, $z_2\leq b$, and $\ell_{bot}\in\Omega(-s_4, s, z_1, z_2)$,
	\item $b \leq z_3 \leq b + (s_4 - s)$, and $\ell_{top} \in\Omega(s, s_4, b, z_3)$,
	\item if $s < s' \leq s_4$, then $\ell_{top}(s') -ps' < (6r+22)(2r+10)^{1/2}(M+1)N^{\alpha/2}$.
\end{enumerate}
It is clear that $D(M)$ is countable. The five conditions above together imply that 
$$\bigcup_{(a,b,s,\ell_{top},\ell_{bot}) \in D(M)}  E(a,b,s,\ell_{top},\ell_{bot}) = G(M) \setminus E(M),$$
and what remains to be shown to prove (\ref{WR3}) is that $E(a,b,s,\ell_{top},\ell_{bot})$ are pairwise disjoint. 

On the intersection of $E(a,b,s,\ell_{top},\ell_{bot})$ and $E(\tilde{a},\tilde{b},\tilde{s},\tilde{\ell}_{top},\tilde{\ell}_{bot})$ we must have $\tilde{a} = L_1^N(-s_4) = a$ so that $a = \tilde{a}$. Furthermore, we have by properties (2) and (5) that $s \geq \tilde{s}$ and $\tilde{s} \geq s$ from which we conclude that $s= \tilde{s}$ and then we conclude $\tilde{b} = b$, $\ell_{top} = \tilde{\ell}_{top}$, $\ell_{bot} = \tilde{\ell}_{bot}$. In summary, if $E(a,b,s,\ell_{top},\ell_{bot})$ and $E(\tilde{a},\tilde{b},\tilde{s},\tilde{\ell}_{top},\tilde{\ell}_{bot})$ have a non-trivial intersection then $(a,b,s,\ell_{top},\ell_{bot}) = (\tilde{a},\tilde{b},\tilde{s},\tilde{\ell}_{top},\tilde{\ell}_{bot})$, which proves (\ref{WR3}).\\

{\raggedleft \bf Step 2.} In this step we prove that  we can find an $N_2$ so that for $N \geq N_2$
\begin{equation}\label{WR1}
	\mathbb{P}\left( \sup_{s \in [0,t_3] }\big[ L^N_1(s) - p s \big] \geq  (6r+22)(2r+10)^{1/2}(M+1)N^{\alpha/2} \right) \leq \mathbb{P}(G(M)) < \epsilon/2.
\end{equation}
A similar argument, which we omit, proves the same inequality with $[-t_3,0]$ in place of $[0,t_3]$ and then the statement of the lemma holds for all $N\geq N_2$, with $R_1 = (6r+22)(2r+10)^{1/2}(M+1)$.

We claim that we can find $\tilde{N}_2 \in \mathbb{N}$ sufficiently large so that if $N \geq \tilde{N}_2$ and $(a,b,s,\ell_{top},\ell_{bot})\in D(M)$ satisfies $\mathbb{P}( E(a,b,s,\ell_{top},\ell_{bot})) > 0$ then we have
\begin{equation}\label{WR7}
	\mathbb{P}^{-s_4,s,a,b,\infty,\ell_{bot}}_{avoid, Ber}\left(\ell(-s_3) > -ps_3 + MN^{\alpha/2}\right) \geq \frac{1}{3}.
\end{equation}
We will prove (\ref{WR7}) in Step 3. For now we assume its validity and conclude the proof of (\ref{WR1}).\\

Let $(a,b,s,\ell_{top},\ell_{bot})\in D(M)$ be such that $\mathbb{P}( E(a,b,s,\ell_{top},\ell_{bot})) > 0$. By the Schur Gibbs property, see Definition \ref{DefSGP}, we have for any $\ell_0\in\Omega(-s_4,s,a,b)$ that
\begin{equation}\label{WR9}
	\mathbb{P}\left(L_1^N\llbracket -s_4,s \rrbracket = \ell_0\,|\,E(a,b,s,\ell_{top}, \ell_{bot}) \right) = \mathbb{P}^{-s_4,s,a,b,\infty,\ell_{bot}}_{avoid, Ber}(\ell = \ell_0),
\end{equation}
where $L_1^N\llbracket -s_4,s \rrbracket$ denotes the restriction of $L_1^N$ to the set $\llbracket -s_4,s \rrbracket$.

Combining (\ref{WR7}) and (\ref{WR9}) we get for $N \geq \tilde{N}_2$
\begin{equation}\label{WR10}
	\begin{split}
		&\mathbb{P}\left( L_1^N(-s_3) > -ps_3 + MN^{\alpha/2} \vert E(a,b,s, \ell_{top}, \ell_{bot})\right) = \\
		&\mathbb{P}^{-s_4,s,a,b,\infty,\ell_{bot}}_{avoid, Ber}\left(\ell(-s_3) > -ps_3 + MN^{\alpha/2}\right) \geq \frac{1}{3}.
	\end{split}
\end{equation}

It follows from (\ref{WR10}) that for $N \geq \tilde{N}_2$ we have
\begin{equation}\label{WR11}
	\begin{split}
		& \epsilon/12 > \mathbb{P}(F(M)) \geq \sum_{\substack{(a,b,s,\ell_{top}, \ell_{bot})\in D(M), \\ \mathbb{P}(E(a,b,s, \ell_{top}, \ell_{bot})) > 0}} \mathbb{P}\left(F(M)\cap E(a,b,s, \ell_{top}, \ell_{bot})\right) = \\
		&\sum_{\substack{(a,b,s,\ell_{top}, \ell_{bot})\in D(M), \\ \mathbb{P}(E(a,b,s, \ell_{top}, \ell_{bot})) > 0}} \hspace{-5mm} \mathbb{P}\left( L_1^N(-s_3) > -ps_3 + MN^{\alpha/2} \vert  E(a,b,s,\ell_{top}, \ell_{bot}) \right )\mathbb{P}\left(E(a,b,s,\ell_{top}, \ell_{bot}) \right) \geq \\
		& \sum_{\substack{(a,b,s,\ell_{top}, \ell_{bot})\in D(M), \\ \mathbb{P}(E(a,b,s, \ell_{top}, \ell_{bot})) > 0}} \frac{1}{3} \cdot \mathbb{P}\left(E(a,b,s,\ell_{top}, \ell_{bot}) \right)  =  \frac{1}{3} \cdot \mathbb{P}(G(M)\setminus E(M)),
	\end{split}
\end{equation}
where in the last equality we used (\ref{WR3}). From (\ref{4.2EFbounds}) and (\ref{WR11}) we have for $N \geq N_2 = \max(\tilde{N}_1, \tilde{N}_2)$  
$$\mathbb{P}(G(M)) \leq \mathbb{P}(E(M)) + \mathbb{P}(G(M)\setminus E(M))< \epsilon/4 + \epsilon/4,$$
which proves (\ref{WR1}).\\


{\bf \raggedleft Step 3.} In this step we prove (\ref{WR7}) and in the sequel we let $(a,b,s,\ell_{top},\ell_{bot})\in D(M)$ be such that $\mathbb{P}( E(a,b,s,\ell_{top},\ell_{bot})) > 0$. We remark that the condition $\mathbb{P}( E(a,b,s,\ell_{top},\ell_{bot})) >0 $ implies that $\Omega_{avoid}({-s_4,s,a,b,\infty,\ell_{bot}})$ is not empty. By Lemma \ref{MCLfg} we know that 
$$\mathbb{P}^{-s_4,s,a,b,\infty,\ell_{bot}}_{avoid, Ber}\left(\ell(-s_3) > -ps_3 + MN^{\alpha/2}\right) \geq \mathbb{P}^{-s_4,s,a,b}_{ Ber}\left(\ell(-s_3) > -ps_3 + MN^{\alpha/2}\right),$$
and so it suffices to show that 
\begin{equation}\label{WT0}
	\mathbb{P}^{-s_4,s,a,b}_{Ber}\left(\ell(-s_3) > -ps_3 + MN^{\alpha/2}\right) \geq \frac{1}{3}.
\end{equation}

One directly observes that
\begin{equation}\label{WT2}
	\begin{split}
		&\mathbb{P}^{-s_4,s,a,b}_{Ber}\left( \ell(-s_3) > -ps_3 + MN^{\alpha/2}\right) = \mathbb{P}^{0,s+s_4,0,b-a}_{Ber}\left(\ell(s_4-s_3) + a \geq -ps_3 + MN^{\alpha/2}\right) \geq\\
		& \mathbb{P}^{0,s+s_4,0,b-a}_{Ber}\left(\ell(s_4-s_3) \geq p(s_4-s_3) + 2MN^{\alpha/2}\right),
	\end{split}
\end{equation}
where the inequality follows from the assumption in (2) that $a+ps_4 \geq -MN^{\alpha/2}$. Moreover, since $b-ps \geq (6r+22)(2r+10)^{1/2}(M+1)N^{\alpha/2}$ and $a+ps_4 \leq MN^{\alpha/2}$, we have 
$$b-a \geq p(s+s_4) + (6r+21)(2r+10)^{1/2}(M+1)N^{\alpha/2} \geq p(s+s_4) + (6r+21)(M+1)(s+s_4)^{1/2}.$$
The second inequality follows since $s+s_4 \leq 2s_4 \leq (2r+10)N^{\alpha}$. 

It follows from Lemma \ref{LemmaHalfS4} with $M_1 = 0$, $M_2 = (6r+21)(M+1)$ that for sufficiently large $N$
\begin{equation}\label{WT3}
	\mathbb{P}^{0,s+s_4,0,b-a}_{Ber}\Big(\ell(s_4-s_3) \geq \frac{s_4-s_3}{s+s_4}[p(s+s_4) + M_2 (s+s_4)^{1/2}] - (s+s_4)^{1/4}\Big) \geq 1/3.
\end{equation}

Note that $\frac{s_4-s_3}{s+s_4} \geq \frac{N^\alpha }{(2r+10)N^\alpha} = \frac{1}{2r+10}$ and so for all $N \in \mathbb{N}$ we have
\begin{equation}\label{WT4}
	\begin{split}
		& \frac{s_4-s_3}{s+s_4}[p(s+s_4) + M_2 (s+s_4)^{1/2}] - (s+s_4)^{1/4} \geq \\
		& p(s_4-s_3) + \frac{(6r+21)(M+1) (s+s_4)^{1/2}}{2r+10}  - (s+s_4)^{1/4} \geq p(s_4-s_3) + 2MN^{\alpha/2}.
	\end{split}
\end{equation}
Combining (\ref{WT2}), (\ref{WT3}) and (\ref{WT4}) we conclude that we can find $\tilde{N}_2 \in \mathbb{N}$ such that if $N \geq \tilde{N}_2$ we have (\ref{WT0}). This suffices for the proof.


%-------------------------------------------------------------------------------------------------------------------------------------------------------------------------------------------------
%    Section 5.2
%
%-------------------------------------------------------------------------------------------------------------------------------------------------------------------------------------------------
\subsection{Proof of Lemma \ref{PropSup2}}\label{Section5.2}
We begin by proving the following important lemma, which shows that it is unlikely that the curve $L_{k-1}^N$ falls uniformly very low on a large interval.
\begin{lemma}\label{21}
	Under the same conditions as in Lemma \ref{PropSup2} the following holds. For any $r,\epsilon>0$ there exist $R>0$ and $N_5 \in \mathbb{N}$ such that for all $N\geq N_5$
	\begin{equation}\label{E21}
		\mathbb{P}\left(\sup_{x\in[r,R]} \left(L_{k-1}^N(xN^\alpha) - pxN^\alpha\right) \leq -(\lambda R^2 + \phi(\epsilon/16) + 1)N^{\alpha/2}\right) < \epsilon,
	\end{equation}
	where $\lambda, \phi$ are as in the definition of an $(\alpha,p,\lambda)$-good sequence of line ensembles, see Definition \ref{Def1}. The same statement holds if $[r,R]$ is replaced with $[-R,-r]$ and the constants $N_5, R$ depend on $\epsilon, r$ as well as the parameters $\alpha, p, \lambda, k$ and the functions $\phi, \psi$ from Definition \ref{Def1}. 
\end{lemma}
\begin{proof}
	Before we go into the proof we give an informal description of the main ideas. The key to this lemma is the parabolic shift implicit in the definition of an $(\alpha,p,\lambda)$-good sequence. This shift requires that the deviation of the top curve $L_1^N$ from the line of slope $p$ to appear roughly parabolic. On the event in equation (\ref{E21}) we have that the $(k-1)$-th curve dips very low uniformly on the interval $[r,R]$ and we will argue that on this event the top $k-2$ curves essentially do not feel the presence of the $(k-1)$-th curve. After a careful analysis using the monotone coupling lemmas from Section \ref{Section3.1} we will see that the latter statement implies that the curve $L_1^N$ behaves like a free bridge between its end-points that have been slighly raised. Consequently, we would expect the midpoint $L_1^N \left( N^{\alpha} (R+r)/2 \right)$ to be close (on scale $N^{\alpha/2}$) to $[L_1^N(rN^{\alpha}) + L^N_1(RN^{\alpha})]/2.$ However, with high probability $[L_1^N(rN^{\alpha}) + L^N_1(RN^{\alpha})]/2$ lies much lower than the inverted parabola $-\lambda(R+r)^2 N^{\alpha/2}/4 $ (due to the concavity of the latter), and so it is very unlikely for $L_1^N \left( N^{\alpha} (R+r)/2 \right)$ to be near it by our assumption. The latter would imply that the event in (\ref{E21}) is itself unlikely, since conditional on it an unlikely event suddenly became likely. 
	
	We proceed to fill in the details of the above sketch of the proof in the following steps. In total there are six steps and we will only prove the statement of the lemma for the interval $[r,R]$, since the argument for $[-R,-r]$ is very similar. \\
	
	\noindent\textbf{Step 1.} We begin by specifying the choice of $R$ in the statement of the lemma, fixing some notation and making a few simplifying assumptions. 
	
	Fix $r , \epsilon > 0$ as in the statement of the lemma. Note that for any $R>r$,
	$$\sup_{r\leq x\leq R} \big(L_{k-1}^N(xN^\alpha) - pxN^\alpha\big) \geq \sup_{\lceil r\rceil \leq x \leq R} \big(L_{k-1}^N(xN^\alpha) - pxN^\alpha\big).$$
	Thus by replacing $r$ with $\lceil r\rceil$, we can assume that $r\in\mathbb{Z}$, which we do in the sequel. Notice that by our assumption that $\mathfrak{L}^N$ is $(\alpha, p,\lambda)$-good we know that (\ref{E21}) holds trivially if $k = 2$ (with the right side of (\ref{E21}) being any number greater than $\epsilon/16$ and in particular $\epsilon$) and so in the sequel we assume that $k \geq 3$. 
	
	Define constants
	\begin{equation}\label{21Cdef}
		C = \sqrt{ 8p(1-p) \log\frac{3}{1-(11/12)^{1/(k-2)}}},
	\end{equation}
	and $R_0 > r$ sufficiently large so that for $R \geq R_0$ and $N \in \mathbb{N}$ we have
	\begin{equation}\label{21Rdef2}
		\frac{\lambda(R-r)^2}{4} \geq  2\phi(\epsilon/16) + 2 + k \lceil C  \lceil RN^{\alpha} \rceil -  \lfloor rN^{\alpha} \rfloor \rceil N^{-\alpha/2}.
	\end{equation}
	We define $R = \lceil R_0\rceil + \mathbf{1}_{\lceil R_0\rceil + r\;\mathrm{odd}}$, so that $R\geq R_0$ and the midpoint $(R+r)/2$ are integers. This specifies our choice of $R$ and for convenience we denote $m = (R+r)/2$. 
	
	In the following, we always assume $N$ is large enough so that $\psi(N) > R$, hence $L_i^N$ are defined at $RN^\alpha$ for $1\leq i\leq k$. We may do so by the second condition in the definition of an $(\alpha,p,\lambda)$-good sequence (see Definition \ref{Def1}). \\
	
	With the choice of $R$ as above we define the events
	\begin{equation}\label{21AB}
		\begin{split}
			A &= \left\{L_1^N\left(mN^{\alpha}\right) - pm N^\alpha  + \lambda m^2 N^{\alpha/2} < -\phi(\epsilon/16)N^{\alpha/2}\right\},\\
			B &= \left\{\sup_{x\in[r,R]} \left(L_{k-1}^N(xN^\alpha) - pxN^\alpha\right) \leq -[\lambda R^2 + \phi(\epsilon/16) + 1] N^{\alpha/2} \right\}.
		\end{split}
	\end{equation}
	The goal of the lemma is to prove that we can find $N_5\in\mathbb{N}$ so that for all $N\geq N_5$
	\begin{equation}\label{21Bbound}
		\mathbb{P}(B) < \epsilon,
	\end{equation}
	which we accomplish in the steps below.\\
	
	\noindent\textbf{Step 2.} In this step we introduce some notation that will be used throughout the next steps. Let $\gamma = \lfloor rN^{\alpha} \rfloor$ and $\Gamma = \lceil RN^{\alpha} \rceil$. We also define the event
	\begin{equation}\label{21x1y1}
		\begin{split}
			& F = \left\{ \sup_{s \in \{\gamma,  \Gamma\}} \left|L_1^N(s) - ps + \lambda s^2N^{-\alpha/2}\right|< [\phi(\epsilon/16) + 1] N^{\alpha/2} \right\}.
		\end{split}
	\end{equation}
	In the remainder of this step we show that $F \cap B$ can be written as a \textit{countable disjoint union}
	\begin{equation}\label{21F}
		F \cap B = \bigsqcup_{(\vec{x},\vec{y},\ell_{bot})\in D}  E(\vec{x},\vec{y},\ell_{bot}),
	\end{equation} 
	where the sets $E(\vec{x},\vec{y},\ell_{bot})$ and $D$ are defined below.
	
	For $\vec{x},\vec{y}\in\mathfrak{W}_{k-2}$, $z_1,z_2\in\mathbb{Z}$, and $\ell_{bot}\in\Omega(\gamma, \Gamma ,z_1,z_2)$, let $E(\vec{x},\vec{y},\ell_{bot})$ denote the event that $L_i^N(\gamma) = x_i$ and $L_i^N(\Gamma)=y_i$ for $1\leq i\leq k-2$, and $L_{k-1}^N$ agrees with $\ell_{bot}$ on $[\gamma, \Gamma]$. Let $D$ denote the set of triples $(\vec{x},\vec{y},\ell_{bot})$ satisfying
	\begin{enumerate}[label=(\arabic*)]
		\item $0\leq y_i - x_i \leq \Gamma - \gamma$ for $1\leq i\leq k-2$,
		\item $|x_1 - p\gamma + \lambda \gamma^2 N^{-3\alpha/2}| < \phi(\epsilon/16) N^{\alpha/2}$ and $|y_1- p \Gamma  + \lambda \Gamma^2N^{-3\alpha/2}| < \phi(\epsilon/16)N^{\alpha/2}$,
		\item $z_1\leq x_{k-2}$, $z_2\leq y_{k-2}$, and $\ell_{bot} \in \Omega(\gamma , \Gamma,z_1,z_2)$,
		\item $\sup_{x \in [r, R]} [\ell_{bot}(x N^{\alpha}) - pxN^{\alpha} ] \leq -[\lambda R^2 + \phi(\epsilon/16) + 1]N^{\alpha/2}$.
	\end{enumerate}
	It is clear that $D$ is countable, the events $E(\vec{x},\vec{y},\ell_{bot})$ are pairwise disjoint for different elements in $D$ and (\ref{21F}) is satisfied. \\
	
	
	\noindent\textbf{Step 3.} We claim that we can find $\tilde{N}_0$ so that for $N\geq \tilde{N}_0$ we have
	\begin{equation}\label{21AEbound}
		\mathbb{P}(A| E(\vec{x},\vec{y},\ell_{bot})) \geq 1/4
	\end{equation}
	for all $(\vec{x}, \vec{y}, \ell_{bot}) \in D$ such that $\mathbb{P}(E(\vec{x},\vec{y},\ell_{bot})) > 0$. We will prove (\ref{21AEbound}) in the steps below. In this step we assume its validity and conclude the proof of (\ref{21Bbound}). \\
	
	It follows from \eqref{21F} and \eqref{21AEbound} that for $N\geq\tilde{N}_0$ and $\mathbb{P}(F\cap B) > 0$ we have
	\begin{equation*}
		\begin{split}
			&\mathbb{P}(A|F \cap B) = \sum_{(\vec{x},\vec{y},\ell_{bot})\in D, \mathbb{P}(E(\vec{x},\vec{y},\ell_{bot}))) > 0} \frac{\mathbb{P}(A| E(\vec{x},\vec{y},\ell_{bot})\mathbb{P}( E(\vec{x},\vec{y},\ell_{bot}))}{\mathbb{P}(F \cap B)} \geq \\
			& \frac{1}{4}\cdot\frac{\sum_{(\vec{x},\vec{y},\ell_{bot})\in D, \mathbb{P}(E(\vec{x},\vec{y},\ell_{bot}))) > 0} \mathbb{P}( E(\vec{x},\vec{y},\ell_{bot}))}{\mathbb{P}(F\cap B)} = \frac{1}{4}.
		\end{split}
	\end{equation*}
	
	From the third condition in the definition of an $(\alpha,p,\lambda)$-good sequence, see Definition \ref{Def1}, we can find $\tilde{N}_1$ so that $\mathbb{P}(A) < \epsilon/8$ for $N\geq\tilde{N}_1$. Hence if $N \geq \max(\tilde{N}_1, \tilde{N}_2)$ and $\mathbb{P}(F\cap B) > 0$ we have
	\begin{equation}\label{NM1}
		\mathbb{P}(F \cap B) = \frac{\mathbb{P}(A\cap F \cap B)}{\mathbb{P}(A|F \cap B)} \leq 4\mathbb{P}(A) < \epsilon/2.
	\end{equation} 
	Lastly, by the same condition in Definition \ref{Def1} we can find $\tilde{N}_2$ so that for $N\geq\tilde{N}_2$ we have
	\begin{equation}\label{NM2}
		\mathbb{P}(F^c) = 2 \cdot \epsilon/8 = \epsilon/4.
	\end{equation}
	In deriving (\ref{NM2}) we used the fact that $|L_1^N(\gamma)  - L_1^N(rN^{\alpha})| \leq 1$, $|L_1^N(\Gamma)  - L_1^N(RN^{\alpha})| \leq 1$ and $p \in [0,1]$. Combining (\ref{NM1}) and (\ref{NM2}) we conclude that if $N \geq N_5 = \max(\tilde{N}_0, \tilde{N}_1, \tilde{N_2})$
	$$\mathbb{P}(B) \leq \mathbb{P}(F\cap B) + \mathbb{P}(F^c) \leq \epsilon/2 + \epsilon/4 < \epsilon,$$
	which proves (\ref{21Bbound}).\\
	
	\noindent\textbf{Step 4.} In this step we prove \eqref{21AEbound}. We define $\vec{x}\,',\vec{y}\,'\in\mathfrak{W}_{k-2}$ through
	\begin{equation}\label{21xybar}
		\begin{split}
			&x_i' = \overline{x} + (k-1-i)\lceil C\sqrt{T}\,\rceil, \quad y_i' = \overline{y} + (k-1-i)\lceil C\sqrt{T}\,\rceil \mbox{ for $i = 1, \dots, k-2$ with } \\
			&\overline{x} = \lceil  p\gamma  - \lambda \gamma^2 N^{-3\alpha/2} + [\phi(\epsilon/16) + 1]N^{\alpha/2} \rceil, \hspace{1mm} \overline{y} = \lceil p \Gamma - \lambda \Gamma^2 N^{-3\alpha/2} + [\phi(\epsilon/16) + 1]N^{\alpha/2} \rceil,
		\end{split}
	\end{equation} 
	where $C$ is as in (\ref{21Cdef}) and $T = \Gamma - \gamma$. Note that for any $(\vec{x}, \vec{y}, \ell_{bot}) \in D$ we have
	$$x_i' \geq \overline{x} \geq x_1 \geq x_i \mbox{ and }y_i' \geq \overline{y} \geq y_1 \geq y_i$$
	for each $i = 1, \dots, k-2$. Furthermore, 
	$$x_i' - x_{i+1}' \geq C\sqrt{T} \mbox{ and }y_i' - y_{i+1}' \geq C\sqrt{T}$$
	for all $i = 1, \dots, k-2$ with the convention $x_{k-1}' = \overline{x}$ and $y_{k-1}' = \overline{y}$. \\
	
	We claim that we can find $\tilde{N}_0$ so that for all $N\geq\tilde{N}_0$ and $(\vec{x}, \vec{y}, \ell_{bot}) \in D$ such that $\mathbb{P}(E(\vec{x},\vec{y},\ell_{bot})) > 0$ we have $\prod_{i = 1}^{k-2} |\Omega(\gamma, \Gamma, x_i', y_i')| \geq |\Omega_{avoid}(\gamma, \Gamma, \vec{x}', \vec{y}', \infty, \ell_{bot})| \geq 1$ and moreover we have
	\begin{equation}\label{21 1/3}
		\mathbb{P}^{\gamma, \Gamma,\vec{x}',\vec{y}'}_{Ber} \left(  Q_1\left(mN^{\alpha}\right) - pm N^\alpha  + \lambda m^2 N^{\alpha/2} < -\phi(\epsilon/16)N^{\alpha/2}  \right) \geq 1/3,
	\end{equation}
	\begin{equation}\label{21 1/12}
		\mathbb{P}^{\gamma, \Gamma,\vec{x}',\vec{y}'}_{Ber} \left( Q_1 \geq \cdots \geq Q_{k-1} \right) \geq 11/12,
	\end{equation}
	where $\mathfrak{Q} = (Q_1, \dots, Q_{k-2})$ is $\mathbb{P}^{\gamma, \Gamma,\vec{x}',\vec{y}'}_{Ber}$-distributed and we used the convention that $Q_{k-1} = \ell_{bot}$. We prove (\ref{21 1/3}) and (\ref{21 1/12}) in the steps below. In this step we assume their validity and conclude the proof of (\ref{21AEbound}).\\
	
	Observe that for any $(\vec{x}, \vec{y}, \ell_{bot}) \in D$ such that $\mathbb{P}(E(\vec{x},\vec{y},\ell_{bot})) > 0$ we the following tower of inequalities provided that $N \geq \tilde{N}_0$
	\begin{equation}\label{21xyest}
		\begin{split}
			&\mathbb{P}(A| E(\vec{x},\vec{y},\ell_{bot})) = \mathbb{P}^{\gamma, \Gamma,\vec{x},\vec{y},\infty,\ell_{bot}}_{avoid,Ber} \left( Q_1\left(mN^{\alpha}\right) - pm N^\alpha  + \lambda m^2 N^{\alpha/2} < -\phi(\epsilon/16)N^{\alpha/2} \right) \geq  \\
			& \mathbb{P}^{\gamma, \Gamma,\vec{x}',\vec{y}',\infty,\ell_{bot}}_{avoid,Ber} \left( Q_1\left(mN^{\alpha}\right) - pm N^\alpha  + \lambda m^2 N^{\alpha/2} < -\phi(\epsilon/16)N^{\alpha/2} \right) = \\
			&\frac{\mathbb{P}^{\gamma, \Gamma,\vec{x}',\vec{y}'}_{Ber} \left( \{ Q_1\left(mN^{\alpha}\right) - pm N^\alpha  + \lambda m^2 N^{\alpha/2} < -\phi(\epsilon/16)N^{\alpha/2} \} \cap \{Q_1 \geq \cdots \geq Q_{k-1} \} \right) }{\mathbb{P}^{\gamma, \Gamma,\vec{x}',\vec{y}'}_{Ber} \left( Q_1 \geq \cdots \geq Q_{k-1} \right) }.
		\end{split}
	\end{equation}
	Let us elaborate on (\ref{21xyest}) briefly. The condition that $\mathbb{P}(E(\vec{x},\vec{y},\ell_{bot})) > 0$ is required to ensure that the probabilities on the first line of (\ref{21xyest}) are well-defined and $N \geq \tilde{N}_0$ ensures that all other probabilities are also well-defined. The equality on the first line of (\ref{21xyest}) follows from the definition of $A$ and the Schur Gibbs property, see Definition \ref{DefSGP}, and $\mathfrak{Q} = (Q_1, \dots, Q_{k-2})$ is $\mathbb{P}^{\gamma, \Gamma,\vec{x},\vec{y},\infty,\ell_{bot}}_{avoid,Ber} $-distributed. The inequality in the first line of (\ref{21xyest}) follows from  Lemma \ref{MCLxy}, while the equality in the second line follows from Definition \ref{DefAvoidingLawBer}, and now $\mathfrak{Q} = (Q_1, \dots, Q_{k-2})$ is $\mathbb{P}^{\gamma, \Gamma,\vec{x}',\vec{y}'}_{Ber}$-distributed with the convention that $Q_{k-1} = \ell_{bot}$. 
	
	Combining (\ref{21 1/3}), (\ref{21 1/12}) and (\ref{21xyest}) we conclude that 
	$$\mathbb{P}(A| E(\vec{x},\vec{y},\ell_{bot}))  \geq 1/3- 1/12 = 1/4,$$
	which proves (\ref{21AEbound}).\\
	
	\noindent\textbf{Step 5.} In this step we prove \eqref{21 1/3}. We observe that since $\mathbb{P}(E(\vec{x},\vec{y},\ell_{bot})) > 0$ we know that $|\Omega_{avoid}(\gamma, \Gamma, \vec{x}, \vec{y}, \infty, \ell_{bot})| \geq 1$ and then we conclude from Lemma \ref{LemmaWD} that there exist $\hat{N}_1 \in \mathbb{N}$ such that for $N \geq \hat{N}_1$ we have $|\Omega_{avoid}(\gamma, \Gamma, \vec{x}', \vec{y}', \infty, \ell_{bot})| \geq 1$.
	
	Below $\ell$ will be used for a generic random variable with law $\mathbb{P}^{\cdot,\cdot,\cdot,\cdot}_{Ber}$, where the boundary data changes from line to line. With $\overline{x},\overline{y}$ as in \eqref{21xybar}, write $\overline{z} = \overline{y}-\overline{x}$ and recall that $T = \Gamma - \gamma$. Then
	\begin{equation} \label{21convex}
		\begin{split}
			& \mathbb{P}^{\gamma, \Gamma,x_1',y_1'}_{Ber} \left(\ell\left(mN^{\alpha}\right) - pm N^\alpha  + \lambda m^2 N^{\alpha/2} < -\phi(\epsilon/16)N^{\alpha/2}  \right) = \\
			& \mathbb{P}^{0,T,x_1',y_1'}_{Ber} \left(\ell(T/2) - pm N^\alpha  + \lambda m^2 N^{\alpha/2} < -\phi(\epsilon/16)N^{\alpha/2}\right) =\\
			& \mathbb{P}^{0,T,\overline{x},\overline{y}}_{Ber}\left(\ell(T/2) - pm N^\alpha  + \lambda m^2 N^{\alpha/2} < -\phi(\epsilon/16)N^{\alpha/2} - (k-2)\lceil C\sqrt{T}\rceil\right) \geq\\
			& \mathbb{P}^{0,T,\overline{x},\overline{y}}_{Ber}\left(\ell(T/2)   - \frac{\overline{x} + \overline{y}}{2}  <  \lambda \left(\frac{\gamma^2 + \Gamma^2}{2 N^{3\alpha/2}} \right)-[2\phi(\epsilon/16) + 1 +  \lambda m^2]N^{\alpha/2} - k \lceil C\sqrt{T} \rceil\right) = \\
			&\mathbb{P}^{0,T,0,\overline{z}}_{Ber}\left(\ell(T/2) - \overline{z}/2 < \lambda \left(\frac{\gamma^2 + \Gamma^2}{2 N^{3\alpha/2}} \right)-[2\phi(\epsilon/16) + 1 +  \lambda m^2]N^{\alpha/2} - k \lceil C\sqrt{T} \rceil\right).
		\end{split}
	\end{equation}
	The equalities in (\ref{21convex}) follow from shifting the boundary data of the curve $\ell$, while the inequality on the third line follows from the definition of $\overline{x},\overline{y}$ as in \eqref{21xybar}.
	
	From our choice of $R$ in Step 1 and the definition of $\gamma, \Gamma$ we know that 
	$$\lambda \frac{\gamma^2+\Gamma^2}{2N^{2\alpha}} -  \lambda m^2 \geq \lambda \frac{(R-r)^2}{4} - \frac{r\lambda}{N^{\alpha}} \geq 2\phi(\epsilon/16) + 2 + k \lceil C\sqrt{T} \rceil N^{-\alpha/2}- \frac{r\lambda}{N^{\alpha}} .$$
	$$ .$$
	The last inequality and (\ref{21convex}) imply 
	\begin{equation} \label{22convex}
		\begin{split}
			& \mathbb{P}^{\gamma, \Gamma,x_1',y_1'}_{Ber} \left(\ell\left(mN^{\alpha}\right) - pm N^\alpha  + \lambda m^2 N^{\alpha/2} < -\phi(\epsilon/16)N^{\alpha/2}  \right) \geq \\
			&\mathbb{P}^{0,T,0,\overline{z}}_{Ber}\left(\ell(T/2) - \overline{z}/2 < N^{\alpha/2} -  r\lambda N^{-\alpha/2} \right).
		\end{split}
	\end{equation}
	
	Let $\tilde{\mathbb{P}}$ be the probability measure on the space afforded by Theorem \ref{KMT}, supporting a random variable $\ell^{(T,\overline{z})}$ with law $\mathbb{P}^{0,T,0,\overline{z}}_{Ber}$ and a Brownian bridge $B^\sigma$ with variance $\sigma^2 = p(1-p)$. Then the probability in the last line of \eqref{21convex} is equal to
	\begin{equation} \label{23convex}
		\begin{split}
			&\mathbb{P}^{0,T,0,\overline{z}}_{Ber}\left(\ell(T/2) - \overline{z}/2 < N^{\alpha/2} -  r\lambda N^{-\alpha/2} \right) = \tilde{\mathbb{P}} \left( \ell^{(T,\overline{z})}(T/2) - \overline{z}/2 <N^{\alpha/2} -  r\lambda N^{-\alpha/2}    \right) \geq \\
			& \tilde{\mathbb{P}} \left( \sqrt{T}B^\sigma_{1/2} < 0 \mbox{ and }\Delta(T,\overline{z}) <N^{\alpha/2} -  r\lambda N^{-\alpha/2}     \right) \geq \frac{1}{2} - \tilde{\mathbb{P}} \left( \Delta(T,\overline{z}) \geq N^{\alpha/2} -  r\lambda N^{-\alpha/2} \right),
		\end{split}
	\end{equation}
	where we recall that $\Delta(T,\overline{z})$ is as in \eqref{KMTeq}. Since as $N \rightarrow \infty$ we have 
	\begin{equation*}
		T \sim (R-r) N^{\alpha} \mbox{ and }\frac{|\overline{z} - pT|^2}{T} \sim (R+r),
	\end{equation*}
	we conclude from Lemma \ref{Cheb} that there exists $\hat{N}_2 \in \mathbb{N}$ such that if $N \geq \max(\hat{N}_1, \hat{N}_2)$ we have
	\begin{equation} \label{24convex}
		\begin{split}
			\tilde{\mathbb{P}} \left( \Delta(T,\overline{z}) \geq N^{\alpha/2} -  r\lambda N^{-\alpha/2} \right) \leq \frac{1}{6}.
		\end{split}
	\end{equation}
	Combining (\ref{22convex}), (\ref{23convex}) and (\ref{24convex}) we obtain \eqref{21 1/3}. \\
	
	
	\noindent\textbf{Step 6.} In this last step, we prove \eqref{21 1/12}. Let $\overline{\ell}_{bot}$ be the straight segment connecting $\overline{x}$ and $\overline{y}$, defined in \eqref{21xybar}. By construction, we have that there is $\hat{N}_3 \in \mathbb{N}$ such that if $N \geq \hat{N}_3$ we have for any $(\vec{x}, \vec{y}, \ell_{bot}) \in D$ that $\ell_{bot}$ lies uniformly below the line segment $\overline{\ell}_{bot}$, which in turn lies at least $C \sqrt{T}$ below the straight segment connecting $x_{k-2}'$ and $y_{k-2}'$. If $\hat{N}_1$ is as in Step 5 we conclude from Lemma \ref{CurveSeparation} that there exists $\hat{N}_4 \in \mathbb{N}$ such that if $N \geq \max(\hat{N}_1, \hat{N}_3, \hat{N}_4)$ and $\mathbb{P}(E(\vec{x},\vec{y},\ell_{bot})) > 0$ 
	\begin{equation}\label{21UP}
		\mathbb{P}^{\gamma, \Gamma,\vec{x}',\vec{y}'}_{Ber} \left( Q_1 \geq \cdots \geq Q_{k-1} \right) \geq \left(1-3e^{-C^2/8p(1-p)}\right)^{k-2} = \frac{11}{12}.
	\end{equation}
	where the condition that $N \geq \hat{N}_1 $ is included to ensure that the probability $\mathbb{P}^{\gamma, \Gamma,\vec{x}',\vec{y}'}_{Ber} $ is well-defined. In deriving (\ref{21UP}) we also used (\ref{21Cdef}), which implies
	$$C = \sqrt{ 8p(1-p) \log\frac{3}{1-(11/12)^{1/(k-2)}}} \geq \sqrt{8p(1-p) \log 3}.$$
	We see that (\ref{21UP}) implies (\ref{21 1/12}), which concludes the proof of the lemma.
\end{proof}


In the remainder of this section we use Lemma \ref{21} to prove Lemma \ref{PropSup2}.
\begin{proof} (Lemma \ref{PropSup2}) For clarity we split the proof into five steps.\\ 
	
	\noindent\textbf{Step 1.} In this step we specify the choice of $R_2$ in the statement of the lemma and introduce some notation that will be used in the proof of the lemma, which is given in Steps 2,3 and 4 below. Throughout we fix $r, \epsilon > 0$. Define the constant
	\begin{equation}\label{4.3Cdef}
		C_1 = \sqrt{16p(1-p)\log\frac{3}{1-2^{-1/(k-1)}}}.
	\end{equation}
	Let $R > r + 3$, $M > 0$ and $\tilde{N}_1 \in \mathbb{N}$ be such that for $N \geq \tilde{N}_1$ we have that the event
	\begin{equation}\label{S5EventB}
		B = \left\{ \sup_{x\in [r+3, R] \cup [-R,-r-3]} \big(L^N_{k-1}(xN^\alpha) - pxN^\alpha\big) \geq -MN^{\alpha/2} \right\}
	\end{equation}
	satisfies 
	\begin{equation}\label{S5EventBineq}
		\mathbb{P} \left( B\right) \geq 1 - \epsilon/2.
	\end{equation}
	Such a choice of $R, M, \tilde{N}_1$ is possible by Lemma \ref{21}.
	
	Let us set 
	$$s^-_1 = \lceil - R \cdot N^{\alpha} \rceil, \hspace{2mm} s^-_2 =  \lfloor -(r+3) \cdot N^{\alpha} \rfloor, \hspace{2mm} s^+_1 = \lceil (r+3) \cdot N^{\alpha} \rceil, \hspace{2mm} s^+_2 =  \lfloor R \cdot N^{\alpha} \rfloor,$$
	and for $a \in \llbracket s_1^-, s_2^- \rrbracket$ and $b \in \llbracket s_1^+, s_2^+ \rrbracket$ we define $\vec{x}\,',\vec{y}\,' \in \mathfrak{W}_{k-1}$ by
	\begin{equation}\label{S5Boundary}
		\begin{split}
			x_i' &= \lfloor pa - MN^{\alpha/2}\rfloor - (i-1)\lceil C_1N^{\alpha/2}\rceil,\\
			y_i' &= \lfloor pb - MN^{\alpha/2}\rfloor - (i-1)\lceil C_1N^{\alpha/2}\rceil,
		\end{split}
	\end{equation}  
	for $i = 1, \dots, k-1$. We will write $\vec{z} = \vec{y}' - \vec{x}'$, and we note that $z_{k-1} \geq p (b-a) -1$ and also $2RN^{\alpha} \geq b-a \geq 2(r+3) N^{\alpha}$. The latter and Lemma \ref{LemmaMinFreeS4} imply that there exists $R_2 > 0$ and $\tilde{N}_2 \in \mathbb{N}$ such that if $N \geq \tilde{N}_2$ we have
	\begin{equation}\label{S5DefR2}
		\mathbb{P}^{0, b - a, 0, z_{k-1}}_{Ber} \left( \inf_{s\in[0,b-a]} \big(\ell(s) - ps\big) \leq -(R_2 - M - C_1k) N^{\alpha/2} \right) < \epsilon/4.
	\end{equation}
	This fixes our choice of $R_2$ in the statement of the lemma. \\
	
	With the above choice of $R_2$ we define the event
	\begin{equation}\label{S5EventA}
		A = \left\{\inf_{s \in [ -t_3, t_3 ]}\big[L^N_{k-1}(s) - p s \big] \leq - R_2N^{\alpha/2}\right\},
	\end{equation}
	and then to prove the lemma it suffices to show that there exists $N_4 \in \mathbb{N}$ such that for $N \geq N_4$ 
	\begin{equation}\label{4.3Abound}
		\mathbb{P}(A) < \epsilon
	\end{equation}
	
	
	\noindent\textbf{Step 2.} In this step, we prove that the event $B$ from (\ref{S5EventB}) can be written as a \textit{countable disjoint union} of the form
	\begin{equation}\label{4.3disj}
		B = \bigsqcup_{(a,b,\vec{x},\vec{y},\ell_{bot}, \ell_{top}^-, \ell_{top}^+)\in D} E(a,b,\vec{x},\vec{y},\ell_{bot},\ell_{top}^-, \ell_{top}^+),
	\end{equation}
	where the set $D$ and events $E(a,b,\vec{x},\vec{y},\ell_{bot},\ell_{top}^-, \ell_{top}^+)$ are defined below.
	
	For $a \in \llbracket s_1^-, s_2^- \rrbracket$ and $b \in \llbracket s_1^+, s_2^+ \rrbracket$, $\vec{x},\vec{y}\in\mathfrak{W}_{k-1}$, $z_1,z_2,z_1^-, z_2^+\in\mathbb{Z}$, $\ell_{bot} \in \Omega(a, b, z_1, z_2)$, $\ell_{top}^- \in \Omega(s_1^-, a, z_1^-, x_{k-1})$ , $\ell_{top}^+ \in \Omega(b, s_2^+, y_{k-1}, z_2^+)$ we define $E(a,b,\vec{x},\vec{y},\ell_{bot}, \ell_{top}^-, \ell_{top}^+)$ to be the event that $L_i^N(a) = x_i$ and $L_i^N(b) = y_i$ for $1\leq i\leq k-1$, and $L_k^N$ agrees with $\ell_{bot}$ on $\llbracket a,b \rrbracket$, $L_{k-1}^N$ agrees with $\ell_{top}^-$ on $\llbracket s_1^-, a \rrbracket$ and with $\ell_{top}^+$ on $\llbracket b, s_2^+ \rrbracket$.
	
	
	We also let $D$ be the collection of tuples $(a,b,\vec{x},\vec{y},\ell_{bot}, \ell_{top}^-, \ell_{top}^+)$ satisfying:
	\begin{enumerate}[label=(\arabic*)]
		\item $a \in \llbracket s_1^-, s_2^- \rrbracket$, $b \in \llbracket s_1^+, s_2^+ \rrbracket$;
		\item $\vec{x}, \vec{y} \in \mathfrak{W}_{k-1}$, $0 \leq y_i - x_i \leq b- a$, $x_{k-1} - pa > - MN^{\alpha/2}$, and $y_{k-1} - pb > - MN^{\alpha/2}$;
		\item if $c \in \llbracket s_1^-, s_2^- \rrbracket \cap (- \infty, a)$ then $\ell_{top}^-(c) - pc \leq -MN^{\alpha/2}$;
		\item if $d \in \llbracket s_1^+, s_2^+ \rrbracket \cap (b, \infty)$ then $\ell_{top}^+(d) - pd\geq -MN^{\alpha/2}$;
		\item $z_1\leq x_{k-1}$, $z_2\leq y_{k-1}$, and $\ell_{bot}\in\Omega(a,b,z_1,z_2)$.
	\end{enumerate} 
	It is clear that $D$ is countable, and that 
	$$B = \bigcup_{(a,b,\vec{x},\vec{y},\ell_{bot})\in D} E(a,b,\vec{x},\vec{y},\ell_{bot}, \ell_{top}^-, \ell_{top}^+),$$
	so to prove (\ref{4.3disj}) it suffices to show that the events $E(a,b,\vec{x},\vec{y},\ell_{bot}, \ell_{top}^-, \ell_{top}^+)$ are pairwise disjoint. Observe that on the intersection of $E(a,b,\vec{x},\vec{y},\ell_{bot}, \ell_{top}^-, \ell_{top}^+)$ and $E(\tilde a,\tilde b,\tilde{\vec{x}},\tilde{\vec{y}},\tilde{\ell}_{bot}, \tilde{\ell}_{top}^-, \tilde{\ell}_{top}^+)$, conditions (2) and (3) imply that $a = \tilde{a}$, while conditions (2) and (4) that $b = \tilde{b}$. Afterwards, we conclude that $\vec{x}=\tilde{\vec{x}}$, $\vec{y} = \tilde{\vec{y}}$, $\ell_{bot} = \tilde{\ell}_{bot}$, $\ell_{top}^- = \tilde{\ell}^-_{top}$ and $\ell_{top}^+ = \tilde{\ell}^+_{top}$, confirming \eqref{4.3disj}.\\
	
	
	\noindent\textbf{Step 3.} In this step we prove (\ref{4.3Abound}). We claim that we can find $\tilde{N}_3 \in \mathbb{N}$ such that if $N \geq \tilde{N}_3$ and $(a,b,\vec{x},\vec{y},\ell_{bot}, \ell_{top}^-, \ell_{top}^+) \in D$ is such that $\mathbb{P} \left( E(a,b,\vec{x},\vec{y},\ell_{bot}, \ell_{top}^-, \ell_{top}^+) \right) > 0$ we have
	\begin{equation}\label{4.3AEbound}
		\mathbb{P}(A\,|\,E(a,b,\vec{x},\vec{y},\ell_{bot}, \ell_{top}^-, \ell_{top}^+)) < \epsilon/2.
	\end{equation}
	We will prove (\ref{4.3AEbound}) in the steps below. Here we assume its validity and conclude the proof of (\ref{4.3Abound}).\\
	
	If $N \geq \max ( \tilde{N}_1, \tilde{N}_2, \tilde{N}_3)$ we have in view of \eqref{4.3disj} and \eqref{4.3AEbound} that 
	\begin{equation*}
		\begin{split}
			&\mathbb{P}(A) \leq \mathbb{P}(A \cap B) + \mathbb{P}(B^c) =\mathbb{P}(B^c) +  \sum_{\substack{(a,b,\vec{x},\vec{y},\ell_{bot}, \ell_{top}^-, \ell_{top}^+)\in D\\ \mathbb{P} \left( E(a,b,\vec{x},\vec{y},\ell_{bot}, \ell_{top}^-, \ell_{top}^+) \right) > 0}}  \mathbb{P}(A|E(a,b,\vec{x},\vec{y},\ell_{bot}, \ell_{top}^-, \ell_{top}^+)) \times \\
			&\mathbb{P}(E(a,b,\vec{x},\vec{y},\ell_{bot}, \ell_{top}^-, \ell_{top}^+)) \leq \mathbb{P}(B^c) + \frac{\epsilon}{2}\sum_{\substack{(a,b,\vec{x},\vec{y},\ell_{bot}, \ell_{top}^-, \ell_{top}^+)\in D\\ \mathbb{P} \left( E(a,b,\vec{x},\vec{y},\ell_{bot}, \ell_{top}^-, \ell_{top}^+) \right) > 0}}  \mathbb{P}(E(a,b,\vec{x},\vec{y},\ell_{bot}, \ell_{top}^-, \ell_{top}^+)) =\\
			&\mathbb{P}(B^c) + \frac{\epsilon}{2}\cdot  \mathbb{P}(B) < \epsilon,
		\end{split}
	\end{equation*}
	where in the last inequality we used (\ref{S5EventBineq}). The above inequality clearly implies (\ref{4.3Abound}).\\
	
	\noindent\textbf{Step 4.} In this step we prove \eqref{4.3AEbound}. We claim that there exists $\tilde{N}_4 \in \mathbb{N}$ such that if $N \geq \tilde{N}_4$, $a \in \llbracket s_1^-, s_2^- \rrbracket$ and $b \in \llbracket s_1^+, s_2^+ \rrbracket$ we have that $\prod_{i = 1}^{k-1} |\Omega(a, b,x_i', y_i')| \geq 1$ and
	\begin{equation}\label{S5T1}
		\mathbb{P}^{a, b, \vec{x}', \vec{y}'}_{Ber}(Q_1 \geq \cdots \geq Q_{k-1}) \geq \frac{1}{2},
	\end{equation}
	where $\mathfrak{Q} = (Q_1, \dots, Q_{k-1})$ is $\mathbb{P}^{a, b, \vec{x}', \vec{y}'}_{Ber}$-distributed and we recall that $\vec{x}'$, $\vec{y}'$ were defined in (\ref{S5Boundary}).
	We will prove (\ref{S5T1}) in Step 5 below. Here we assume its validity and conclude the proof of (\ref{4.3AEbound}).\\
	
	
	Observe that by condition (2) in Step 2, we have that $x_i'\leq pa - MN^{\alpha/2} \leq x_{k-1} \leq x_i$, and similarly $y_i' \leq pb - MN^{\alpha/2} \leq y_{k-1} \leq y_i$ for $i = 1, \dots, k-1$. From this observation we conclude that if $N \geq \tilde{N}_4$ is sufficiently large and $(a,b,\vec{x},\vec{y},\ell_{bot}, \ell_{top}^-, \ell_{top}^+) \in D$ is such that $\mathbb{P} \left( E(a,b,\vec{x},\vec{y},\ell_{bot}, \ell_{top}^-, \ell_{top}^+) \right) > 0$ we have
	\begin{equation}\label{4.3main}
		\begin{split}
			&\mathbb{P}(A|E(a,b,\vec{x},\vec{y},\ell_{bot},\ell_{top}^-, \ell_{top}^+)) \leq \\
			&\mathbb{P}\left( \inf_{s\in[a, b]} \left(L_{k-1}^N(s) - ps\right) \leq -R_2N^{\alpha/2}\, \big| \, E(a,b,\vec{x},\vec{y},\ell_{bot},\ell_{top}^-, \ell_{top}^+) \right) = \\
			&\mathbb{P}^{a,b, \vec{x}, \vec{y},\infty,\ell_{bot}}_{avoid, Ber} \left( \inf_{s\in[a, b]} \left(Q_{k-1}(s) - ps\right) \leq -R_2N^{\alpha/2} \right) \leq\\
			& \mathbb{P}^{a, b, \vec{x}', \vec{y}'}_{avoid, Ber} \left( \inf_{s\in[a,b]} \big(Q_{k-1}(s) - ps\big) \leq -R_2N^{\alpha/2} \right) = \\
			&\frac{\mathbb{P}^{a, b, \vec{x}', \vec{y}'}_{Ber} \left( \{ \inf_{s\in[a,b]} \big(Q_{k-1}(s) - ps\big) \leq -R_2 N^{\alpha/2} \}  \cap \{ Q_1 \geq \cdots \geq Q_{k-1} \} \right)}{\mathbb{P}^{a, b, \vec{x}\,', \vec{y}\,'}_{Ber}(Q_1 \geq \cdots \geq Q_{k-1})} \leq \\
			&\frac{\mathbb{P}^{a, b, \vec{x}', \vec{y}'}_{Ber} \left( \inf_{s\in[a,b]} \big(Q_{k-1}(s) - ps\big) \leq -R_2 N^{\alpha/2} \right)}{\mathbb{P}^{a, b, \vec{x}\,', \vec{y}\,'}_{Ber}(Q_1 \geq \cdots \geq Q_{k-1})}.
		\end{split}
	\end{equation}
	Let us elaborate on (\ref{4.3main}) briefly. The first inequality in (\ref{4.3main}) follows from the definition of $A$ and the fact that $a \leq -t_3$ while $b \geq t_3$ by construction. The condition $\mathbb{P} \left( E(a,b,\vec{x},\vec{y},\ell_{bot}, \ell_{top}^-, \ell_{top}^+) \right) > 0$ ensures that the first three probabilities in (\ref{4.3main}) are all well-defined. The equality on the second line follows from the Schur Gibbs property and the inequality on the third line follows from Lemmas \ref{MCLxy} and \ref{MCLfg} since $x_i' \leq x_i$ and $y_i' \leq y_i$ by construction. To ensure that the probability in the fourth line is well-defined (and hence Lemmas \ref{MCLxy} and \ref{MCLfg} are applicable) it suffices to assume that $N \geq \tilde{N}_4$, in view of Lemma \ref{LemmaWD}. The equality on the fourth line follows from the definition of $\mathbb{P}^{a, b, \vec{x}', \vec{y}'}_{avoid, Ber} $, see Definition \ref{DefAvoidingLawBer} and the last inequality is trivial.\\
	
	By our choice of $R_2$, see (\ref{S5DefR2}), we know that there is $\tilde{N}_5 \in \mathbb{N}$ such that if $N \geq \tilde{N}_5$
	\begin{equation}\label{4.3main2}
		\begin{split}
			&\mathbb{P}^{a, b, \vec{x}', \vec{y}'}_{Ber} \left( \inf_{s\in[a,b]} \big(Q_{k-1}(s) - ps\big) \leq -R_2 N^{\alpha/2} \right) = \\
			&\mathbb{P}^{0, b-a, 0, z_{k-1}}_{Ber} \left( \inf_{s\in[0,b-a]} \big(\ell(s) - ps\big) \leq -R_2 N^{\alpha/2} - x_{k-1}'\right) \leq \\
			&\mathbb{P}^{0, b-a, 0, z_{k-1}}_{Ber} \left( \inf_{s\in[0,b-a]} \big(\ell(s) - ps\big) \leq -(R_2 - M - C_1k) N^{\alpha/2} \right) < \epsilon/4.
		\end{split}
	\end{equation}
	Combining (\ref{S5T1}), (\ref{4.3main}) and (\ref{4.3main}) we conclude that for $N \geq \tilde{N}_3 = \max(\tilde{N}_4, \tilde{N}_5)$ we have
	$$\mathbb{P}(A|E(a,b,\vec{x},\vec{y},\ell_{bot},\ell_{top}^-, \ell_{top}^+)) < 2 \cdot \epsilon/4 = \epsilon/2,$$
	which implies (\ref{4.3AEbound}).\\
	
	\noindent\textbf{Step 5.} In this final step we prove (\ref{S5T1}).
	
	Lastly, we prove that we can enlarge $\tilde{N}_1$ so that \eqref{4.3intersect} holds for $N\geq\tilde{N}_1$. Write $a = a'N^\alpha, b = b'N^\alpha$, and $T = a+b = (a'+b')N^\alpha$. Also let $C' = C/\sqrt{a'+b'}$ with $C$ as in \eqref{4.3Cdef}, so that $x_i' - x_{i+1}' \geq CN^{\alpha/2} = C'\sqrt{T}$ and likewise for $y_i'$. Note that $|z_{k-1}-pT| \leq 1$. It follows from Lemma \ref{CurveSeparation}, applied with $\ell_{bot} = -\infty$ and $C'$ in place of $C$, that for $T$ larger than some $T_0$, 
	\begin{equation}\label{4.3avoid}
		\begin{split}
			& \mathbb{P}^{-a, b, \vec{x}\,', \vec{y}\,'}_{Ber}(L_1 \geq \cdots \geq L_{k-1}) = \mathbb{P}^{0, a+b, \vec{x}\,', \vec{y}\,'}_{Ber}(L_1 \geq \cdots \geq L_{k-1}) \geq\\ & \left(1 - 3e^{-(C')^2/8p(1-p)}\right)^{k-1} \geq \left(1 - 3e^{-C^2/16p(1-p)R}\right)^{k-1}.
		\end{split}
	\end{equation}
	Here, we used the fact that $a'+b' \leq 2R$, hence $C' \geq C/\sqrt{2R}$. The constant $T_0$ depends in particular on $C'$, hence possibly on $a+b$. Referring to the proofs of Lemmas \ref{CurveSeparation} and \ref{Cheb}, we see that the dependency of $T_0$ on $C'$ amounts to requiring that $e^{-C'\sqrt{T_0}}$ be sufficiently small. But $C' \geq C/\sqrt{2R}$, so for this it suffices to choose $T_0$ depending on $C$ and $R$. Moreover, $T\geq 2rN^\alpha$, so as long as $\tilde{N}_1 \geq (T_0/2r)^{1/\alpha}$, we have the bound in \eqref{4.3avoid} for $N\geq\tilde{N}_1$ independent $a,b,\vec{x},\vec{y}$. Our choice of $C$ in \eqref{4.3Cdef} ensures that the expression on the right in \eqref{4.3avoid} is at least $1/2$, proving \eqref{4.3intersect}.
	
	
	In this step, we fix $R_2 > 0$ and $\tilde{N}_1$ so that for $N\geq\tilde{N}_1$, we have
	\begin{align}
		&\mathbb{P}^{-a, b, \vec{x}\,', \vec{y}\,'}_{Ber} \left( \inf_{s\in[0,a+b]} \big(L_{k-1}(s) - ps\big) \leq -R_2 N^{\alpha/2} \right) < \epsilon/4, \label{4.3free}\\
		&\mathbb{P}^{-a, b, \vec{x}\,', \vec{y}\,'}_{Ber}(L_1 \geq \cdots \geq L_{k-1}) \geq 1/2. \label{4.3intersect}
	\end{align}
	Let us first prove \eqref{4.3free}. Writing $\vec{z} = \vec{y}\,' - \vec{x}\,'$, and using the fact that $L_1,\dots,L_{k-1}$ are independent under $\mathbb{P}^{-a, b, \vec{x}\,', \vec{y}\,'}_{Ber}$, we can rewrite the left hand side of \eqref{4.3free} as
	\begin{equation}\label{4.3R2}
		\begin{split}
			& \mathbb{P}^{0, a+b, x_{k-1}', y_{k-1}'}_{Ber} \left( \inf_{s\in[0,a+b]} \big(\ell(s) - p(s-a)\big) \leq -R_2 N^{\alpha/2} \right) =\\
			&\mathbb{P}^{0, a+b, 0, z_{k-1}}_{Ber} \left( \inf_{s\in[0,a+b]} \left(\ell(s) - ps + pa - \lceil pa + MN^{\alpha/2}\rceil - (k-2)\lceil CN^{\alpha/2}\rceil\right) \leq -R_2 N^{\alpha/2} \right) \leq \\
			& \mathbb{P}^{0, a+b, 0, z_{k-1}}_{Ber} \left( \inf_{s\in[0,a+b]} \big(\ell(s) - ps\big) \leq -(R_2 - M - Ck) N^{\alpha/2} \right).
		\end{split}
	\end{equation}
	
	
	
\end{proof}