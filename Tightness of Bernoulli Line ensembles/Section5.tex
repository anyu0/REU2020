%-------------------------------------------------------------------------------------------------------------------------------------------------------------------------------------------------
%    Section 5
%
%-------------------------------------------------------------------------------------------------------------------------------------------------------------------------------------------------
\section{Bounding the max and min}\label{Section5}

In this section we prove Lemmas \ref{PropSup} and \ref{PropSup2} and we assume the same notation as in the statements of these lemmas. In particular, we assume that $k \in \mathbb{N}$, $k \geq 2$, $p \in (0,1)$, $\alpha, \lambda > 0$ are all fixed and 
\begin{equation*}
\big\{\mathfrak{L}^N = (L^N_1,L^N_2, \dots, L^N_k)\big\}_{N=1}^{\infty},
\end{equation*}
is an $(\alpha,p,\lambda)$-good sequence of $\llbracket 1, k\rrbracket$-indexed Bernoulli line ensembles as in Definition \ref{Def1} that are all defined on a probability space with measure $\mathbb{P}$. The proof of Lemma \ref{PropSup} is given in Section \ref{Section5.1} and Lemma \ref{PropSup2} is proved in Section \ref{Section5.2}.


%-------------------------------------------------------------------------------------------------------------------------------------------------------------------------------------------------
%    Section 5.1
%
%-------------------------------------------------------------------------------------------------------------------------------------------------------------------------------------------------
\subsection{Proof of Lemma \ref{PropSup}}\label{Section5.1}

Our proof of Lemma \ref{PropSup} is similar to that of \cite[Lemma 5.2]{CD}. For clarity we split the proof into three steps. In the first step we introduce some notation that will be required in the proof of the lemma, which is presented in Steps 2 and 3. \\

{\bf \raggedleft Step 1.} We write $s_4 = \lceil r+4 \rceil N^\alpha$, $s_3 = \lfloor r+3 \rfloor N^\alpha$, so that $s_3 \leq t_3 \leq s_4$, and assume that $N$ is large enough so that $\psi(N)N^{\alpha}$ from Definition \ref{Def1} is at least $s_4$. Notice that such a choice is possible by our assumption that $\mathfrak{L}^N$ is an $(\alpha,p,\lambda)$-good sequence and in particular, we know that $L_i^N$ are defined at $\pm s_4$ for $i \in \llbracket 1, k \rrbracket$. We define events 
$$E(M) = \Big\{\big|L_1^N(-s_4) + ps_4\big| > MN^{\alpha/2}\Big\}, \quad F(M) = \Big\{L_1^N(-s_3) > -ps_3 + MN^{\alpha/2} \Big\},$$
$$G(M) = \Bigg\{\sup_{s\in[0,s_4]} \big[L_1^N(s) - ps \big] \geq (6r+22)(2r+10)^{1/2}(M+1)N^{\alpha/2} \Bigg\}.$$

If $\epsilon > 0$ is as in the statement of the lemma, we note by (\ref{globalParabola}) that we can find $M$ and $\tilde{N}_1$ sufficiently large so that if $N \geq \tilde{N}_1$ we have 
\begin{equation}\label{4.2EFbounds}
\mathbb{P}(E(M)) < \epsilon/4 \mbox{ and } \mathbb{P}(F(M)) < \epsilon/12.
\end{equation}
In the remainder of this step we show that the event $G(M) \setminus E(M)$ can be written as a {\em countable disjoint union} of certain events, i.e. we show that
\begin{equation}\label{WR3}
\bigsqcup\limits_{(a,b,s,\ell_{top},\ell_{bot}) \in D(M)}  E(a,b,s,\ell_{top},\ell_{bot}) = G(M) \setminus E(M),
\end{equation}
where the sets $E(a,b,s,\ell_{top},\ell_{bot})$ and $D(M)$ are described below. 

For $a,b,z_1,z_2,z_3 \in\mathbb{Z}$ with $z_1\leq a$, $z_2\leq b$, $s\in\llbracket 0, s_4 \rrbracket$, $\ell_{bot}\in\Omega(-s_4,s,z_1,z_2)$ and $\ell_{top}\in\Omega(s,s_4,b,z_3)$ we define $E(a,b,s,\ell_{top},\ell_{bot})$ to be the event that $L_1^N(-s_4) = a$, $L_1^N(s) = b$, $L_1^N$ agrees with $\ell_{top}$ on $\llbracket s,s_4\rrbracket$, and $L_2^N$ agrees with $\ell_{bot}$ on $\llbracket -s_4,s \rrbracket$.  Let $D(M)$ be the set of tuples $(a,b,s, \ell_{top}, \ell_{bot})$ satisfying
\begin{enumerate}[label=(\arabic*)]
	\item $0\leq s\leq s_4$,
	\item $0\leq b-a \leq s + s_4$, $|a + ps_4| \leq MN^{\alpha/2}$, and $b-ps \geq (6r+22)(2r+10)^{1/2}(M+1)N^{\alpha/2}$,
	\item $z_1\leq a$, $z_2\leq b$, and $\ell_{bot}\in\Omega(-s_4, s, z_1, z_2)$,
	\item $b \leq z_3 \leq b + (s_4 - s)$, and $\ell_{top} \in\Omega(s, s_4, b, z_3)$,
	\item if $s < s' \leq s_4$, then $\ell_{top}(s') -ps' < (6r+22)(2r+10)^{1/2}(M+1)N^{\alpha/2}$.
\end{enumerate}
It is clear that $D(M)$ is countable. The five conditions above together imply that 
$$\bigcup_{(a,b,s,\ell_{top},\ell_{bot}) \in D(M)}  E(a,b,s,\ell_{top},\ell_{bot}) = G(M) \setminus E(M),$$
and what remains to be shown to prove (\ref{WR3}) is that $E(a,b,s,\ell_{top},\ell_{bot})$ are pairwise disjoint. 

On the intersection of $E(a,b,s,\ell_{top},\ell_{bot})$ and $E(\tilde{a},\tilde{b},\tilde{s},\tilde{\ell}_{top},\tilde{\ell}_{bot})$ we must have $\tilde{a} = L_1^N(-s_4) = a$ so that $a = \tilde{a}$. Furthermore, we have by properties (2) and (5) that $s \geq \tilde{s}$ and $\tilde{s} \geq s$ from which we conclude that $s= \tilde{s}$ and then we conclude $\tilde{b} = b$, $\ell_{top} = \tilde{\ell}_{top}$, $\ell_{bot} = \tilde{\ell}_{bot}$. In summary, if $E(a,b,s,\ell_{top},\ell_{bot})$ and $E(\tilde{a},\tilde{b},\tilde{s},\tilde{\ell}_{top},\tilde{\ell}_{bot})$ have a non-trivial intersection then $(a,b,s,\ell_{top},\ell_{bot}) = (\tilde{a},\tilde{b},\tilde{s},\tilde{\ell}_{top},\tilde{\ell}_{bot})$, which proves (\ref{WR3}).\\

{\raggedleft \bf Step 2.} In this step we prove that  we can find an $N_2$ so that for $N \geq N_2$
\begin{equation}\label{WR1}
\mathbb{P}\left( \sup_{s \in [0,t_3] }\big[ L^N_1(s) - p s \big] \geq  (6r+22)(2r+10)^{1/2}(M+1)N^{\alpha/2} \right) \leq \mathbb{P}(G(M)) < \epsilon/2.
\end{equation}
A similar argument, which we omit, proves the same inequality with $[-t_3,0]$ in place of $[0,t_3]$ and then the statement of the lemma holds for all $N\geq N_2$, with $R_1 = (6r+22)(2r+10)^{1/2}(M+1)$.

We claim that we can find $\tilde{N}_2 \in \mathbb{N}$ sufficiently large so that if $N \geq \tilde{N}_2$ and $(a,b,s,\ell_{top},\ell_{bot})\in D(M)$ satisfies $\mathbb{P}( E(a,b,s,\ell_{top},\ell_{bot})) > 0$ then we have
\begin{equation}\label{WR7}
\mathbb{P}^{-s_4,s,a,b,\infty,\ell_{bot}}_{avoid, Ber}\left(\ell(-s_3) > -ps_3 + MN^{\alpha/2}\right) \geq \frac{1}{3}.
\end{equation}
We will prove (\ref{WR7}) in Step 3. For now we assume its validity and conclude the proof of (\ref{WR1}).\\

Let $(a,b,s,\ell_{top},\ell_{bot})\in D(M)$ be such that $\mathbb{P}( E(a,b,s,\ell_{top},\ell_{bot})) > 0$. By the Schur Gibbs property, see Definition \ref{DefSGP}, we have for any $\ell_0\in\Omega(-s_4,s,a,b)$ that
\begin{equation}\label{WR9}
\mathbb{P}\left(L_1^N\llbracket -s_4,s \rrbracket = \ell_0\,|\,E(a,b,s,\ell_{top}, \ell_{bot}) \right) = \mathbb{P}^{-s_4,s,a,b,\infty,\ell_{bot}}_{avoid, Ber}(\ell = \ell_0),
\end{equation}
where $L_1^N\llbracket -s_4,s \rrbracket$ denotes the restriction of $L_1^N$ to the set $\llbracket -s_4,s \rrbracket$.

Combining (\ref{WR7}) and (\ref{WR9}) we get for $N \geq \tilde{N}_2$
\begin{equation}\label{WR10}
\begin{split}
&\mathbb{P}\left( L_1^N(-s_3) > -ps_3 + MN^{\alpha/2} \vert E(a,b,s, \ell_{top}, \ell_{bot})\right) = \\
&\mathbb{P}^{-s_4,s,a,b,\infty,\ell_{bot}}_{avoid, Ber}\left(\ell(-s_3) > -ps_3 + MN^{\alpha/2}\right) \geq \frac{1}{3}.
\end{split}
\end{equation}

It follows from (\ref{WR10}) that for $N \geq \tilde{N}_2$ we have
\begin{equation}\label{WR11}
\begin{split}
& \epsilon/12 > \mathbb{P}(F(M)) \geq \sum_{\substack{(a,b,s,\ell_{top}, \ell_{bot})\in D(M), \\ \mathbb{P}(E(a,b,s, \ell_{top}, \ell_{bot})) > 0}} \mathbb{P}\left(F(M)\cap E(a,b,s, \ell_{top}, \ell_{bot})\right) = \\
&\sum_{\substack{(a,b,s,\ell_{top}, \ell_{bot})\in D(M), \\ \mathbb{P}(E(a,b,s, \ell_{top}, \ell_{bot})) > 0}} \hspace{-5mm} \mathbb{P}\left( L_1^N(-s_3) > -ps_3 + MN^{\alpha/2} \vert  E(a,b,s,\ell_{top}, \ell_{bot}) \right )\mathbb{P}\left(E(a,b,s,\ell_{top}, \ell_{bot}) \right) \geq \\
& \sum_{\substack{(a,b,s,\ell_{top}, \ell_{bot})\in D(M), \\ \mathbb{P}(E(a,b,s, \ell_{top}, \ell_{bot})) > 0}} \frac{1}{3} \cdot \mathbb{P}\left(E(a,b,s,\ell_{top}, \ell_{bot}) \right)  =  \frac{1}{3} \cdot \mathbb{P}(G(M)\setminus E(M)),
\end{split}
\end{equation}
where in the last equality we used (\ref{WR3}). From (\ref{4.2EFbounds}) and (\ref{WR11}) we have for $N \geq N_2 = \max(\tilde{N}_1, \tilde{N}_2)$  that 
$$\mathbb{P}(G(M)) \leq \mathbb{P}(E(M)) + \mathbb{P}(G(M)\setminus E(M))< \epsilon/4 + \epsilon/4,$$
which proves (\ref{WR1}).\\


{\bf \raggedleft Step 3.} In this step we prove (\ref{WR7}) and in the sequel we let $(a,b,s,\ell_{top},\ell_{bot})\in D(M)$ be such that $\mathbb{P}( E(a,b,s,\ell_{top},\ell_{bot})) > 0$. We remark that the condition $\mathbb{P}( E(a,b,s,\ell_{top},\ell_{bot})) >0 $ implies that $\Omega_{avoid}({-s_4,s,a,b,\infty,\ell_{bot}})$ is not empty. By Lemma \ref{MCLfg} we know that 
$$\mathbb{P}^{-s_4,s,a,b,\infty,\ell_{bot}}_{avoid, Ber}\left(\ell(-s_3) > -ps_3 + MN^{\alpha/2}\right) \geq \mathbb{P}^{-s_4,s,a,b}_{ Ber}\left(\ell(-s_3) > -ps_3 + MN^{\alpha/2}\right),$$
and so it suffices to show that 
\begin{equation}\label{WT0}
\mathbb{P}^{-s_4,s,a,b}_{Ber}\left(\ell(-s_3) > -ps_3 + MN^{\alpha/2}\right) \geq \frac{1}{3}.
\end{equation}

One directly observes that
\begin{equation}\label{WT2}
\begin{split}
&\mathbb{P}^{-s_4,s,a,b}_{Ber}\left( \ell(-s_3) > -ps_3 + MN^{\alpha/2}\right) = \mathbb{P}^{0,s+s_4,0,b-a}_{Ber}\left(\ell(s_4-s_3) + a \geq -ps_3 + MN^{\alpha/2}\right) \geq\\
& \mathbb{P}^{0,s+s_4,0,b-a}_{Ber}\left(\ell(s_4-s_3) \geq p(s_4-s_3) + 2MN^{\alpha/2}\right),
\end{split}
\end{equation}
where the inequality follows from the assumption in (2) that $a+ps_4 \geq -MN^{\alpha/2}$. Moreover, since $b-ps \geq (6r+22)(2r+10)^{1/2}(M+1)N^{\alpha/2}$ and $a+ps_4 \leq MN^{\alpha/2}$, we have 
$$b-a \geq p(s+s_4) + (6r+21)(2r+10)^{1/2}(M+1)N^{\alpha/2} \geq p(s+s_4) + (6r+21)(M+1)(s+s_4)^{1/2}.$$
The second inequality follows since $s+s_4 \leq 2s_4 \leq (2r+10)N^{\alpha}$. 

It follows from Lemma \ref{LemmaHalfS4} with $M_1 = 0$, $M_2 = (6r+21)(M+1)$ that for sufficiently large $N$
\begin{equation}\label{WT3}
\mathbb{P}^{0,s+s_4,0,b-a}_{Ber}\Big(\ell(s_4-s_3) \geq \frac{s_4-s_3}{s+s_4}[p(s+s_4) + M_2 (s+s_4)^{1/2}] - (s+s_4)^{1/4}\Big) \geq 1/3.
\end{equation}

Note that $\frac{s_4-s_3}{s+s_4} \geq \frac{N^\alpha }{(2r+10)N^\alpha} = \frac{1}{2r+10}$ and so for all $N \in \mathbb{N}$ we have
\begin{equation}\label{WT4}
\begin{split}
& \frac{s_4-s_3}{s+s_4}[p(s+s_4) + M_2 (s+s_4)^{1/2}] - (s+s_4)^{1/4} \geq \\
& p(s_4-s_3) + \frac{(6r+21)(M+1) (s+s_4)^{1/2}}{2r+10}  - (s+s_4)^{1/4} \geq p(s_4-s_3) + 2MN^{\alpha/2}.
\end{split}
\end{equation}
Combining (\ref{WT2}), (\ref{WT3}) and (\ref{WT4}) we conclude that we can find $\tilde{N}_2 \in \mathbb{N}$ such that if $N \geq \tilde{N}_2$ we have (\ref{WT0}). This suffices for the proof.


%-------------------------------------------------------------------------------------------------------------------------------------------------------------------------------------------------
%    Section 5.2
%
%-------------------------------------------------------------------------------------------------------------------------------------------------------------------------------------------------
\subsection{Proof of Lemma \ref{PropSup2}}\label{Section5.2}

	We begin by proving the following important lemma, which allows us to prevent the curve $L_{k-1}^N$ from falling too low on a large enough interval.
	
	\begin{lemma}\label{21}
		Fix $p\in (0,1)$, $k\in\mathbb{N}$, and $\alpha,\lambda > 0$. Suppose that $\mathfrak{L}^N = (L_1^N, \dots, L_k^N)$ is a $(\alpha,p,\lambda)$-good sequence of $\llbracket 1, k\rrbracket$-indexed Bernoulli line ensembles. Then for any $r,\epsilon>0$, there exist $R>0$ and $N_5 \in \mathbb{N}$ such that for all $N\geq N_5$,
		\[
		\mathbb{P}\left(\max_{x\in[r,R]} \left(L_{k-1}^N(xN^\alpha) - pxN^\alpha\right) \leq -(\lambda R^2 + \phi(\epsilon/16))N^{\alpha/2}\right) < \epsilon.
		\]
		The same statement holds if $[r,R]$ is replaced with $[-R,-r]$.
	\end{lemma}

	\begin{remark}
		The key to this lemma is the parabolic shift implicit in the definition of an $(\alpha,p,\lambda)$-good sequence. This requires the deviation of the top curve from the line of slope $p$ to appear roughly parabolic. Using monotone coupling, we separate the curves of the ensemble so that $L_1^N$ is nearly independent of the other curves. Then we would expect the value of $L_1^N$ at the midpoint of $r$ and $R$ to be close to the midpoint of the straight line segment connecting two points of the parabola. But the parabola is convex, so for large enough $R$ this violates the one-point tightness assumpion at $(R+r)/2$.
	\end{remark}
	
	\begin{proof}
		
		For clarity we split this proof into five steps. The proof of the analogous statement with $[r,R]$ replaced by $[-R,-r]$ is very similar. In particular, only trivial modifications to Steps 4 and 5 are necessary. In the first step we introduce notation that will be used in the proof of the lemma in the subsequent steps.\\
		
		\noindent\textbf{Step 1.} Fix $r>0$. Note that for any $R>r$,
		\[
		\max_{r\leq x\leq R} \big(L_{k-1}^N(xN^\alpha) - pxN^\alpha\big) \geq \max_{\lceil r\rceil \leq x \leq R} \big(L_{k-1}^N(xN^\alpha) - pxN^\alpha\big).
		\]
		Thus by replacing $r$ with $\lceil r\rceil$, we can assume that $r\in\mathbb{Z}$. Define constants
		\begin{align}
		C &= \sqrt{ 8p(1-p) \log\frac{3}{1-(11/12)^{1/(k-2)}}}\,,\label{21Cdef}\\
		R_0 &= \frac{2}{\lambda}\left(Ck+\sqrt{C^{2}k^{2}+2\lambda\phi(\epsilon/16)}\right) + r + 1. \label{21Rdef}
		\end{align}
		Note that $R_0\geq r$. We define $R = \lceil R_0\rceil + \mathbf{1}_{\lceil R_0\rceil + r\;\mathrm{odd}}$, so that $R\geq R_0$ and the midpoint $(R+r)/2$ are integers. In the following, we always assume $N$ is large enough so that $\psi(N) > R$, hence $L_i^N$ are defined at $RN^\alpha$ for $1\leq i\leq k$. We may do so by the second condition in the definition of an $(\alpha,p,\lambda)$-good sequence (see Definition \ref{Def1}). Define events
		\begin{equation}\label{21AB}
		\begin{split}
		A &= \left\{L_1^N\left(\frac{R+r}{2}\,N^\alpha\right) - pN^\alpha\,\frac{R+r}{2} + \lambda\left(\frac{R+r}{2}\right)^2 N^{\alpha/2} < -\phi(\epsilon/16)N^{\alpha/2}\right\},\\
		B &= \left\{\max_{x\in[r,R]} \left(L_{k-1}^N(xN^\alpha) - pxN^\alpha\right) \leq -(\lambda R^2 + \phi(\epsilon/16)) N^{\alpha/2} \right\}.
		\end{split}
		\end{equation}
		We will argue in the subsequent steps that we can find $N_5\in\mathbb{N}$ so that for all $N\geq N_5$,
		\begin{equation}\label{21Bbound}
		\mathbb{P}(B) < \epsilon,
		\end{equation}
		which will prove the lemma. Let $F$ denote the subset of $B$ for which the inequalities
		\begin{equation}\label{21x1y1}
		\begin{split}
		& N^{-\alpha/2}\left|L_1^N(rN^\alpha) - prN^\alpha + \lambda r^2N^{\alpha/2}\right| < \phi(\epsilon/16),\\
		& N^{-\alpha/2}\left|L_1^N(RN^\alpha) - pRN^\alpha + \lambda R^2N^{\alpha/2}\right| < \phi(\epsilon/16)
		\end{split}
		\end{equation}
		hold. In the remainder of this step we show that $F$ can be written as a \textit{countable disjoint union} of the form
		\begin{equation}\label{21F}
		F = \bigsqcup_{(\vec{x},\vec{y},\ell_{bot})\in D} \left( F \cap E(\vec{x},\vec{y},\ell_{bot})\right),
		\end{equation} 
		with the sets $E(\vec{x},\vec{y},\ell_{bot})$ and $D$
		defined in the following.
		
		For $\vec{x},\vec{y}\in\mathfrak{W}_{k-2}$, $z_1,z_2\in\mathbb{Z}$, and $\ell_{bot}\in\Omega(rN^\alpha,RN^\alpha,z_1,z_2)$, let $E(\vec{x},\vec{y},\ell_{bot})$ denote the event that $L_i^N(rN^\alpha) = x_i$ and $L_i^N(RN^\alpha)=y_i$ for $1\leq i\leq k-2$, and $L_{k-1}^N$ agrees with $\ell_{bot}$ on $[rN^\alpha,RN^\alpha]$. Let $D$ denote the set of triples $(\vec{x},\vec{y},\ell_{bot})$ satisfying
		\begin{enumerate}[label=(\arabic*)]
			
			\item $0\leq y_i - x_i \leq (R-r)N^\alpha$ for $1\leq i\leq k-2$,
			
			\item $N^{-\alpha/2}|x_1 - prN^\alpha + \lambda r^2N^{\alpha/2}| < \phi(\epsilon/16)$ and $N^{-\alpha/2}|y_1-pRN^\alpha + \lambda R^2N^{\alpha/2}| < \phi(\epsilon/16)$,
			
			\item $z_1\leq x_{k-2}$, $z_2\leq y_{k-2}$, and $\ell_{bot} \in \Omega(rN^\alpha,RN^\alpha,z_1,z_2)$.
			
		\end{enumerate}
		Then $D$ is countable, since there are finitely many integers $x_i,y_i$ satisfying (1), countably many $z_1,z_2\in\mathbb{Z}$ satisfying (3), and at most finitely many $\ell_{bot}$ satisfying (3) for each choice of $z_1,z_2$. Moreover, the $E(\vec{x},\vec{y},\ell_{bot})$ are pairwise disjoint, and the three conditions together show that $F = \bigcup_{(\vec{x},\vec{y},\ell_{bot})\in D} (F\cap E(\vec{x},\vec{y},\ell_{bot}))$. This proves \eqref{21F}.\\
		
		\noindent\textbf{Step 2.} We will argue in the following steps that we can find $\tilde{N}_0$ so that for $N\geq \tilde{N}_0$ we have
		\begin{equation}\label{21AEbound}
		\mathbb{P}(A\,|\, F\cap E(\vec{x},\vec{y},\ell_{bot})) > 1/4
		\end{equation}
		uniformly in $\vec{x},\vec{y},\ell_{bot}$. In this step, we prove \eqref{21Bbound} assuming this fact. 
		
		It follows from \eqref{21F} and \eqref{21AEbound} that for $N\geq\tilde{N}_0$,
		\begin{align*}
		&\mathbb{P}(A\,|\,F) = \sum_{(\vec{x},\vec{y},\ell_{bot})\in D} \frac{\mathbb{P}(A\,|\,F\cap E(\vec{x},\vec{y},\ell_{bot}))\mathbb{P}(F\cap E(\vec{x},\vec{y},\ell_{bot}))}{\mathbb{P}(F)} \geq \\ &\frac{1}{4}\cdot\frac{\sum_{(\vec{x},\vec{y},\ell_{bot})\in D} \mathbb{P}(F\cap E(\vec{x},\vec{y},\ell_{bot}))}{\mathbb{P}(F)} = \frac{1}{4}.
		\end{align*}
		From the third condition in the definition of an $(\alpha,p,\lambda)$-good sequence, Definition \ref{Def1}, we can find $\tilde{N}_1$ so that $\mathbb{P}(A) < \epsilon/8$ for $N\geq\tilde{N}_1$. Hence
		\[
		\mathbb{P}(F) = \frac{\mathbb{P}(A\cap F)}{\mathbb{P}(A\,|\,F)} \leq 4\mathbb{P}(A) < \epsilon/2.
		\]
		Lastly, we can find $\tilde{N}_2$ so that for $N\geq\tilde{N}_2$, the two inequalities in \eqref{21x1y1} hold with probability $>1-\epsilon/2$. We conclude that
		\[
		\mathbb{P}(B) \leq \mathbb{P}(F) + \epsilon/2 \leq \epsilon
		\]
		for $N\geq N_5 = \max(\tilde{N}_0,\tilde{N}_1,\tilde{N}_2)$.\\
		
		\noindent\textbf{Step 3.} We will now prove \eqref{21AEbound}, assuming results from Steps 3 and 4 below. We first note that by Lemma \ref{MCLxy}, if we raise the curves at the endpoints $rN^\alpha,RN^\alpha$, then the probability of the event $A$ will decrease. In particular, let us write
		\begin{equation}\label{21xybar}
		\overline{x} = \lceil prN^\alpha - (\lambda r^2 - \phi(\epsilon/16))N^{\alpha/2}\,\rceil, \quad \overline{y} = \lceil pRN^\alpha - (\lambda R^2 - \phi(\epsilon/16))N^{\alpha/2}\,\rceil,
		\end{equation} 
		and $T = (R-r)N^\alpha$, and define $\vec{x}\,',\vec{y}\,'\in\mathfrak{W}_{k-2}$ by
		\begin{equation*}
		x_i' = \overline{x} + (k-1-i)\lceil C\sqrt{T}\,\rceil, \quad
		y_i' = \overline{y} + (k-1-i)\lceil C\sqrt{T}\,\rceil.
		\end{equation*}
		Note that $x_i' \geq \overline{x} \geq x_1 \geq x_i$ for each $i$ by condition (2) above. Furthermore, $x_i' - x_{i+1}' \geq C\sqrt{T}$. The same observations hold for $y_i'$. Using the Schur Gibbs property (Definition \ref{DefSGP}) and Lemma \ref{MCLxy}, we have
		\begin{equation}\label{21xyest}
		\begin{split}
		&\mathbb{P}(A\,|\,F\cap E(\vec{x},\vec{y},\ell_{bot})) = \mathbb{P}^{rN^\alpha, RN^\alpha,\vec{x},\vec{y},\infty,\ell_{bot}}_{avoid,Ber} (A\,|\,F) \geq \mathbb{P}^{rN^\alpha, RN^\alpha,\vec{x}\,',\vec{y}\,',\infty,\ell_{bot}}_{avoid,Ber} (A\,|\,F) \geq \\
		&\mathbb{P}^{rN^\alpha, RN^\alpha,\vec{x}\,',\vec{y}\,'}_{Ber} \left(\tilde{A}\cap\left\{L_1 \geq \cdots \geq L_{k-2} \geq \ell_{bot}\right\}\,\big|\,\tilde F\right) \geq \\
		&\mathbb{P}^{rN^\alpha, RN^\alpha,\vec{x}\,',\vec{y}\,'}_{Ber} \big(\tilde A\,\big|\,\tilde F\big) - \left( 1 - \mathbb{P}^{rN^\alpha, RN^\alpha,\vec{x}\,',\vec{y}\,'}_{Ber} \left(L_1 \geq \cdots \geq L_{k-2} \geq \ell_{bot}\,\big|\,\tilde F\right)\right). 
		\end{split}
		\end{equation}
		Here, we have written $(L_1,\dots,L_{k-2})$ for the line ensemble with law $\mathbb{P}^{rN^\alpha, RN^\alpha,\vec{x}\,',\vec{y}\,'}_{Ber}$, and we define the events $\tilde A$, $\tilde F$ in the same way as $A$, $F$ in Step 1 but replacing $L_1^N, L_{k-1}^N$ with $L_1, \ell_{bot}$. In Step 4, we will find $\tilde{N}_0$ so that for $N\geq\tilde{N}_0$, 
		\begin{align}
		& \mathbb{P}^{rN^\alpha, RN^\alpha,\vec{x}\,',\vec{y}\,'}_{Ber} \big(\tilde A\,\big|\,\tilde F \big) \geq 1/3, \label{21 1/3}\\
		& \mathbb{P}^{rN^\alpha, RN^\alpha,\vec{x}\,',\vec{y}\,'}_{Ber} \left(L_1 \geq \cdots \geq L_{k-2} \geq \ell_{bot}\,\big|\,\tilde F \right) \geq 11/12 \label{21 1/12},
		\end{align}
		independent of $\vec{x},\vec{y},\ell_{bot}$. Then \eqref{21xyest} implies $\mathbb{P}(A\,|\,F\cap E(\vec{x},\vec{y},\ell_{bot})) \geq 1/3 - 1/12 = 1/4$ for $N\geq\tilde{N}_0$, proving \eqref{21AEbound}.\\
		
		\noindent\textbf{Step 4.} In this step we prove \eqref{21 1/3}. A very similar argument proves the same inequality if $[r,R]$ is replaced by $[-R,-r]$ in the definition of $B$. 
		
		Let $\ell$ be the random variable with law $\mathbb{P}^{rN^\alpha,RN^\alpha,x_1',y_1'}_{Ber}$, and define $\tilde{A}_\ell$ in the same way as $A$ in \eqref{21AB} but with $L_1^N$ replaced by $\ell$. Since $L_1,\dots,L_{k-2},\ell_{bot}$ are independent under $\mathbb{P}^{rN^\alpha, RN^\alpha,\vec{x}\,',\vec{y}\,'}_{Ber}$, and the event $\tilde A$ in \eqref{21xyest} depends only on $L_1$, we have
		\begin{equation}\label{21indep}
		\mathbb{P}^{rN^\alpha, RN^\alpha,\vec{x}\,',\vec{y}\,'}_{Ber}\big(\tilde A\,\big|\,\tilde F \big) = \mathbb{P}^{rN^\alpha, RN^\alpha,x_1',y_1'}_{Ber}\big(\tilde{A}_\ell\big).
		\end{equation} 
		With $\overline{x},\overline{y}$ as in \eqref{21xybar}, write $\overline{z} = \overline{y}-\overline{x}$. We have
		\begin{equation} \label{21convex}
		\begin{split}
		& \mathbb{P}^{rN^\alpha, RN^\alpha,x_1',y_1'}_{Ber} \big(\tilde A_\ell\big)
		= \mathbb{P}^{0,T,x_1',y_1'}_{Ber} \left(\ell(T/2) - pN^\alpha \frac{R+r}{2} + \lambda\Big(\frac{R+r}{2}\Big)^2 N^{\alpha/2} < -\phi(\epsilon/16)N^{\alpha/2}\right) =\\
		& \mathbb{P}^{0,T,\overline{x},\overline{y}}_{Ber}\left(\ell(T/2) - pN^\alpha\frac{R+r}{2} + \lambda\Big(\frac{R+r}{2}\Big)^2 N^{\alpha/2} < -\big(\phi(\epsilon/16) + (k-2)\lceil C\sqrt{R-r}\,\rceil\big)N^{\alpha/2}\right) \geq\\
		& \mathbb{P}^{0,T,\overline{x},\overline{y}}_{Ber}\left(\ell(T/2) - \frac{\overline{x} + \overline{y}}{2} < \Big( \lambda\Big(\frac{R^2+r^2}{2}\Big) - \lambda\Big(\frac{R+r}{2}\Big)^2 - Ck\sqrt{R-r} - 2\phi(\epsilon/16)\Big)N^{\alpha/2}\right) = \\
		&\mathbb{P}^{0,T,0,\overline{z}}_{Ber}\left(\ell(T/2) - \overline{z}/2 < \gamma\sqrt{T}\right).
		\end{split}
		\end{equation}
		The inequality in the second last line follows from the definitions of $\overline{x},\overline{y}$, and in the last line, we have shifted the domain and written $\gamma$ for the constant multiplying $N^{\alpha/2}$ in the previous line, divided by $\sqrt{R-r}$. We claim that our choice of $R_0$ in \eqref{21Rdef} and the fact that $R\geq R_0$ imply that $\gamma > 0$. To see this observe that
		\[
		\frac{R^2+r^2}{2} - \left(\frac{R+r}{2}\right)^2 = \frac{R^2 + r^2-2rR}{4} = \frac{(R-r)^2}{4}
		\]
		for fixed $r$, and thus since $R-r \geq R_0-r > 1$,
		\begin{align*}
		&\gamma\sqrt{R-r} = \frac{\lambda}{4}(R-r)^2 - Ck\sqrt{R-r} - 2\phi(\epsilon/16) > \frac{\lambda}{4}(R_0-r)^2 - Ck(R_0-r) - 2\phi(\epsilon/16) \geq 0.
		\end{align*} 
		The last inequality is easily checked by computing the zeros of the quadratic function in $R_0$.
		
		Now by Theorem \ref{KMT}, there is a probability measure $\tilde{\mathbb{P}}$ supporting a random variable $\ell^{(T,\overline{z})}$ with law $\mathbb{P}^{0,T,0,\overline{z}}_{Ber}$ and a Brownian bridge $B^\sigma$ with variance $\sigma^2 = p(1-p)$. Then the probability in the last line of \eqref{21convex} is equal to
		\begin{align*}
		& \tilde{\mathbb{P}}\left( \ell^{(T,\overline{z})}(T/2) - \overline{z}/2 < \gamma\sqrt{T}\right) = \tilde{\mathbb{P}}\left(\left[\ell^{(T,\overline{z})}(T/2) - \overline{z}/2 - \sqrt{T}B^\sigma_{1/2}\right] + \sqrt{T}B^\sigma_{1/2} < \gamma\sqrt{T}\right) \geq \\
		& \tilde{\mathbb{P}}\left(\sqrt{T}B^\sigma_{1/2} < 0\quad\mathrm{and}\quad \Delta(T,\overline{z}) < \gamma\sqrt{T}\right) \geq \frac{1}{2} - \tilde{\mathbb{P}}\left(\Delta(T,\overline{z}) \geq \gamma\sqrt{T}\right).
		\end{align*}
		Here, $\Delta(T,\overline{z})$ is as defined in \eqref{KMTeq}. Observe that
		\begin{equation}\label{21zpT}
		\frac{|\overline{z} - pT|^2}{T} \leq \frac{(\lambda(R^2-r^2) N^{\alpha/2} + 1)^2}{(R-r)N^\alpha} \leq 4\lambda^2(R+r)^2(R-r).
		\end{equation}
		Thus Corollary \ref{Cheb} gives us $\tilde{N}_0$ such that $\mathbb{P}(\Delta(T,\overline{z})\geq \gamma\sqrt{T})<1/6$ for $N\geq \tilde{N}_0$. This gives a lower bound on $\mathbb{P}^{rN^\alpha, RN^\alpha,x_1',y_1'}_{Ber} (A)$ of $1/2 - 1/6 = 1/3$ as desired.\\
		
		\noindent\textbf{Step 5.} In this last step, we prove \eqref{21 1/12} for $N\geq \tilde{N}_0$, after possibly enlarging $\tilde{N}_0$ from Step 4. The argument with $[-R,-r]$ in place of $[r,R]$ is essentially the same. 
		
		Note that on the event $\tilde F$, $\ell_{bot}$ lies uniformly below the line segment $\overline{\ell}_{bot}$  on $[rN^\alpha,RN^\alpha]$ connecting the points $\overline{x}$ and $\overline{y}$ defined in \eqref{21xybar}. This segment in turn lies uniformly at a distance of at least $C\sqrt{T}$ below the segment connecting $x'_{k-2} \geq \overline{x} + C\sqrt{T}$ and $y'_{k-2} \geq \overline{y} + C\sqrt{T}$. Moreover, the endpoints of all adjacent curves are at least $C\sqrt{T}$ apart. Since $L_1,\dots,L_{k-2},\overline \ell_{bot}$ are independent of $\tilde F$ under $\mathbb{P}^{rN^\alpha, RN^\alpha,\vec{x}\,',\vec{y}\,'}_{Ber}$, we have 
		\[
		\mathbb{P}^{rN^\alpha, RN^\alpha,\vec{x}\,',\vec{y}\,'}_{Ber} \left(L_1 \geq \cdots \geq L_{k-2} \geq \ell_{bot} \,\big|\,\tilde F \right) \geq \mathbb{P}^{rN^\alpha, RN^\alpha,\vec{x}\,',\vec{y}\,'}_{Ber} \left(L_1 \geq \cdots \geq L_{k-2} \geq \overline \ell_{bot}\right).
		\]
		In view of \eqref{21zpT}, we see from Lemma \ref{CurveSeparation} that we can enlarge $\tilde{N}_0$ depending on $\lambda, r, R, C$ so that for $N\geq\tilde{N}_0$, the probability on right is bounded below by
		\[
		\left(1-3e^{-C^2/8p(1-p)}\right)^{k-2}.
		\]
		Our choice of $C$ in \eqref{21Cdef} implies that this quantity is at least 11/12, proving \eqref{21 1/12}.
		
	\end{proof}

	We now prove Lemma \ref{PropSup2}. We exploit Lemma \ref{21} in order to find two points on intervals $[r,R]$ and $[-R,-r]$ where $L_{k-1}^N(xN^\alpha)$ cannot be too low. After separating the curves in order to treat $L_{k-1}^N$ as a free Bernoulli bridge as in the previous argument, we employ Lemma \ref{LemmaMinFreeS4} to bound the deviation of $L_{k-1}^N$ below the line of slope $p$.
	
	\begin{proof} We split the proof into four steps for clarity. In the first step we introduce notation to be used in the latter steps.\\
	
	\noindent\textbf{Step 1.} Define events
		\[
		A = \left\{\inf_{s \in [ -t_3, t_3 ]}\big(L^N_{k-1}(s) - p s \big) \leq - R_2N^{\alpha/2}\right\},
		\]
		\[
		B = \left\{ \max_{x\in [r+3, R] \cup [-R,-r-3]} \big(L^N_{k-1}(xN^\alpha) - pxN^\alpha\big) > -MN^{\alpha/2} \right\}.
		\]
		Here, we define $M,R$ via Lemma \ref{21}, taking $R$ large enough so that with $M = \lambda R^2 + \phi(\epsilon/64)$, we have 
		\begin{equation}\label{4.3Bbound}
		\mathbb{P}(B) \geq 1 - \epsilon/2
		\end{equation} 
		for $N$ larger than some $\tilde{N}_0$.
		Lastly, define the constant
		\begin{equation}\label{4.3Cdef}
		C = \sqrt{16p(1-p)\log\frac{3}{1-2^{-1/(k-1)}}}.
		\end{equation}
		In the following steps, we will find $N_4\in\mathbb{N}$ and $R_2 > 0$ depending on $M,C,k,r,\epsilon$ so that 
		\begin{equation}\label{4.3Abound}
		\mathbb{P}(A) < \epsilon
		\end{equation}
		for $N\geq N_4$, which will prove the lemma. In particular, we specify $R_2$ in Step 3.
		
		In the remainder of this step, we prove that $B$ can be written as a \textit{countable disjoint union} of the form
		\begin{equation}\label{4.3disj}
		B = \bigsqcup_{(a,b,\vec{x},\vec{y},\ell_{bot})\in D} E(a,b,\vec{x},\vec{y},\ell_{bot}),
		\end{equation}
		where the sets $D$ and $E(a,b,\vec{x},\vec{y},\ell_{bot})$ are defined below.
		
		For $0<a,b\in\mathbb{Z}$, $\vec{x},\vec{y}\in\mathfrak{W}_{k-1}$, $z_1,z_2\in\mathbb{Z}$, and $\ell_{bot}\in\Omega(-a,b,z_1,z_2)$, we define $E(a,b,\vec{x},\vec{y},\ell_{bot})$ to be the event that $L_i^N(-a) = x_i$ and $L_i^N(b) = y_i$ for $1\leq i\leq k-1$, and $L_k^N$ agrees with $\ell_{bot}$ on $[-a,b]$. Let $D$ be the collection of tuples $(a,b,\vec{x},\vec{y},\ell_{bot})$ satisfying 
		\begin{enumerate}[label=(\arabic*)]
			
			\item $a,b\in[(r+3)N^\alpha,RN^\alpha]$.
			
			\item $0 \leq y_i - x_i \leq b+a$, $x_{k-1} + pa > - MN^{\alpha/2}$, and $y_{k-1} - pb > - MN^{\alpha/2}$.
			
			\item If $c,d\in\mathbb{Z}$, $c > a$, and $d > b$, then $L_{k-1}^N(-c) + pc \leq -MN^{\alpha/2}$ and $L_{k-1}^N(d) - pd \leq -MN^{\alpha/2}$,
			
			\item $z_1\leq x_{k-1}$, $z_2\leq y_{k-1}$, and $\ell_{bot}\in\Omega(-a,b,z_1,z_2)$.
			
		\end{enumerate} 
		Since there are finitely many integers $a,b$ satisfying (1), countably many $x_i,y_i$ and $z_1,z_2$ satisfying (2) and (4), and finitely many $L_i^N,\ell_{bot}$ satisfying (3) and (4) given $a,b,x_i,y_i,z_1,z_2$, we see that $D$ is countable. The four conditions together imply that $B = \bigcup_{(a,b,\vec{x},\vec{y},\ell_{bot})\in D} E(a,b,\vec{x},\vec{y},\ell_{bot})$.
		
		To see that the $E(a,b,\vec{x},\vec{y},\ell_{bot})$ are pairwise disjoint for $(a,b,\vec{x},\vec{y},\ell_{bot})\in D$, observe that on the intersection of $E(a,b,\vec{x},\vec{y},\ell_{bot})$ and $E(\tilde a,\tilde b,\tilde{\vec{x}},\tilde{\vec{y}},\tilde{\ell}_{bot})$, conditions (2) and (3) imply that $a=\tilde{a}$ and $b=\tilde{b}$. We then conclude from the definition of $E(a,b,\vec{x},\vec{y},\ell_{bot})$ that $\vec{x}=\tilde{\vec{x}}$, $\vec{y} = \tilde{\vec{y}}$, and $\ell_{bot} = \tilde{\ell}_{bot}$. This proves \eqref{4.3disj}.\\
		
		\noindent\textbf{Step 2.} We will prove in the steps below that we can find $\tilde{N}_1$ and $R_2$ so that for $N\geq\tilde{N}_1$,
		\begin{equation}\label{4.3AEbound}
		\mathbb{P}(A\,|\,E(a,b,\vec{x},\vec{y},\ell_{bot})) < \epsilon/2
		\end{equation}
		uniformly in $a,b,\vec{x},\vec{y},\ell_{bot}$. In this step, we prove \eqref{4.3Abound} assuming this fact.
		
		By \eqref{4.3AEbound} and \eqref{4.3disj}, we obtain
		\begin{align*}
		&\mathbb{P}(A \cap B) = \sum_{(a,b,\vec{x},\vec{y},\ell_{bot})\in D} \mathbb{P}(A\,|\,E(a,b,\vec{x},\vec{y},\ell_{bot}))\mathbb{P}(E(a,b,\vec{x},\vec{y},\ell_{bot})) \leq\\
		& \frac{\epsilon}{2}\sum_{(a,b,\vec{x},\vec{y},\ell_{bot})\in D} \mathbb{P}(E(a,b,\vec{x},\vec{y},\ell_{bot})) \leq \frac{\epsilon}{2}
		\end{align*}
		for $N\geq\tilde{N}_1$. It follows from \eqref{4.3Bbound} that
		\begin{equation*}
		\mathbb{P}(A) \leq \mathbb{P}(A\cap B) + \epsilon/2 < \epsilon
		\end{equation*}
		for $N\geq N_4 = \max(\tilde{N}_0,\tilde{N}_1)$.\\
		
		\noindent\textbf{Step 3.} We next prove \eqref{4.3AEbound}, assuming results from Steps 3 below. Define $\vec{x}\,',\vec{y}\,' \in \mathfrak{W}_{k-1}$ by
		\begin{align*}
		x_i' &= \lfloor - pa - MN^{\alpha/2}\rfloor - (i-1)\lceil CN^{\alpha/2}\rceil,\\
		y_i' &= \lfloor pb - MN^{\alpha/2}\rfloor - (i-1)\lceil CN^{\alpha/2}\rceil.
		\end{align*}  
		Observe that by condition (2) above, $x_i'\leq -pa - MN^{\alpha/2} \leq x_{k-1} \leq x_i$, and similarly for $\vec{y}$. It follows from the Schur Gibbs property (see Definition \ref{DefSGP}) and Lemmas \ref{MCLxy} and \ref{MCLfg} that
		\begin{equation}\label{4.3main}
		\begin{split}
		&\mathbb{P}(A\,|\,E(a,b,\vec{x},\vec{y},\ell_{bot})) \leq \mathbb{P}\left( \inf_{s\in[-a, b]} \left(L_{k-1}^N(s) - ps\right) \leq -R_2N^{\alpha/2}\, \big| \, E(a,b,\vec{x},\vec{y},\ell_{bot}) \right) = \\
		&\mathbb{P}^{-a,b, \vec{x}, \vec{y},\infty,\ell_{bot}}_{avoid, Ber} \left( \inf_{s\in[-a, b]} \left(Q_{k-1}(s) - ps\right) \leq -R_2N^{\alpha/2} \right) \leq\\
		& \mathbb{P}^{-a, b, \vec{x}\,', \vec{y}\,'}_{avoid, Ber} \left( \inf_{s\in[0,a+b]} \big(Q_{k-1}(s) - ps\big) \leq -R_2N^{\alpha/2} \right) \leq \\
		&\frac{\mathbb{P}^{-a, b, \vec{x}\,', \vec{y}\,'}_{Ber} \left( \inf_{s\in[0,a+b]} \big(L_{k-1}(s) - ps\big) \leq -R_2 N^{\alpha/2} \right)}{\mathbb{P}^{-a, b, \vec{x}\,', \vec{y}\,'}_{Ber}(L_1 \geq \cdots \geq L_{k-1})}.
		\end{split}
		\end{equation}
		In Step 3 below, we will find $\tilde{N}_1$ and $R_2 > 0$ such that the numerator and denominator in the last line are $<\epsilon/4$ and $\geq1/2$ for $N\geq\tilde{N}_1$. It then follows that the first expression in \eqref{4.3main} is bounded above by $\epsilon/2$, proving \eqref{4.3AEbound}.\\
		
		\noindent\textbf{Step 4.} In this step, we fix $R_2 > 0$ and $\tilde{N}_1$ so that for $N\geq\tilde{N}_1$, we have
		\begin{align}
		&\mathbb{P}^{-a, b, \vec{x}\,', \vec{y}\,'}_{Ber} \left( \inf_{s\in[0,a+b]} \big(L_{k-1}(s) - ps\big) \leq -R_2 N^{\alpha/2} \right) < \epsilon/4, \label{4.3free}\\
		&\mathbb{P}^{-a, b, \vec{x}\,', \vec{y}\,'}_{Ber}(L_1 \geq \cdots \geq L_{k-1}) \geq 1/2. \label{4.3intersect}
		\end{align}
		Let us first prove \eqref{4.3free}. Writing $\vec{z} = \vec{y}\,' - \vec{x}\,'$, and using the fact that $L_1,\dots,L_{k-1}$ are independent under $\mathbb{P}^{-a, b, \vec{x}\,', \vec{y}\,'}_{Ber}$, we can rewrite the left hand side of \eqref{4.3free} as
		\begin{equation}\label{4.3R2}
		\begin{split}
		& \mathbb{P}^{0, a+b, x_{k-1}', y_{k-1}'}_{Ber} \left( \inf_{s\in[0,a+b]} \big(\ell(s) - p(s-a)\big) \leq -R_2 N^{\alpha/2} \right) =\\
		&\mathbb{P}^{0, a+b, 0, z_{k-1}}_{Ber} \left( \inf_{s\in[0,a+b]} \left(\ell(s) - ps + pa - \lceil pa + MN^{\alpha/2}\rceil - (k-2)\lceil CN^{\alpha/2}\rceil\right) \leq -R_2 N^{\alpha/2} \right) \leq \\
		& \mathbb{P}^{0, a+b, 0, z_{k-1}}_{Ber} \left( \inf_{s\in[0,a+b]} \big(\ell(s) - ps\big) \leq -(R_2 - M - Ck) N^{\alpha/2} \right).
		\end{split}
		\end{equation}
		In the first line, we have implicitly shifted the domain of the paths from $[-a,b]$ to $[0,a+b]$. Since $z_k\geq p(a+b)$, Lemma \ref{LemmaMinFreeS4} allows us to find $R_2>0$ depending on $M,C,k,r,\epsilon$ so that this probability is $<\epsilon/4$ as long as $a+b$ is larger than some constant $W_1$. Since $a+b \geq 2rN^\alpha$, so it suffices to take $\tilde{N}_1 \geq (W_1/2r)^{1/\alpha}$. Thus we obtain \eqref{4.3free}, \textit{independent} of $a,b,\vec{x},\vec{y}$.
		
		Lastly, we prove that we can enlarge $\tilde{N}_1$ so that \eqref{4.3intersect} holds for $N\geq\tilde{N}_1$. Write $a = a'N^\alpha, b = b'N^\alpha$, and $T = a+b = (a'+b')N^\alpha$. Also let $C' = C/\sqrt{a'+b'}$ with $C$ as in \eqref{4.3Cdef}, so that $x_i' - x_{i+1}' \geq CN^{\alpha/2} = C'\sqrt{T}$ and likewise for $y_i'$. Note that $|z_{k-1}-pT| \leq 1$. It follows from Lemma \ref{CurveSeparation}, applied with $\ell_{bot} = -\infty$ and $C'$ in place of $C$, that for $T$ larger than some $T_0$, 
		\begin{equation}\label{4.3avoid}
		\begin{split}
		& \mathbb{P}^{-a, b, \vec{x}\,', \vec{y}\,'}_{Ber}(L_1 \geq \cdots \geq L_{k-1}) = \mathbb{P}^{0, a+b, \vec{x}\,', \vec{y}\,'}_{Ber}(L_1 \geq \cdots \geq L_{k-1}) \geq\\ & \left(1 - 3e^{-(C')^2/8p(1-p)}\right)^{k-1} \geq \left(1 - 3e^{-C^2/16p(1-p)R}\right)^{k-1}.
		\end{split}
		\end{equation}
		Here, we used the fact that $a'+b' \leq 2R$, hence $C' \geq C/\sqrt{2R}$. The constant $T_0$ depends in particular on $C'$, hence possibly on $a+b$. Referring to the proofs of Lemmas \ref{CurveSeparation} and \ref{Cheb}, we see that the dependency of $T_0$ on $C'$ amounts to requiring that $e^{-C'\sqrt{T_0}}$ be sufficiently small. But $C' \geq C/\sqrt{2R}$, so for this it suffices to choose $T_0$ depending on $C$ and $R$. Moreover, $T\geq 2rN^\alpha$, so as long as $\tilde{N}_1 \geq (T_0/2r)^{1/\alpha}$, we have the bound in \eqref{4.3avoid} for $N\geq\tilde{N}_1$ independent $a,b,\vec{x},\vec{y}$. Our choice of $C$ in \eqref{4.3Cdef} ensures that the expression on the right in \eqref{4.3avoid} is at least $1/2$, proving \eqref{4.3intersect}.
		
	\end{proof}
	
	
	

		
	
	
