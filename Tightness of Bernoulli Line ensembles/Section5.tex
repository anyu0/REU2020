
%-------------------------------------------------------------------------------------------------------------------------------------------------------------------------------------------------
%    Section 5
%
%-------------------------------------------------------------------------------------------------------------------------------------------------------------------------------------------------
\section{Proof of three key lemmas}\label{Section5}
Here we prove the three key lemmas from Section \ref{Section4.1}.

\subsection{Proof of Lemma \ref{PropSup}}

We first establish some notation. Let $a,b,t_1,t_2,z_1,z_2 \in \mathbb{Z}$ be given such that $t_1 + 1 < t_2$, $0\leq z_2 - z_1 \leq t_2 - t_1$, $0\leq b-a \leq t_2 - t_1$, $z_1\leq a$, and $z_2\leq b$. We write $\ell\in\Omega(t_1,t_2,a,b)$ and $\ell_{bot}\in\Omega(t_1,t_2,z_1,z_2)$ for generic paths in these two spaces, and we consider the event $\{\ell \geq \ell_{bot}\} = \{\ell(s) \geq \ell_{bot}(s), s\in[t_1,t_2]\}$. Note that $\mathbb{P}^{t_1,t_2,a,b,\infty,\ell_{bot}}_{avoid,Ber}(\ell) = \mathbb{P}^{t_1,t_2,a,b}_{Ber}(\ell\,|\,\ell \geq \ell_{bot}))$. We now establish some auxiliary results which will be used in the proof of Lemma \ref{PropSup}.

\begin{lemma}\label{pathcounting}
	If $a\leq k_1\leq k_2\leq a + T - t_1$, then with notation as above,
	\[
	\mathbb{P}^{t_1, t_2, a, b}_{Ber}\big( \ell \geq \ell_{bot}\,\big|\,\ell(T) = k_1\big) \leq \mathbb{P}^{t_1, t_2, a, b}_{Ber}\big(\ell \geq \ell_{bot}\,\big|\,\ell(T) = k_2\big).
	\]
\end{lemma}

\begin{remark}
	This lemma essentially states that a path $\ell$ is more likely to lie above $\ell_{bot}$ if its value at a point $T$ is increased. A more general result is proven in \cite[Lemma 4.1]{CD}
\end{remark}

\begin{proof}
	Let $\ell_1$ be a random path distributed according to $\mathbb{P}^{t_1, t_2, a, b}_{Ber}$ conditioned on $\ell_1(T) = k_1$. We can identify $\ell_1$ with a sequence of $+$'s and $-$'s of length $t_2-t_1$, where a $+$ in the $i$th position means that $\ell_1(t_1+i+1)-\ell_1(t_1+i) = 1$, and a $-$ means that $\ell_1(t_1+i+1)-\ell_1(t_1+i) = 0$. [Maybe include Figure 9 from Corwin-Dimitrov here.] In this representation, the value of $\ell_1(T)$ is $a$ plus the number of $+$'s in the first $T-t_1$ slots, and the value of $\ell_1(t_2)$ is $a$ plus the total number of $+$'s. Note that we must have exactly $(k_1-a)$ $+$'s in the first $T-t_1$ slots, and $(b-k_1)$ $+$'s in the last $t_2-T$ slots. We pick uniformly at random $(k_2-k_1)$ $-$'s in the first $T-t_1$ slots and change them to $+$'s, then pick randomly $(k_2-k_1)$ $+$'s in the last $t_2-T$ slots and change them to $-$'s. This defines a new path $\ell_2$. Since there are now $k_2-a$ $+$'s in the first $T-t_1$ slots, we have $\ell_2(T) = k_2$, and we still have $\ell_2(t_2) = b$ since the number of $+$'s is unchanged. Thus we see that $\ell_2$ is distributed according to $\mathbb{P}^{t_1,t_2,a,b}_{Ber}$ conditioned on $\ell_2(T) = k_2$. 
	
	Now suppose $\ell_1 \geq \ell_{bot}$. We claim that $\ell_2 \geq \ell_1$ on all of $[t_1,t_2]$. To see this, note that for any $s\in\llbracket t_1, t_2\rrbracket$, $\ell_2(s) - \ell_1(s)$ is equal to the number of $+$'s in the first $s-t_1$ slots of the sequence representing $\ell_2$, minus the corresponding number for $\ell_1$. If $s\leq T$, this difference is clearly positive by construction. The difference is equal to $k_2 - k_1 \geq 0$ at $s = T$, and the difference then decreases monotonically as $s$ increases to $t_2$, since we have removed exactly $k_2-k_1$ $+$'s from the last $t_2-T$ slots. The difference is of course 0 at $s = t_2$, so this proves the claim. It follows that
	\[
	\mathbf{1}_{\ell_1 \geq \ell_{bot}} \leq \mathbf{1}_{\ell_2 \geq \ell_{bot}}.
	\]
	Now taking expectations of both sides and recalling the distributions of $\ell_1,\ell_2$ proves the lemma.
\end{proof}


\begin{corollary}\label{setpathcounting}
	Let $T\in\llbracket t_1, t_2\rrbracket$, and let $A, B$ be nonempty sets of integers such that $a \leq \alpha \leq \beta \leq a+T-t_1$ for all $\alpha\in A, \beta\in B$. Then
	\[
	\mathbb{P}^{t_1, t_2, a, b}_{Ber}\big(\ell \geq \ell_{bot}\,\big|\,\ell(T) \in A\big) \leq \mathbb{P}^{t_1, t_2, a, b}_{Ber}\big(\ell \geq \ell_{bot}\,\big|\,\ell(T) \in B\big).
	\]
\end{corollary}

\begin{proof}
	We have
	\begin{align*}
	& \mathbb{P}^{t_1, t_2, a, b}_{Ber}\big(\ell \geq \ell_{bot}\,\big|\,\ell(T) \in A\big) = \sum_{\alpha\in A} \mathbb{P}^{t_1, t_2, a, b}_{Ber}\big(\ell \geq \ell_{bot}\,\big|\,\ell(T) = \alpha\big) \cdot \frac{\mathbb{P}^{t_1, t_2, a, b}_{Ber}(\ell(T) = \alpha)}{\mathbb{P}^{t_1, t_2, a, b}_{Ber}(\ell(T) \in A)}\\
	= \; & \sum_{\alpha\in A}\sum_{\beta\in B} \mathbb{P}^{t_1, t_2, a, b}_{Ber}\big(\ell \geq \ell_{bot}\,\big|\,\ell(T) = \alpha\big) \cdot \frac{\mathbb{P}^{t_1, t_2, a, b}_{Ber}(\ell(T) = \alpha)}{\mathbb{P}^{t_1, t_2, a, b}_{Ber}(\ell(T) \in A)} \cdot \frac{\mathbb{P}^{t_1, t_2, a, b}_{Ber}(\ell(T) = \beta)}{\mathbb{P}^{t_1, t_2, a, b}_{Ber}(\ell(T) \in B)}\\
	\leq \; & \sum_{\alpha\in A}\sum_{\beta\in B} \mathbb{P}^{t_1, t_2, a, b}_{Ber}\big(\ell \geq \ell_{bot}\,\big|\,\ell(T) = \beta\big) \cdot \frac{\mathbb{P}^{t_1, t_2, a, b}_{Ber}(\ell(T) = \alpha)}{\mathbb{P}^{t_1, t_2, a, b}_{Ber}(\ell(T) \in A)} \cdot \frac{\mathbb{P}^{t_1, t_2, a, b}_{Ber}(\ell(T) = \beta)}{\mathbb{P}^{t_1, t_2, a, b}_{Ber}(\ell(T) \in B)}\\
	= \; & \sum_{\beta \in B} \mathbb{P}^{t_1, t_2, a, b}_{Ber}\big(\ell \geq \ell_{bot}\,\big|\,\ell(T) = \beta\big) \cdot \frac{\mathbb{P}^{t_1, t_2, a, b}_{Ber}(\ell(T) = \beta)}{\mathbb{P}^{t_1, t_2, a, b}_{Ber}(\ell(T) \in B)} = \mathbb{P}^{t_1, t_2, a, b}_{Ber}\big(\ell \geq \ell_{bot}\,\big|\,\ell(T) \in B\big).
	\end{align*}
	The inequality in the third line follows from Lemma \ref{pathcounting}.
\end{proof}

\begin{corollary}\label{freeavoidbound}
	Let $\alpha \leq a + T - t_1$. Then
	\begin{align*}
	\mathbb{P}^{t_1,t_2,a,b,\infty,\ell_{bot}}_{avoid,Ber} (\ell(T)\geq\alpha) &\geq \mathbb{P}^{t_1,t_2,a,b}_{Ber} (\ell(T)\geq \alpha).
	\end{align*}
\end{corollary}

\begin{proof}
	We write $\mathbb{P} := \mathbb{P}^{t_1,t_2,a,b}_{Ber}$ for brevity. Using Bayes' theorem repeatedly, we find
	\begin{align*}
	\mathbb{P}(\ell(T)\geq\alpha\,|\,\ell \geq \ell_{bot}) &= \frac{\mathbb{P}(\ell \geq \ell_{bot}\,|\,\ell(T)\geq\alpha) \mathbb{P}(\ell(T)\geq\alpha)}{\mathbb{P}(\ell \geq \ell_{bot})}\\ 
	&\geq \frac{\mathbb{P}(\ell \geq \ell_{bot}\,|\,\ell(T) < \alpha) \mathbb{P}(\ell(T)\geq\alpha)}{\mathbb{P}(\ell \geq \ell_{bot})}\\
	&= \big(1 - \mathbb{P}(\ell(T)\geq\alpha\,|\,\ell \geq \ell_{bot})\big)\cdot\frac{\mathbb{P}(\ell(T)\geq\alpha)}{\mathbb{P}(\ell(T) < \alpha)}.
	\end{align*}
	The inequality in the second line follows from Corollary \ref{setpathcounting}. It follows that
	\[
	\mathbb{P}(\ell(T)\geq\alpha\,|\,\ell \geq \ell_{bot}) \geq \frac{\mathbb{P}(\ell(T)\geq\alpha)}{\mathbb{P}(\ell(T)\geq\alpha) + \mathbb{P}(\ell(T) < \alpha)} = \mathbb{P}(\ell(T)\geq\alpha).
	\]
\end{proof}

We are now ready to prove Lemma \ref{PropSup}. The proof is similar to that of \cite[Lemma 5.2]{CD}. We exploit the one-point tightness of $L_1^N$ at two appropriately chosen points, and we use Lemma \ref{LemmaHalfS4} to control the deviation of $L_1^N$ from the line of slope $p$ away from these points.

\begin{proof}
	We write $s_4 = \lceil r+3 \rceil N^\alpha$, $s_3 = \lfloor r+3 \rfloor N^\alpha$, so that $s_3 \leq t_3 \leq s_4$, and take $N$ large enough so that $L_1^N$ is defined at $s_4$. We define events 
	\[
	E(M) = \Big\{\big|L_1^N(-s_4) + ps_4\big| > MN^{\alpha/2}\Big\}, \quad F(M) = \Big\{L_1^N(-s_3) > -ps_3 + MN^{\alpha/2} \Big\},
	\]
	\[
	G(M) = \Bigg\{\sup_{s\in[0,t_3]} \big(L_1^N(s) - ps \big) \geq (6r+22)(2r+6)^{1/2}(M+1)N^{\alpha/2} \Bigg\}.
	\]
	For $a,b\in\mathbb{Z}$, $s\in\llbracket 0, t_3 \rrbracket$, and $\ell_{bot}\in\Omega(-s_4,s,z_1,z_2)$ with $z_1\leq a$, $z_2\leq b$, we also define $E(a,b,s,\ell_{bot})$ to be the event that $L_1^N(-s_4) = a$, $L_1^N(s) = b$, and $L_2^N$ agrees with $\ell_{bot}$ on $[-s_4,s]$. 
	
	We claim that the set $G(M) \setminus E(M)$ can be written as a \textit{countable disjoint} union of sets $E(a,b,s,\ell_{bot})$. Let $D(M)$ be the set of tuples $(a,b,s,\ell_{bot})$ satisfying
	\begin{enumerate}[label=(\arabic*)]
		
		\item $0\leq s\leq t_3$,
		
		\item $0\leq b-a \leq s + s_4$, $|a + ps_4| \leq MN^{\alpha/2}$, and $b-ps > (6r+22)(2r+6)^{1/2}(M+1)N^{\alpha/2}$,
		
		\item $z_1\leq a, z_2\leq b$, and $\ell_{bot}\in\Omega(-s_4, s, z_1, z_2)$.
		
	\end{enumerate}
	Conditions (1) and (2) show that the union of these sets $E(a,b,s,\ell_{bot})$ for $(a,b,s,\ell_{bot})\in D(M)$ is $G(M)\setminus E(M)$. Observe that $D(M)$ is countable, since there are finitely many possible choices of $s$, countably many $a,b$ and $z_1,z_2$ for each $s$, and finitely many $\ell_{bot}$ for each $z_1,z_2$. Moreover, the sets $E(a,b,s,\ell_{bot})$ are clearly pairwise disjoint for distinct tuples in $D(M)$. This proves the claim.
	
	Now by one-point tightness of $L_1^N$ at integer multiples of $N^\alpha$, we can choose $M$ large enough depending on $\epsilon$ so that
	\begin{equation}
	\mathbb{P}(E(M)) < \epsilon/4, \quad \mathbb{P}(F(M)) < \epsilon/12 \label{4.2EFbounds}
	\end{equation}
	for all $N\in\mathbb{N}$. If $(a,b,s,\ell_{bot})\in D(M)$, then
	\begin{align*}
	\mathbb{P}^{-s_4,s,a,b}_{Ber}\Big( \ell(-s_3) > -ps_3 + MN^{\alpha/2}\Big) &= \mathbb{P}^{0,s+s_4,0,b-a}_{Ber}\Big(\ell(s_4-s_3) + a \geq -ps_3 + MN^{\alpha/2}\Big)\\
	&\geq \mathbb{P}^{0,s+s_4,0,b-a}_{Ber}\Big(\ell(s_4-s_3) \geq p(s_4-s_3) + 2MN^{\alpha/2}\Big).
	\end{align*}
	The inequality follows from the assumption in (2) that $a+ps_4 \geq -MN^{\alpha/2}$. Moreover, since $b-ps > (6r+22)(2r+6)^{1/2}(M+1)N^{\alpha/2}$ and $a+ps_4 \leq MN^{\alpha/2}$, we have 
	\[
	b-a \geq p(s+s_4) + (6r+21)(2r+6)^{1/2}(M+1)N^{\alpha/2} \geq p(s+t_3) + (6r+21)(M+1)(s+s_4)^{1/2}.
	\]  
	The second inequality follows since $s+s_4 \leq 2s_4 \leq (2r+6)N^{\alpha}$. It follows from Lemma \ref{LemmaHalfS4} with $M_1 = 0$, $M_2 = (6r+21)(M+1)$ that for sufficiently large $N$, we have
	\begin{equation}
	\mathbb{P}^{0,s+s_4,0,b-a}_{Ber}\Big(\ell(s_4-s_3) \geq \frac{s_4-s_3}{s+s_4}[p(s+s_4) + M_2 N^{\alpha/2}] - (s+s_4)^{1/4}\Big) \geq 1/3, \label{4.2interp}
	\end{equation}
	for all $(a,b,s,\ell_{bot}) \in D(M)$ simultaneously. Note that $\frac{s_4-s_3}{s+s_4} \geq \frac{N^\alpha - 1}{(2r+6)N^\alpha} \geq \frac{1}{2r+7}$
	for large $N$. Hence $\frac{s_4-s_3}{s+s_4}[p(s+t_3) + M_2 N^{\alpha/2}] - (s+s_4)^{1/4} \geq p(s+s_4) + 3(M+1)N^{\alpha/2} - (s+s_4)^{1/4}\geq p(s+s_4) + 2MN^{\alpha/2}$ for all large enough $N$. We conclude from \eqref{4.2interp} that
	\[
	\mathbb{P}^{-s_4,s,a,b}_{Ber}\Big(\ell(-s_3) > -ps_3 + MN^{\alpha/2}\Big) \geq 1/3
	\]
	uniformly in $a,b$ for large $N$. Now by the Gibbs property for $L^N$, we have for any $\ell\in\Omega(-s_4,s,a,b)$ that
	\[
	\mathbb{P}(L_1^N|_{[-s_4,s]} = \ell\,|\,E(a,b,s,\ell_{bot})) = \mathbb{P}^{-s_4,s,a,b,\infty,\ell_{bot}}_{avoid, Ber}(\ell).
	\]
	Hence by Corollary \ref{freeavoidbound},
	\begin{align*}
	&\mathbb{P}\big( L_1^N(-s_3) > -ps_3 + MN^{\alpha/2}\,\big|\,E(a,b,s,\ell_{bot})\big)\\
	= \; & \sum_{\ell\in\Omega(-s_4,s,a,b)} \mathbb{P}^{-s_4,s,a,b,\infty,\ell_{bot}}_{avoid, Ber}(\ell)\cdot \mathbb{P}^{-s_4,s,a,b,\infty,\ell_{bot}}_{avoid, Ber}\big(\ell(-s_3) > -ps_3 + MN^{\alpha/2}\big)\\
	\geq \; & \sum_{\ell\in\Omega(-s_4,s,a,b)} \mathbb{P}^{-s_4,s,a,b,\infty,\ell_{bot}}_{avoid, Ber}(\ell)\cdot \mathbb{P}^{-s_4,s,a,b}_{Ber}\big(\ell(-s_3) > -ps_3 + MN^{\alpha/2}\big)\\
	\geq \; & \frac{1}{3}\sum_{\ell\in\Omega(-s_4,s,a,b)} \mathbb{P}^{-s_4,s,a,b,\infty,\ell_{bot}}_{avoid, Ber}(\ell) = \frac{1}{3}.
	\end{align*}
	Note once again that this bound holds independent of $a,b$ for all sufficiently large $N$. It follows from \eqref{4.2EFbounds} that
	\[
	\epsilon/12 > \mathbb{P}(F(M)) \geq \sum_{(a,b,s,\ell_{bot})\in D(M)} \mathbb{P}(F(M)\cap E(a,b,s,\ell_{bot}))
	\]
	\[
	= \sum_{(a,b,s,\ell_{bot})\in D(M)} \mathbb{P}(F(M)\,|\, E(a,b,s,\ell_{bot}))\mathbb{P}(E(a,b,s,\ell_{bot})) \geq \frac{1}{3}\mathbb{P}(G(M)\setminus E(M))
	\]
	for large $N$. Since in addition $\mathbb{P}(E(M)) < \epsilon/4$, we find that
	\[
	\mathbb{P}\Big( \sup_{s \in [0,t_3] }\big( L^N_1(s) - p s \big) \geq  (6r+22)(2r+6)^{1/2}(M+1)N^{\alpha/2} \Big) = \mathbb{P}(G(M)) < \epsilon/2
	\]
	for large enough $N$. A similar argument proves the same inequality with $[-t_3,0]$ in place of $[0,t_3]$. Thus we can find an $N_2 = N_2(\epsilon)$ so that
	\[
	\mathbb{P}\Big( \sup_{s \in [-t_3,t_3] }\big( L^N_1(s) - p s \big) \geq  R_1N^{\alpha/2} \Big) < \epsilon
	\]
	for all $N\geq N_2$, with $R_1 = (6r+22)(2r+6)^{1/2}(M+1)$.
	
\end{proof}
	
	
\subsection{Proof of Lemma \ref{PropSup2}}

	We begin by proving the following lemma, which allows us to prevent the bottom curve of an ensemble from falling too low on some interval.
	
	\begin{lemma}\label{21}
		Fix $p\in (0,1)$, $k\in\mathbb{N}$, and $\alpha,\lambda > 0$. Suppose that $\mathfrak{L}^N = (L_1^N, \dots, L_k^N)$ is a $(\alpha,p,\lambda)$-good sequence of $\llbracket 1, k\rrbracket$-indexed Bernoulli line ensembles. Then for any $r,\epsilon>0$, there exists $R>0$ depending on $\lambda,k,p,\epsilon,r,\phi$ and $N_0 \in \mathbb{N}$ depending on $\lambda,k,p,\epsilon,r,\phi,\psi,\alpha$ such that for all $N\geq N_0$,
		\[
		\mathbb{P}\Big(\max_{x\in[r,R]} \big(L_k^N(xN^\alpha) - pxN^\alpha\big) \leq -(\lambda R^2 + \phi(\epsilon/16))N^{\alpha/2}\Big) < \epsilon.
		\]
		The same statement holds if $[r,R]$ is replaced with $[-R,-r]$.
	\end{lemma}

	\begin{remark}
		The key to this lemma is the parabolic shift implicit in the definition of an $(\alpha,p,\lambda)$-good sequence. This requires the deviation of the top curve from the line of slope $p$ to appear roughly parabolic. Using monotone coupling, we separate the curves of the ensemble so that $L_1^N$ is nearly independent of the other curves. Then we would expect the value of $L_1^N$ at the midpoint of $r$ and $R$ to be close to the midpoint of the straight line segment connecting two points of the parabola. But the parabola is convex, so for large enough $R$ this violates the one-point tightness assumpion at $(R+r)/2$.
	\end{remark}
	
	\begin{proof}
		
		Fix $r>0$. Note that for any $R>r$,
		\[
		\max_{r\leq x\leq R} \big(L_k^N(xN^\alpha) - pxN^\alpha\big) \geq \max_{\lceil r\rceil \leq x \leq R} \big(L_k^N(xN^\alpha) - pxN^\alpha\big).
		\]
		Thus by replacing $r$ with $\lceil r\rceil$, we can assume that $r\in\mathbb{Z}$. Before beginning the proof, we introduce notation. Define constants
		\begin{align}
		C &= \sqrt{ 8p(1-p) \log\frac{3}{1-(11/12)^{1/(k-1)}}}\,,\label{21Cdef}\\
		R_0 &= 8\left( r + \frac{Ck}{\lambda}\right) \vee \frac{2\phi(\epsilon/16)}{Ck+\lambda r}. \label{21Rdef}
		\end{align}
		Note that $R_0\geq r$. We define $R = R_0 + \mathbf{1}_{R_0 + r\;\mathrm{odd}}$, so that $R\geq R_0$ and the midpoint $(R+r)/2$ is an integer. In the following, we always assume $N$ is large enough depending on $\psi,R$ so that $L_1^N$ is defined at $R$. We may do so by the second condition in the definition of an $(\alpha,p,\lambda)$-good sequence (see Definition \ref{Def1}). Define events
		\begin{align*}
		A &= \left\{L_1^N\left(\frac{R+r}{2}\,N^\alpha\right) - pN^\alpha\,\frac{R+r}{2} + \lambda\left(\frac{R+r}{2}\right)^2 N^{\alpha/2} < -\phi(\epsilon/16)N^{\alpha/2}\right\},\\
		B &= \left\{\max_{x\in[r,R]} \left(L_k^N(xN^\alpha) - pxN^\alpha\right) \leq -(\lambda R^2 + \phi(\epsilon/16)) N^{\alpha/2} \right\}.
		\end{align*}
		Let $F$ denote the subset of $B$ for which the inequalities
		\begin{equation}\label{21x1y1}
		\begin{split}
		& prN^\alpha - (\lambda r^2+\phi(\epsilon/16))N^{\alpha/2} < L_1^N(rN^\alpha) <  prN^\alpha - (\lambda r^2-\phi(\epsilon/16))N^{\alpha/2},\\
		& pRN^\alpha - (\lambda R^2+\phi(\epsilon/16))N^{\alpha/2} < L_1^N(RN^\alpha) <  pRN^\alpha - (\lambda R^2-\phi(\epsilon/16))N^{\alpha/2}
		\end{split}
		\end{equation}
		hold. Let $D$ denote the set of pairs $(\vec{x},\vec{y})$, with $\vec{x},\vec{y}\in\mathfrak{W}_{k-1}$ satisfying 
		\begin{enumerate}[label=(\arabic*)]
			
			\item $0\leq y_i - x_i \leq (R-r)N^\alpha$ for $1\leq i\leq k$,
			
			\item $prN^\alpha - (\lambda r^2+\phi(\epsilon/8))N^{\alpha/2} < x_1 <  prN^\alpha - (\lambda r^2-\phi(\epsilon/8))N^{\alpha/2}$ and $pRN^\alpha - (\lambda R^2+\phi(\epsilon/8))N^{\alpha/2} < y_1 <  pRN^\alpha - (\lambda R^2-\phi(\epsilon/8))N^{\alpha/2}$.
			
		\end{enumerate}
		Let $E(\vec{x},\vec{y})$ denote the subset of $F$ consisting of $L^N$ for which $L_i^N(rN^\alpha) = x_i$ and $L_i^N(RN^\alpha)=y_i$ for $1\leq i\leq k$, and $L_1^N(s) \geq \cdots \geq L_k^N(s)$ for all $s$. Then $D$ is countable, the $E(\vec{x},\vec{y})$ are pairwise disjoint, and $F = \bigcup_{(\vec{x},\vec{y})\in D} E(\vec{x},\vec{y})$.
		
		To prove the lemma, we argue that $\mathbb{P}(B) < \epsilon$ for large $N$. We split the proof into several steps.\\
		
		\noindent\textbf{Step 1.} We will argue in the following steps that for large enough $N$,
		\begin{equation}\label{21AEbound}
		\mathbb{P}(A\,|\, E(\vec{x},\vec{y})) > 1/4
		\end{equation}
		uniformly in $\vec{x},\vec{y}$. In this step, we prove the lemma assuming this fact. 
		
		It follows from \eqref{21AEbound} that
		\begin{equation}
		\mathbb{P}(A\,|\,F) = \sum_{(\vec{x},\vec{y})\in D} \frac{\mathbb{P}(A\,|\,E(\vec{x},\vec{y}))\mathbb{P}(E(\vec{x},\vec{y}))}{\mathbb{P}(F)} \geq \frac{1}{4}\cdot\frac{\sum_{(\vec{x},\vec{y})\in D} \mathbb{P}(E(\vec{x},\vec{y}))}{\mathbb{P}(F)} = \frac{1}{4}.
		\end{equation}
		From the third condition in Definition \ref{Def1}, we have $\mathbb{P}(A) < \epsilon/8$ for large enough $N$. Hence
		\[
		\mathbb{P}(F) = \frac{\mathbb{P}(A\cap F)}{\mathbb{P}(A\,|\,F)} \leq 4\mathbb{P}(A) < \epsilon/2.
		\]
		Now with probability $>1-\epsilon/2$, the two inequalities in \eqref{21x1y1} hold. We conclude that
		\[
		\mathbb{P}(B) \leq \mathbb{P}(F) + \epsilon/2 \leq \epsilon.
		\]
		
		\noindent\textbf{Step 2.} We will now prove \eqref{21AEbound}, assuming results from Steps 3 and 4 below. We first note that by Lemma \ref{MCLxy}, if we raise the endpoints of each curve, then the probability of the event $A$ will decrease. In particular, write $T = (R-r)N^\alpha$, and define $\vec{x}\,',\vec{y}\,'$ by
		\begin{align*}
		x_i' &= \lceil prN^\alpha - (\lambda r^2 - \phi(\epsilon/8))N^{\alpha/2}\,\rceil + (k-i)\lceil C\sqrt{T}\,\rceil,\\
		y_i' &= \lceil pRN^\alpha - (\lambda R^2 - \phi(\epsilon/8))N^{\alpha/2}\,\rceil + (k-i)\lceil C\sqrt{T}\,\rceil.
		\end{align*}
		Note that $x_i' \geq x_1 \geq x_i$ for each $i$ by condition (2) above. Furthermore, $x_i' - x_{i+1}' \geq C\sqrt{T}$. The same observations hold for $y_i'$. Using Lemma \ref{MCLxy}, we have
		\begin{align}
		\mathbb{P}(A\,|\,E(\vec{x},\vec{y})) &= \mathbb{P}^{rN^\alpha, RN^\alpha,\vec{x},\vec{y},\infty,L_k}_{avoid,Ber} (A\,|\,F) \geq \mathbb{P}^{rN^\alpha, RN^\alpha,\vec{x}\,',\vec{y}\,',\infty,L_k}_{avoid,Ber} (A\,|\,F) \nonumber \\
		&\geq \mathbb{P}^{rN^\alpha, RN^\alpha,\vec{x}',\vec{y}'}_{Ber} (A\cap\{L_1 \geq \cdots \geq L_k\}\,|\,F) \nonumber \\
		&\geq \mathbb{P}^{rN^\alpha, RN^\alpha,\vec{x}',\vec{y}'}_{Ber} (A\,|\,F) - \big( 1 - \mathbb{P}^{rN^\alpha, RN^\alpha,\vec{x}',\vec{y}'}_{Ber} (L_1 \geq \cdots \geq L_k\,|\,F)\big) \nonumber\\
		&= \mathbb{P}^{rN^\alpha, RN^\alpha,x_1',y_1'}_{Ber} (A) - \big( 1 - \mathbb{P}^{rN^\alpha, RN^\alpha,\vec{x}',\vec{y}'}_{Ber} (L_1 \geq \cdots \geq L_k\,|\,F)\big). \label{21xyest}
		\end{align}
		For the first term in the last line, we used the Gibbs property and the fact that $A$ and $F$ are independent under $\mathbb{P}^{rN^\alpha, RN^\alpha,\vec{x},\vec{y}}_{Ber}$. In Steps 3 and 4, we will argue that the two probabilities in \eqref{21xyest} are bounded below by 1/3 and 11/12, respectively. Then $\mathbb{P}(A\,|\,E(\vec{x},\vec{y})) \geq 1/3 - 1/12 = 1/4$ for large $N$ independent of $\vec{x},\vec{y}$, proving \eqref{21AEbound}.\\
		
		\noindent\textbf{Step 3.} In this step, we argue that $\mathbb{P}^{rN^\alpha, RN^\alpha,x_1',y_1'}_{Ber} (A) > 1/3$ for sufficiently large $N$. Write $\overline{x} = x_1' - (k-1)\lceil C\sqrt{T}\rceil$, $\overline{y} = y_1' - (k-1)\lceil C\sqrt{T}\rceil$, and $\overline{z} = \overline{y}-\overline{x}$. We have
		\begin{align}
		& \mathbb{P}^{rN^\alpha, RN^\alpha,x_1',y_1'}_{Ber} (A)\nonumber\\
		= \; &\mathbb{P}^{0,T,x_1',y_1'}_{Ber} \Big(L_1(T/2) - pN^\alpha \frac{R+r}{2} + \lambda\Big(\frac{R+r}{2}\Big)^2 N^{\alpha/2} < -\phi(\epsilon/8)N^{\alpha/2}\Big)\nonumber\\
		= \; & \mathbb{P}^{0,T,\overline{x},\overline{y}}_{Ber}\Big(L_1(T/2) - pN^\alpha\frac{R+r}{2} + \lambda\Big(\frac{R+r}{2}\Big)^2 N^{\alpha/2} < -\big(\phi(\epsilon/8) + (k-1)\lceil C\sqrt{R-r}\,\rceil\big)N^{\alpha/2}\Big)\nonumber\\
		\geq \; & \mathbb{P}^{0,T,\overline{x},\overline{y}}_{Ber}\Big(L_1(T/2) - \frac{\overline{x} + \overline{y}}{2} < \Big( \lambda\Big(\frac{R^2+r^2}{2}\Big) - \lambda\Big(\frac{R+r}{2}\Big)^2 - Ck\sqrt{R-r} - 2\phi(\epsilon/8)\Big)N^{\alpha/2}\Big). \label{21convex}
		\end{align}
		The inequality in the last line follows from the definitions of $\overline{x},\overline{y}$. Observe that
		\[
		\frac{R^2+r^2}{2} - \left(\frac{R+r}{2}\right)^2 = \frac{R^2 + r^2}{4} - \frac{rR}{4} = O(R^2)
		\]
		for fixed $r$. Our choice of $R$ from \eqref{21Rdef} ensures that the constant factor multiplying $N^{\alpha/2}$ in \eqref{21Rdef} is positive. Denoting this constant by $\gamma$, we see that \eqref{21convex} is bounded below by
		\[
		\mathbb{P}^{0,T,0,\overline{z}}_{Ber}\Big(L_1(T/2) - \overline{z}/2 < \gamma\sqrt{T}\Big).
		\]
		Let $\ell^{(T,\overline{z})}$ have the same law as $L_1$ under a probability measure $\mathbb{P}$ as in Theorem \ref{KMT}, and let $B^\sigma$, $\sigma^2 = p(1-p)$, be the Brownian bridge provided by the theorem. Then the last probability is
		\begin{align*}
		& \mathbb{P}\Big( \ell^{(T,\overline{z})}(T/2) - \overline{z}/2 < \gamma\sqrt{T}\Big) = \mathbb{P}\Big(\Big[\ell^{(T,\overline{z})}(T/2) - \overline{z}/2 - \sqrt{T}B^\sigma_{1/2}\Big] + \sqrt{T}B^\sigma_{1/2} < \gamma\sqrt{T}\Big) \\
		\geq \; & \mathbb{P}\Big(\sqrt{T}B^\sigma_{1/2} < 0\quad\mathrm{and}\quad \Delta(T,\overline{z}) < \gamma\sqrt{T}\Big) \geq \frac{1}{2} - \mathbb{P}\Big(\Delta(T,\overline{z}) \geq \gamma\sqrt{T}\Big).
		\end{align*}
		Here, $\Delta(T,\overline{z})$ is as defined in \eqref{KMTeq}. Observe that
		\begin{equation}\label{21zpT}
		\frac{|\overline{z} - pT|^2}{T} \leq \frac{(\lambda(R^2-r^2) N^{\alpha/2} + 1)^2}{(R-r)N^\alpha} \leq 4\lambda^2(R+r)^2(R-r).
		\end{equation}
		Thus Corollary \ref{Cheb} shows that $\mathbb{P}(\Delta(T,\overline{z})\geq \gamma\sqrt{T})<1/6$ for large enough $N$. This gives a lower bound on $\mathbb{P}^{rN^\alpha, RN^\alpha,x_1',y_1'}_{Ber} (A)$ of $1/2 - 1/6 = 1/3$ as desired.\\
		
		\noindent\textbf{Step 4.} In this last step, we show that $\mathbb{P}^{rN^\alpha, RN^\alpha,\vec{x},\vec{y}}_{Ber} (L_1 > \cdots > L_k\,|\,F) > 11/12$ for large $N$. Note that on the event $F$, $L_k^N$ lies uniformly below the line segment connecting $L_1^N(rN^\alpha)$ and $L_1^N(RN^\alpha)$. Thus after raising the endpoints to $\vec{x}\,',\vec{y}\,'$, the bottom curve $L_k$ lies uniformly at a distance of at least $C\sqrt{T}$ below the segment connecting $L_{k-1}(rN^\alpha)$ and $L_{k-1}(RN^\alpha)$, and moreover the endpoints of adjacent curves are at least $C\sqrt{T}$ apart. Then in order to have $L_1 \geq \cdots \geq L_k$ given $F$, it suffices to require each $L_i$ to lie within a distance of $C\sqrt{T}/2$ from the line segment connecting its endpoints. That is,
		\begin{align}
		&\mathbb{P}^{rN^\alpha, RN^\alpha,\vec{x}\,',\vec{y}\,'}_{Ber} (L_1 \geq \cdots \geq L_k\,|\,F) \nonumber\\
		\geq \; & \mathbb{P}^{rN^\alpha, RN^\alpha,\vec{x}\,',\vec{y}\,'}_{Ber} \Big(\sup_{x\in[r,R]} \big|L_i(xN^\alpha) - x_i' - (\overline{z}/T)(x-r)N^\alpha\big| \leq C\sqrt{T}/2, \;1\leq i\leq k-1\,\Big|\,F\Big) \nonumber\\
		= \; & \Big[ \mathbb{P}^{0,T,0,\overline{z}}_{Ber} \Big(\sup_{s\in[0,T]} \big|L_1(s+rN^\alpha) - (\overline{z}/T)s\big| \leq C\sqrt{T}/2\Big)\Big]^{k-1} \nonumber\\
		= \; & \Big[ 1 - \mathbb{P}\Big(\sup_{s\in[0,T]} \big|\ell^{(T,\overline{z})} - (\overline{z}/T)s\big| > C\sqrt{T}/2\Big)\Big]^{k-1}, \label{21avoidest}
		\end{align}
		with $\mathbb{P}$ and $\ell^{(T,\overline{z})}$ as in Step 1. In the third line, we used the fact that $L_1,\dots,L_{k-1}$ are independent from each other and from $L_k$ under $\mathbb{P}^{rN^\alpha, RN^\alpha,\vec{x},\vec{y}}_{Ber}$. Let $B^\sigma$ be as in Step 1. Then we have
		\begin{align*}
		&\mathbb{P} \Big(\sup_{s\in[0,T]} \big|\ell^{(T,\overline{z})}(s) - (\overline{z}/T)s\big| > C\sqrt{T}/2\Big)\\
		\leq \; & \mathbb{P}\Big(\sup_{s\in[0,T]} |\sqrt{T}B^{\sigma}_{s/T}| > C\sqrt{T}/4\Big) + \mathbb{P}\Big(\Delta(T,\overline{z}) > C\sqrt{T}/4\Big).
		\end{align*}
		By Lemma \ref{BBmax}, the first term is equal to
		\[
		2\exp\left(-\frac{2}{\sigma^2}\Big(\frac{C}{4}\Big)^2\right) = 2e^{-C^2/8p(1-p)}.
		\]
		For the second term, \eqref{21zpT} and Corollary \ref{Cheb} give an upper bound of $<e^{-C^2/8p(1-p)}$ for sufficiently large $N$. Our choice of $C$ in \eqref{21Cdef} implies a lower bound in \eqref{21avoidest} of
		\[
		(1-3e^{-C^2/8p(1-p)})^{k-1} > 11/12.
		\]
		\noindent Essentially the same argument proves the statement if $[r,R]$ is replaced by $[-R,-r]$.
		
	\end{proof}

	We now prove Lemma \ref{PropSup2}. We exploit Lemma \ref{21} in order to find two far away points where $L_k^N$ cannot be too low. After separating the curves in order to treat $L_k^N$ as a free curve as in the previous argument, we employ Lemma \ref{LemmaMinFreeS4} to bound the deviation of $L_k^N$ below the line of slope $p$.
	
	\begin{proof}
		We first introduce notation used in the proof. Define events
		\[
		A_N(R_2) = \left\{\inf_{s \in [ -t_2, t_2 ]}\big(L^N_k(s) - p s \big) \leq - R_2N^{\alpha/2}\right\},
		\]
		\begin{align*}
		B_N &= \left\{ \max_{x\in [r+2, R]} \big(L^N_k(xN^\alpha) - pxN^\alpha\big) > -MN^{\alpha/2} \right\}\\
		&\qquad \cap \left\{ \max_{x\in [-R, -r-2]} \big(L^N_k(xN^\alpha) - pxN^\alpha\big) > -MN^{\alpha/2} \right\}.
		\end{align*}
		Here, we define $M,R$ via Lemma \ref{21}, taking $R$ large enough so that with $M = \lambda R^2 + \phi(\epsilon/64)$, we have 
		\begin{equation}\label{4.3Bbound}
		\mathbb{P}(B_N^c) < \epsilon/2
		\end{equation} 
		for sufficiently large $N$. 
		
		For $0<a,b\in\mathbb{Z}$ and $\vec{x},\vec{y}\in\mathfrak{W}_k$, we define $E(a,b,\vec{x},\vec{y})$ to be the event that $L_i^N(-a) = x_i$ and $L_i^N(b) = y_i$ for $1\leq i\leq k$, and $L_1^N(s) > \cdots > L_k^N(s)$ for all $s\in[-RN^\alpha,RN^\alpha]$.
		
		We claim that $B_N(M,R)$ can be written as a countable disjoint union of sets $E(a,b,\vec{x},\vec{y})$. Let $D_N(M)$ be the collection of tuples $(a,b,\vec{x},\vec{y})$ satisfying 
		\begin{enumerate}[label=(\arabic*)]
			
			\item $a,b\in[rN^\alpha,RN^\alpha]$.
			
			\item $0 \leq y_i - x_i \leq b+a$, $x_k + pa > - MN^{\alpha/2}$, and $y_k - pb > - MN^{\alpha/2}$.
			
			\item If $c,d\in\mathbb{Z}$, $c > a$, and $d>b$, then $L_k^N(-c) + pc \leq -MN^{\alpha/2}$ and $L_k^N(d) - pd \leq -MN^{\alpha/2}$.
			
		\end{enumerate} 
		Since there are finitely many integers $a,b$ satisfying (1), the $x_i,y_i$ are integers, and there are finitely many choices of $L_i^N$ on $[-aN^\alpha, bN^\alpha]$ given $a,b,x_i,y_i$, we see that $D_N(M)$ is countable. The third condition ensures that the $E(a,b,\vec{x},\vec{y})$ are pairwise disjoint. To see that their union over $D_N(M)$ is all of $B_N(M,R)$, note that $B_N(M,R)$ occurs if and only if there is a first integer time $s=-a$ and a last integer time $s=b$ when $L_k^N(s)-ps$ crosses $-MN^{\alpha/2}$.
		
		Lastly, define the constant
		\begin{equation}\label{4.3Cdef}
		C = \sqrt{16p(1-p)\log\frac{3}{1-2^{-1/(k-1)}}}.
		\end{equation}
		We will prove that $\mathbb{P}(A_N(R_2)) < \epsilon$ for large $N$, if $R_2$ is chosen large enough depending on $M,C,k,r,\epsilon$. We specify how we choose $R_2$ after \eqref{4.3R2} below. We split the proof into steps for clarity.\\
		
		\noindent\textbf{Step 1.} We will prove in the steps below that for large enough $N$,
		\begin{equation}\label{4.3AEbound}
		\mathbb{P}(A_N(R_2)\,|\,E(a,b,\vec{x},\vec{y}) < \epsilon/2
		\end{equation}
		uniformly in $a,b,\vec{x},\vec{y}$. In this step, we prove the lemma assuming this fact.
		
		Since the $E(a,b,\vec{x},\vec{y})$ are disjoint, \eqref{4.3AEbound} implies
		\begin{align*}
		\mathbb{P}(A_N(R_2) \cap B_N(M,R)) &= \sum_{(a,b,\vec{x},\vec{y})\in D_N} \mathbb{P}(A_N(R_2)\,|\,E(a,b,\vec{x},\vec{y}))\mathbb{P}(E(a,b,\vec{x},\vec{y}))\\
		&\leq \frac{\epsilon}{2}\sum_{(a,b,\vec{x},\vec{y})\in D_N} \mathbb{P}(E(a,b,\vec{x},\vec{y})) \leq \frac{\epsilon}{2}.
		\end{align*}
		
		It follows from \eqref{4.3Bbound} that
		\begin{equation*}
		\mathbb{P}(A_N(R_2)) \leq \mathbb{P}(A_N(R_2)\cap B_N) + \epsilon/2 < \epsilon
		\end{equation*}
		for large enough $N$.\\
		
		\noindent\textbf{Step 2.} We next prove \eqref{4.3AEbound}, assuming results from Steps 3 and 4 below. Define $\vec{x}\,',\vec{y}\,'$ by
		\begin{align*}
		x_i' &= \lfloor - pa - MN^{\alpha/2}\rfloor - (i-1)\lceil CN^{\alpha/2}\rceil,\\
		y_i' &= \lfloor pb - MN^{\alpha/2}\rfloor - (i-1)\lceil CN^{\alpha/2}\rceil.
		\end{align*}  
		Observe that by condition (2) above, $x_i'\leq -pa - MN^{\alpha/2} \leq x_k \leq x_i$, and similarly for $\vec{y}$. It follows from Lemma \ref{MCLxy} that
		\begin{align}
		&\mathbb{P}(A_N(R_2)\,|\,E(a,b,\vec{x},\vec{y})) \leq \mathbb{P}^{-a,b, \vec{x}, \vec{y}}_{avoid, Ber} \Big( \inf_{s\in[-a, b]} \big(L_k(s) - ps\big) \leq -R_2N^{\alpha/2} \Big) \label{4.3main}\\
		= \; & \mathbb{P}^{0, a+b, \vec{x}, \vec{y}}_{avoid, Ber} \Big( \inf_{s\in[0,a+b]} \big(L_k(s-a) - p(s-a)\big) \leq -R_2N^{\alpha/2} \Big) \nonumber\\
		\leq \; & \mathbb{P}^{0, a+b, \vec{x}\,', \vec{y}\,'}_{avoid, Ber} \Big( \inf_{s\in[0,a+b]} \big(L_k'(s) - p(s-a)\big) \leq -R_2N^{\alpha/2} \Big) \nonumber.
		\end{align}
		In the last line, we have written $L_k'(s) = L_k(s-a)$. The last probability is bounded above by
		\begin{equation}\label{4.3bayes}
		\frac{\mathbb{P}^{0, a+b, \vec{x}\,', \vec{y}\,'}_{Ber} \Big( \inf_{s\in[0,a+b]} \big(\ell(s) - p(s-a)\big) \leq -R_2 N^{\alpha/2} \Big)}{\mathbb{P}^{0, a+b, \vec{x}\,', \vec{y}\,'}_{Ber}(F)},
		\end{equation}
		where
		\[
		F = \{L_1'(s) > \cdots > L_k'(s), \,s\in [0, a+b]\}.
		\]
		In Steps 3 and 4 below, we will prove that the numerator and denominator in \eqref{4.3bayes} are $<\epsilon/4$ and $>1/2$ for sufficiently large $N$. It then follows that \eqref{4.3bayes} is bounded above by $\epsilon/2$, proving \eqref{4.3AEbound}.\\
		
		\noindent\textbf{Step 3.} We now argue that
		\begin{equation}\label{4.3num}
		\mathbb{P}^{0, a+b, \vec{x}\,', \vec{y}\,'}_{Ber} \Big( \inf_{s\in[0,a+b]} \big(\ell(s) - p(s-a)\big) \leq -R_2 N^{\alpha/2} \Big) < \epsilon/4
		\end{equation}
		for sufficiently large $N$. Writing $\vec{z} = \vec{y}\,' - \vec{x}\,'$, \eqref{4.3num} is equal to
		\begin{align}
		& \mathbb{P}^{0, a+b, x_k', y_k'}_{Ber} \Big( \inf_{s\in[0,a+b]} \big(\ell(s) - p(s-a)\big) \leq -R_2 N^{\alpha/2} \Big) \nonumber\\
		= \; & \mathbb{P}^{0, a+b, 0, z_k}_{Ber} \Big( \inf_{s\in[0,a+b]} \big(\ell(s) - ps + pa - \lceil pa + MN^{\alpha/2}\rceil - (k-1)\lceil CN^{\alpha/2}\rceil\big) \leq -R_2 N^{\alpha/2} \Big) \nonumber\\
		\leq \; & \mathbb{P}^{0, a+b, 0, z_k}_{Ber} \Big( \inf_{s\in[0,a+b]} \big(\ell(s) - ps\big) \leq -(R_2 - M - Ck) N^{\alpha/2} \Big).\label{4.3R2}
		\end{align}
		Since $z_k\geq p(a+b)$, Lemma \ref{LemmaMinFreeS4} allows us to find $R_2>0$ depending on $M,C,k,r,\epsilon$ so that this probability is $<\epsilon/4$ for all large $N$, such that $a+b$ is larger than some constant $W_1$. But observe that $a+b \geq 2rN^\alpha$, so it suffices to take $N > (W_1/2r)^{1/\alpha}$. Thus we obtain \eqref{4.3num}, \textit{independent} of $a,b,\vec{x},\vec{y}$.\\
		
		\noindent\textbf{Step 4.} Lastly, we argue that
		\begin{equation}\label{4.3denom}
		\mathbb{P}^{0, a+b, \vec{x}\,', \vec{y}\,'}_{Ber}(F) > 1/2
		\end{equation}
		for large $N$. The argument is very similar to that in the proof of Lemma \ref{21}. Write $a = a'N^\alpha, b = b'N^\alpha$, $T = a+b = (a'+b')N^\alpha$, and $z = y_k' - x_k'$. Let $\ell^{(T,z)}$ be a random variable with the same law as the $L'_i$ shifted down by $x_i\,'$ under a measure $\mathbb{P}$, as provided by Theorem 3.3. Let $B^{\sigma}$ be a Brownian bridge with variance $\sigma^2 = p(1-p)$ coupled with $\ell^{(T,z)}$. Then
		\begin{align}
		\mathbb{P}^{0,T,\vec{x}\,', \vec{y}\,'}_{Ber}(F) &\geq \mathbb{P}^{0,T,\vec{x}\,', \vec{y}\,'}_{Ber} \Big( \sup_{s\in[0,T]} \Big|L_i'(s) - x_i' - (z/T)s\Big| < \frac{CN^{\alpha/2}}{2},\; 1\leq i\leq k \Big)\nonumber\\
		&= \Big[ 1 - \mathbb{P} \Big( \sup_{s\in[0,T]} \Big|\ell^{(T,z)}(s) - (z/T)s\Big| \geq C'\sqrt{T} \Big) \Big]^k,\label{4.3avoid}
		\end{align}
		where in the last line we have written $C' = C/2\sqrt{a'+b'}$. Now
		\begin{align}
		&\mathbb{P} \Big( \sup_{s\in[0,T]} \Big|\ell^{(T,z)}(s) - (z/T)s\Big| \geq C'\sqrt{T} \Big) \nonumber\\
		\leq \; & \mathbb{P} \Big( \sup_{s\in[0,T]} |\sqrt{T}\,B^{\sigma}_{s/T}| \geq C'\sqrt{T}/2\Big) + \mathbb{P} \Big( \Delta(T,z) \geq C'\sqrt{T}/2 \Big), \label{4.3BB}
		\end{align}
		where $\Delta(T,z)$ is as defined in Theorem 3.3. Since $a' + b' \leq 2R$, Lemma \ref{BBmax} implies that the first term in \eqref{4.3BB} is equal to
		\[
		2e^{-C^2/8\sigma^2(a'+b')} \leq 2e^{-C^2/16p(1-p)R}.
		\]
		Since $|z-pT| < 1$, Lemma \ref{Cheb} gives a lower bound for the second term in \eqref{4.3BB} of $e^{-C^2/16p(1-p)R}$ for $T\geq T_0$, where $T_0$ is some constant depending in particular on $C'$, hence possibly on $a+b$. But note that $C'\sqrt{T} = CN^{\alpha/2}$ is independent of $a,b$, so the $C'$ dependency in Lemma \ref{Cheb} reduces to a dependence on $C$. Moreover, $T\geq 2rN^\alpha$, so as long as $N \geq (T_0/2r)^{1/\alpha}$, we have the bound independent $a,b,\vec{x},\vec{y}$. Our choice of $C$ in \eqref{4.3Cdef} then gives a lower bound in \eqref{4.3avoid} of
		\[
		\big(1 - 3e^{-C^2/16p(1-p)R}\big)^k \geq 1/2.
		\]
		This proves \eqref{4.3denom}.
		
	\end{proof}
	
	
	
	\subsection{Proof of Lemma \ref{LemmaAP1}}
	
	We will require the following lemma, which essentially states that if we move the endpoints of a line ensemble closer together, then the acceptance probability decreases.
	
	\begin{lemma}\label{APsqueezing}
		Fix $T_0,T_1\in\mathbb{N}$ and $\vec{x},\vec{y}\in\mathfrak{W}_{k-1}$, $k\geq 3$, with $T_1 - T_0 \geq y_i - x_i \geq 0$ for $1\leq i\leq k-1$. Also let $z_0, z_1 \in\mathbb{N}$ satisfy $T_1 - T_0 \geq z_1 - z_0 \geq 0$, and let $\ell_{bot}\in\Omega(T_0,T_1,x_k,y_k)$. Suppose $\vec{x}\,', \vec{y}\,' \in\mathfrak{W}_{k-1}$ satisfy $T_1 - T_0 \geq y_i' - x_i' \geq 0$ for $1\leq i\leq k-1$ and
		\begin{enumerate}[label = (\arabic*)]
			
			\item $x_i' \leq x_i$ and $y_i' \leq y_i$ for $1\leq i\leq k-1$,
			
			\item $x_i' - x_{i+1}' \leq x_i - x_{i+1}$ and $y_i' - y_{i+1}' \leq y_i - y_{i+1}$ for $1\leq i\leq k-2$.
			
		\end{enumerate}
		Then
		\[
		Z(-t_1,t_1,\vec{x},\vec{y},\ell_{bot}) \geq Z(-t_1,t_1,\vec{x}\,', \vec{y}\,', \ell_{bot}).
		\]
	\end{lemma}
	
	\begin{proof}
		Without loss of generality, we can assume that $T_0 = 0$ and $T_1 = T$. Let $\mathfrak{L}'$ be a random line ensemble distributed according to $\mathbb{P}^{0, T, \vec{x}\,', \vec{y}\,'}_{Ber}$. Then we can represent each of the curves $L_i'$ in $\mathfrak{L}'$ as a sequence of $+$'s and $-$'s of length $T$, where a $+$ in position $t$ indicates that $L_i'(t+1) = L_i'(t) + 1$, and a $-$ indicates that $L_i'(t+1) = L_i'(t)$. For $s\in\llbracket 0, T\rrbracket$, let $P_i'(s)$ denote the number of $+$'s in the first $s$ slots of $L_i'$. Then particular, $P_i'(T) = y_i' - x_i'$.
		
		For each $i$, write $\Delta x_i = x_i - x_i'$, $\Delta y_i = y_i - y_i'$, and $\Delta_i = \Delta y_i - \Delta x_i$. If $\Delta_i \geq 0$, draw uniformly at random $\Delta_i$ of the $-$'s in the sequence representing $L_i'$ and change them to $+$'s. Otherwise, randomly change $\Delta_i$ of the $+$'s to $-$'s. Now let $P_i(s)$ denote the number of $+$'s in the first $s$ slots of this new sequence. Since $P_i(T) = (y_i' - x_i') + (y_i - y_i') - (x_i - x_i') = y_i - x_i$, this defines a new random line ensemble $\mathfrak{L} = (L_1,\dots,L_{k-1})$ distributed according to $\mathbb{P}^{0,T,\vec{x},\vec{y}}_{Ber}$.
		
		We claim that if $L_1' \geq \cdots \geq L_{k-1}' \geq \ell_{bot}$ on $[T_0, T_1]$, then also $L_1 \geq \cdots \geq L_{k-1} \geq \ell_{bot}$. Let us write $L_k = L_k' = \ell_{bot}$ for simplicity. Fix $i$ with $1\leq i\leq k - 1$, so that $L_i' \geq L_{i+1}'$ by assumption. Note that $L_{i+1}(s) > L_i(s)$ if and only if $P_{i+1}(s) - P_i(s) > x_i - x_{i+1}$. We have
		\begin{align*}
		P_{i+1}(s) - P_i(s) &= (P_{i+1}'(s) + \Delta_{i+1}) - (P_i'(s) + \Delta_i)\\
		&= (P_{i+1}'(s) - P_i'(s)) + (\Delta y_{i+1} - \Delta x_{i+1}) - (\Delta y_i - \Delta x_i)\\
		&\leq (x_i' - x_{i+1}') + (y_{i+1} - y_{i+1}') - (x_{i+1} - x_{i+1}') - (y_i - y_i') + (x_i - x_i')\\
		&= (x_i - x_{i+1}) + (y_i' - y_{i+1}') - (y_i - y_{i+1})\\
		&\leq x_i - x_{i+1}.
		\end{align*}
		In the third line, we used the assumption that $L_i'(s) \geq L_{i+1}'(s)$, and in the last line, we used condition (2) in the hypothesis. This proves the claim. It follows that
		\[
		\mathbf{1}_{L_1 \geq \cdots \geq L_{k-1} \geq \ell_{bot}} \geq \mathbf{1}_{L_1' \geq \cdots \geq L_{k-1}' \geq \ell_{bot}}.
		\]
		Taking expectations of both sides and recalling the distributions of $\mathfrak{L}$ and $\mathfrak{L}'$ proves the lemma.
		
		\end{proof}
		
	We will prove Lemma \ref{LemmaAP1} as a consequence of the following intermediate result.
	
	\begin{lemma}\label{LemmaAP2} Fix $k \in \mathbb{N}$, $p \in (0,1)$, $M_1, M_2 > 0$ . Suppose that $\ell_{bot}: \llbracket -t_2, t_2 \rrbracket \rightarrow \mathbb{R} \cup \{ - \infty \}$, and $\vec{x}, \vec{y} \in \mathfrak{W}_{k-1}$ are such that $2t_2 \geq y_i-x_i \geq 0$ for $i = 1, \dots, k-1$. Suppose further that
		\begin{enumerate}
			\item $\sup_{s \in [ -t_2,t_2]}\big(\ell_{bot}(s)  - ps \big)  \leq M_2 (2t_2)^{1/2}$,
			\item  $ x_{k-1} \geq \max\left(\ell_{bot}(-t_2), -pt_2- M_1 (2t_2)^{1/2}\right),$
			\item $ y_{k-1} \geq  \max \left( \ell_{bot}(t_2),  p t_2- M_1(2t_2)^{1/2} \right).$
		\end{enumerate}
		Let $g,h$ be as defined in Lemma \ref{LemmaAP1}. Denote $\widetilde{\ell}_{bot} = \ell_{bot} - \infty\cdot\mathbf{1}_{\llbracket -t_1 + 1, t_1 - 1\rrbracket}$. Then, there exists $N_5 = N_5(M_1,M_2,k ) \in \mathbb{N}$  such that for any $\tilde{\epsilon}  > 0$ and $N \geq N_5$ we have
		\begin{equation}\label{eqn61}
		\mathbb{P}^{-t_2, t_2, \vec{x},\vec{y}, \infty, \widetilde{\ell}_{bot} }_{avoid, Ber} \Big( Z\big(  -t_1, t_1, \widetilde{Q}(-t_1) , \widetilde{Q}(t_1), \ell_{bot}\llbracket -t_1, t_1\rrbracket\big) \geq g   \Big)  \geq h.
		\end{equation}
		Here, $\widetilde{Q}$ is a line ensemble with law $\mathbb{P}^{-t_2, t_2, \vec{x},\vec{y}, \infty, \widetilde{\ell}_{bot} }_{avoid, Ber}$, and $\widetilde{Q}(a) = \big(\widetilde{Q}_1(a),\dots,\widetilde{Q}_{k-1}(a)\big)$.
		
	\end{lemma}

	\begin{proof}
	
	Define 
	\begin{align*}
	\overline{x} &= \lceil (M_2+1)(2t_2)^{1/2} - pt_1\rceil, \quad
	\overline{y} = \lceil(M_2+1)(2t_2)^{1/2} + pt_1\rceil.
	\end{align*}
	Fix $\delta \in (0,1)$, and define the event 
	\[
	E = \big\{\widetilde{Q}_{k-1}(-t_1) \geq \overline{x} \big\} \cap \big\{\widetilde{Q}_{k-1}(t_1) \geq \overline{y} \big\} \cap \left\{\min_{1\leq i\leq k-2} \big(\widetilde{Q}_i(\pm t_1) - \widetilde{Q}_{i+1}(\pm t_1)\big) \geq \delta(2t_2)^{1/2}\right\}.
	\]
	By condition (1) in the hypothesis, and since $\delta < 1$, we have in addition on $E$ that
	\begin{align*}
	\widetilde{Q}_{k-1}(-t_1) - \ell_{bot}(- t_1) &\geq \overline{x} - \ell_{bot}(- t_1) \geq (2t_2)^{1/2} \geq \delta(2t_2)^{1/2},\\
	\widetilde{Q}_{k-1}(t_1) - \ell_{bot}(t_1) &\geq \overline{y} - \ell_{bot}(t_1) \geq (2t_2)^{1/2} \geq \delta(2t_2)^{1/2}.
	\end{align*}
	We split the remainder of the proof into several steps for clarity.\\
		
	\noindent\textbf{Step 1.} We will prove in the subsequent steps that 
		\begin{equation}\label{4.4Ebound}
		\mathbb{P}^{-t_2, t_2, \vec{x},\vec{y}, \infty, \widetilde{\ell}_{bot} }_{avoid, Ber}(E) \geq h
		\end{equation} 
		for large enough $N$. Here, we prove \eqref{eqn61} as a consequence by showing that
		\begin{equation}\label{4.4APbound}
		E \subset \left\{Z(-t_1,t_1,\widetilde{Q}(-t_1),\widetilde{Q}(t_1),\ell_{bot}) \geq g\right\}
		\end{equation}
		for large $N$. Define $\vec{x}\,', \vec{y}\,' \in \mathfrak{W}_{k-1}$ by
		\begin{align*}
		x_i' &= \overline{x} + (k-1-i)\lceil \delta(2t_2)^{1/2}\rceil,\\
		y_i' &= \overline{y} + (k-1-i)\lceil \delta(2t_2)^{1/2}\rceil.
		\end{align*}
		Observe that $x_i' \leq \widetilde{Q}_{k-1}(-t_1) + (k-1-i)\lceil \delta(2t_2)^{1/2}\rceil \leq \widetilde{Q}_i(-t_1)$, and similarly for $y_i'$. Also, $x_{i+1}' - x_i' \leq \widetilde{Q}_{i+1}(-t_1) - \widetilde{Q}_i(-t_1)$ and $y_{i+1}' - y_i' \leq \widetilde{Q}_{i+1}(t_1) - \widetilde{Q}_i(t_1)$ for $1\leq i\leq k-2$. By Lemma \ref{APsqueezing}, we have
		\begin{equation}\label{4.4squeeze}
		Z(-t_1,t_1,\widetilde{Q}(-t_1),\widetilde{Q}(t_1),\ell_{bot}) \geq Z(-t_1,t_1,\vec{x}\,', \vec{y}\,', \ell_{bot}).
		\end{equation}
		Observe that $x_{k-1}' = \overline{x}$ and $y_{k-1}' = \overline{y}$ lie at least $\delta(2t_2)^{1/2}$ above the points $\overline{x}\,' := M_2(2t_2)^{1/2} - pt_1$ and $\overline{y}\,' := M_2(2t_2)^{1/2} + pt_1$, respectively, and moreover, $\ell_{bot}$ lies uniformly below the line segment connecting $\overline{x}\,'$ and $\overline{y}\,'$. Since we also have $x_i' - x_{i+1}' \geq \delta(2t_2)^{1/2}$ and $y_i' - y_{i+1}' \geq \delta(2t_2)^{1/2}$ for each $i$, an acceptance occurs as long as each curve $\widetilde{Q}_i$ remains within a distance $\delta(2t)^{1/2}/2$ of the line segment connecting $x_i'$ and $y_i'$. Thus, writing $\overline{z} = \overline{y} - \overline{x}$ and $T = 2t_1$, we have
		\begin{equation}\label{4.4tube}
		Z(-t_1, t_1, \vec{x}\,', \vec{y}\,', \ell_{bot}) \geq \Big[1 - \mathbb{P}\Big(\sup_{s\in[0,T]}\big|\ell^{(T,\overline{z})}(s) - (\overline{z}/T)s\big| \geq \delta\sqrt{T}/2\Big)\Big]^{k-1}.
		\end{equation} 
		The probability on the right hand side is bounded above by
		\[
		\mathbb{P}\Big(\max_{s\in[0,T]} \big|B^\sigma_{s/T}\big| \geq \delta/4\Big) + \mathbb{P}\Big(\Delta(T,\overline{z}) \geq \delta\sqrt{T}/4\Big).
		\]
		The first term is equal to $2e^{-\delta^2/8p(1-p)}$ by Lemma \ref{BBmax}, and since $|\overline{z} - pT| \leq 1$, it follows from Lemma \ref{Cheb} that the second term is $< e^{-\delta^2/8p(1-p)}$ for large enough $N$ depending on $r,p,\delta$. In view of \eqref{4.4squeeze} and \eqref{4.4tube}, we obtain \eqref{4.4APbound}.
		
		\end{proof}
		
	We are now equipped to prove Lemma \ref{LemmaAP1}.\\ 
	
	First, define $S:=[t_1^-, t_1^+]$ and $T:=[t_2^-, t_1^-]\cup [t_1^+,t_2^+]$. Define $\pr_{\mathfrak L'}$ and $\pr_{\tilde{\mathfrak{L}}'}$ as the measures on Bernoulli line ensembles $\mathfrak L', \tilde{\mathfrak{L}}': T\to \R$ with $\mathfrak L'(t_2^-)=\tilde{\mathfrak{L}}':(t_2^-)=\vec x$ and $\mathfrak L'(t_2^+)=\tilde{\mathfrak{L}}'(t_2^+)=\vec y$, which are the restrictions of the previous measures $\pr_{\mathfrak L}$ and $\pr_{\tilde{\mathfrak{L}}}$ to $T$. The Radon-Nikodym derivative on line ensembles $\mathfrak B: T\to \R^k$ is \begin{equation}\label{propRadon}\frac{d\pr_{\mathfrak L'}}{d\pr_{\tilde{\mathfrak L}'}}\left(\mathfrak B\right)=\left(Z'\right)^{-1}Z\left(t_1^-,t_1^+,\mathfrak{B}\left(t_1^-\right),\mathfrak{B}\left(t_1^+\right),\ell_{bot};T\right)
	\end{equation}
	with $Z':=\ex_{\tilde{\mathfrak{L}}'}\left(Z\left(t_1^-, t_1^+, \mathfrak B(t_1^-), \mathfrak B(t_1^+), \ell_{bot}; T\right)\right)$ Now, note that $Z\left(t_1^-, t_1^+, \mathfrak B(t_1^-), \mathfrak B(t_1^+), \ell_{bot}; T\right)$ is a function of $(\mathfrak B(t_1^,), \mathfrak B(t_1^+))$, and in fact because of the manner of our restriction implies the equality of the laws of $\left(\mathfrak B(t_1^,), \mathfrak B(t_1^+)\right)$ and $(\tilde{\mathfrak L}(t_1^,), \tilde{\mathfrak L}(t_1^+))$ alongside inequality (get citation) we have
	\[
	Z'=\ex_{\tilde{\mathfrak{L}}'}\left[Z(t_1^-, t_1^+, \mathfrak B(t_1^-),\mathfrak B(t_1^+),\ell_{bot};S)\right]=\ex_{\tilde{\mathfrak{L}}}\left[Z(t_1^-, t_1^+, \tilde{\mathfrak L}(t_1^-),\tilde{\mathfrak L}(t_1^+),\ell_{bot};S)\right]\geq gh
\]
which gives us 
\begin{equation}
	\label{Zineq} \left(Z'\right)^{-1}\leq \frac{1}{gh}
\end{equation}

	For the same reasons as above, the law of $\mathfrak B(t_1^-), \mathfrak{B}(t_1^+)$ under $\pr_{\mathfrak L'}$ is the same as $\mathfrak L(t_1^,), \mathfrak L(t_1^+)$ under $\pr_{\mathfrak L}$ Hence,
	\begin{equation}
	\pr_{\mathfrak L}\left(Z\left(t_1^-,t_1^+,\mathfrak L(t_1^-),\mathfrak L(t_1^+),\ell_{bot};S\right)\leq gh\tilde \epsilon\right)=\pr_{\mathfrak L'}\left(Z\left(t_1^-,t_1^+,\mathfrak B(t_1^-),\mathfrak B(t_1^+),\ell_{bot};S\right)\leq gh\tilde \epsilon\right)
	\end{equation}
	
	Now if we let $E=\{Z(t_1^-,t_1^+,\mathfrak{B}(t_1^-),\mathfrak{B}(t_1^+),\ell_{bot};S)\leq gh\tilde\epsilon\}\subset \Omega$ with $\Omega$ defined as the probability space of paths of $\mathfrak B$. As a result, we have 
	\[
	\pr_{\mathfrak L'}(E)=\int_\Omega \indic_E\cdot d\pr_{\mathfrak L'}(B)=(Z')^{-1}\int_\Omega\indic_E\cdot Z(t_1^-,t_1^+,B(t_1^-),B(t_1^+,\ell_{bot};S)\cdot d\pr_{\tilde{\mathfrak{L}}'}
	\]
	using the Radon-Nikodym derivative from \ref{propRadon}. By the definition of $E$, we have that the acceptance probability is bounded by $gh\tilde{\epsilon}$, which gives the inequality \[\pr_{\mathfrak L'}(E)\leq (Z')^{-1}\int_\Omega \indic_E\cdot gh\tilde{\epsilon}\cdot d\pr_{\tilde{\mathfrak{L}}'}(B)\leq \frac{1}{gh}\int_\Omega\indic_E\cdot gh\tilde\epsilon\cdot d\pr_{\tilde{\mathfrak{L}}'}(B)\leq \tilde{\epsilon}\] with \ref{Zineq}  providing the middle inequality and the fact that $\indic_E\leq 1$ providing the final inequality. This inequality is the desired result.












