
%-------------------------------------------------------------------------------------------------------------------------------------------------------------------------------------------------
%    Section 5
%
%-------------------------------------------------------------------------------------------------------------------------------------------------------------------------------------------------
\section{Bounding the max and min}\label{Section5}

\subsection{Proof of Lemma \ref{PropSup}}

Our proof of Lemma \ref{PropSup} is similar to that of \cite[Lemma 5.2]{CD}. We exploit the one-point tightness of $L_1^N$ at two appropriately chosen points, and we use Lemma \ref{LemmaHalfS4} to control the upward deviation of $L_1^N$ from the line of slope $p$ away from these points.

\begin{proof}
	We write $s_4 = \lceil r+3 \rceil N^\alpha$, $s_3 = \lfloor r+3 \rfloor N^\alpha$, so that $s_3 \leq t_3 \leq s_4$, and take $N$ large enough so that $L_1^N$ is defined at $s_4$. We define events 
	\[
	E(M) = \Big\{\big|L_1^N(-s_4) + ps_4\big| > MN^{\alpha/2}\Big\}, \quad F(M) = \Big\{L_1^N(-s_3) > -ps_3 + MN^{\alpha/2} \Big\},
	\]
	\[
	G(M) = \Bigg\{\sup_{s\in[0,t_3]} \big(L_1^N(s) - ps \big) \geq (6r+22)(2r+6)^{1/2}(M+1)N^{\alpha/2} \Bigg\}.
	\]
	For $a,b\in\mathbb{Z}$, $s\in\llbracket 0, t_3 \rrbracket$, and $\ell_{bot}\in\Omega(-s_4,s,z_1,z_2)$ with $z_1\leq a$, $z_2\leq b$, we also define $E(a,b,s,\ell_{bot})$ to be the event that $L_1^N(-s_4) = a$, $L_1^N(s) = b$, and $L_2^N$ agrees with $\ell_{bot}$ on $[-s_4,s]$. 
	
	We claim that the set $G(M) \setminus E(M)$ can be written as a \textit{countable disjoint} union of sets $E(a,b,s,\ell_{bot})$. Let $D(M)$ be the set of tuples $(a,b,s,\ell_{bot})$ satisfying
	\begin{enumerate}[label=(\arabic*)]
		
		\item $0\leq s\leq t_3$,
		
		\item $0\leq b-a \leq s + s_4$, $|a + ps_4| \leq MN^{\alpha/2}$, and $b-ps \geq (6r+22)(2r+6)^{1/2}(M+1)N^{\alpha/2}$,
		
		\item If $0\leq s' < s$, then $b-ps' < (6r+22)(2r+6)^{1/2}(M+1)N^{\alpha/2}$,
		
		\item $z_1\leq a, z_2\leq b$, and $\ell_{bot}\in\Omega(-s_4, s, z_1, z_2)$.
		
	\end{enumerate}
	Conditions (1), (2), and (3) show that the union of these sets $E(a,b,s,\ell_{bot})$ for $(a,b,s,\ell_{bot})\in D(M)$ is $G(M)\setminus E(M)$. In particular, to see that condition (3) is not too restrictive, note that if $L^N$ is contained in $G(M)$, then there must be a first time $s\in\llbracket 0,t_3\rrbracket$ when $L_1^N(s)-ps \geq (6r+22)(2r+6)^{1/2}(M+1)N^\alpha/2$. Observe that $D(M)$ is countable, since there are finitely many possible choices of $s$, countably many $a,b$ and $z_1,z_2$ for each $s$, and finitely many $\ell_{bot}$ for each $z_1,z_2$. Moreover, condition (3) implies that the sets $E(a,b,s,\ell_{bot})$ are pairwise disjoint for distinct tuples in $D(M)$. This proves the claim.
	
	Now by one-point tightness of $L_1^N$ at integer multiples of $N^\alpha$, we can choose $M$ large enough depending on $\epsilon$ so that
	\begin{equation}
	\mathbb{P}(E(M)) < \epsilon/4, \quad \mathbb{P}(F(M)) < \epsilon/12 \label{4.2EFbounds}
	\end{equation}
	for all $N\in\mathbb{N}$. If $(a,b,s,\ell_{bot})\in D(M)$, then
	\begin{align*}
	\mathbb{P}^{-s_4,s,a,b}_{Ber}\Big( \ell(-s_3) > -ps_3 + MN^{\alpha/2}\Big) &= \mathbb{P}^{0,s+s_4,0,b-a}_{Ber}\Big(\ell(s_4-s_3) + a \geq -ps_3 + MN^{\alpha/2}\Big)\\
	&\geq \mathbb{P}^{0,s+s_4,0,b-a}_{Ber}\Big(\ell(s_4-s_3) \geq p(s_4-s_3) + 2MN^{\alpha/2}\Big).
	\end{align*}
	The inequality follows from the assumption in (2) that $a+ps_4 \geq -MN^{\alpha/2}$. Moreover, since $b-ps > (6r+22)(2r+6)^{1/2}(M+1)N^{\alpha/2}$ and $a+ps_4 \leq MN^{\alpha/2}$, we have 
	\[
	b-a \geq p(s+s_4) + (6r+21)(2r+6)^{1/2}(M+1)N^{\alpha/2} \geq p(s+s_4) + (6r+21)(M+1)(s+s_4)^{1/2}.
	\]  
	The second inequality follows since $s+s_4 \leq 2s_4 \leq (2r+6)N^{\alpha}$. It follows from Lemma \ref{LemmaHalfS4} with $M_1 = 0$, $M_2 = (6r+21)(M+1)$ that for sufficiently large $N$, we have
	\begin{equation}
	\mathbb{P}^{0,s+s_4,0,b-a}_{Ber}\Big(\ell(s_4-s_3) \geq \frac{s_4-s_3}{s+s_4}[p(s+s_4) + M_2 N^{\alpha/2}] - (s+s_4)^{1/4}\Big) \geq 1/3, \label{4.2interp}
	\end{equation}
	for all $(a,b,s,\ell_{bot}) \in D(M)$ simultaneously. Note that $\frac{s_4-s_3}{s+s_4} \geq \frac{N^\alpha - 1}{(2r+6)N^\alpha} \geq \frac{1}{2r+7}$
	for large $N$. Hence $\frac{s_4-s_3}{s+s_4}[p(s+s_4) + M_2 N^{\alpha/2}] - (s+s_4)^{1/4} \geq p(s_4-s_3) + 3(M+1)N^{\alpha/2} - (s+s_4)^{1/4}\geq p(s_4-s_3) + 2MN^{\alpha/2}$ for all large enough $N$. We conclude from \eqref{4.2interp} that
	\[
	\mathbb{P}^{-s_4,s,a,b}_{Ber}\Big(\ell(-s_3) > -ps_3 + MN^{\alpha/2}\Big) \geq 1/3
	\]
	uniformly in $a,b$ for large $N$. Now by the Schur Gibbs property for $L^N$, we have for any $\ell\in\Omega(-s_4,s,a,b)$ that
	\[
	\mathbb{P}(L_1^N|_{[-s_4,s]} = \ell\,|\,E(a,b,s,\ell_{bot})) = \mathbb{P}^{-s_4,s,a,b,\infty,\ell_{bot}}_{avoid, Ber}(\ell).
	\]
	Also observe that the event that $\ell(-s_3) > -ps_3 + MN^{\alpha/2}$ decreases in probability if $\ell$ is lowered. It follows from Lemma \ref{MCLfg} that
	\[
	\mathbb{P}^{-s_4,s,a,b,\infty,\ell_{bot}}_{avoid, Ber}\big(\ell(-s_3) > -ps_3 + MN^{\alpha/2}\big) \geq \mathbb{P}^{-s_4,s,a,b}_{Ber}\big(\ell(-s_3) > -ps_3 + MN^{\alpha/2}\big).
	\]
	Therefore
	\begin{align*}
	&\mathbb{P}\big( L_1^N(-s_3) > -ps_3 + MN^{\alpha/2}\,\big|\,E(a,b,s,\ell_{bot})\big)\\
	= \; & \sum_{\ell\in\Omega(-s_4,s,a,b)} \mathbb{P}^{-s_4,s,a,b,\infty,\ell_{bot}}_{avoid, Ber}(\ell)\cdot \mathbb{P}^{-s_4,s,a,b,\infty,\ell_{bot}}_{avoid, Ber}\big(\ell(-s_3) > -ps_3 + MN^{\alpha/2}\big)\\
	\geq \; & \sum_{\ell\in\Omega(-s_4,s,a,b)} \mathbb{P}^{-s_4,s,a,b,\infty,\ell_{bot}}_{avoid, Ber}(\ell)\cdot \mathbb{P}^{-s_4,s,a,b}_{Ber}\big(\ell(-s_3) > -ps_3 + MN^{\alpha/2}\big)\\
	\geq \; & \frac{1}{3}\sum_{\ell\in\Omega(-s_4,s,a,b)} \mathbb{P}^{-s_4,s,a,b,\infty,\ell_{bot}}_{avoid, Ber}(\ell) = \frac{1}{3}.
	\end{align*}
	Note once again that this bound holds independent of $a,b$ for all sufficiently large $N$. It follows from \eqref{4.2EFbounds} that
	\[
	\epsilon/12 > \mathbb{P}(F(M)) \geq \sum_{(a,b,s,\ell_{bot})\in D(M)} \mathbb{P}(F(M)\cap E(a,b,s,\ell_{bot}))
	\]
	\[
	= \sum_{(a,b,s,\ell_{bot})\in D(M)} \mathbb{P}(F(M)\,|\, E(a,b,s,\ell_{bot}))\mathbb{P}(E(a,b,s,\ell_{bot})) \geq \frac{1}{3}\mathbb{P}(G(M)\setminus E(M))
	\]
	for large $N$. Since in addition $\mathbb{P}(E(M)) < \epsilon/4$, we find that
	\[
	\mathbb{P}\Big( \sup_{s \in [0,t_3] }\big( L^N_1(s) - p s \big) \geq  (6r+22)(2r+6)^{1/2}(M+1)N^{\alpha/2} \Big) = \mathbb{P}(G(M)) < \epsilon/2
	\]
	for large enough $N$. A similar argument proves the same inequality with $[-t_3,0]$ in place of $[0,t_3]$. Thus we can find an $N_2 = N_2(\epsilon)$ so that
	\[
	\mathbb{P}\Big( \sup_{s \in [-t_3,t_3] }\big( L^N_1(s) - p s \big) \geq  R_1N^{\alpha/2} \Big) < \epsilon
	\]
	for all $N\geq N_2$, with $R_1 = (6r+22)(2r+6)^{1/2}(M+1)$.
	
\end{proof}
	
	
\subsection{Proof of Lemma \ref{PropSup2}}

	We begin by proving the following importnat lemma, which allows us to prevent the bottom curve of an ensemble from falling too low on a large enough interval.
	
	\begin{lemma}\label{21}
		Fix $p\in (0,1)$, $k\in\mathbb{N}$, and $\alpha,\lambda > 0$. Suppose that $\mathfrak{L}^N = (L_1^N, \dots, L_k^N)$ is a $(\alpha,p,\lambda)$-good sequence of $\llbracket 1, k\rrbracket$-indexed Bernoulli line ensembles. Then for any $r,\epsilon>0$, there exists $R>0$ depending on $\lambda,k,p,\epsilon,r,\phi$ and $N_0 \in \mathbb{N}$ depending on $\lambda,k,p,\epsilon,r,\phi,\psi,\alpha$ such that for all $N\geq N_0$,
		\[
		\mathbb{P}\Big(\max_{x\in[r,R]} \big(L_k^N(xN^\alpha) - pxN^\alpha\big) \leq -(\lambda R^2 + \phi(\epsilon/16))N^{\alpha/2}\Big) < \epsilon.
		\]
		The same statement holds if $[r,R]$ is replaced with $[-R,-r]$.
	\end{lemma}

	\begin{remark}
		The key to this lemma is the parabolic shift implicit in the definition of an $(\alpha,p,\lambda)$-good sequence. This requires the deviation of the top curve from the line of slope $p$ to appear roughly parabolic. Using monotone coupling, we separate the curves of the ensemble so that $L_1^N$ is nearly independent of the other curves. Then we would expect the value of $L_1^N$ at the midpoint of $r$ and $R$ to be close to the midpoint of the straight line segment connecting two points of the parabola. But the parabola is convex, so for large enough $R$ this violates the one-point tightness assumpion at $(R+r)/2$.
	\end{remark}
	
	\begin{proof}
		
		Fix $r>0$. Note that for any $R>r$,
		\[
		\max_{r\leq x\leq R} \big(L_k^N(xN^\alpha) - pxN^\alpha\big) \geq \max_{\lceil r\rceil \leq x \leq R} \big(L_k^N(xN^\alpha) - pxN^\alpha\big).
		\]
		Thus by replacing $r$ with $\lceil r\rceil$, we can assume that $r\in\mathbb{Z}$. Before beginning the proof, we introduce notation. Define constants
		\begin{align}
		C &= \sqrt{ 8p(1-p) \log\frac{3}{1-(11/12)^{1/(k-1)}}}\,,\label{21Cdef}\\
		R_0 &= Ck+\sqrt{C^{2}k^{2}+2\phi(\epsilon/16)}+r. \label{21Rdef}
		\end{align}
		Note that $R_0\geq r$. We define $R = R_0 + \mathbf{1}_{R_0 + r\;\mathrm{odd}}$, so that $R\geq R_0$ and the midpoint $(R+r)/2$ is an integer. In the following, we always assume $N$ is large enough depending on $\psi,R$ so that $L_1^N$ is defined at $R$. We may do so by the second condition in the definition of an $(\alpha,p,\lambda)$-good sequence (see Definition \ref{Def1}). Define events
		\begin{align*}
		A &= \left\{L_1^N\left(\frac{R+r}{2}\,N^\alpha\right) - pN^\alpha\,\frac{R+r}{2} + \lambda\left(\frac{R+r}{2}\right)^2 N^{\alpha/2} < -\phi(\epsilon/16)N^{\alpha/2}\right\},\\
		B &= \left\{\max_{x\in[r,R]} \left(L_k^N(xN^\alpha) - pxN^\alpha\right) \leq -(\lambda R^2 + \phi(\epsilon/16)) N^{\alpha/2} \right\}.
		\end{align*}
		Let $F$ denote the subset of $B$ for which the inequalities
		\begin{equation}\label{21x1y1}
		\begin{split}
		& prN^\alpha - (\lambda r^2+\phi(\epsilon/16))N^{\alpha/2} < L_1^N(rN^\alpha) <  prN^\alpha - (\lambda r^2-\phi(\epsilon/16))N^{\alpha/2},\\
		& pRN^\alpha - (\lambda R^2+\phi(\epsilon/16))N^{\alpha/2} < L_1^N(RN^\alpha) <  pRN^\alpha - (\lambda R^2-\phi(\epsilon/16))N^{\alpha/2}
		\end{split}
		\end{equation}
		hold. Let $D$ denote the set of pairs $(\vec{x},\vec{y})$, with $\vec{x},\vec{y}\in\mathfrak{W}_{k-1}$ satisfying 
		\begin{enumerate}[label=(\arabic*)]
			
			\item $0\leq y_i - x_i \leq (R-r)N^\alpha$ for $1\leq i\leq k$,
			
			\item $prN^\alpha - (\lambda r^2+\phi(\epsilon/16))N^{\alpha/2} < x_1 <  prN^\alpha - (\lambda r^2-\phi(\epsilon/16))N^{\alpha/2}$ and $pRN^\alpha - (\lambda R^2+\phi(\epsilon/16))N^{\alpha/2} < y_1 <  pRN^\alpha - (\lambda R^2-\phi(\epsilon/16))N^{\alpha/2}$.
			
		\end{enumerate}
		Let $E(\vec{x},\vec{y})$ denote the subset of $F$ consisting of $L^N$ for which $L_i^N(rN^\alpha) = x_i$ and $L_i^N(RN^\alpha)=y_i$ for $1\leq i\leq k$, and $L_1^N(s) \geq \cdots \geq L_k^N(s)$ for all $s$. Then $D$ is countable, the $E(\vec{x},\vec{y})$ are pairwise disjoint, and $F = \bigcup_{(\vec{x},\vec{y})\in D} E(\vec{x},\vec{y})$.
		
		To prove the lemma, we argue that $\mathbb{P}(B) < \epsilon$ for large $N$. We split the proof into several steps.\\
		
		\noindent\textbf{Step 1.} We will argue in the following steps that for large enough $N$,
		\begin{equation}\label{21AEbound}
		\mathbb{P}(A\,|\, E(\vec{x},\vec{y})) > 1/4
		\end{equation}
		uniformly in $\vec{x},\vec{y}$. In this step, we prove the lemma assuming this fact. 
		
		It follows from \eqref{21AEbound} that
		\begin{equation}
		\mathbb{P}(A\,|\,F) = \sum_{(\vec{x},\vec{y})\in D} \frac{\mathbb{P}(A\,|\,E(\vec{x},\vec{y}))\mathbb{P}(E(\vec{x},\vec{y}))}{\mathbb{P}(F)} \geq \frac{1}{4}\cdot\frac{\sum_{(\vec{x},\vec{y})\in D} \mathbb{P}(E(\vec{x},\vec{y}))}{\mathbb{P}(F)} = \frac{1}{4}.
		\end{equation}
		From the third condition in Definition \ref{Def1}, we have $\mathbb{P}(A) < \epsilon/8$ for large enough $N$. Hence
		\[
		\mathbb{P}(F) = \frac{\mathbb{P}(A\cap F)}{\mathbb{P}(A\,|\,F)} \leq 4\mathbb{P}(A) < \epsilon/2.
		\]
		Now with probability $>1-\epsilon/2$, the two inequalities in \eqref{21x1y1} hold. We conclude that
		\[
		\mathbb{P}(B) \leq \mathbb{P}(F) + \epsilon/2 \leq \epsilon.
		\]
		
		\noindent\textbf{Step 2.} We will now prove \eqref{21AEbound}, assuming results from Steps 3 and 4 below. We first note that by Lemma \ref{MCLxy}, if we raise the endpoints of each curve, then the probability of the event $A$ will decrease. In particular, write $T = (R-r)N^\alpha$, and define $\vec{x}\,',\vec{y}\,'$ by
		\begin{align*}
		x_i' &= \lceil prN^\alpha - (\lambda r^2 - \phi(\epsilon/8))N^{\alpha/2}\,\rceil + (k-i)\lceil C\sqrt{T}\,\rceil,\\
		y_i' &= \lceil pRN^\alpha - (\lambda R^2 - \phi(\epsilon/8))N^{\alpha/2}\,\rceil + (k-i)\lceil C\sqrt{T}\,\rceil.
		\end{align*}
		Note that $x_i' \geq x_1 \geq x_i$ for each $i$ by condition (2) above. Furthermore, $x_i' - x_{i+1}' \geq C\sqrt{T}$. The same observations hold for $y_i'$. Using Lemma \ref{MCLxy}, we have
		\begin{align}
		\mathbb{P}(A\,|\,E(\vec{x},\vec{y})) &= \mathbb{P}^{rN^\alpha, RN^\alpha,\vec{x},\vec{y},\infty,L_k}_{avoid,Ber} (A\,|\,F) \geq \mathbb{P}^{rN^\alpha, RN^\alpha,\vec{x}\,',\vec{y}\,',\infty,L_k}_{avoid,Ber} (A\,|\,F) \nonumber \\
		&\geq \mathbb{P}^{rN^\alpha, RN^\alpha,\vec{x}',\vec{y}'}_{Ber} (A\cap\{L_1 \geq \cdots \geq L_k\}\,|\,F) \nonumber \\
		&\geq \mathbb{P}^{rN^\alpha, RN^\alpha,\vec{x}',\vec{y}'}_{Ber} (A\,|\,F) - \big( 1 - \mathbb{P}^{rN^\alpha, RN^\alpha,\vec{x}',\vec{y}'}_{Ber} (L_1 \geq \cdots \geq L_k\,|\,F)\big) \nonumber\\
		&= \mathbb{P}^{rN^\alpha, RN^\alpha,x_1',y_1'}_{Ber} (A) - \big( 1 - \mathbb{P}^{rN^\alpha, RN^\alpha,\vec{x}',\vec{y}'}_{Ber} (L_1 \geq \cdots \geq L_k\,|\,F)\big). \label{21xyest}
		\end{align}
		For the first term in the last line, we used the Schur Gibbs property and the fact that $A$ and $F$ are independent under $\mathbb{P}^{rN^\alpha, RN^\alpha,\vec{x},\vec{y}}_{Ber}$. In Step 3, we will argue that the two probabilities in \eqref{21xyest} are bounded below by 1/3 and 11/12, respectively. Then $\mathbb{P}(A\,|\,E(\vec{x},\vec{y})) \geq 1/3 - 1/12 = 1/4$ for large $N$ independent of $\vec{x},\vec{y}$, proving \eqref{21AEbound}.\\
		
		\noindent\textbf{Step 3.} We first argue that $\mathbb{P}^{rN^\alpha, RN^\alpha,x_1',y_1'}_{Ber} (A) > 1/3$ for sufficiently large $N$. Write 
		\[
		\overline{x} = x_1' - (k-1)\lceil C\sqrt{T}\rceil, \quad \overline{y} = y_1' - (k-1)\lceil C\sqrt{T}\rceil,
		\] 
		and $\overline{z} = \overline{y}-\overline{x}$. We have
		\begin{align}
		& \mathbb{P}^{rN^\alpha, RN^\alpha,x_1',y_1'}_{Ber} (A)\nonumber\\
		= \; &\mathbb{P}^{0,T,x_1',y_1'}_{Ber} \Big(L_1(T/2) - pN^\alpha \frac{R+r}{2} + \lambda\Big(\frac{R+r}{2}\Big)^2 N^{\alpha/2} < -\phi(\epsilon/16)N^{\alpha/2}\Big)\nonumber\\
		= \; & \mathbb{P}^{0,T,\overline{x},\overline{y}}_{Ber}\Big(L_1(T/2) - pN^\alpha\frac{R+r}{2} + \lambda\Big(\frac{R+r}{2}\Big)^2 N^{\alpha/2} < -\big(\phi(\epsilon/16) + (k-1)\lceil C\sqrt{R-r}\,\rceil\big)N^{\alpha/2}\Big)\nonumber\\
		\geq \; & \mathbb{P}^{0,T,\overline{x},\overline{y}}_{Ber}\Big(L_1(T/2) - \frac{\overline{x} + \overline{y}}{2} < \Big( \lambda\Big(\frac{R^2+r^2}{2}\Big) - \lambda\Big(\frac{R+r}{2}\Big)^2 - Ck\sqrt{R-r} - 2\phi(\epsilon/16)\Big)N^{\alpha/2}\Big). \label{21convex}
		\end{align}
		The inequality in the last line follows from the definitions of $\overline{x},\overline{y}$. Observe that
		\[
		\frac{R^2+r^2}{2} - \left(\frac{R+r}{2}\right)^2 = \frac{R^2 + r^2}{4} - \frac{rR}{2} = O(R^2)
		\]
		for fixed $r$. Our choice of $R$ from \eqref{21Rdef} ensures that the constant factor multiplying $N^{\alpha/2}$ in \eqref{21Rdef} is positive. Denoting this constant by $\gamma_0$ and letting $\gamma = \gamma_0/\sqrt{R-r}$, we see that \eqref{21convex} is equal to
		\[
		\mathbb{P}^{0,T,0,\overline{z}}_{Ber}\Big(L_1(T/2) - \overline{z}/2 < \gamma\sqrt{T}\Big).
		\]
		Let $\ell^{(T,\overline{z})}$ have the same law as $L_1$ under a probability measure $\mathbb{P}$ as in Theorem \ref{KMT}, and let $B^\sigma$, $\sigma^2 = p(1-p)$, be the Brownian bridge provided by the theorem. Then the last probability is
		\begin{align*}
		& \mathbb{P}\Big( \ell^{(T,\overline{z})}(T/2) - \overline{z}/2 < \gamma\sqrt{T}\Big) = \mathbb{P}\Big(\Big[\ell^{(T,\overline{z})}(T/2) - \overline{z}/2 - \sqrt{T}B^\sigma_{1/2}\Big] + \sqrt{T}B^\sigma_{1/2} < \gamma\sqrt{T}\Big) \\
		\geq \; & \mathbb{P}\Big(\sqrt{T}B^\sigma_{1/2} < 0\quad\mathrm{and}\quad \Delta(T,\overline{z}) < \gamma\sqrt{T}\Big) \geq \frac{1}{2} - \mathbb{P}\Big(\Delta(T,\overline{z}) \geq \gamma\sqrt{T}\Big).
		\end{align*}
		Here, $\Delta(T,\overline{z})$ is as defined in \eqref{KMTeq}. Observe that
		\begin{equation}\label{21zpT}
		\frac{|\overline{z} - pT|^2}{T} \leq \frac{(\lambda(R^2-r^2) N^{\alpha/2} + 1)^2}{(R-r)N^\alpha} \leq 4\lambda^2(R+r)^2(R-r).
		\end{equation}
		Thus Corollary \ref{Cheb} shows that $\mathbb{P}(\Delta(T,\overline{z})\geq \gamma\sqrt{T})<1/6$ for large enough $N$. This gives a lower bound on $\mathbb{P}^{rN^\alpha, RN^\alpha,x_1',y_1'}_{Ber} (A)$ of $1/2 - 1/6 = 1/3$ as desired.
		
		Lastly, we show that $\mathbb{P}^{rN^\alpha, RN^\alpha,\vec{x}^{\prime},\vec{y}^{\prime}}_{Ber} (L_1 \geq \cdots \geq L_k\,|\,F) > 11/12$ for large $N$. Note that on the event $F$, $L_k^N$ lies uniformly below the line segment connecting $L_1^N(rN^\alpha)$ and $L_1^N(RN^\alpha)$. Thus after raising the endpoints to $\vec{x}\,',\vec{y}\,'$, the bottom curve $L_k$ lies uniformly at a distance of at least $C\sqrt{T}$ below the segment $\ell_{k-1}$ connecting $L_{k-1}(rN^\alpha)$ and $L_{k-1}(RN^\alpha)$, and moreover the endpoints of all adjacent curves are at least $C\sqrt{T}$ apart. Let $\ell_{bot}$ denote the segment $\ell_{k-1} - C\sqrt{T}$, so that $L_k \leq \ell_{bot}$. Then since $L_1,\dots,L_{k-1},\ell_{bot}$ are independent of $F$ under $\mathbb{P}^{rN^\alpha, RN^\alpha,\vec{x},\vec{y}}_{Ber}$, we have 
		\[
		\mathbb{P}^{rN^\alpha, RN^\alpha,\vec{x},\vec{y}}_{Ber} (L_1 \geq \cdots \geq L_k\,|\,F) \geq \mathbb{P}^{rN^\alpha, RN^\alpha,\vec{x},\vec{y}}_{Ber} (L_1 \geq \cdots \geq L_{k-1} \geq \ell_{bot}).
		\]
		In view of \eqref{21zpT}, we see from Lemma \ref{CurveSeparation} that for large enough $N$ depending on $\lambda, r, R, C$, the probability on right is bounded below by
		\[
		\big(1-3e^{-C^2/8p(1-p)}\big)^{k-1}.
		\]
		Our choice of $C$ in \eqref{21Cdef} implies that this quantity is at least 11/12.\\
		
		\noindent Essentially the same argument proves the statement if $[r,R]$ is replaced by $[-R,-r]$.
		
	\end{proof}

	We now prove Lemma \ref{PropSup2}. We exploit Lemma \ref{21} in order to find two far away points where $L_k^N$ cannot be too low. After separating the curves in order to treat $L_k^N$ as a free curve as in the previous argument, we employ Lemma \ref{LemmaMinFreeS4} to bound the deviation of $L_k^N$ below the line of slope $p$.
	
	\begin{proof}
		We first introduce notation used in the proof. Define events
		\[
		A_N(R_2) = \left\{\inf_{s \in [ -t_2, t_2 ]}\big(L^N_k(s) - p s \big) \leq - R_2N^{\alpha/2}\right\},
		\]
		\begin{align*}
		B_N &= \left\{ \max_{x\in [r+2, R]} \big(L^N_k(xN^\alpha) - pxN^\alpha\big) > -MN^{\alpha/2} \right\}\\
		&\qquad \cap \left\{ \max_{x\in [-R, -r-2]} \big(L^N_k(xN^\alpha) - pxN^\alpha\big) > -MN^{\alpha/2} \right\}.
		\end{align*}
		Here, we define $M,R$ via Lemma \ref{21}, taking $R$ large enough so that with $M = \lambda R^2 + \phi(\epsilon/64)$, we have 
		\begin{equation}\label{4.3Bbound}
		\mathbb{P}(B_N^c) < \epsilon/2
		\end{equation} 
		for sufficiently large $N$. 
		
		For $0<a,b\in\mathbb{Z}$ and $\vec{x},\vec{y}\in\mathfrak{W}_k$, we define $E(a,b,\vec{x},\vec{y})$ to be the event that $L_i^N(-a) = x_i$ and $L_i^N(b) = y_i$ for $1\leq i\leq k$, and $L_1^N(s) > \cdots > L_k^N(s)$ for all $s\in[-RN^\alpha,RN^\alpha]$.
		
		We claim that $B_N(M,R)$ can be written as a countable disjoint union of sets $E(a,b,\vec{x},\vec{y})$. Let $D_N(M)$ be the collection of tuples $(a,b,\vec{x},\vec{y})$ satisfying 
		\begin{enumerate}[label=(\arabic*)]
			
			\item $a,b\in[(r+2)N^\alpha,RN^\alpha]$.
			
			\item $0 \leq y_i - x_i \leq b+a$, $x_k + pa > - MN^{\alpha/2}$, and $y_k - pb > - MN^{\alpha/2}$.
			
			\item If $c,d\in\mathbb{Z}$, $c > a$, and $d>b$, then $L_k^N(-c) + pc \leq -MN^{\alpha/2}$ and $L_k^N(d) - pd \leq -MN^{\alpha/2}$.
			
		\end{enumerate} 
		Since there are finitely many integers $a,b$ satisfying (1), the $x_i,y_i$ are integers, and there are finitely many choices of $L_i^N$ on $[-aN^\alpha, bN^\alpha]$ given $a,b,x_i,y_i$, we see that $D_N(M)$ is countable. The third condition ensures that the $E(a,b,\vec{x},\vec{y})$ are pairwise disjoint. To see that their union over $D_N(M)$ is all of $B_N(M,R)$, note that $B_N(M,R)$ occurs if and only if there is a first integer time $s=-a$ and a last integer time $s=b$ when $L_k^N(s)-ps$ crosses $-MN^{\alpha/2}$.
		
		Lastly, define the constant
		\begin{equation}\label{4.3Cdef}
		C = \sqrt{16p(1-p)\log\frac{3}{1-2^{-1/(k-1)}}}.
		\end{equation}
		We will prove that $\mathbb{P}(A_N(R_2)) < \epsilon$ for large $N$, if $R_2$ is chosen large enough depending on $M,C,k,r,\epsilon$. We specify how we choose $R_2$ after \eqref{4.3R2} below. We split the proof into steps for clarity.\\
		
		\noindent\textbf{Step 1.} We will prove in the steps below that for large enough $N$,
		\begin{equation}\label{4.3AEbound}
		\mathbb{P}(A_N(R_2)\,|\,E(a,b,\vec{x},\vec{y})) < \epsilon/2
		\end{equation}
		uniformly in $a,b,\vec{x},\vec{y}$. In this step, we prove the lemma assuming this fact.
		
		Since the $E(a,b,\vec{x},\vec{y})$ are disjoint, \eqref{4.3AEbound} implies
		\begin{align*}
		\mathbb{P}(A_N(R_2) \cap B_N(M,R)) &= \sum_{(a,b,\vec{x},\vec{y})\in D_N} \mathbb{P}(A_N(R_2)\,|\,E(a,b,\vec{x},\vec{y}))\mathbb{P}(E(a,b,\vec{x},\vec{y}))\\
		&\leq \frac{\epsilon}{2}\sum_{(a,b,\vec{x},\vec{y})\in D_N} \mathbb{P}(E(a,b,\vec{x},\vec{y})) \leq \frac{\epsilon}{2}.
		\end{align*}
		
		It follows from \eqref{4.3Bbound} that
		\begin{equation*}
		\mathbb{P}(A_N(R_2)) \leq \mathbb{P}(A_N(R_2)\cap B_N) + \epsilon/2 < \epsilon
		\end{equation*}
		for large enough $N$.\\
		
		\noindent\textbf{Step 2.} We next prove \eqref{4.3AEbound}, assuming results from Steps 3 below. Define $\vec{x}\,',\vec{y}\,'$ by
		\begin{align*}
		x_i' &= \lfloor - pa - MN^{\alpha/2}\rfloor - (i-1)\lceil CN^{\alpha/2}\rceil,\\
		y_i' &= \lfloor pb - MN^{\alpha/2}\rfloor - (i-1)\lceil CN^{\alpha/2}\rceil.
		\end{align*}  
		Observe that by condition (2) above, $x_i'\leq -pa - MN^{\alpha/2} \leq x_k \leq x_i$, and similarly for $\vec{y}$. It follows from Lemma \ref{MCLxy} that
		\begin{align}
		&\mathbb{P}(A_N(R_2)\,|\,E(a,b,\vec{x},\vec{y})) \leq \mathbb{P}^{-a,b, \vec{x}, \vec{y}}_{avoid, Ber} \Big( \inf_{s\in[-a, b]} \big(L_k(s) - ps\big) \leq -R_2N^{\alpha/2} \Big) \label{4.3main}\\
		= \; & \mathbb{P}^{0, a+b, \vec{x}, \vec{y}}_{avoid, Ber} \Big( \inf_{s\in[0,a+b]} \big(L_k(s-a) - p(s-a)\big) \leq -R_2N^{\alpha/2} \Big) \nonumber\\
		\leq \; & \mathbb{P}^{0, a+b, \vec{x}\,', \vec{y}\,'}_{avoid, Ber} \Big( \inf_{s\in[0,a+b]} \big(L_k'(s) - p(s-a)\big) \leq -R_2N^{\alpha/2} \Big) \nonumber.
		\end{align}
		In the last line, we have written $L_k'(s) = L_k(s-a)$. The last probability is bounded above by
		\begin{equation}\label{4.3bayes}
		\frac{\mathbb{P}^{0, a+b, \vec{x}\,', \vec{y}\,'}_{Ber} \Big( \inf_{s\in[0,a+b]} \big(\ell(s) - p(s-a)\big) \leq -R_2 N^{\alpha/2} \Big)}{\mathbb{P}^{0, a+b, \vec{x}\,', \vec{y}\,'}_{Ber}(F)},
		\end{equation}
		where
		\[
		F = \{L_1'(s) > \cdots > L_k'(s), \,s\in [0, a+b]\}.
		\]
		In Step 3 below, we will prove that the numerator and denominator in \eqref{4.3bayes} are $<\epsilon/4$ and $>1/2$ for sufficiently large $N$. It then follows that \eqref{4.3bayes} is bounded above by $\epsilon/2$, proving \eqref{4.3AEbound}.\\
		
		\noindent\textbf{Step 3.} We first argue that
		\begin{equation}\label{4.3num}
		\mathbb{P}^{0, a+b, \vec{x}\,', \vec{y}\,'}_{Ber} \Big( \inf_{s\in[0,a+b]} \big(\ell(s) - p(s-a)\big) \leq -R_2 N^{\alpha/2} \Big) < \epsilon/4
		\end{equation}
		for sufficiently large $N$. Writing $\vec{z} = \vec{y}\,' - \vec{x}\,'$, \eqref{4.3num} is equal to
		\begin{align}
		& \mathbb{P}^{0, a+b, x_k', y_k'}_{Ber} \Big( \inf_{s\in[0,a+b]} \big(\ell(s) - p(s-a)\big) \leq -R_2 N^{\alpha/2} \Big) \nonumber\\
		= \; & \mathbb{P}^{0, a+b, 0, z_k}_{Ber} \Big( \inf_{s\in[0,a+b]} \big(\ell(s) - ps + pa - \lceil pa + MN^{\alpha/2}\rceil - (k-1)\lceil CN^{\alpha/2}\rceil\big) \leq -R_2 N^{\alpha/2} \Big) \nonumber\\
		\leq \; & \mathbb{P}^{0, a+b, 0, z_k}_{Ber} \Big( \inf_{s\in[0,a+b]} \big(\ell(s) - ps\big) \leq -(R_2 - M - Ck) N^{\alpha/2} \Big).\label{4.3R2}
		\end{align}
		Since $z_k\geq p(a+b)$, Lemma \ref{LemmaMinFreeS4} allows us to find $R_2>0$ depending on $M,C,k,r,\epsilon$ so that this probability is $<\epsilon/4$ for all large $N$, such that $a+b$ is larger than some constant $W_1$. But observe that $a+b \geq 2rN^\alpha$, so it suffices to take $N > (W_1/2r)^{1/\alpha}$. Thus we obtain \eqref{4.3num}, \textit{independent} of $a,b,\vec{x},\vec{y}$.
		
		Lastly, we argue that
		\begin{equation}\label{4.3denom}
		\mathbb{P}^{0, a+b, \vec{x}\,', \vec{y}\,'}_{Ber}(F) > 1/2
		\end{equation}
		for large $N$. Write $a = a'N^\alpha, b = b'N^\alpha$, $T = a+b = (a'+b')N^\alpha$, and $z = y_k' - x_k'$. Also let $C' = C/\sqrt{a'+b'}$, so that $x_i' - x_{i+1}' \geq CN^{\alpha/2} = C'\sqrt{T}$ and likewise for $y_i'$. Note that $|z-pT| < 1$. It follows from Lemma \ref{CurveSeparation}, applied with $k+1$ in place of $k$, $\ell_{bot} = -\infty$, and $C'$ in place of $C$, that for $T$ larger than some $T_0$, 
		\begin{equation}\label{4.3avoid}
		\mathbb{P}^{0, a+b, \vec{x}\,', \vec{y}\,'}_{Ber}(F) \geq \big(1 - 3e^{-(C')^2/8p(1-p)}\big)^k \geq \big(1 - 3e^{-C^2/16p(1-p)R}\big)^k.
		\end{equation}
		Here, we used the fact that $a'+b' \leq 2R$, hence $C' \geq C/\sqrt{2R}$. The constant $T_0$ depends in particular on $C'$, hence possibly on $a+b$. Referring to the proofs of Lemmas \ref{CurveSeparation} and \ref{Cheb}, we see that the dependency of $T_0$ on $C'$ amounts to requiring that $e^{-C'\sqrt{T_0}}$ be sufficiently small. But $C' \geq C/\sqrt{2R}$, so for this it suffices to choose $T_0$ depending on $C$ and $R$. Moreover, $T\geq 2rN^\alpha$, so as long as $N \geq (T_0/2r)^{1/\alpha}$, we have the bound in \eqref{4.3avoid} independent $a,b,\vec{x},\vec{y}$. Our choice of $C$ in \eqref{4.3Cdef} ensures that the expression on the right in \eqref{4.3avoid} is $> 1/2$, proving \eqref{4.3denom}
		
	\end{proof}
	
	
	

		
	
	
