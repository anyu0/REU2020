
%-------------------------------------------------------------------------------------------------------------------------------------------------------------------------------------------------
% Section 4
%
%-------------------------------------------------------------------------------------------------------------------------------------------------------------------------------------------------
\section{Proof of Theorem \ref{PropTightGood} }\label{Section4}


The goal of this section is to prove Theorem \ref{PropTightGood} and for the remainder we assume that $k \in \mathbb{N}$ with $k \geq 2$, $p \in (0,1)$, $\alpha, \lambda > 0$ are all fixed and 
\begin{equation}\label{eqalphagood}
\big\{\mathfrak{L}^N = (L^N_1,L^N_2, \dots, L^N_k)\big\}_{N=1}^{\infty},
\end{equation}
 is an $(\alpha,p,\lambda)$-good sequence of $\llbracket 1, k\rrbracket$-indexed Bernoulli line ensembles as in Definition \ref{Def1} that are all defined on a probability space with measure $\mathbb{P}$. The main technical result we will require is contained in Proposition \ref{PropMain} below and its proof is the content of Section \ref{Section4.1}. The proof of Theorem \ref{PropTightGood} is given in Section \ref{Section4.2}.
%-------------------------------------------------------------------------------------------------------------------------------------------------------------------------------------------------
% Section 4.1
%
%-------------------------------------------------------------------------------------------------------------------------------------------------------------------------------------------------
\subsection{Bounds on the acceptance probability}\label{Section4.1}
 The main result in this section is presented as Proposition \ref{PropMain} below. In order to formulate it and some of the lemmas below it will be convenient to adopt the following notation for any $r > 0$:
\begin{equation}\label{eqsts}
t_1 =\lfloor (r+1) N^{\alpha} \rfloor,\quad t_2 = \lfloor (r+2)N^{\alpha} \rfloor,\quad \textrm{and } t_3 = \lfloor (r+3)N^{\alpha} \rfloor.
\end{equation}
\begin{proposition}\label{PropMain} For any $\epsilon > 0$, $r > 0$ and any $(\alpha,p,\lambda)$-good sequence of Bernoulli line ensembles $\big\{ \mathfrak{L}^N  = (L^N_1,L^N_2, \dots, L^N_k)\big\}_{N=1}^{\infty}$
there exist $\delta > 0$ and $N_1$ (both depending on $\epsilon, r$ as well as $ \alpha, p, \lambda$ and the functions $\phi, \psi$ in Definition \ref{Def1}) such that for all $N \geq N_1$
we have 
$$\mathbb{P}\Big(Z\big( -t_1, t_1, \vec{x}, \vec{y} , L_{k}\llbracket -t_1, t_1\rrbracket\big) < \delta\Big) < \epsilon,$$
where $\vec{x} = (L_1^N(-t_1), \dots, L_{k-1}^N(-t_1)$, $\vec{y} = (L_1^N(t_1), \dots, L^N_{k-1}(t_1))$,  $ L_{k}\llbracket -t_1, t_1\rrbracket$ is the restriction of $L^N_k$ to the set $\llbracket -t_1, t_1\rrbracket$, and $Z$ is the acceptance probability of Definition \ref{DefAP}. $\mathbb{P}$ is the measure on a probability space that supports $\big\{ \mathfrak{L}^N \big\}_{N = 1}^\infty$.
\end{proposition}

The general strategy we use to prove Proposition \ref{PropMain} is inspired by the proof of Proposition 6.5 in \cite{CorHamK}. We begin by stating three key lemmas that will be required. Their proofs are postponed to Section \ref{Section5}. All constants in the statements below will depend implicitly on $\alpha$, $r$, $p$, $\lambda$, and the functions $\phi, \psi$ from Definition \ref{Def1}, which are fixed throughout. We will not list this dependence explicitly.

Lemma \ref{PropSup} controls the deviation of the curve $L^N_1(s)$ from the line $ps$ in the scale $N^{\alpha/2}$.
\begin{lemma}\label{PropSup} For each $\epsilon > 0$ there exist $R_1=R_1(\epsilon) > 0$ and $N_2= N_2(\epsilon)$ such that for $N \geq N_2$ 
$$\mathbb{P}\Big( \sup_{s \in [ -t_3, t_3] }\big( L^N_1(s) - p s \big) \geq  R_1N^{\alpha/2} \Big) < \epsilon.$$
\end{lemma}

Lemma \ref{PropSup2} controls the upper deviation of the curve $L^N_2(s)$ from the line $ps$ in the scale $N^{\alpha/2}$.
\begin{lemma}\label{PropSup2} For each $\epsilon > 0$ there exist $R_2=R_2( \epsilon) > 0$ and $N_3=N_3(\epsilon)$ such that for $N \geq N_3$
$$\mathbb{P}\Big( \inf_{s \in [ -t_2, t_2 ]}\big(L^N_k(s) - p s \big) \leq - R_2N^{\alpha/2} \Big) < \epsilon.$$
\end{lemma}

\begin{lemma}\label{LemmaAP1} Fix $k \in \mathbb{N}$, $p \in (0,1)$, $M_1, M_2 > 0$ . Suppose that $\ell_{bot}: \llbracket -t_2, t_2 \rrbracket \rightarrow \mathbb{R} \cup \{ - \infty \}$, and $\vec{x}, \vec{y} \in \mathfrak{W}_{k-1}$ are such that $2t_2 \geq y_i-x_i \geq 0$ for $i = 1, \dots, k-1$. Suppose further that
	\begin{enumerate}
		\item $\sup_{s \in [ -t_2,t_2]}\big(\ell_{bot}(s)  - ps \big)  \leq M_2 (2t_2)^{1/2}$,
		\item  $ x_{k-1} \geq \max\left(\ell_{bot}(t_2), -pt_2- M_1 (2t_2)^{1/2}\right),$
		\item $ y_{k-1} \geq  \max \left( \ell_{bot}(t_2),  p t_2- M_1(2t_2)^{1/2} \right).$
	\end{enumerate}
	Define the constants $g$ and $h$ (depending on $ M_1, M_2, p , k, r$) via
	$$g =  \cdots \mbox{ and } h = \cdots .$$
	Then, there exists $N_4 = N_4(M_1,M_2,k ) \in \mathbb{N}$  such that for any $\tilde{\epsilon}  > 0$ and $N \geq N_4$ we have
	\begin{equation}\label{eqn60}
	\mathbb{P}^{-t_2, t_2, \vec{x},\vec{y}, \infty, \ell_{bot} }_{avoid, Ber} \Big( Z\big(  -t_1, t_1, Q(-t_1) ,Q(t_1), \ell_{bot}\llbracket -t_1, t_1\rrbracket\big) \leq  gh \tilde{\epsilon}   \Big)  \leq \tilde{\epsilon},
	\end{equation}
	where $\ell_{bot}\llbracket -t_1, t_1\rrbracket$ is the vector, whose coordinates match those of $\ell_{bot}$ on $\llbracket -t_1, t_1\rrbracket$ and $Q(a) = (Q_1(a), \dots, Q_{k-1}(a))$ is the value of the line ensemble $Q$ whose law is $\mathbb{P}^{-t_2, t_2, \vec{x},\vec{y}, \infty, \ell_{bot} }_{avoid, Ber}$ at location $a$.
\end{lemma}


\begin{proof}[Almost proof of Proposition \ref{PropMain}] Let $\epsilon > 0$ be given. Define the event
	\begin{equation*}
	\begin{split}
	&E_N = \Big\{    L_{k-1}^N(  \pm t_2) \mp pt_2 \geq  - M_1 (2t_2)^{1/2}\Big\} \cap \Big\{ \sup_{s \in [ -t_2, t_2]} \big( {L}^N_{k}(s) - p s \big)\leq M_2  (2t_2)^{1/2} \Big\},
	\end{split}
	\end{equation*}
	where $M_1$ and $M_2$ are sufficiently large so that for all large $N$ we have $\mathbb{P}(E_N^c) <  \epsilon / 2$. The existence of such $M_1$ and $M_2$ is assured from Lemmas \ref{PropSup} and \ref{PropSup2}. 
	
	Let $\delta = (\epsilon/2) \cdot g h$, where $g,h$ are as in Lemma \ref{LemmaAP1} for the values $M_1, M_2$ as above and $r$ as in the statement of the proposition.
	We denote
	$$V = \Big\{Z\big( -t_1,t_1, \vec{x}, \vec{y} , L_{k}\llbracket -t_1, t_1\rrbracket\big)< \delta\Big\}$$
	and make the following deduction
	\begin{equation*}
	\begin{split}
	&\mathbb{P}\big( V \cap E_N \big) =\mathbb{E} \bigg[    \mathbb{E}\Big[{\bf 1}_{E_N} \cdot {\bf 1}_{V} \Big{|} \mathcal{F}_{ext} \big( \{1, \dots, k-1\} \times \llbracket -t_2 + 1,t_2 - 1\rrbracket \big)\Big] \bigg] = \\
	&\mathbb{E} \bigg[ {\bf 1}_{E_N} \cdot   \mathbb{E}\Big[ {\bf 1} \{ Z\big( -t_1,t_1, \vec{x}, \vec{y} , L_{k}\llbracket -t_1, t_1\rrbracket \big) < \delta\}   \Big{|} \mathcal{F}_{ext} \big( \{1, \dots, k-1\} \times \llbracket -t_2 + 1,t_2 - 1\rrbracket \big)\Big] \bigg]  = \\
	&\mathbb{E} \left[ {\bf 1}_{E_N} \cdot  \mathbb{E}^{-t_2, t_2, L^N(-t_2), L^N(t_2), \infty, L^N_k\llbracket -t_2, t_2\rrbracket }_{avoid, Ber}\left[ {\bf 1} \{ Z\big( -t_1, t_1, \ell(-t_1),\ell(t_1),{L}^N_k\llbracket -t_1, t_1\rrbracket \big) < \delta\} \right] \right] \leq \\
	&  \mathbb{E} \left[ {\bf 1}_{E_N} \cdot  \epsilon/2 \right] \leq \epsilon/2.
	\end{split}
	\end{equation*}
	The first equality follows from the tower property for conditional expectations. The second equality uses the fact that ${\bf 1}_{E_N} $ is $\mathcal{F}_{ext} \big( \{1\} \times \llbracket -t_2 + 1,t_2 - 1\rrbracket$-measurable and can thus be taken outside of the conditional expectation as well as the definition of $V$. The third equality uses the Schur Gibbs property. The inequality on the third line uses Lemma \ref{LemmaAP1} with $\tilde{\epsilon} = \epsilon/2$ as well as the fact that on the event $E_N^c$ the random variables $L^N(-t_2), L^N(t_2)$ and $L^N_k \llbracket -t_2, t_2 \rrbracket$ (that play the roles of $\vec{x}, \vec{y}$ and $\ell_{bot}$) satisfy the inequalities 
	$$L^N_{k-1}(-t_2) \geq  -pt_2- M_1 (2t_2)^{1/2},  L^N_{k-1}(t_2) \geq  p t_2- M_1(2t_2)^{1/2}, \sup_{s \in [ -t_2,t_2]} \hspace{-2mm}\big(L^N_k(s)  - ps \big)  \leq M_2(2t_2)^{1/2}.$$
	The last inequality is trivial.
	
	Combining the above inequality with $\mathbb{P}(E_N^c) <  \epsilon/2$, we see that for all large $N$ we have
	$$\mathbb{P}\left( V  \right) = \mathbb{P}(V \cap E_N) + \mathbb{P}(V \cap E_N^c) \leq \epsilon/2 + \mathbb{P}(E_N^c) < \epsilon.$$
\end{proof}


%-------------------------------------------------------------------------------------------------------------------------------------------------------------------------------------------------
% Section 4.2
%
%-------------------------------------------------------------------------------------------------------------------------------------------------------------------------------------------------
\subsection{Proof of Theorem \ref{PropTightGood} }\label{Section4.2}
	
	By Lemma \ref{2Tight}, it suffices to verify the following two conditions for all $1\leq i\leq k$, $R>0$, and $\epsilon>0$:
	\begin{align}
	\lim_{a\to\infty} &\limsup_{N\to\infty} \pr(|f^N_i(0)|\geq a) = 0 \label{ThmCond1}\\
	\lim_{\delta\to 0} &\limsup_{N\to\infty} \pr\bigg(\sup_{\substack{x,y\in [-R,R], \\ |x-y|\leq\delta}} |f^N_i(x) - f^N_i(y)| \geq \epsilon\bigg)= 0. \label{ThmCond2}
	\end{align}
	For the sake of clarity, we will prove these conditions in two separate steps.\\
	
	\noindent\textbf{Step 1.} We first prove condition \eqref{ThmCond1}, making use of Lemmas \ref{PropSup} and \ref{PropSup2} in order to obtain upper and lower bounds for the top and bottom curves respectively, thus bounding all curves.
	
	Fix $\epsilon>0$. We show that there exists an $a>0$ and $N'$ such that $N>N'$ implies
	$$\pr(|f^N_i(0)|\geq a) = \pr(|L_i^N(0)|\geq a N^{\alpha/2})<\epsilon.$$
	By Lemmas \ref{PropSup} and \ref{PropSup2}, there exist $R_1 := R_1(\epsilon/2)>0$, $R_2 := R_2(\epsilon/2)>0$ and $N_2 := N_2(\epsilon/2),N_3 := N_3(\epsilon/2)$ such that 
	\begin{align*}
	N\geq N_2 &\text{ implies } \pr\Big(\sup_{s\in[-t_3,t_3]}\left(L_1^N(s)-ps\right)\geq R_1N^{\alpha/2}\Big)<\epsilon/2,\\
	N\geq N_3 &\text{ implies } \pr\Big(\inf_{s\in[-t_2,t_2]}\left(L_k^N(s)-ps\right)\leq -R_2 N^{\alpha/2}\Big)<\epsilon/2.
	\end{align*}
	In particular, taking $s=0$, we find that for $N\geq N' := N_2 \vee N_3$, 
	\begin{align*}
	\pr\big(L_1^N(0)\geq R_1N^{\alpha/2}\big) &\leq \pr\Big(\sup_{s\in[-t_3,t_3]}\left(L_1^N(s)-ps\right)\geq R_1N^{\alpha/2}\Big)< \epsilon/2,\\
	\pr\big(L_k^N(0)\leq -R_2N^{\alpha/2}\big) &\leq \pr\Big(\inf_{s\in[-t_2,t_2]}\left(L_k^N(s)-ps\right)\leq -R_2N^{\alpha/2}\Big)< \epsilon/2.
	\end{align*}
	Letting $a = R_1 \vee R_2$ and noting that $L_1^N(0)>L_2^N(0)>...>L_k^N(0)$, we find that for $1\leq i\leq k$ and $N \geq N'$,
	$$\pr\big(|L_i^N(0)|\geq a N^{\alpha/2}\big)\leq \pr\big(L_1^N(0)\geq R_1 N^{\alpha/2}\big) + \pr\big(L_k^N(0)\leq -R_2N^{\alpha/2}\big) < \epsilon.$$
	This proves \eqref{ThmCond1}.\\
	
	\noindent\textbf{Step 2.} Here, we will prove condition \eqref{ThmCond2} for a fixed $i$. We must show that for all $\epsilon,\eta>0$ and $R>0$, there exists a $\delta$ and $N_0$ such that $N>N_0$ implies 
	\[
	\pr\bigg(\sup_{\substack{x,y\in [-R,R],\\ |x-y|\leq\delta}} |f^N_i(x) - f^N_i(y)| \geq \epsilon\bigg)<\eta.
	\]
	We rewrite the left hand side as
	\begin{align}
	\pr\bigg(\sup_{\substack{x,y\in [-R,R],\\ |x-y|\leq\delta}} \abs*{N^{-\alpha/2}\left(L^N_i(xN^{\alpha}) - L^N_i(yN^\alpha)\right)-p(x-y)N^{\alpha/2}+\lambda(x^2-y^2)} \geq \epsilon\bigg).
	\end{align}
	Given that $|x-y|<\delta$ and $x,y\in [-R,R]$, we know that $|x+y|\leq 2R$ and $|x-y|<\delta$, hence $|x^2-y^2|\leq 2R\delta$. Thus if we take $\delta < \frac{\epsilon}{8\lambda R}$, then the last probability is bounded below by
	\begin{align*}
	&\pr\bigg(\sup_{\substack{x,y\in [-R,R],\\ |x-y|\leq\delta}} N^{-\alpha/2}\abs*{L_i^N(xN^\alpha)-L_i^N(yN^\alpha)-p(x-y)N^\alpha}+2\lambda R\delta\geq \epsilon\bigg)\\
	\leq \; & \pr\bigg(\sup_{\substack{x,y\in [-R,R],\\ |x-y|\leq\delta}} \abs*{L_i^N(xN^\alpha)-L_i^N(yN^\alpha)-p(x-y)N^\alpha}\geq \frac{3N^{\alpha/2}\epsilon}4\bigg)\\
	= \; & \pr\bigg(\sup_{\substack{x,y\in [-RN^\alpha,RN^\alpha],\\ |x-y|\leq \delta N^\alpha}} \abs*{L_i^N(x)-L_i^N(y)-p(x-y)}\geq \frac{3N^{\alpha/2}\epsilon}4\bigg).
	\end{align*}
	We denote the event in the last line by $A_\delta$, and we now bound $\mathbb{P}(A_\delta)$ by size-biasing.
	
	Define events
	\begin{align*}
	E_1&=\left\{\max_{1\leq j \leq i}\abs*{f_j(\pm R)}\leq M_1\right\},\\
	E_2&=\left\{Z(-RN^\alpha, RN^\alpha, \vec x, \vec y, \infty, L_{i+1}^N[-RN^\alpha, RN^\alpha])>\delta_1\right\}.
	\end{align*}
	Here, $\vec{x} = (L_1^N(-RN^\alpha),\dots,L_i(-RN^\alpha))$ and $\vec{y} = (L_1^N(RN^\alpha),\dots,L_i(RN^\alpha))$. We argue that $E_1,E_2$ have high probability for appropriately chosen $M_1,\delta_1$, and it then suffices to bound the probability of $A_\delta$ on these events.
	
	Firstly, we observe that $L_j^N(\pm RN^\alpha)> L_{j+1}^N(\pm RN^\alpha)$, so $f_j^N(\pm R)>f_{j+1}^N(\pm R)$ as well. Thus
	\begin{align*}
	E_1^c &= \{f_1(\pm R)> M_1\} \cup \{f_i(\pm R)<-M_1\} \\
	&= \left\{ \left(L_1^N(\pm RN^\alpha)\mp pRN^\alpha\right)> (M_1-\lambda R^2)N^{\alpha/2}\right\}\\
	&\qquad \cup \left\{\left(L_i^N(\pm RN^\alpha)\mp pRN^\alpha\right)< -(\lambda R^2+M_1)N^{\alpha/2}\right\}.
	\end{align*}
	Now take $r>R$. Then in particular $RN^\alpha \leq t_3$, so we have $$
	\pr\left(L_1^N(\pm RN^\alpha)\mp prN^\alpha>(M_1-\lambda R^2)N^{\alpha/2}\right)
	\leq\pr\Big(\sup_{s\in[-t_3,t_3]}L_1^N(s)-ps>(M_1-\lambda R^2)N^{\alpha/2}\Big).
	$$ By Lemma \ref{PropSup}, we find that if $M_1>R_1(\eta/8)+\lambda R^2$ and $N>N_1(\eta/8)$, then this probability is less than $\eta/8$. Next, we have
	\begin{align*}
	\pr \left(L_i^N(\pm RN^\alpha)\mp pRN^\alpha< -(\lambda R^2+M_1)N^{\alpha/2}\right)&\leq \pr \left(L_i^N(\pm RN^\alpha)\mp pRN^\alpha< -M_1N^{\alpha/2}\right)\\
	&\leq \pr\Big(\inf_{s\in[-t_2,t_2]}\big(L_i^N(s)-ps\big)<-M_1N^{\alpha/2}\Big),
	\end{align*}
	and this last probability is $<\eta/8$ for $M_1\geq R_2(\eta/8)$ and $N>N_2(\eta/8)$ by Lemma \ref{PropSup2}. Therefore taking $M_1=\max\{R_1(\eta/8)+\lambda R^2,R_2(\eta/8)\}$, we find $$\pr(E_1^c)<\frac{\eta}{4}.$$
	Now by Proposition \ref{PropMain} with $r=R-1$, there exist $\delta_1(\eta/4)$ and $N_1(\eta/4)$ such that $N\geq N_1$ implies 
	\[
	\pr\left( E_2^c\right)<\frac{\eta}{4}.
	\]
	In summary, for $N>N_{01} := \max\{N_1(\eta/4),N_2(\eta/8),N_3(\eta/8)\}$, 
	\begin{equation}
	\pr(A_\delta)= \pr(A_\delta\cap E_1\cap E_2)+\pr(A_\delta\cap\left(E_1^c\cup E_2^c\right))\leq \pr(A_\delta\cap E_1\cap E_2)+\frac{\eta}{2}. \label{ThmBias}
	\end{equation}
	It remains to bound the first term. We define a $\sigma$-algebra 
	$$\mathcal{F}=\sigma\left(L_{i+1}^N,L_1^N(\pm RN^\alpha ), L_2^N(\pm RN^\alpha ),\dots, L_i^N(\pm RN^\alpha )\right).$$
	Clearly $E_1, E_2\in \mathcal{F}$, so the indicator random variables $\indic_{E_1}$ and $\indic_{E_2}$ are $\mathcal{F}$-measurable. It follows from the tower property of conditional expectation that
	\begin{align}
	\pr(A_\delta\cap E_1\cap E_2)&=\ex[\indic_{A_\delta} \indic_{E_1} \indic_{E_2}] =\ex[\indic_{E_1} \indic_{E_2}\ex[\indic_{A_\delta}\mid \mathcal{F}]\,]. \label{tower}
	\end{align}
	By the Schur-Gibbs property (see Definition \ref{DefSGP}),
	\[
	\ex[\indic_{A_\delta}\mid \mathcal{F}]=\ex_{avoid,Ber}^{-RN^\alpha,RN^\alpha,\vec x, \vec y, \infty, L_{i+1}^N}[\indic_{A_\delta}].
	\]
	We now observe that the Radon-Nikodym derivative of $\pr_{avoid,Ber}^{-RN^\alpha,RN^\alpha,\vec x, \vec y, \infty, L_{m+1}^N}$ with respect to $\pr_{Ber}^{-RN^\alpha, RN^\alpha,\vec x,\vec y}$ is given by 
	\begin{equation}
	\frac{d\pr_{avoid,Ber}^{-RN^\alpha,RN^\alpha,\vec x, \vec y, \infty, L_{i+1}^N}}{d\pr_{Ber}^{-RN^\alpha, RN^\alpha,\vec x,\vec y}} = \frac{\indic_{\left\{L_1 \geq \cdots \geq L_{i+1}\right\}}}{Z(-RN^\alpha,RN^\alpha,\vec x, \vec y, L_{i+1}^N)}. \label{RN}
	\end{equation}
	To see this, note that for any event $A$,
	\begin{align*}
	&\pr_{avoid,Ber}^{-RN^\alpha,RN^\alpha,\vec x, \vec y, \infty, L_{i+1}^N}(A) = \frac{\pr_{Ber}^{-RN^\alpha,RN^\alpha,\vec x, \vec y}(A\cap\left\{L_1 \geq \cdots \geq L_{i+1}\right\})}{\pr_{Ber}^{-RN^\alpha,RN^\alpha,\vec x, \vec y}(L_1\geq\cdots\geq L_{i+1})}\\
	= \; & \frac{\ex_{Ber}^{-RN^\alpha,RN^\alpha,\vec x, \vec y}\left[\indic_A \indic_{\left\{L_1\geq\cdots\geq L_{i+1}\right\}}\right]}{Z(-RN^\alpha,RN^\alpha,\vec{x},\vec{y},L^N_{i+1})} = \int_A \frac{\indic_{\left\{L_1\geq\cdots\geq L_{i+1}\right\}}}{Z(-RN^\alpha,RN^\alpha,\vec x, \vec y, L_{i+1}^N)}\,d\pr_{Ber}^{-RN^\alpha,RN^\alpha,\vec x, \vec y}.
	\end{align*}
	It follows from \eqref{tower}, \eqref{RN}, and the definition of $E_2$ that
	\begin{align*}
	\pr(A_\delta\cap E_1\cap E_2) &=\ex\left[\indic_{E_1}\indic_{E_2} \ex_{Ber}^{-RN^\alpha, RN^\alpha,\vec x,\vec y}\left[\frac{\indic_{A_\delta}\cdot \indic_{\left\{L_1\geq\cdots\geq L_{i+1}\right\}}}{Z(-RN^\alpha,RN^\alpha,\vec x, \vec y, L_{i+1}^N)}\right]\right]\\
	&\leq \ex\left[\indic_{E_1}\ex_{Ber}^{-RN^\alpha,RN^\alpha,\vec x,\vec y}\left[\frac{\indic_{A_\delta}}{\delta_1}\right]\right]\\
	&\leq \frac{1}{\delta_1}\,\pr_{Ber}^{-RN^\alpha,RN^\alpha,\vec x,\vec y}(A_\delta).
	\end{align*}
	By Lemma \ref{MOCLemmaS4}, there exist $N_4$ and $\delta$ such that $N>N_4$ implies
	\[
	\pr_{Ber}^{-RN^\alpha,RN^\alpha,\vec x,\vec y}(A_\delta)<\frac{\eta\,\delta_1}2,
	\] 
	and hence $$\pr\left(A_\delta\cap E_1\cap E_2\right)\leq \frac{\eta}{2}.$$ We conclude from \eqref{ThmBias} that $\mathbb{P}(A_\delta) < \eta$ for $N\geq N_0 := N_{01} \vee N_4$. This completes the proof.

