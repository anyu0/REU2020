%-------------------------------------------------------------------------------------------------------------------------------------------------------------------------------------------------
% Section 4
%
%-------------------------------------------------------------------------------------------------------------------------------------------------------------------------------------------------
\section{Proof of Theorem \ref{PropTightGood} }\label{Section4}


The goal of this section is to prove Theorem \ref{PropTightGood}. Throughout this section, we assume that we have fixed $k \in \mathbb{N}$ with $k \geq 2$, $p \in (0,1)$, $\alpha, \lambda > 0$, and
\begin{equation*}
\big\{\mathfrak{L}^N = (L^N_1,L^N_2, \dots, L^N_k)\big\}_{N=1}^{\infty}
\end{equation*}
an $(\alpha,p,\lambda)$-good sequence of $\llbracket 1, k\rrbracket$-indexed Bernoulli line ensembles as in Definition \ref{Def1}, all defined on a probability space with measure $\mathbb{P}$. The proof of Theorem \ref{PropTightGood} depends on three results -- Proposition \ref{PropMain} and Lemmas \ref{PropSup} and \ref{PropSup2}. In these three statements we establish various properties of the sequence of line ensembles $\mathfrak{L}^N$. The constants in these statements depend implicitly on $\alpha$, $p$, $\lambda$, $k$, and the functions $\phi, \psi$ from Definition \ref{Def1}, which are fixed throughout. We will not list these dependencies explicitly. The proof of Proposition \ref{PropMain} is given in Section \ref{Section4.1} while the proofs of Lemmas \ref{PropSup} and \ref{PropSup2} are in Section \ref{Section5}. Theorem \ref{PropTightGood} (i) and (ii) are proved in Sections \ref{Section4.2} and \ref{Section4.3} respectively.
%-------------------------------------------------------------------------------------------------------------------------------------------------------------------------------------------------
% Section 4.1
%
%-------------------------------------------------------------------------------------------------------------------------------------------------------------------------------------------------
\subsection{Bounds on the acceptance probability}\label{Section4.1}
The main result in this section is presented as Proposition \ref{PropMain} below. In order to formulate it and some of the lemmas below, it will be convenient to adopt the following notation for any $r > 0$ and $m \in \mathbb{N}$:
\begin{equation}\label{eqsts}
t_m =\lfloor (r+m) N^{\alpha} \rfloor.
\end{equation}
\begin{proposition}\label{PropMain} Let $\mathbb{P}$ be the measure from the beginning of this section. For any $\epsilon > 0$, $r > 0$ there exist $\delta = \delta(\epsilon, r) > 0$ and $N_1 = N_1(\epsilon, r)$ such that for all $N \geq N_1$
	we have 
	$$\mathbb{P}\Big(Z\big( -t_1, t_1, \vec{x}, \vec{y} , \infty,  L^N_{k}\llbracket -t_1, t_1\rrbracket\big) < \delta\Big) < \epsilon,$$
	where $\vec{x} = (L_1^N(-t_1), \dots, L_{k-1}^N(-t_1))$, $\vec{y} = (L_1^N(t_1), \dots, L^N_{k-1}(t_1))$,  $ L^N_{k}\llbracket -t_1, t_1\rrbracket$ is the restriction of $L^N_k$ to the set $\llbracket -t_1, t_1\rrbracket$, and $Z$ is the acceptance probability of Definition \ref{DefAP}. 
\end{proposition}

The general strategy we use to prove Proposition \ref{PropMain} is inspired by the proof of \cite[Proposition 6.5]{CorHamK}. We begin by stating three key lemmas that will be required. The proofs of Lemmas \ref{PropSup} and \ref{PropSup2} are postponed to Section \ref{Section5} and Lemma \ref{LemmaAP1} is proved in Section \ref{Section6}.


\begin{lemma}\label{PropSup}  Let $\mathbb{P}$ be the measure from the beginning of this section. For any $\epsilon > 0$, $r > 0$ there exist $R_1=R_1(\epsilon,r) > 0$ and $N_2= N_2(\epsilon,r)$ such that for $N \geq N_2$ 
	$$\mathbb{P} \left( \sup_{s \in [ -t_3, t_3] }\big[ L^N_1(s) - p s \big] \geq  R_1N^{\alpha/2} \right) < \epsilon.$$
\end{lemma}

\begin{lemma}\label{PropSup2}  Let $\mathbb{P}$ be the measure from the beginning of this section.  For any $\epsilon > 0$, $r > 0$ there exist $R_2=R_2( \epsilon,r) > 0$ and $N_3=N_3(\epsilon,r)$ such that for $N \geq N_3$
	$$\mathbb{P}\left( \inf_{s \in [ -t_3, t_3 ]}\big[L^N_{k-1}(s) - p s \big] \leq - R_2N^{\alpha/2} \right) < \epsilon.$$
\end{lemma}

\begin{lemma}\label{LemmaAP1} Fix $k \in \mathbb{N}$, $k \geq 2$, $p \in (0,1)$, $r, \alpha, M_1, M_2 > 0$ . Suppose that $\ell_{bot}: \llbracket -t_3, t_3 \rrbracket \rightarrow \mathbb{R} \cup \{ - \infty \}$, and $\vec{x}, \vec{y} \in \mathfrak{W}_{k-1}$ are such that $|\Omega_{avoid}(-t_3, t_3, \vec{x}, \vec{y}, \infty, \ell_{bot})| \geq 1$. Suppose further that
	\begin{enumerate}
		\item $\sup_{s \in [- t_3,t_3]}\big[\ell_{bot}(s)  - ps \big]  \leq M_2 (2t_3)^{1/2}$,
		\item  $-pt_3 + M_1 (2t_3)^{1/2} \geq  x_1 \geq  x_{k-1} \geq \max\left(\ell_{bot}(-t_3), -pt_3- M_1 (2t_3)^{1/2}\right),$
		\item $pt_3 + M_1 (2t_3)^{1/2} \geq y_1 \geq y_{k-1} \geq  \max \left( \ell_{bot}(t_3),  p t_3- M_1(2t_3)^{1/2} \right).$
	\end{enumerate}
	Then there exist constants $g,h$ and $N_4 \in \mathbb{N}$ all depending on $ M_1, M_2, p , k, r, \alpha$   such that for any $\tilde{\epsilon}  > 0$ and $N \geq N_4$ we have
	\begin{equation}\label{eqn60}
	\mathbb{P}^{-t_3, t_3, \vec{x},\vec{y}, \infty, \ell_{bot} }_{avoid, Ber} \Big( Z\big(  -t_1, t_1, \mathfrak{Q}(-t_1) ,\mathfrak{Q}(t_1), \infty,  \ell_{bot}\llbracket -t_1, t_1\rrbracket\big) \leq  gh \tilde{\epsilon}   \Big)  \leq \tilde{\epsilon},
	\end{equation}
	where $Z$ is the acceptance probability of Definition \ref{DefAP}, $\ell_{bot}\llbracket -t_1, t_1\rrbracket$ is the vector, whose coordinates match those of $\ell_{bot}$ on $\llbracket -t_1, t_1\rrbracket$ and $\mathfrak{Q}(a) = (Q_1(a), \dots, Q_{k-1}(a))$ is the value of the line ensemble $\mathfrak{Q} = (Q_1, \dots, Q_{k-1})$ whose law is $\mathbb{P}^{-t_3, t_3, \vec{x},\vec{y}, \infty, \ell_{bot} }_{avoid, Ber}$ at location $a$.
\end{lemma}


\begin{proof}[Proof of Proposition \ref{PropMain}] Let $\epsilon > 0$ be given. Define the event
	\begin{equation*}
	\begin{split}
	E_N = &\left\{    L_{k-1}^N(  \pm t_3) \mp pt_3 \geq  - M_1 (2t_3)^{1/2}\right\} \cap \left\{    L_{1}^N(  \pm t_3) \mp pt_3 \leq   M_1 (2t_3)^{1/2}\right\} \cap \\
	& \left\{ \sup_{s \in [ -t_3, t_3]} [{L}^N_{k}(s) - p s ]\leq M_2  (2t_3)^{1/2} \right\}.
	\end{split}
	\end{equation*}
	In view of  Lemmas \ref{PropSup} and \ref{PropSup2} and the fact that $\mathbb{P}$-almost surely $L_1^N(s) \geq L_k^N(s)$ for all $s \in [ -t_3, t_3]$ we can find sufficiently large $M_1, M_2$ and $N_2$ such that for $N \geq N_2$ we have 
	\begin{equation}\label{UBEC}
	\mathbb{P}(E_N^c) <  \epsilon / 2.
	\end{equation}
	
	
	Let $g,h, N_4$ be as in Lemma \ref{LemmaAP1} for the values $M_1, M_2$ as above, the values $\alpha, p, k$ from the beginning of the section and $r$ as in the statement of the proposition. For $\delta = (\epsilon/2) \cdot g h$ we denote
	$$V = \Big\{Z\big( -t_1,t_1, \vec{x}, \vec{y} , \infty,  L^N_{k}\llbracket -t_1, t_1\rrbracket\big)< \delta\Big\}$$
	and make the following deduction for $N \geq N_4$
	\begin{equation}\label{boundoneEN}
	\begin{split}
	&\mathbb{P}\big( V \cap E_N \big) =\mathbb{E} \bigg[    \mathbb{E}\Big[{\bf 1}_{E_N} \cdot {\bf 1}_{V} \Big{|} \sigma \big( \mathfrak{L}^N(-t_3),  \mathfrak{L}^N(t_3),L^N_k\llbracket -t_3, t_3\rrbracket   \big)\Big] \bigg] = \\
	&\mathbb{E} \bigg[ {\bf 1}_{E_N} \cdot   \mathbb{E}\Big[ {\bf 1} \{ Z\big( -t_1,t_1, \vec{x}, \vec{y} , \infty,  L^N_{k}\llbracket -t_1, t_1\rrbracket \big) < \delta\}   \Big{|} \sigma \big( \mathfrak{L}^N(-t_3),  \mathfrak{L}^N(t_3),L^N_k\llbracket -t_3, t_3\rrbracket   \big) \Big] \bigg]  = \\
	&\mathbb{E} \left[ {\bf 1}_{E_N} \cdot  \mathbb{E}_{avoid}\left[ {\bf 1} \{ Z\big( -t_1, t_1, \mathfrak{L}(-t_1),\mathfrak{L}(t_1), \infty, {L}^N_k\llbracket -t_1, t_1\rrbracket \big) < \delta\} \right] \right] \leq  \mathbb{E} \left[ {\bf 1}_{E_N} \cdot  \epsilon/2 \right] \leq \epsilon/2.
	\end{split}
	\end{equation}
	In (\ref{boundoneEN}) we have written $\mathbb{E}_{avoid}$ in place of $\mathbb{E}^{-t_3, t_3, \mathfrak{L}^N(-t_3), \mathfrak{L}^N(t_3), \infty, L^N_k\llbracket -t_3, t_3\rrbracket }_{avoid, Ber}$ to ease the notation; in addition, we have that $\mathfrak{L}^N(a) = (L_1^N(a), \dots, L_{k-1}^N(a))$ and $\mathfrak{L}$ on the last line is distributed according to $\mathbb{P}^{-t_3, t_3, \mathfrak{L}^N(-t_3), \mathfrak{L}^N(t_3), \infty, L^N_k\llbracket -t_3, t_3\rrbracket }_{avoid, Ber}$. We elaborate on (\ref{boundoneEN}) in the paragraph below.
	
	The first equality in (\ref{boundoneEN}) follows from the tower property for conditional expectations. The second equality uses the definition of $V$ and the fact that ${\bf 1}_{E_N} $ is $\sigma \big( \mathfrak{L}^N(-t_3),  \mathfrak{L}^N(t_3),L^N_k\llbracket -t_3, t_3\rrbracket   \big)$-measurable and can thus be taken outside of the conditional expectation. The third equality uses the Schur Gibbs property, see Definition \ref{DefSGP}. The first inequality on the third line holds if $N \geq N_4$ and uses Lemma \ref{LemmaAP1} with $\tilde{\epsilon} = \epsilon/2$  as well as the fact that on the event $E_N$ the random variables $\mathfrak{L}^N(-t_3), \mathfrak{L}^N(t_3)$ and $L^N_k \llbracket -t_3, t_3 \rrbracket$ (that play the roles of $\vec{x}, \vec{y}$ and $\ell_{bot}$) satisfy the inequalities 
	\begin{enumerate}
		\item $\sup_{s \in [- t_3,t_3]}\big[L^N_k(s)  - ps \big]  \leq M_2 (2t_3)^{1/2}$,
		\item  $-pt_3 + M_1 (2t_3)^{1/2} \geq  {L}^N_1(-t_3) \geq  {L}^N_{k-1}(-t_3) \geq \max\left(L^N_k(-t_3), -pt_3- M_1 (2t_3)^{1/2}\right),$
		\item $pt_3 + M_1 (2t_3)^{1/2} \geq {L}^N_1(t_3) \geq {L}^N_{k-1}(t_3) \geq  \max \left(L^N_k(t_3),  p t_3- M_1(2t_3)^{1/2} \right).$
	\end{enumerate}
	The last inequality in (\ref{boundoneEN}) is trivial.
	
	Combining (\ref{boundoneEN}) with (\ref{UBEC}), we see that for all $N \geq \max(N_2, N_4)$ we have
	$$\mathbb{P}\left( V  \right) = \mathbb{P}(V \cap E_N) + \mathbb{P}(V \cap E_N^c) \leq \epsilon/2 + \mathbb{P}(E_N^c) < \epsilon,$$
	which proves the proposition.
\end{proof}


%-------------------------------------------------------------------------------------------------------------------------------------------------------------------------------------------------
% Section 4.2
%
%-------------------------------------------------------------------------------------------------------------------------------------------------------------------------------------------------
\subsection{Proof of Theorem \ref{PropTightGood} (i) }\label{Section4.2}

By Lemma \ref{2Tight}, it suffices to verify the following two conditions for all $i \in \llbracket 1,k-1\rrbracket$, $r>0$, and $\epsilon>0$:
\begin{equation}\label{ThmCond1}
\lim_{a\to\infty} \limsup_{N\to\infty} \pr(|f^N_i(0)|\geq a) = 0 
\end{equation}
\begin{equation}\label{ThmCond2}
\lim_{\delta\to 0} \limsup_{N\to\infty} \pr\left(\sup_{{x,y\in [-r,r], |x-y|\leq\delta}} |f^N_i(x) - f^N_i(y)| \geq \epsilon \right)= 0.
\end{equation}
For the sake of clarity, we will prove these conditions in several steps.\\

\noindent\textbf{Step 1.} In this step we prove \eqref{ThmCond1}. Let $\epsilon > 0$ be given. Then by Lemmas \ref{PropSup} and \ref{PropSup2} we can find $N_2, N_3$ and $R_1, R_2$ such that for $N \geq \max(N_1, N_2)$
\begin{align*}
\pr\left(\sup_{s\in[-t_3,t_3]} [L_1^N(s)-ps]\geq R_1N^{\alpha/2}\right)<\epsilon/2,\\
\pr\left(\inf_{s\in[-t_3,t_3]}[L_{k-1}^N(s)-ps]\leq -R_2 N^{\alpha/2}\right)<\epsilon/2.
\end{align*}
In particular, if we set $R  = \max(R_1, R_2)$ and utilize the fact that $L_1^N(0) \geq \cdots \geq L_{k-1}^N(0)$ we conclude that for any $i \in \llbracket 1, k-1 \rrbracket$ we have
$$\pr\big(|L_i^N(0)|\geq R N^{\alpha/2}\big)\leq \pr\big(L_1^N(0)\geq R_1 N^{\alpha/2}\big) + \pr\big(L_{k-1}^N(0)\leq -R_2N^{\alpha/2}\big) < \epsilon,$$
which implies \eqref{ThmCond1}.\\

\noindent\textbf{Step 2.} In this step we prove (\ref{ThmCond2}). In the sequel we fix $r, \epsilon > 0$ and $i \in \llbracket 1, k-1 \rrbracket$. To prove (\ref{ThmCond2}) it suffices to show that for any $\eta>0$, there exists a $\delta > 0$ and $N_0$ such that $N \geq N_0$ implies 
\begin{equation}\label{S2R1}
\pr\left(\sup_{{x,y\in [-r,r], |x-y|\leq\delta}} |f^N_i(x) - f^N_i(y)| \geq \epsilon\right)<\eta.
\end{equation}

For $\delta > 0$ we define the event
\begin{equation}\label{Adelta}
A^N_\delta =\left\{\sup_{{x,y\in [-t_1,t_1], |x-y|\leq \delta N^\alpha}} \abs*{L_i^N(x)-L_i^N(y)-p(x-y)}\geq \frac{3N^{\alpha/2}\epsilon}4\right\},
\end{equation}
where we recall that $t_1 = \lfloor (r + 1)N^{\alpha} \rfloor$ from (\ref{eqsts}). We claim that there exist $\delta_0 > 0$ and $N_0 \in \mathbb{N}$ such that for $\delta \in (0, \delta_0]$ and $N \geq N_0$ we have
\begin{equation}\label{Abound}
\mathbb{P}(A^N_\delta) < \eta.
\end{equation}
We prove (\ref{Abound}) in the steps below. Here we assume its validity and conclude the proof of (\ref{S2R1}).\\

Observe that if $\delta \in \left (0, \min \left(\delta_0, \epsilon \cdot (8 \lambda r)^{-1} \right) \right)$, where $\lambda$ is as in the statement of the theorem, we have the following tower of inequalities
\begin{equation}\label{Term1}
\begin{split}
&\pr\left(\sup_{{x,y\in [-r,r], |x-y|\leq\delta}} |f^N_i(x) - f^N_i(y)| \geq \epsilon\right) = \\
&\pr\left(\sup_{{x,y\in [-r,r],|x-y|\leq\delta}} \abs*{N^{-\alpha/2}\left(L^N_i(xN^{\alpha}) - L^N_i(yN^\alpha)\right)-p(x-y)N^{\alpha/2}+\lambda(x^2-y^2)} \geq \epsilon\right) \leq \\
&\pr\left(\sup_{{x,y\in [-r,r], |x-y|\leq\delta}} N^{-\alpha/2}\abs*{L_i^N(xN^\alpha)-L_i^N(yN^\alpha)-p(x-y)N^\alpha}+2\lambda r\delta\geq \epsilon\right) \leq \\
& \pr\left(\sup_{{x,y\in [-r,r], |x-y|\leq\delta}} \abs*{L_i^N(xN^\alpha)-L_i^N(yN^\alpha)-p(x-y)N^\alpha}\geq \frac{3N^{\alpha/2}\epsilon}4\right)\leq \mathbb{P}(A^N_\delta) < \eta .
\end{split}
\end{equation}
In (\ref{Term1}) the first equality follows from the definition of $f_i^N$, and the inequality on the second line follows from the inequality $|x^2-y^2|\leq 2r\delta$, which holds for all $x,y \in [-r,r]$ such that $|x-y| \leq \delta$. The inequality in the third line of (\ref{Term1}) follows from our assumption that $\delta <   \epsilon \cdot (8 \lambda r)^{-1}$ and the first inequality on the last line follows from the definition of $A^N_\delta$ in (\ref{Adelta}), and the fact that $t_1 \geq r N^{\alpha}$. The last inequality follows from our assumption that $\delta < \delta_0$ and (\ref{Abound}). In view of (\ref{Term1}) we conclude (\ref{S2R1}).\\

{\bf \raggedleft Step 3.} In this step we prove (\ref{Abound}) and fix $\eta > 0$ in the sequel. For $\delta_1, M_1 > 0$ and $N \in \mathbb{N}$ we define the events
\begin{equation}
\begin{split}
E_1= \hspace{-0.5mm}\left\{\max_{1\leq j \leq k-1}\abs*{L_j^N(\pm t_1) \mp pt_1}\leq M_1N^{\alpha/2} \right\}, E_2=\hspace{-0.5mm}\left\{Z(-t_1, t_1, \vec x, \vec y, \infty, L_{k}^N\llbracket -t_1, t_1 \rrbracket)>\delta_1\right\},
\end{split}
\end{equation}
where we used the same notation as in Proposition \ref{PropMain} (in particular $\vec x = (L^N_1(-t_1), \dots, L_{k-1}^N(-t_1))$ and $\vec y = (L_1^N(t_1), \dots, L_{k-1}^N(t_1))$). Combining Lemmas \ref{PropSup}, \ref{PropSup2} and Proposition \ref{PropMain} we know that we can find $\delta_1 > 0$ sufficiently small, $M_1$ sufficiently large and $\tilde{N} \in \mathbb{N}$ such that for $N \geq \tilde{N}$ we know
\begin{equation}\label{BoundECup}
\mathbb{P} \left(E_1^c \cup E_2^c \right) < \eta/2.
\end{equation}
We claim that we can find $\delta_0 > 0$ and $N_0 \geq \tilde{N}$ such that for $N \geq N_0$ and $\delta \in (0, \delta_0)$ we have
\begin{equation}\label{Abound2}
\mathbb{P}(A^N_\delta \cap E_1 \cap E_2 ) < \eta/2.
\end{equation}
Since
$$\pr(A^N_\delta)= \pr(A^N_\delta\cap E_1\cap E_2)+\pr(A^N_\delta\cap\left(E_1^c\cup E_2^c\right))\leq \pr(A^N_\delta\cap E_1\cap E_2)+\mathbb{P} \left(E_1^c \cup E_2^c \right),$$
we see that (\ref{BoundECup}) and (\ref{Abound2}) together imply (\ref{Abound}).\\

{\bf \raggedleft Step 4.} In this step we prove (\ref{Abound2}). We define the $\sigma$-algebra 
$$\mathcal{F}=\sigma\left(L_{k}^N \llbracket -t_1, t_1 \rrbracket ,L_1^N(\pm t_1 ), L_2^N(\pm t_1 ),\dots, L_{k-1}^N(\pm t_1 )\right).$$
Clearly $E_1, E_2\in \mathcal{F}$, so the indicator random variables $\indic_{E_1}$ and $\indic_{E_2}$ are $\mathcal{F}$-measurable. It follows from the tower property of conditional expectation that
\begin{equation}\label{tower}
\pr\left(A^N_\delta\cap E_1\cap E_2\right)=\ex\left[\indic_{A^N_\delta} \indic_{E_1} \indic_{E_2}\right] =\ex\left[\indic_{E_1} \indic_{E_2}\ex\left[\indic_{A^N_\delta}\mid \mathcal{F}\right]\right]. 
\end{equation}
By the Schur-Gibbs property (see Definition \ref{DefSGP}), we know that $\mathbb{P}$-almost surely
\begin{equation}\label{tower2}
\ex\left[\indic_{A^N_\delta}\mid \mathcal{F}\right]=\ex_{avoid,Ber}^{-t_1,t_1,\vec x, \vec y, \infty, L_{k}^N \llbracket -t_1, t_1 \rrbracket}\left[\indic_{A^N_\delta}\right].
\end{equation}
We now observe that the Radon-Nikodym derivative of $\pr_{avoid,Ber}^{-t_1,t_1,\vec x, \vec y, \infty, L_{k}^N \llbracket -t_1, t_1 \rrbracket}$ with respect to $\pr_{Ber}^{-t_1, t_1,\vec x,\vec y}$ is given by 
\begin{equation}\label{RN}
\frac{d\pr_{avoid,Ber}^{-t_1,t_1,\vec x, \vec y, \infty, L_{k}^N\llbracket -t_1, t_1 \rrbracket } (Q_1, \dots, Q_{k-1})}{d\pr_{Ber}^{-t_1, t_1,\vec x,\vec y}} = \frac{\indic_{\left\{Q_1 \geq \cdots \geq Q_{k-1} \geq Q_k \right\}}}{Z(-t_1,t_1,\vec x, \vec y, \infty,  L_{k}^N\llbracket -t_1, t_1 \rrbracket )},
\end{equation}
where $\mathfrak{Q} = (Q_1, \dots, Q_{k-1})$ is $\pr_{Ber}^{-t_1, t_1,\vec x,\vec y}$-distributed and $Q_k = L_{k}^N\llbracket -t_1, t_1 \rrbracket$. To see this, note that by Definition \ref{DefAvoidingLawBer} we have for any set $A \subset  \prod_{i = 1}^{k-1}\Omega(-t_1, t_1, x_i, y_i )$ that 
\begin{equation*}
\begin{split}
&\pr_{avoid,Ber}^{-t_1,t_1,\vec x, \vec y, \infty, L_{k}^N\llbracket -t_1, t_1 \rrbracket}(A) = \frac{\pr_{Ber}^{-t_1,t_1,\vec x, \vec y}(A\cap\left\{Q_1 \geq \cdots \geq Q_{k-1} \geq Q_k\right\})}{\pr_{Ber}^{-t_1,t_1,\vec x, \vec y}(Q_1\geq\cdots\geq Q_{k-1} \geq Q_k)} = \\
& \frac{\ex_{Ber}^{-t_1,t_1,\vec x, \vec y}\left[\indic_A \cdot  \indic_{\left\{Q_1\geq\cdots\geq Q_{k-1} \geq Q_k \right\}}\right]}{Z(-t_1,t_1,\vec{x},\vec{y}, \infty, L^N_{k} \llbracket -t_1, t_1 \rrbracket)} = \int_A \frac{\indic_{\left\{Q_1\geq\cdots\geq Q_{k-1} \geq Q_k \right\}}}{Z(-t_1,t_1,\vec x, \vec y, \infty, L_{k}^N \llbracket -t_1, t_1 \rrbracket )}\,d\pr_{Ber}^{-t_1,t_1,\vec x, \vec y}.
\end{split}
\end{equation*}
It follows from \eqref{tower}, \eqref{RN}, and the definition of $E_2$ that
\begin{equation}\label{RT1}
\begin{split}
&\pr(A^N_\delta\cap E_1\cap E_2) =\ex\left[\indic_{E_1}\indic_{E_2} \ex_{Ber}^{-t_1, t_1,\vec x,\vec y}\left[\frac{\indic_{B^N_\delta}\cdot \indic_{\left\{Q_1\geq\cdots\geq Q_{k}\right\}}}{Z(-t_1,t_1,\vec x, \vec y, L^N_{k} \llbracket -t_1, t_1 \rrbracket)}\right]\right] \leq \\
&\leq \ex\left[\indic_{E_1}\indic_{E_2}\ex_{Ber}^{-t_1,t_1,\vec x,\vec y}\left[\frac{\indic_{B^N_\delta}}{\delta_1}\right]\right] \leq \frac{1}{\delta_1} \ex\left[ \indic_{E_1} \cdot\pr_{Ber}^{-t_1,t_1,\vec{x}, \vec{y} }(B^N_\delta) \right],
\end{split}
\end{equation}
where 
$$B^N_\delta = \left\{\sup_{{x,y\in [-t_1,t_1], |x-y|\leq \delta N^\alpha}} \abs*{Q_i(x)-Q_i(y)-p(x-y)}\geq \frac{3N^{\alpha/2}\epsilon}4\right\}.$$
Notice that under $\pr_{Ber}^{-t_1,t_1,\vec{x}, \vec{y} }$ the law of $Q_i$ is precisely $\pr_{Ber}^{-t_1,t_1,x_i, y_i }$, and so we conclude that 
\begin{equation}\label{RT2}
\begin{split}
& \pr_{Ber}^{-t_1,t_1,\vec{x}, \vec{y} }(B^N_\delta) = \pr_{Ber}^{0,2t_1,0, y_i - x_i }\left( \sup_{{x,y\in [0,2t_1], |x-y|\leq \delta N^\alpha}} \abs*{\ell(x)-\ell(y)-p(x-y)}\geq \frac{3N^{\alpha/2}\epsilon}4  \right),
\end{split}
\end{equation}
where $\ell$ has law $\pr_{Ber}^{0,2t_1,0, y_i - x_i }$ (note that in (\ref{RT2}) we implicitly translated the path $\ell$ to the right by $t_1$ and up by $-x_i$, which does not affect the probability in question). Since on the event $E_1$ we know that $|y_i -x_i - 2 p t_1| \leq 2M_1 N^{\alpha}$  we conclude from Lemma \ref{MOCLemmaS4} that we can find $N_0$ and $\delta_0 > 0$ depending on $M_1, r, \alpha$ such that for $N \geq N_0$ and $\delta \in (0, \delta_0)$ we have
\begin{equation}\label{RT3}
\begin{split}
& \indic_{E_1} \cdot \pr_{Ber}^{0,2t_1,0, y_i - x_i }\left( \sup_{{x,y\in [0,2t_1], |x-y|\leq \delta N^\alpha}} \abs*{\ell(x)-\ell(y)-p(x-y)}\geq \frac{3N^{\alpha/2}\epsilon}4  \right) < \frac{\delta_1 \eta}{2}.
\end{split}
\end{equation}
Combining (\ref{RT1}), (\ref{RT2}) and (\ref{RT3}) we conclude (\ref{Abound2}), and hence statement (i) of the theorem.


\subsection{Proof of Theorem \ref{PropTightGood} (ii)}\label{Section4.3}

In this section we fix a subsequential limit $\mathcal{L}^\infty = (f_1^\infty,\dots,f_{k-1}^\infty)$ of the sequence $\tilde{\mathbb{P}}_N$ as in the statement of Theorem \ref{PropTightGood}, and we prove that $\mathcal{L}^\infty$ possesses the partial Brownian Gibbs property. Our approach is similar to that in \cite[Sections 5.1 and 5.2]{DimMat}. We first give a definition of measures on scaled free and avoiding Bernoulli random walks. These measures will appear when we apply the Schur Gibbs property to the scaled line ensembles $f^N$. 

\begin{definition}\label{scaledRW}
	Let $a,b\in N^{-\alpha}\mathbb{Z}$ with $a<b$ and $x,y\in N^{-\alpha/2}\mathbb{Z}$ satisfy $0\leq y-x \leq (b-a)N^{\alpha/2}$. Let $\ell^{(T,z)}$ denote a random variable with law $\mathbb{P}^{0,T,0,z}_{Ber}$ as in Definition \ref{DefAvoidingLawBer}. We define $\mathbb{P}^{a,b,x,y}_{free,N}$ to be the law of the $C([a,b])$-valued random variable $Y$ given by
	\[
	Y(t) = \frac{x + N^{-\alpha/2}\left[\ell^{((b-a)N^\alpha,\,(y-x)N^{\alpha/2}))}_{(t-a)N^\alpha} - ptN^\alpha\right]}{\sqrt{p(1-p)}}, \quad t\in [a,b].
	\]
	Now for $i\in\llbracket 1,k\rrbracket$, let $\ell^{(N,z),i}$ denote iid random variables with laws $\mathbb{P}^{0,N,0,z}_{Ber}$. Let $\vec{x},\vec{y}\in(N^{-\alpha/2}\mathbb{Z})^k$ satisfy $0\leq y_i-x_i \leq (b-a)N^{\alpha/2}$ for $i\in\llbracket 1,k\rrbracket$. We define the $\llbracket 1,k\rrbracket$-indexed line ensemble $\mathcal{Y}^N$ by
	\[
	\mathcal{Y}^N_i(t) = \frac{x_i + N^{-\alpha/2}\left[\ell^{((b-a)N^\alpha,\,(y_i-x_i)N^{\alpha/2})),i}_{(t-a)N^\alpha} - ptN^\alpha\right]}{\sqrt{p(1-p)}}, \quad i\in\llbracket 1,k\rrbracket, t\in [a,b].
	\]
	We let $\mathbb{P}^{a,b,\vec{x},\vec{y}}_{free,N}$ denote the law of $\mathcal{Y}^N$. Suppose $\vec{x},\vec{y}\in (N^{-\alpha/2}\mathbb{Z})^k\cap W_k^\circ$ and $f : [a,b] \to (-\infty,\infty]$, $g:[a,b]\to [-\infty,\infty)$ are continuous functions. We define the probability measure $\mathbb{P}^{a,b,\vec{x},\vec{y},f,g}_{avoid,N}$ to be $\mathbb{P}^{a,b,\vec{x},\vec{y}}_{free,N}$ conditioned on the event
	\[
	E = \{f(r) \geq \mathcal{Y}^N_1(r) \geq \cdots \geq \mathcal{Y}^N_k(r) \geq g(r) \mbox{ for } r\in[a,b]\}.
	\]
	This measure is well-defined if $E$ is nonempty.
	
\end{definition}

Next, we state two lemmas whose proofs we give in Section \ref{BGPapp}. The first lemma proves weak convergence of the scaled avoiding random walk measures in Definition \ref{scaledRW}. It states roughly that if the data of these measures converge, then the measures converge weakly to the law of avoiding Brownian bridges with the limiting data.

\begin{lemma}\label{scaledavoidBB}
	Fix $k\in\mathbb{N}$ and $a,b\in\mathbb{R}$ with $a<b$, and let $f:[a-1,b+1]\to(-\infty,\infty]$, $g:[a-1,b+1]\to[-\infty,\infty)$ be continuous functions such that $f(t) > g(t)$ for all $t\in[a-1,b+1]$. Let $\vec{x},\vec{y}\in W_k^\circ$ be such that $f(a) > x_1$, $f(b) > y_1$, $g(a) < x_k$, and $g(b) < y_k$. Let $a_N = \lfloor aN^\alpha\rfloor N^{-\alpha}$ and $b_N = \lceil bN^\alpha\rceil N^{-\alpha}$, and let $f_N : [a-1,b+1]\to(-\infty,\infty]$ and $g_N : [a-1,b+1]\to[-\infty,\infty)$ be continuous functions such that $f_N\to f$ and $g_N\to g$ uniformly on $[a-1,b+1]$. Lastly, let $\vec{x}\,^N, \vec{y}\,^N \in (N^{-\alpha/2}\mathbb{Z})^k \cap W_k^\circ$, write $\tilde{x}^N_i = (x_i^N - pa_N N^{\alpha/2})/\sqrt{p(1-p)}$, $\tilde{y}^N_i = (y_i^N - pb_N N^{\alpha/2})/\sqrt{p(1-p)}$, and suppose that $\tilde{x}^N_i \to x_i$ and $\tilde{y}^N_i \to y_i$ as $N\to\infty$ for each $i\in\llbracket 1,k\rrbracket$. Then there exists $N_0 \in \mathbb{N}$ so that $\mathbb{P}^{a_N,b_N,\vec{x}\,^N,\vec{y}\,^N,f_N,g_N}_{avoid,N}$ is well-defined for $N\geq N_0$. Moreover, if $\mathcal{Y}^N$ have laws $\mathbb{P}^{a_N,b_N,\vec{x}\,^N,\vec{y}\,^N,f_N,g_N}_{avoid,N}$ and $\mathcal{Z}^N = \mathcal{Y}^N|_{\Sigma\times[a,b]}$, then the law of $\mathcal{Z}^N$ converges weakly to $\mathbb{P}^{a,b,\vec{x},\vec{y},f,g}_{avoid}$ as $N\to\infty$.
\end{lemma}

The next lemma shows that at any given point, the values of the $k-1$ curves in $\mathcal{L}^\infty$ are each distinct, so that Lemma \ref{scaledavoidBB} may be applied.

\begin{lemma}\label{inftydistinct}
	For any $s\in\mathbb{R}$, we have $\mathcal{L}^\infty(s) = (f_1^\infty(s),\dots,f_{k-1}^\infty(s)) \in W^\circ_{k-1}$, $\mathbb{P}$-a.s.
\end{lemma}

Using these two lemmas, we now give the proof of Theorem \ref{PropTightGood} (ii).

\begin{proof}
	We will write $\Sigma = \llbracket 1,k\rrbracket$. Let us write $\mathcal{Y}^N = (Y^N_1,\dots,Y^N_{k-1})$ with $Y^N_i(s) = N^{-\alpha/2}(L^N_i(sN^\alpha)-psN^\alpha)/\sqrt{p(1-p)}$. We may assume without loss of generality that $\mathcal{Y}^N \implies \mathcal{L}^\infty$ as $N\to\infty$. Fix a set $K = \llbracket k_1,k_2\rrbracket \subseteq \llbracket 1, k-2\rrbracket$ and $a,b\in\mathbb{R}$ with $a<b$. We also fix a bounded Borel-measurable function $F:C(K\times[a,b])\to\mathbb{R}$. It suffices to prove that $\mathbb{P}$-a.s.,
	\begin{equation}\label{BGPcondex}
		\ex[F(\mathcal{L}^\infty|_{K\times[a,b]})\,|\,\mathcal{F}_{ext}(K\times(a,b))] = \ex^{a,b,\vec{x},\vec{y},f,g}_{avoid}[F(\mathcal{Q})],
	\end{equation}
	where $\vec{x} = (f^\infty_{k_1}(a),\dots,f^\infty_{k_2}(a))$, $\vec{y} = (f^\infty_{k_1}(b),\dots,f^\infty_{k_2}(b))$, $f=f^\infty_{k_1-1}$ (with $f^\infty_0 = +\infty$), $g=f^\infty_{k_2+1}$, and the $\sigma$-algebra $\mathcal{F}_{ext}(K\times(a,b))$ is as in Definition \ref{DefBGP}. We prove \eqref{BGPcondex} in two steps.\\
	
	\noindent\textbf{Step 1. } Fix $m\in\mathbb{N}$, $n_1,\dots,n_m\in\Sigma$, $t_1,\dots,t_m\in\mathbb{R}$, and $h_1,\dots,h_m : \mathbb{R}\to\mathbb{R}$ bounded continuous functions. Define $S = \{i\in\llbracket 1,m\rrbracket : n_i \in K, t_i \in [a,b]\}$. In this step we prove that
	\begin{equation}\label{BBcondexsplit}
		\ex\left[\prod_{i=1}^m h_i(f^\infty_{n_i}(t_i))\right] = \ex\left[\prod_{s\notin S} h_s(f^\infty_{n_s}(t_s))\cdot\ex^{a,b,\vec{x},\vec{y},f,g}_{avoid}\left[\prod_{s\in S} h_s(Q_{n_s}(t_s))\right]\right],
	\end{equation}
	where $Q$ denotes a random variable with law $\pr^{a,b,\vec{x},\vec{y},f,g}_{avoid}$. By assumption, we have
	\begin{equation}\label{BGPweak}
		\lim_{N\to\infty}\ex\left[\prod_{i=1}^m h_i(Y^N_{n_i}(t_i))\right] = \ex\left[\prod_{i=1}^m h_i(f^\infty_{n_i}(t_i))\right].
	\end{equation}
	We define the sequences $a_N = \lfloor aN^\alpha\rfloor N^{-\alpha}$, $b_N = \lceil bN^\alpha\rceil N^{-\alpha}$, $\vec{x}\,^N = (L_{k_1}^N(a_N),\dots,L_{k_2}^N(a_N))$, $\vec{y}\,^N = (L_{k_1}^N(b_N),\dots,L_{k_2}^N(b_N))$, $f_N = Y_{k_1-1}^N$ (where $Y_0 = +\infty$), $g_N = Y_{k_2+1}^N$. Since $a_N \to a$, $b_N\to b$, we may choose $N_0$ sufficiently large so that if $N\geq N_0$, then $t_s < a_N$ or $t_s > b_N$ for all $s\notin S$ with $n_s \in K$. Since the line ensemble $(L_1^N,\dots,L_{k-1}^N)$ in the definition of $\mathcal{Y}^N$ satisfies the Schur Gibbs property (see Definition \ref{DefSGP}), we see from Definition \ref{scaledRW} that the law of $\mathcal{Y}^N|_{K\times[a,b]}$ conditioned on the $\sigma$-algebra $\mathcal{F} = \sigma\left(Y^N_{k_1-1}, Y^N_{k_2+1}, Y^N_{k_1}(a_N), Y^N_{k_1}(b_N),\dots,Y^N_{k_2}(a_N),Y^N_{k_2}(b_N)\right)$ is precisely $\mathbb{P}^{a_N,b_N,\vec{x}\,^N,\vec{y}\,^N,f_N,g_N}_{avoid,N}$. Therefore, writing $Z^N$ for a random variable with this law, we have
	\begin{equation}\label{BBschur}
		\ex\left[\prod_{i=1}^m h_i(Y^N_{n_i}(t_i))\right] = \ex\left[\prod_{s\notin S} h_s(Y^N_{n_s}(t_s))\cdot\ex^{a_N,b_N,\vec{x}\,^N,\vec{y}\,^N,f_N,g_N}_{avoid,N}\left[\prod_{s\in S} h_s(Z^N_{n_s-k_1+1}(t_s))\right]\right].
	\end{equation}
	Now by Lemma \ref{inftydistinct}, we have $\mathbb{P}$-a.s. that $\vec{x},\vec{y} \in W_{k_2-k_1+1}^\circ$, where we recall that $\vec{x} = \mathcal{L}^\infty(a)$, $\vec{y} = \mathcal{L}^\infty(b)$. By the Skorohod representation theorem, there is a probability space $(\Omega,\mathcal{F},\mathbb{P})$ supporting random variables with the laws of $\mathcal{Y}^N$, $\mathcal{L}^\infty$ (which we denote by the same symbols), such that $\mathcal{Y}^N \to \mathcal{L}^\infty$ uniformly on compact sets at every point of $\Omega$. In particular, $f_N\to f = f^\infty_{k_2+1}$ and $g_N\to g = f^\infty_{k_1-1}$ uniformly on $[a-1,b+1]\supseteq [a_N,b_N]$, and $(x_i\,^N - pa_N N^{\alpha/2})/\sqrt{p(1-p)}\to\vec{x}$, $(y_i\,^N-pb_N N^{\alpha/2})/\sqrt{p(1-p)}\to\vec{y}$ for $i\in\llbracket 1,k-1\rrbracket$. It follows from Lemma \ref{scaledavoidBB} that 
	\begin{equation}\label{BGPNweak}
		\lim_{N\to\infty} \ex^{a_N,b_N,\vec{x}\,^N,\vec{y}\,^N,f_N,g_N}_{avoid,N}\left[\prod_{s\in S} h_s(Z^N_{n_s-k_1+1}(t_s))\right] = \ex^{a,b,\vec{x},\vec{y},f,g}_{avoid}\left[\prod_{s\in S} h_s(Q_{n_s}(t_s))\right].
	\end{equation}
	Lastly, the continuity of the $h_i$ implies that
	\begin{equation}\label{BGPuniform}
		\lim_{N\to\infty}\prod_{s\notin S} h_s(Y_{n_s}^N(t_s)) = \prod_{s\notin S} h_s(f^\infty_{n_s}(t_s)).
	\end{equation}
	Combining \eqref{BGPweak}, \eqref{BBschur}, \eqref{BGPNweak}, and \eqref{BGPuniform} and applying the bounded convergence theorem proves \eqref{BBcondexsplit}.\\
	
	\noindent\textbf{Step 2. } In this step we prove \eqref{BGPcondex} as a consequence of \eqref{BBcondexsplit}. For $n\in\mathbb{N}$ we define piecewise linear functions
	\[
	\chi_n(x,r) = \begin{cases}
		0, & x > r + 1/n,\\
		1-n(x-r), & x\in[r,r+1/n],\\
		1, & x < r.
	\end{cases}
	\]
	We fix $m_1,m_2\in\mathbb{N}$, $n^1_1,\dots,n^1_{m_1},n^2_1,\dots,n^2_{m_2}\in\Sigma$, $t^1_1,\dots,t^1_{m_1},t^2_1,\dots,t^2_{m_2}\in\mathbb{R}$, such that $(n^1_i,t^1_i)\notin K\times[a,b]$ and $(n^2_i,t^2_i)\in K\times[a,b]$ for all $i$. Then \eqref{BBcondexsplit} implies that
	\[
	\ex\left[\prod_{i=1}^{m_1} \chi_n(f_{n_i^1}^\infty(t_i^1),a_i)\prod_{i=1}^{m_2}\chi_n(f_{n_i^2}^\infty(t_i^2),b_i)\right] = \ex\left[\prod_{i=1}^{m_1} \chi_n(f_{n_i^1}^\infty(t_i^1),a_i)\ex^{a,b,\vec{x},\vec{y},f,g}_{avoid}\left[\prod_{i=1}^{m_2} \chi_n(Q_{n_i^2}(t_i^2),b_i)\right]\right].
	\]
	Letting $n\to\infty$, we have $\chi_n(x,r)\to \chi(x,r)=\mathbf{1}_{x\leq r}$, and the bounded convergence theorem implies that
	\[
	\ex\left[\prod_{i=1}^{m_1} \chi(f_{n_i^1}^\infty(t_i^1),a_i)\prod_{i=1}^{m_2}\chi(f_{n_i^2}^\infty(t_i^2),b_i)\right] = \ex\left[\prod_{i=1}^{m_1} \chi(f_{n_i^1}^\infty(t_i^1),a_i)\ex^{a,b,\vec{x},\vec{y},f,g}_{avoid}\left[\prod_{i=1}^{m_2} \chi(Q_{n_i^2}(t_i^2),b_i)\right]\right].
	\]
	Let $\mathcal{H}$ denote the space of bounded Borel measurable functions $H:C(K\times[a,b])\to\mathbb{R}$ satisfying
	\begin{equation}\label{BGPH}
		\ex\left[\prod_{i=1}^{m_1} \chi(f_{n_i^1}^\infty(t_i^1),a_i)H(\mathcal{L}^\infty|_{K\times[a,b]})\right] = \ex\left[\prod_{i=1}^{m_1} \chi(f_{n_i^1}^\infty(t_i^1),a_i)\ex^{a,b,\vec{x},\vec{y},f,g}_{avoid}\left[H(\mathcal{Q})\right]\right].
	\end{equation}
	The above shows that $\mathcal{H}$ contains all functions $\mathbf{1}_A$ for sets $A$ contained in the $\pi$-system $\mathcal{A}$ consisting of sets of the form
	\[
	\{h\in C(K\times[a,b]) : h(n_i^2,t_i^2) \leq b_i \mbox{ for } i\in\llbracket 1,m_2\rrbracket\}.
	\]
	We note that $\mathcal{H}$ is closed under linear combinations simply by linearity of expectation, and if $H_n\in\mathcal{H}$ are nonnegative bounded measurable functions converging monotonically to a bounded function $H$, then $H\in\mathcal{H}$ by the monotone convergence theorem. Thus by the monotone class theorem \cite[Theorem 5.2.2]{Durrett}, $\mathcal{H}$ contains all bounded $\sigma(\mathcal{A})$-measurable functions. Since the finite dimensional sets in $\mathcal{A}$ generate the full Borel $\sigma$-algebra $\mathcal{C}_K$ (see for instance \cite[Lemma 3.1]{DimMat}), we have in particular that $F\in\mathcal{H}$.
	
	Now let $\mathcal{B}$ denote the collection of sets $B\in\mathcal{F}_{ext}(K\times(a,b))$ such that
	\begin{equation}\label{BGPB}
		\ex[\mathbf{1}_B \cdot F(\mathcal{L}^\infty|_{K\times[a,b]})] = \ex[\mathbf{1}_B \cdot \ex^{a,b,\vec{x},\vec{y},f,g}_{avoid}[F(\mathcal{Q})]].
	\end{equation}
	We observe that $\mathcal{B}$ is a $\lambda$-system. Indeed, since \eqref{BGPH} holds for $H=F$, taking $a_i,b_i\to\infty$ and applying the bounded convergence theorem shows that \eqref{BGPB} holds with $\mathbf{1}_B = 1$. Thus if $B\in\mathcal{B}$ then $B^c\in\mathcal{B}$ since $\mathbf{1}_{B^c} = 1-\mathbf{1}_B$. If $B_i\in\mathcal{B}$, $i\in\mathbb{N}$, are pairwise disjoint and $B=\bigcup_i B_i$, then $\mathbf{1}_B = \sum_i \mathbf{1}_{B_i}$, and it follows from the monotone convergence theorem that $B\in\mathcal{B}$. Moreover, \eqref{BGPH} with $H=F$ implies that $\mathcal{B}$ contains the $\pi$-system $P$ of sets of the form
	\[
	\{h\in C(\Sigma\times\mathbb{R}) : h(n_i,t_i) \leq a_i \mbox{ for } i \in\llbracket 1,m_1\rrbracket, \mbox{ where } (n_i,t_i)\notin K\times(a,b)\}.
	\]
	By the $\pi$-$\lambda$ theorem \cite[Theorem 2.1.6]{Durrett} it follows that $\mathcal{B}$ contains $\sigma(P) = \mathcal{F}_{ext}(K\times(a,b))$. Thus \eqref{BGPB} holds for all $B\in\mathcal{F}_{ext}(K\times(a,b))$. It is proven in \cite[Lemma 3.4]{DimMat} that $\ex^{a,b,\vec{x},\vec{y},f,g}_{avoid}[F(\mathcal{Q})]$ is an $\mathcal{F}_{ext}(K\times(a,b))$-measurable function. Therefore \eqref{BGPcondex} follows from \eqref{BGPB} by the definition of conditional expectation.
	
\end{proof}

