%-------------------------------------------------------------------------------------------------------------------------------------------------------------------------------------------------
% AppendixB
%
%-------------------------------------------------------------------------------------------------------------------------------------------------------------------------------------------------
\section{Appendix B} \label{Section9}

The goal of this section is to establish the weak convergence of scaled avoiding Bernoulli line ensemble. We consider the $\llbracket 1,k\rrbracket$-indexed line ensembles with distribution given by $\mathbb{P}^{0,T,\vec{x},\vec{y},\infty,-\infty}_{avoid,Ber}$ in the sense of Definition \ref{DefAvoidingLawBer}. Recall that this is just the law of $k$ independent Bernoulli random walks that have been conditioned to start from $\vec{x}=(x_{1},\dots,x_{k})$ at time $0$ and and at $\vec{y}=(y_1,\cdots,y_{k})$ at time $T$ and are always ordered. Here $\vec{x}$, $\vec{y}\in\mathfrak{W}_{k}$ satisfy $T\geq y_{i}-x_{i}\geq 0$ for $i=1,\dots,k$. We will drop the infinities and simply write $\mathbb{P}^{0,T,\vec{x},\vec{y}}_{avoid,Ber}$ for the measure.

This section will be divided into $6$ subsections. In Section \ref{DefMainRes}, we introduce some definitions, formulate the precise statements of two main results we want to prove as Proposition \ref{WeakConvDistinct} and Proposition \ref{WeakConvCollide} and introduce Lemma \ref{LemmaDensity} and Lemma \ref{Integrable} that help to prove them. In Section \ref{SkewSchurPoly}, we introduce some fundamental knowledge about Skew Schur Polynomials and give the distribution of avoiding Bernoulli line ensembles at integer points using Skew Schur Polynomials as Lemma \ref{BerDist}. In Section \ref{ProofProp1}, we will prove our first main result Proposition \ref{WeakConvDistinct} assuming Lemma \ref{LemmaDensity}. In Section \ref{multivar} we introduce some notations and results about multi-indices and multivariate functions which paves the way for proof of Proposition \ref{WeakConvCollide}. Section \ref{ProofProp2} will prove our second main result Proposition \ref{WeakConvCollide} assuming Lemma \ref{Integrable}. Finally, in Section \ref{Pf2Lemma}, we prove Lemma \ref{LemmaDensity} and Lemma \ref{Integrable} and complete our proof.

\subsection{Definitions and Main Results}{\label{DefMainRes}}
We start by introducing some helpful notations.
\begin{definition}\label{DefScaled}
	Fix $p,t\in(0,1)$, $k\in\mathbb{N}$, $\vec{a}$, $\vec{b}\in \mathbb{W}_{k}$ are two vectors in Weyl chamber defined in Definition \ref{DefAvoidingLaw}. Suppose that $\vec{x}^{T}=(x_{1}^{T},\cdots,x_{k}^{T})$ and $\vec{y}^{T}=(y_{1}^{T},\cdots,y_{k}^{T})$ are two sequences of $k$-dimensional vectors in $\mathfrak{W}_{k}$ such that $$\lim_{T\rightarrow\infty}\frac{x_{i}^{T}}{\sqrt{T}}=a_{i} \text{ and } \lim_{T\rightarrow\infty}\frac{y_{i}^{T}-pT}{\sqrt{T}}=b_{i}$$ for $i=1,\dots,k$. Define the sequence of random $k$-dimensional vectors $Z^{T}$ by 
	\begin{align}{\label{ZTandL}}
		\begin{split}
		Z^{T}=\big(\frac{L_{1}(tT)-ptT}{\sqrt{T}},\cdots,\frac{L_{k}(tT)-ptT}{\sqrt{T}}\big)
	\end{split}
	\end{align}
	where $(L_{1},\cdots,L_{k})$ is $\mathbb{P}^{0,T,\vec{x}^{T},\vec{y}^{T}}_{avoid,Ber}$-distributed.
\end{definition}

We introduce some constants below
\begin{align}{\label{WCconst}}
	\begin{split}
	&c_{1}(p,t)=\frac{1}{p(p+1)t}, \quad c_{2}(p,t)=\frac{1}{p(p+1)(1-t)}, \quad c_{3}(p,t)=\frac{1}{2p(p+1)t(1-t)}\\
	&Z=(2\pi)^{\frac{k}{2}}(p(p+1)t(1-t))^{\frac{k}{2}}\cdot e^{c_{1}(t,p)\sum_{i=1}^{k}a_{i}^{2}}\cdot e^{c_{2}(t,p)\sum_{i=1}^{k}b_{i}^{2}}\det\left(e^{-\frac{1}{2p(p+1)}(b_{i}-a_{j})^{2}}\right)_{i,j=1}^{k}
	\end{split}
\end{align}
and define the function $\rho(z_{1},\cdots, z_{k})\equiv\rho(\vec{z})$ as the following:
\begin{align}{\label{Density}}
	\rho(z_{1},\cdots,z_{k})=\frac{1}{Z}\cdot \mathbbm{1}_{\{z_{1}>\cdots >z_{k}\}}\cdot \det\big[e^{c_{1}(t,p)a_{i}z_{j}}\big]_{i,j=1}^{k}\det\big[e^{c_{2}(t,p)b_{i}z_{j}}\big]_{i,j=1}^{k}\prod_{i=1}^{k}e^{-c_{3}(t,p)z_{i}^{2}}
\end{align}

We will prove that the function $\rho(z)$ defined in (\ref{Density}) is a probability density function, meaning that it is non-negative and integrates to $1$ over $\mathbb{R}^{k}$. Since this is an important ingredient of our results, we isolate it as Lemma \ref{PfDensity} and will prove it in Section \ref{Pf2Lemma}. For now, we assume that $\rho(\vec{z})$ in (\ref{Density}) is a density so that we can state our first main result in the following, which gives the limiting distribution of $Z^{T}$ when vectors $\vec{a}$ and $\vec{b}$ contain distinct values.

\begin{proposition}{\label{WeakConvDistinct}}
Assume the same notation as in the Definition \ref{DefScaled}. When $a_{1}> \cdots > a_{k}$ and $b_{1}> \cdots > b_{k}$ are all distinct, the random vector $Z^{T}$ converges weakly to a continuous distribution with the density in (\ref{Density}).	
\end{proposition}

Proposition \ref{WeakConvDistinct} states the result when $\vec{a}$ and $\vec{b}$ consist of distinct values. When the values in $\vec{a}$ and $\vec{b}$ start to collide, the three determinants in the density function (\ref{Density}) will vanish(one in constant $Z$ in equation (\ref{WCconst}) and the other two are in the expression of equation (\ref{Density})). In the following, we are going to formulate the result under this new situation. We will construct a modified density function and the random vector $Z^{T}$ will weakly converge to this new density function.

Suppose vectors $\vec{a}$ and $\vec{b}$ cluster as the following:
\begin{align}{\label{Block}}
\begin{split}
	&\vec{a}_{0}=(a_{1},\cdots,a_{k})=(\underbrace{\alpha_{1},\cdots,\alpha_{1}}_{m_{1}},\cdots,\underbrace{\alpha_{p},\cdots,\alpha_{p}}_{m_{p}})\\
	&\vec{b}_{0}=(b_{1},\cdots,b_{k})=(\underbrace{\beta_{1},\cdots,\beta_{1}}_{n_{1}},\cdots,\underbrace{\beta_{q},\cdots,\beta_{q}}_{n_{q}})
\end{split}
\end{align}
where $\alpha_{1}>\alpha_{2}>\cdots>\alpha_{p}$, $\beta_{1}>\beta_{2}>\cdots>\beta_{q}$ and $\sum_{i=1}^{p}m_{i}=\sum_{i=1}^{q}n_{i}=k$. Denote $\vec{m}=(m_{1},\cdots,m_{p})$, $\vec{n}=(n_{1},\cdots,n_{q})$ and define two determinants $\varphi(\vec{a}_{0},\vec{z},\vec{m})$ and $\psi(\vec{b}_{0},\vec{z},\vec{n})$ below:
\begin{equation}{\label{TwoDet}}
\begin{split}
	\varphi(\vec{a}_{0},\vec{z},\vec{m})= \det
	\left[ \begin{array}{ccc}
		((c_{1}(t,p)z_{j})^{i-1}e^{c_{1}(t,p)\alpha_{1}z_{j}})_{\substack{i=1,\cdots,m_{1}\\j=1,\cdots,k}}\\
	\vdots\\
	((c_{2}(t,p)z_{j})^{i-1}e^{c_{1}(t,p)\alpha_{p}z_{j}})_{\substack{i=1,\cdots,m_{p} \\j=1,\cdots,k}}
	\end{array}
	\right]
\\
	\psi(\vec{b}_{0},\vec{z},\vec{n})= \det
	\left[ \begin{array}{ccc}
		((c_{2}(t,p)z_{j})^{i-1}e^{c_{2}(t,p)\beta_{1}z_{j}})_{\substack{i=1,\cdots,n_{1}\\j=1,\cdots,k}}\\
	\vdots\\
	((c_{2}(t,p)z_{j})^{i-1}e^{c_{2}(t,p)\beta_{q}z_{j}})_{\substack{i=1,\cdots, n_{q} \\j=1,\cdots,k}}
	\end{array}
	\right]
\end{split}
\end{equation}

Then define the function 
\begin{align}{\label{PreDensity17}}
	H(\vec{z})=\varphi(\vec{a}_{0},\vec{z},\vec{m})\psi(\vec{b}_{0},\vec{z},\vec{n})\prod_{i=1}^{k}e^{-c_{3}(t,p)z_{i}^{2}}
\end{align}
we can prove that $H(\vec{z})$ in (\ref{PreDensity17}) is non-negative and integrable over $\mathbb{R}^{k}$,so that we can multiply it with the normalized constant $Z_{0}=\int_{\mathbb{R}^{k}}H(\vec{z})\mathbbm{1}_{\{z_1>\cdots>z_{k}\}}dz<\infty$ and make it a probability density function:
\begin{align}{\label{Density17}}
\rho_{\vec{a}_{0},\vec{b}_{0}}(z_{1},\cdots,z_{k})=\frac{1}{Z_{0}}\cdot\mathbf{1}_{\{z_{1}>\cdots> z_{k}\}}\cdot \varphi(\vec{a}_{0},\vec{z},\vec{m})\psi(\vec{b}_{0},\vec{z},\vec{n})\prod_{i=1}^{k}e^{-c_{3}(t,p)z_{i}^{2}}	
\end{align}

Now we are ready to state our second main result, which gives the weak convergence of $Z^{T}$ when $\vec{a}$ and $\vec{b}$ have collided values.
\begin{proposition}{\label{WeakConvCollide}}
Assume the same notation as in the Definition \ref{DefScaled} and suppose vectors $\vec{a}_{0}$, $\vec{b}_{0}$ has the form in (\ref{Block}). Then, the random vector $Z^{T}$ converges weakly to a continuous distribution with density in (\ref{Density17}).
\end{proposition}

\subsection{Skew Schur polynomials and distribution of avoiding Bernoulli line ensembles}{\label{SkewSchurPoly}}
We give some definitions and elementary results regarding skew Schur polynomials first, which are mainly from \cite[Chapter 1]{Mac}.
\begin{definition}{\label{DefPar}} \emph{Partition, Interlaced}
\begin{enumerate}
	\item A \emph{partition} is any infinite sequence $\lambda=(\lambda_{1}, \lambda_{2}, \cdots, \lambda_{r}, \cdots)$ of non-negative integers in decreasing order $\lambda_{1}\geq \lambda_{2}\geq \cdots\geq \lambda_{r}\geq \cdots$ and containing only finitely many non-zero terms. The non-zero $\lambda_{i}$ are called \emph{parts} of $\lambda$, and the sum of the parts is the \emph{weight} of $\lambda$, denoted by $|\lambda|$.
	\item Suppose $\lambda$ and $\mu$ are two partitions, we denote $\lambda\supset\mu$ if $\lambda_{i}\geq \mu_{i}$ for all $i\in \mathbb{Z}^{+}$, and we can define a new partition $\lambda-\mu=(\lambda_{1}-\mu_{1},\lambda_{2}-\mu_{2},\cdots)$.
	\item Partitions $\lambda=(\lambda_{1}, \lambda_{2},\cdots)$ and $\mu=(\mu_{1}, \mu_{2},\cdots)$ are call \emph{interlaced}, denoted by $\mu\preceq \lambda$, if $\lambda_1\geq \mu_1\geq \lambda_2\geq \mu_2\geq\cdots$. By definition, if $\mu\preceq\lambda$, then $\mu\subset\lambda$.
\end{enumerate}
\end{definition}

\begin{definition} \emph{Complete Symmetric Function}{\label{CompSymFunc}}\\
	For each $r\geq 0$ the $r$-th \emph{complete symmetric function} $h_{r}$ is the sum of all monomials of total degree $r$ in the variables $x_1, x_2, \cdots$ so that
	\begin{align}
		h_{r}=\sum_{|\lambda|=r}m_{\lambda}
	\end{align}
	 where $m_{\lambda}=x_1^{\lambda_1}x_2^{\lambda_2}\cdots$ is the monomial with parts of partition $\lambda$ as corresponding powers. For $r<0$, we  define $h_r$ to be zero. In particular, when $x_1=x_2=\cdots=x_n=1$, $x_{n+1}=x_{n+2}=\cdots=0$, $h_r$ is just the number of partitions that have weight $r$ and $n$ nonzero parts. Thus, we have: 
	 \begin{align}
	 	h_{r}(1^{n})=\binom{r+n-1}{r}
	 \end{align}m
\end{definition}

Next, we introduce Skew Schur Polynomial based on \cite[Chapter 1, (5.4), (5.11), (5.12)]{Mac}.
\begin{definition} \emph{Skew Schur Polynomial, Jacob-Trudi Formula}{\label{DefSkewSchurPoly}}
\begin{enumerate}
	\item Suppose $\mu\preceq\lambda$ are two interlaced partitions, then the \emph{skew Schur polynomial} $s_{\lambda/\mu}$ with single variable $x$ is defined by $s_{\lambda/\mu}(x)=x^{|\lambda-\mu|}$. Otherwise, we define $s_{\lambda/\mu}=0$.
	\item Suppose $\lambda\supset\mu$ are two partitions, define the \emph{skew Schur polynomial} $s_{\lambda/\mu}$ with respect to variables $x_1, x_2, \cdots, x_{n}$ by
	\begin{align}
		s_{\lambda/\mu}(x_1,\cdots,x_n)=\sum_{(\nu)}\prod_{i=1}^{n}s_{\nu^{i}/\nu^{i-1}}(x_i)=\sum_{(\nu)}\prod_{i=1}^{n}x_{i}^{|\nu^{i}-\nu^{i-1}|}
	\end{align}
	summed over all sequences $(\nu)=(\nu^{0},\nu^{1},\cdots,\nu^{n})$ of partitions, such that $\nu^{0}=\mu$, $\nu^{n}=\lambda$ and $\nu^{0}\preceq\nu^{1}\preceq\cdots\preceq\nu^{n}$. In particular, when $x_1=x_2=\cdots=x_{n}=1$, the skew Schur polynomial is just the number of such sequences of interlaced partitions $(\nu)$.
	\item We also have the following \emph{Jacob-Trudi Formula}\cite[Chapter 1, (5.4)]{Mac} for the skew Schur polynomial:
	\begin{align}{\label{J-TFormula}}
		s_{\lambda/\mu}=\det\left(h_{\lambda_{i}-\mu_{j}-i+j}\right)_{1\leq i,j\leq n}
	\end{align}
	where $h_r$ is the complete symmetric function in Definition \ref{CompSymFunc}.
\end{enumerate}
\end{definition}

Based on the above preparation, we are ready to state the following lemma giving the distribution of avoiding Bernoulli line ensembles at time $\lfloor tT \rfloor$.
\begin{lemma}{\label{BerDist}}
Assume the same notations as in Section \ref{DefMainRes}, denote $m=\lfloor tT \rfloor$, $n=T-\lfloor tT \rfloor$. Then, the avoiding Bernoulli line ensemble at time $m$ has the following distribution: 
\begin{align}{\label{ProbMassFunc}}
\mathbb{P}(L_{1}(m) = \lambda_{1}, \cdots, L_{k}(m) = \lambda_{k})=\frac{s_{\lambda/\mu}(1^{m})\cdot s_{\kappa/\lambda}(1^{n})}{s_{\kappa/\mu}(1^{T})}	
\end{align}
where $\lambda_{1}\geq\lambda_{2}\geq\cdots\geq\lambda_{k}$ are positive integers, $s_{\lambda/\mu}$ denote skew Schur polynomials and they are specialized in all parameters equal to $1$. The $\mu$ partition is just the vector $\vec{x}^{T}$ and the $\kappa$ partition should be $\vec{y}^{T}$.
\end{lemma}
\begin{remark}
	Here we let $\lambda_{1}\geq\lambda_{2}\geq\cdots\geq\lambda_{k}$ be positive integers, although they could potentially be negative. However, we can shift all the endpoints up such that all possible $\lambda_{i}$ are positive. Also, in the proof we treat finite dimensional vectors as partitions because as long as we add infinitely many zeros at their ends we can and make them  ``partitions'' in Definition \ref{DefPar}.
\end{remark}
\begin{proof} Let $\Omega(0,T,\vec{x}^T, \vec{y}^T)$ be the set of all non-intersecting Bernoulli line ensembles from $\vec{x}^T$ to $\vec{y}^T$. For each line ensemble $\mathfrak{B}\in \Omega(0,T,\bar x^T,\bar y^T)$ with $\mathfrak B=(B_1,...,B_k)$, we may define $\lambda^{i}(\mathfrak B):=(B_1(i),B_2(i),...,B_k(i))$, where $1 \leq i\leq T$ is an integer. The $\lambda^{i}(\mathfrak{B})$ form partitions since by the definition of avoiding Bernoulli line ensembles, we have the inequality $B_\alpha(i)\geq B_\beta(i)$ if $\alpha<\beta$. Now we claim that partitions $\lambda^{i}(\mathfrak{B})$, $1\leq i \leq T$ are interlaced. By definition of Bernoulli random walk, we have $B_{\alpha}(i+1)=B_{\alpha}(i)+\xi$ and $B_{\alpha+1}(i+1)=B_{\alpha+1}(i)+\eta$, where $\xi$, $\eta\in\{0,1\}$, so $B_{\alpha}(i+1)\geq B_{\alpha}(i)$. If $B_{\alpha}(i)\geq B_{\alpha+1}(i)+1$, then $B_{\alpha}(i)\geq B_{\alpha+1}(i+1)$; If $B_{\alpha}(i)=B_{\alpha+1}(i)$, in order to assure $B_{\alpha}(i+1)\geq B_{\alpha+1}(i+1)$, $\xi$ and $\eta$ have to be $0$ or $1$ at the same time, or $\xi=1$ and $\eta=0$, all of which deduce that $B_{\alpha}(i)\geq B_{\alpha+1}(i+1)$. Therefore,
\[B_{\alpha+1}(i+1)\leq B_\alpha(i)\leq B_\alpha(i+1)\]
and $\lambda^{1}(\mathfrak{B})\preceq\cdots\preceq\lambda^{T}(\mathfrak{B})$ are interlaced. Note that when $i=0$, we get $\lambda^0(\mathfrak{B})=\bar x^T$ and $\lambda^T(\mathfrak{B})=\bar y^T$.

Now, let us define the set 
\[TB_{\lambda/\mu}^T:=\{(\lambda^0,...,\lambda^T)\mid \lambda^0=\mu, \lambda^T=\lambda, \lambda^i\preceq\lambda^{i+1}\}\] 
Now, if we take $f:\Omega(0,T,\bar x^T, \bar y ^T)\to TB_{\kappa/\mu}^T$ with $f(\mathfrak{B})= (\lambda_0(\mathfrak{B}),\cdots \lambda_T(\mathfrak{B}))$. We find that this function is in fact a bijection. 

First, to show injectivity, suppose that there are two Bernoulli line ensembles, $\mathfrak{B}, \widetilde{\mathfrak{B}}\in \Omega(0,T,\bar x^T, \bar y ^T)$ such that $\mathfrak{B}\neq \widetilde{\mathfrak{B}}$.
Bernoulli line ensembles are determined by their values at integer times, so this would imply that there exists some $(q,r)$ such that $0\leq r\leq T$, $0\leq q \leq k$ and $B_q(r)\neq \widetilde{B}_q(r)$ where $B_q$ and $\widetilde{B}_q$ are components of $\mathfrak{B}$ and $\widetilde{\mathfrak{B}}$ respectively. 
This implies that $\lambda^r(\mathfrak B)\neq \lambda^r(\widetilde{\mathfrak{B}})$, so we have injectivity. 

Now, surjectivity follows since for any $\bar\lambda=(\lambda^0,...,\lambda^T)$ we may define $\mathfrak{B}(\bar{\lambda})=(B_1(\bar\lambda),...,B_k(\bar\lambda))$ where $B_r(\bar\lambda)(i)=\lambda^i_r$ where $\lambda^i_r$ is the $r$\textit{-th} entry of $\lambda^i$. The restrictions on $TB_{\kappa/\mu}^T$ ensure that each $\mathfrak{B}(\bar\lambda)\in \Omega(0,T,\bar x^T,\bar y^T)$, and so $f(\mathfrak B(\bar{\lambda}))=(\lambda^0,\cdots \lambda^T)$ by the definition $\mathfrak{B}(\bar\lambda)$. 

Applying the result regarding the relationship between number of sequences of interlaced partitions and skew Schur polynomial(Definition \ref{DefSkewSchurPoly}, (2)), we have $\lvert TB_{\lambda/\mu}^T\rvert =s_{\lambda/\mu}(1^T)$.

Therefore, we conclude that 
\begin{align*}
\pr(L_1(m)=\lambda_1,\cdots, L_k(m)=\lambda_k)
=&{}\frac{\lvert \Omega(0, m, \vec{x}^T, \lambda)\rvert\cdot \lvert \Omega(m, T, \lambda, \vec{y}^T)\rvert}{\lvert \Omega(0, T, \vec{x}^T, \vec{y}^T)\rvert}\\
=&{}\frac{s_{\lambda/\vec{x}^T}(1^{m})\cdot s_{\vec{y}^T/\lambda}(1^{n})}{s_{\vec{y}^T/\vec{x}^T}(1^T)}
\end{align*} where $\lambda=(\lambda_{1},\cdots,\lambda_{k})$ is a partition such that $\vec{x}^{T}\preceq\lambda\preceq\vec{y}^{T}$.
\end{proof}

By Jacob-Trudi formula (\ref	{J-TFormula}) and Lemma \ref	{BerDist}, we further get
\begin{align}{\label{BerDistJT}}
	\begin{split}
		\pr(L_1(m)=\lambda_1,\cdots, L_k(m)=\lambda_k)=\frac{\det\left[h_{\lambda_{i}-x_{j}^{T}+j-i}(1^{m})\right]_{i,j=1}^{k}\cdot\det\left[h_{y_{i}^{T}-\lambda_{j}+j-i}(1^{n})\right]_{i,j=1}^{k}}{\det\left[h_{y_{i}^{T}-x_{j}^{T}+j-i}(1^{T})\right]_{i,j=1}^{k}}
	\end{split}
\end{align}
\subsection{Proof of Proposition \ref{WeakConvDistinct}}{\label{ProofProp1}} In this section, we first prove Lemma \ref{PfDensity} which claims the function in (\ref{Density}) is a density and then prove the weak convergence result in Proposition \ref{WeakConvDistinct}. The fact that (\ref{Density}) is a density is formulated in the following lemma.

\begin{lemma}{\label{PfDensity}}
	Assume the same notations as in Section \ref{DefMainRes}. Denote the function 
	\begin{align}
		\widetilde{\rho}(z_1,\cdots,z_k)=\mathbf{1}_{\{z_1>z_2>\cdots>z_k\}}\det\left[e^{c_{1}(t,p)a_i z_j}\right]_{i,j=1}^{k}\cdot\det\left[e^{c_{2}(t,p)b_i z_j}\right]_{i,j=1}^{k}\cdot\prod_{i=1}^{k}e^{-c_3(t,p)z_{i}^{2}}
	\end{align}
	Then $\widetilde{\rho}(z_1,\cdots,z_k)\geq 0$ for all $\vec{z}=(z_1,\cdots,z_k)\in\mathbb{R}^{k}$ and $\widetilde{\rho}(z_1,\cdots,z_k)>0$ if $z_1>z_2>\cdots>z_k$. Moreover, the function $\widetilde{\rho}$ is integrable on $\mathbb{R}^{k}$ and we have
	\begin{align}
		\int_{\mathbb{R}^{k}}\widetilde{\rho}(z_1,\cdots,z_k)dz_1\cdots dz_k=Z
	\end{align}
	where the constant $Z$ is defined in (\ref{WCconst}), thus implying the function $\rho(\vec{z})$ in (\ref{Density}) is a density.
\end{lemma}

We start with finding the asymptotic formula for probability mass function (\ref{BerDistJT}), and it is equivalent to find the asymptotic formula for complete symmetric functions $h_{\lambda_{i}-x_{j}^{T}+j-i}(1^{m})$, $h_{y_{i}^{T}-\lambda_{j}+j-i}(1^{n})$ and $h_{y_{i}^{T}-x_{j}^{T}+j-i}(1^{T})$. By the definition of random vector $Z^{T}$ in (\ref{ZTandL}), we find that 
\begin{align}
	\left\{Z^{T}_{1}=z_1,\cdots, Z^{T}_{k}=z_k\right\}\equiv\left\{L_{1}(tT)=\lambda_1,\cdots,L_{k}(tT)=\lambda_k\right\}
\end{align}
where $\lambda_{i}=z_{i}\sqrt{T}+ptT$ for $i\in\{1,\cdots,k\}$. In addition, $x_{i}^{T}\sim a_{i}\sqrt{T}+o\left(\sqrt{T}\right)$ and $y_{i}^{T}\sim b_{i}\sqrt{T}+pT+o\left(\sqrt{T}\right)$ by Definition \ref{DefScaled}. Therefore, we have
\begin{align}
\begin{split}
	\lambda_{i}-x_{j}^{T}+j-i&=pm+O\left(T^{1/2}\right),\ y_{i}^{T}-\lambda_{j}+j-i=pn+O\left(T^{1/2}\right)\\
	&y_{i}^{T}-x_{j}^{T}+j-i=pT+O\left(T^{1/2}\right)
\end{split}
\end{align}
Thus, we only need to consider the complete symmetric functions in the form $h_{N}(1^{n})$, where $N=pn+x\sqrt{n}$ and $x\in[-R,R]$ is bounded. In this case, we have the following lemma giving the asymptotic behavior of $h_{N}(1^{n})$.
\begin{lemma}
	Suppose that $p\in(0,1)$ and $R>0$ are given. Suppose that $x\in[-R,R]$ and $N=pn+\sqrt{n}x$ is an integer. Then 
\begin{align}
\begin{split}{\label{AsympForh}}
&h_{N}(1^{n})=(\sqrt{2\pi})^{-1}\cdot \exp\left(-\frac{x^2}{2(1+p)p}\right)\cdot \exp\left(N\log(\frac{1+p}{p})\right)\cdot \exp\left(O(n^{-1/2})\right)\\
& \exp\left(n\log(1+p)-(1/2)\log n-(1/2)\log(p(1+p))\right)
\end{split}
\end{align}
where the constant in the big $O$ notation depends on $p$ and $R$ alone. Moreover, there exist positive constants $C,c>0$ depending on $p$ alone such that for all large enough $n\in\mathbb{N}$ and $N\in[0,(1+p)n]$
\begin{align}{\label{hbounded}}
	h_{N}(1^n)\leq C\cdot \exp\left(N\log((1+p)p)+n\log(1+p)-(1/2)\log n\right)\cdot \exp\left(-cn^{-1}(N-pn)^2\right).
\end{align}
\end{lemma}
\begin{remark}
	Notice that when $R>0$ is fixed, $N\in[pn-R\sqrt{n},pn+R\sqrt{n}]$. However, we specify the range of $N$ by $[0,(1+p)n]$. First, it is because when $N<0$ the complete symmetric function $h_{N}(1^{n})$ would be zero by Definition \ref{CompSymFunc} and the situation becomes trivial. Second, when $n$ is sufficiently large, the interval $[0,(1+p)n]$ will contain $[pn-R\sqrt{n},pn+R\sqrt{n}]$, so it's sufficient to consider the case when $N\in[0,(1+p)n]$.
\end{remark}
\begin{proof}
\textbf{Step 1.} In this step we prove (\ref	{AsympForh}). Using the formula for complete symmetric function (\ref{CompSymFunc}), we obtain
\begin{align}{\label{CompSymFactorial}}
	h_{N}(1^n)=\frac{(N+n-1)!}{N!(n-1)!}
\end{align}

We have the following Stirling's formula \cite{stirling} that for $n\geq 1$
$$n!=\sqrt{2\pi n}n^ne^{-n}e^{r_{n}}\text{, where }\frac{1}{12n+1}<r_{n}<\frac{1}{12n}$$
Applying the Stirling formula to equation (\ref{CompSymFactorial}) implies that
\begin{align}{\label{hStirling}}
\begin{split}
	& h_{N}(1^n)=\frac{\exp\left( (N+n-1/2)\log(N+n-1)-(n-1/2)\log(n-1)-(N+1/2)\log(N)+O(n^{-1}) \right)}{\sqrt{2\pi}}\\
	& = (\sqrt{2\pi})^{-1}\cdot \exp\left((N+n-1/2)\log\left(\frac{N+n-1}{(1+p)n}\right)-(n-1/2)\log(n-1)-(N+1/2)\log\left(\frac{N}{pn}\right)\right)\\
	&\cdot \exp\left((N+n-1/2)\log((1+p)n)-(N+1/2)\log(pn)+O\left(n^{-1}\right)\right).
\end{split}
\end{align}

Denote $\Delta=\sqrt{n}x$, and we now use the Taylor expansion of the logarithm and the expression for $N$ to get
\begin{align*}
	\log\left(\frac{N+n-1}{(1+p)n}\right)&=\log\left(\frac{pn+\Delta+n-1}{(1+p)n}\right)=\log\left(1+\frac{\Delta-1}{(1+p)n}\right)\\
	& = \frac{\Delta-1}{(1+p)n}-\frac{1}{2}\frac{\Delta^2}{n^2(1+p)^{2}}+O\left(n^{-3/2}\right)
\end{align*}
Analogously, we have $$\log\left(\frac{N}{pn}\right)=\log\left(1+\frac{\Delta}{pn}\right)=\frac{\Delta}{pn}-\frac{1}{2}\frac{\Delta^2}{p^2 n^2}+O\left(n^{-3/2}\right)$$
Plugging the two equations above to equation (\ref{hStirling}) we get
\begin{align*}
	& h_{N}(1^n)=(\sqrt{2\pi e})^{-1}\cdot \exp\left(\frac{\Delta(N+n)}{(1+p)n}-\frac{1}{2}\frac{\Delta^{2}(N+n)}{n^2(1+p)^2}\right)\exp\left(-\frac{\Delta N}{pn}+\frac{1}{2}\frac{\Delta^2 N}{p^2 n^2}\right)\\
	& \cdot \exp\left((N+n-1/2)\log((1+p)n)-(n-1/2)\log(n-1)-(N+1/2)\log(pn)+O(n^{-1/2})\right)
\end{align*}

We next observe that 
$$\frac{\Delta(N+n)}{(1+p)n}-\frac{\Delta N}{pn}=\Delta\left(\frac{(1+p)n+\Delta}{(1+p)n}-\frac{pn+\Delta}{pn}\right)=-\frac{\Delta^2}{(1+p)pn}$$
 and $$-\frac{1}{2}\frac{\Delta^2(N+n)}{n^2(1+p)^2}+\frac{1}{2}\frac{\Delta^2N}{p^2 n^2}=-\frac{\Delta^2}{2n}\left(\frac{N+n}{(1+p)^2n}-\frac{N}{p^2n}\right)=\frac{\Delta^2}{2(1+p)pn}+O\left(n^{-1/2}\right)$$
Combining the above we conclude that 
\begin{align}{\label{hfinal}}
\begin{split}
	& h_{N}(1^n)=(\sqrt{2\pi e})^{-1}\cdot \exp\left(-\frac{\Delta^2}{2(1+p)pn}\right)\cdot\exp\left(N\log\left(\frac{1+p}{p}\right)\right)\cdot\exp\left(O\left(n^{-1/2}\right)\right)\\
	&\cdot \exp\left((n-1/2)\log((1+p)n)-(n-1/2)\log(n-1)-(1/2)\log(pn)\right).
\end{split}
\end{align}
We finally observe that 
\begin{align*}
\begin{split}
	& (n-1/2)\log((1+p)n)-(n-1/2)\log(n-1)-(1/2)\log(pn)=(n-1/2)\log(1+p)\\
	& +(n-1/2)\log n-(n-1/2)\log(n)-(n-1/2)\log(1-n^{-1})-(1/2)\log p-(1/2)\log n\\
	& = n\log(1+p)+1-(1/2)\log n -(1/2)\log(p(1+p))+O\left(n^{-1}\right)
\end{split}
\end{align*}
and putting this in (\ref{hfinal}) we arrive at (\ref{AsympForh}).\\
\textbf{Step 2.} In this step we prove (\ref{hbounded}).
\end{proof}

Before proving Lemma \ref{PfDensity}, we need to introduce another result regarding the non-vanishing of determinant.

\begin{lemma}{\label{NonVanish}}
	Suppose the vector $\vec{m}=(m_1,\cdots,m_p)$ satisfies $k=\sum_{i=1}^{p}m_{i}=\sum_{j=1}^{q}n_{j}$, then the following determinant
\[ U= \det
	\left[ \begin{array}{ccc}
		(z_{j}^{i-1}e^{\alpha_{1}z_{j}})_{\substack{i=1,\cdots,m_{1}\\j=1,\cdots,k}}\\
	\vdots\\
	(z_{j}^{i-1}e^{\alpha_{p}z_{j}})_{\substack{i=m_1+\cdots+m_{p-1}+1,\\ \cdots ,m_{1}+\cdots +m_{p} \\j=1,\cdots,k}}
	\end{array}
	\right]
\]
is non-zero for any $(z_{1},\cdots,z_{k})$ whose elements are distinct.
\end{lemma}

\begin{proof} We claim that, the following equation with respect to $\vec{z}$:
$$(\xi_{1}+\xi_{2}z+\cdots+\xi_{m_1}z^{m_{i}-1})e^{\alpha_{1}z}+\cdots(\xi_{m_{1}+\cdots+m_{p-1}+1}+\cdots+\xi_{k}z^{m_{p}-1})e^{\alpha_{p}z}=0$$ has at most $(k-1)$ distinct roots, where $\sum_{i=1}^{p}m_i=k$ and $(\xi_{1},\cdots,\xi_{k})\in\mathbb{R}^{k}$ is non-zero.\\
Denote the above determinant by $\det\Bigl[\begin{smallmatrix} v_{1}\\\vdots\\ v_{k} \end{smallmatrix}\Bigr]$. If this claim holds, we can conclude that we cannot find non-zero $(\xi_{1},\cdots,\xi_{k})\in\mathbb{R}^{k}$ such that $\xi_{1}v_{1}+\cdots+\xi_{k}v_{k}=0$. Thus, the $k$ row vectors of the determinant are linear independent and the determinant is non-zero. Then we prove the claim by induction on $k$.\\
$1^{\circ}$ If $k=2$, the equation is $(\xi_{1}+\xi_{2}z)e^{\alpha_{1}z}=0$ or $\xi_{1}e^{\alpha_{1}z}+\xi_{2}e^{\alpha_{2}z}=0$, where $\xi_{1},\xi_{2}\in\mathbb{R}$ cannot be zero at the same time. Then, it's easy to see that the equation has at most $1$ root in two scenarios.\\
$2^{\circ}$ Suppose the claim holds for $k\leqslant n$.\\
$3^{\circ}$ When $k=n+1$, we have the equation $$(\xi_{1}+\xi_{2}z+\cdots+\xi_{m_1}z^{m_{i}-1})e^{\alpha_{1}z}+\cdots(\xi_{m_{1}+\cdots+m_{p-1}+1}+\cdots+\xi_{k}z^{m_{p}-1})e^{\alpha_{p}z}=0$$ but now $\sum_{i=1}^{p}m_{i}=n+1$. WLOG, suppose $(\xi_{1},\cdots,\xi_{m_{1}})$ has a non-zero element and $\xi_{\ell}$ is the first non-zero element. Notice that the above equation has the same roots as the following one:
$$F(z)=(\xi_{\ell}z^{\ell-1}+\cdots+\xi_{m_1}z^{m_1-1})+\cdots+(\xi_{m_1+\cdots+m_{p-1}+1}+\cdots+\xi_{k}z^{m_{p}-1})e^{(\alpha_{p}-\alpha_{1})z}=0$$
Assume it has at least $(n+1)$ distinct roots $\eta_{1}<\eta_{2}<\cdots<\eta_{n+1}$. Then $F^{\prime}(z)=0$ has at least $n$ distinct roots $\delta_{1}<\cdots<\delta_{n}$ such that $\eta_{1}<\delta_{1}<\eta_{2}<\cdots<\delta_{n}<\eta_{n+1}$, by Rolle's Theorem. Actually, $F^{\prime}(z)=(\xi_{\ell}(\ell-1))z^{\ell-2}+\cdots+\xi_{m_1}(m_1-1)z^{m_{1}-2})+\cdots+[\xi_{m_1+\cdots+m_{p-1}+1}^{\prime}+\cdots+\xi_{k}^{\prime}z^{m_{p}-1})]e^{(\alpha_{p}-\alpha_{1})z}=0$
where $\xi_{m_1+\cdots+m_{p-1}+1}^{\prime}$ and $\xi_{k}^{\prime}$ are coefficients that can be calculated. This equation has at most $(m_1-1)+m_2+\cdots+m_{p}-1=n-1$ roots by $2^{\circ}$, which leads to a contradiction. Therefore, our claim holds and we proved Lemma \ref{NonVanish}.
\end{proof}

Now, we are ready to prove Lemma \ref{PfDensity}.
\begin{proof}[Proof of Lemma \ref{PfDensity}] For clarity we split the proof into several steps.\\
\textbf{Step 1. }In this step we show that $\widetilde{\rho}(z_1,\cdots,z_{k})\geq 0$ and $\widetilde{\rho}(z_1,\cdots,z_k)>0$ if $z_1>z_2>\cdots>z_k$. Because of the indicator function in $\widetilde{\rho(\vec{z})}$, we know $\widetilde{\rho}(z_1,\cdots,z_{k})=0$ unless $z_1>\cdots>z_k$. Therefore, it suffices to show that 
\begin{align}{\label{RhoPositive}}
\widetilde{\rho}(z_1,\cdots,z_k)>0 \text{ if } z_{1}>\cdots>z_k	
\end{align}
Fix $\vec{z}\in\mathbb{R}^{k}$ such that $z_1>\cdots>z_k$. Suppose that $T_{0}\in\mathbb{N}$ is sufficiently large so that for $T\geq T_{0}$ we have
$$z_{k}\sqrt{T}+ptT\geq a_1\sqrt{T}+k+1 \text{ and } b_{k}\sqrt{T}+pT\geq z_{1}\sqrt{T}+ptT+k+1,$$ which ensures that $\lambda_{i}-x^{T}_{j}+j-i$ and $y^{T}_{i}-\lambda_{j}+j-i$ are positive. For a signature $\nu$ of length $k$ we define
\begin{align}
	A_{\nu}(T)=s_{\nu/\vec{x}}(1^{m})\cdot s_{\vec{y}/\nu}(1^{n}),\text{ where } m=\lfloor tT \rfloor \text{ and } n=T-m
\end{align} 
\begin{align}{\label{DefBLambda}}
\begin{split}
	& B_{\nu}(T)=(\sqrt{2\pi})^{k}\exp\left(-kT\log(1+p)+k\log T +(k/2)\log(p(1+p))\right)\\
	& \cdot \exp\left(-\log\left(\frac{1+p}{p}\right)\sum_{i=1}^{k}(y_{i}^{T}-x_{i}^{T})\right)\cdot A_{\nu}(T)
\end{split}
\end{align}
We claim that
\begin{align}{\label{BLambda}}
	\lim_{T\rightarrow\infty} B_{\lambda}(T)=\widetilde{\rho}(z_1,\cdots,z_k)\cdot (2\pi p(p+1)t(1-t))^{-k/2}\cdot\prod_{i=1}^{k}\exp(-c_1(t,p)a_i^{2}-c_2(t,p)b_i^2)
\end{align}
We prove (\ref{BLambda}) in Step 2 below. For now we assume its validity and conclude the proof of (\ref{RhoPositive}). In fact, since $B_{\lambda}(T)\geq 0$ for all $T\geq T_{0}$ we see from (\ref{DefBLambda}) that $\widetilde{\rho}(z_1,\cdots,z_k)\geq 0$. Also, by Lemma \ref{NonVanish} we know that $\widetilde{\rho}(z_1,\cdots,z_{k})\neq 0$ so (\ref{RhoPositive}) holds.\\
\textbf{Step 2.} In this step we prove the claim (\ref{BLambda}) in Step 1. From the Jacob-Trudi formula for skew Schur polynomials (\ref{J-TFormula}) and Lemma \ref{AsympForh} we have
\begin{align}
	\begin{split}
		& s_{\lambda/\vec{x}}(1^{m})=\det\left[\exp\left(-\frac{(\lambda_{i}-x_{j}^{T}+j-i-pm)^{2}}{2(1+p)pm}\right)\exp(O\left( T^{-1/2}\right)\right]\cdot (\sqrt{2\pi})^{-k}\cdot\\
		& \exp\left(km\log(1+p)-(k/2)\log m-(k/2)\log(p(1+p))+\log\left(\frac{1+p}{p}\right)\sum_{i=1}^{k}(\lambda_{i}-x^{T}_{i})\right)
	\end{split}
\end{align}
\begin{align}
	\begin{split}
		& s_{\vec{y}/\lambda}(1^{n})=\det\left[\exp\left(-\frac{(\vec{y}-x_{j}^{T}+j-i-pm)^{2}}{2(1+p)pm}\right)\exp(O\left( T^{-1/2}\right)\right]\cdot (\sqrt{2\pi})^{-k}\cdot\\
		& \exp\left(km\log(1+p)-(k/2)\log m-(k/2)\log(p(1+p))+\log\left(\frac{1+p}{p}\right)\sum_{i=1}^{k}(\lambda_{i}-x^{T}_{i})\right)
	\end{split}
\end{align}
\begin{align}
	\begin{split}
		& s_{\lambda/\vec{x}}(1^{m})=\det\left[\exp\left(-\frac{(\lambda_{i}-x_{j}^{T}+j-i-pm)^{2}}{2(1+p)pm}\right)\exp(O\left( T^{-1/2}\right)\right]\cdot (\sqrt{2\pi})^{-k}\cdot\\
		& \exp\left(km\log(1+p)-(k/2)\log m-(k/2)\log(p(1+p))+\log\left(\frac{1+p}{p}\right)\sum_{i=1}^{k}(\lambda_{i}-x^{T}_{i})\right)
	\end{split}
\end{align}
\end{proof}
\begin{lemma}{\label{PointConvLemma}}
	Let $\mathbb{W}_{k}^{o}$ denote the open Weyl chamber in $\mathbb{R}^{N}$:
$$\mathbb{W}_{N}^{o}=\{(x_{1},\cdots,x_{k})\in\mathbb{R}^{N}:x_{1}>x_{2}>\cdots>x_{k}\}$$ Fix a real number $A>0$, $p,t\in(0,1)$, take $z=(z_{1},\cdots,z_{k})\in\mathbb{W}_{k}^{o}$ such that $A>z_{1}>\cdots>z_{k}>-A$. Choose sufficiently large $T_{0}$ such that $ptT_{0}-A\sqrt{T_{0}}\geqslant 1$, then for $T\geqslant T_{0}$, define $\lambda_{i}(T)=\lfloor z_{i}\sqrt{T}+ptT\rfloor\geqslant 1$ for $i=1,\cdots,k$. Denote $P_{T}(z)=(\sqrt{T})^{k}\pr(L_1(\lfloor tT\rfloor)=\lambda_1(T),\cdots, L_k(\lfloor tT\rfloor)=\lambda_k(T))$, then we have for almost every $z\in[-A,A]^{k}$:
$$\lim_{T\rightarrow\infty}P_{T}(z)=\rho(z_{1},\cdots,z_{k})$$ and $P_{T}(z)$ is bounded on $[-A,A]^{k}$, where $\rho(z)$ is given in Proposition \ref{WeakConvDistinct}.
\end{lemma}

\begin{proof}
	(\romannumeral 1) First, we discuss the pointwise convergence of $P_{T}(z)$. By the Jacobi-Trudi formula(\cite[Chapter 1, (5.4)]{Mac} ), we conclude:
\begin{align}{\label{J-T}}
\pr(L_1(\lfloor tT\rfloor)=\lambda_1,\cdots, L_k(\lfloor tT\rfloor)=\lambda_k)=\frac{\det\left(h_{\lambda_i-x_j^T+j-i}(1^{\floor{tT}})\right)_{i,j=1}^k\cdot\det\left(h_{y_i^T-\lambda_j+j-i}(1^{T-\floor{tT}})\right)_{i,j=1}^k}{\det\left(h_{y_i^{T}-x_j^T+j-i}(1^T)\right)_{i,j=1}^k}	
\end{align}
We write $S_{1}(p,t,T)=\big((p+1)tT)\log((p+1)tT\big)-(ptT)\log(ptT)-(tT)\log(tT)$ and further calculate the determinant 
\begin{align*} 
\det(h_{\lambda_{i}-x_{j}-i+j}(1^{\lfloor tT \rfloor}))_{i,j=1}^{k} =&\left[(\sqrt{2\pi})^{-1}\sqrt{\frac{p+1}{pt}}\sqrt{T}^{-1}e^{S_{1}(p,t,T)-\log(p+1)}\right]^{k}\\
&\cdot
	 \det\left(e^{(-i+j)\log(\frac{p+1}{p})+(z_{i}-a_{j})\sqrt{T}\cdot \log(\frac{p+1}{p})-\frac{1}{2}\frac{(z_{i}-a_{j})^2}{p(p+1)t}+o(1)}\right)_{i,j=1}^{k}\\
	 =& \left[(\sqrt{2\pi})^{-1}\sqrt{\frac{p+1}{pt}}\sqrt{T}^{-1}e^{S_{1}(p,t,T)-\log(p+1)}\right]^{k}\left(\frac{p+1}{p}\right)^{\sum_{i=1}^{k}(z_{i}-a_{i})\cdot\sqrt{T}} \\
	&\cdot e^{-\frac{1}{2p(p+1)t}\sum_{i=1}^{k}(a_{i}^2+z_{i}^2)}\cdot \det\left(e^{c_{1}(p,t)z_{i}a_{j}+o(1)}\right)_{i,j=1}^{k}
\end{align*}
where the constant $c_{1}(p,t)=\frac{1}{p(p+1)t}$.

We calculate the other two determinants in equation \ref{J-T} analogously:
\begin{align*}
	\det \left(h_{y_{i}-\lambda_{j}-i+j}(1^{T-\lfloor tT \rfloor})\right)&= \left[(\sqrt{2\pi})^{-1}\sqrt{\frac{p+1}{p(1-t)}}\sqrt{T}^{-1}\cdot e^{S_{2}(p,t,T)-\log(p+1)}\right]^{k}\left(\frac{p+1}{p}\right)^{\sum_{i=1}^{k}(b_{i}-z_{i})\cdot\sqrt{T}} \\
	& \cdot e^{-\frac{1}{2p(p+1)(1-t)}\sum_{i=1}^{k}(b_{i}^2+z_{i}^2)}\cdot \det\left(e^{c_{2}(p,t)b_{i}z_{j}+o(1)}\right)_{i,j=1}^{k}\\
	\det(h_{y_{i}^{T}-x_{j}^{T}-i+j}(1^{T}))=& \left[(\sqrt{2\pi})^{-1}\sqrt{\frac{p+1}{p}}\sqrt{T}^{-1}\cdot e^{S_{3}(p,t,T)-\log(p+1)}\right]^{k}\left(\frac{p+1}{p}\right)^{\sum_{i=1}^{k}(b_{i}-a_{i})\cdot\sqrt{T}} \\
	&\cdot \det\left(e^{-\frac{1}{2p(p+1)}(b_{i}-a_{j})^2+o(1)}\right)_{i,j=1}^{k}
	\end{align*}
where the constants $S_{2}(p,t,T)$, $S_{3}(p,t,T)$, $c_{2}(p,t)$ are:
\begin{align*}
S_{2}(p,t,T)&=((p+1)(1-t)T)\log((p+1)tT)-(p(1-t)T)\log(ptT)-((1-t)T)\log((1-t)T)\\ 
S_{3}(p,t,T)&=((p+1)T)\log((p+1)tT)-(pT)\log(pT)-T\log T,\quad c_{2}(p,t)=\frac{1}{p(p+1)(1-t)}\end{align*}
Notice that $S_{1}(p,t,T)+S_{2}(p,t,T)-S_{3}(p,t,T)=0$. Plugging the above three determinants into equation \ref{J-T}, we get
\begin{equation}{\label{PointConv}}
\begin{split}
	& \pr(L_1(\lfloor tT\rfloor)=\lambda_1,\cdots, L_k(\lfloor tT\rfloor)=\lambda_k)\\
	= & (2\pi)^{-\frac{k}{2}}\left[\sqrt{\frac{p+1}{pt(1-t)}}\right]^{k}\cdot T^{-\frac{k}{2}} \cdot e^{-\frac{1}{2p(p+1)t}\sum_{i=1}^{k}(a_{i}^2+z_{i}^2)-\frac{1}{2p(p+1)(1-t)}\sum_{i=1}^{k}(b_{i}^2+z_{i}^{2})}\\
	& \cdot e^{-k\log(p+1)}\cdot \frac{\det(e^{c_{1}(p,t)z_{i}a_{j}+o(1)})_{i,j=1}^{k}\cdot \det(e^{c_{2}(p,t)b_{i}z_{j}+o(1)})_{i,j=1}^{k}}{\det(e^{-\frac{1}{2p(p+1)}(b_{i}-a_{j})^{2}+o(1)})_{i,j=1}^{k}}\\
	= &(2\pi)^{-\frac{k}{2}}\left[\sqrt{\frac{1}{p(p+1)t(1-t)}}\right]^{k}\cdot T^{-\frac{k}{2}}\cdot e^{-\frac{1}{2p(p+1)t}\sum_{i=1}^{k}a_{i}^2-\frac{1}{2p(p+1)(1-t)}\sum_{i=1}^{k}b_{i}^2}\\
	& \cdot\frac{\det\left(e^{c_{1}(p,t)z_{i}a_{j}}\right)_{i,j=1}^{k}\cdot \det\left(e^{c_{2}(p,t)b_{i}z_{j}}\right)_{i,j=1}^{k}}{\det\left(e^{-\frac{1}{2p(p+1)}(b_{i}-a_{j})^{2}}\right)_{i,j=1}^{k}}\cdot \exp\{o(1)\}\cdot \prod_{i=1}^{k}e^{-c_{3}(t,p)z_{i}^2}
\end{split}
\end{equation}
where $c_{3}(t,p)=\frac{1}{2p(p+1)t(1-t)}$, and the constant in little $o$ notation depends on $A,p$.

In conclusion, $P_{T}(z)=(\sqrt{T})^{k}\cdot\pr(L_1(\lfloor tT\rfloor)=\lambda_1,\cdots, L_k(\lfloor tT\rfloor)=\lambda_k)$ converges to $\rho(z_{1},\dots,z_{k})$ in Proposition \ref{WeakConvDistinct} as $T\rightarrow\infty$ and 
\begin{align*}
	&c_{1}(p,t)=\frac{1}{p(p+1)t}, \quad c_{2}(p,t)=\frac{1}{p(p+1)(1-t)}, \quad c_{3}(p,t)=\frac{1}{2p(p+1)t(1-t)}\\
	&Z=(2\pi)^{\frac{k}{2}}(p(p+1)t(1-t))^{\frac{k}{2}}\cdot e^{c_{1}(t,p)\sum_{i=1}^{k}a_{i}^{2}}\cdot e^{c_{2}(t,p)\sum_{i=1}^{k}b_{i}^{2}}\det\left(e^{-\frac{1}{2p(p+1)}(b_{i}-a_{j})^{2}}\right)_{i,j=1}^{k}
\end{align*}
(\romannumeral 2) We now discuss boundedness. By the Equation \ref{PointConv} we just derived, $P_{T}(z)=\rho(z)\cdot\exp\{o(1)\}$ on the compact set $[-A,A]^{k}$, where the constant in little $o$ notation only depends on $A,p$. Since the continuous function $\rho(z)$ is bounded on $[-A,A]^{k}$, and $\exp\{o(1)\}$ is uniformly bounded on $[-A,A]^{k}$, we conclude that $P_{T}(z)$ is bounded as well.
\end{proof}
We are now ready to prove Proposition \ref{WeakConvDistinct}.
\begin{proof}[Proof of Proposition \ref{WeakConvDistinct}]
In the following, we prove the weak convergence of the random vector $Z^{T}$, when $\vec{a}=(a_{1},\cdots,a_{k})$ and $\vec{b}=(b_{1},\cdots,b_{k})$ consist of distinct entries. 
In order to show weak convergence, it is sufficient to show that for every open set $O\in\mathbb{R}^{k}$, we have: 
$$\liminf_{T\rightarrow\infty}\mathbb{P}((Z_{1}^{T},\cdots,Z_{k}^{T})\in O)\geqslant\int_{O}\rho(z_{1},\cdots,z_{k})dz_{1}dz_{2}\cdots dz_{k}$$
according to \cite[Theorem 3.2.11]{Durrett}. It is also sufficient to show that for any open set $U\in\mathbb{W}_{k}^{o}$, we have:
\begin{align}{\label{WeakConv}}
	\liminf_{T\rightarrow\infty}\mathbb{P}((Z_{1}^{T},\cdots,Z_{k}^{T})\in U)\geqslant\int_{U}\rho(z_{1},\cdots,z_{k})dz_{1}dz_{2}\cdots dz_{k}
\end{align}
which implies that:
\begin{align*}
	&\liminf_{T\rightarrow\infty}\mathbb{P}((Z_{1}^{T},\cdots,Z_{k}^{T})\in O)\geqslant\liminf_{T\rightarrow\infty}\mathbb{P}((Z_{1}^{T},\cdots,Z_{k}^{T})\in O\cap\mathbb{W}_{k}^{o})\\
	&\geqslant \int_{\mathbb{W}_{k}^{o}\cap O}\rho(z_{1},\cdots,z_{k})dz_{1}\cdots dz_{k}= \int_{O}\rho(z_{1},\cdots,z_{k})dz_{1}\cdots dz_{k}
\end{align*}
The second inequality uses the above result (\ref{WeakConv}), since $\mathbb{W}_{k}^{o}\cap O$ is an open set in $\mathbb{W}_{k}^{o}$. The last equality is because $\rho(z)$ is zero outside $\mathbb{W}_{k}^{o}$. The rest of the proof will be divided into $4$ steps. In Step 1, we prove that weak convergence holds on every closed rectangle. In Step 2, we prove the result \ref{WeakConv} using Lemma \ref{PointConvLemma}. In Step 3, we prove that $\rho(z)$ is actually a density and conclude weak convergence.\\
\noindent \textbf{Step 1. }In this step, we establish the following result:\\
For any closed rectangle $R=[u_{1},v_{1}]\times [u_{2},v_{2}]\times\cdots\times[u_{N},v_{N}]\in\mathbb{W}_{k}^{o}$, 
\begin{align}
	\lim_{T\rightarrow\infty}\mathbb{P}((Z_{1}^{T},\cdots,Z_{k}^{T})\in R)=\int_{R}\rho(z_{1},\cdots,z_{k})dz_{1}\cdots dz_{k}
\end{align}
where $\rho(z)$ is given in Proposition \ref{WeakConvDistinct}.

Define $m_{i}^{T}=\lceil u_{i}\sqrt{T}+ptT\rceil$ and $M_{i}^{T}=\lfloor v_{i}\sqrt{T}+ptT\rfloor$. Then we have:
\begin{align*}
&\mathbb{P}((Z_{1}^{T},\cdots,Z_{k}^{T})\in R)=\mathbb{P}(u_{1}\leqslant Z_{1}^{T} \leqslant v_{1}, \dots,  u_{k}\leqslant Z_{k}^{T} \leqslant v_{k})\\
&=\mathbb{P}(u_{i}\sqrt{T}+ptT\leqslant L_{i}(\lfloor tT\rfloor) \leqslant v_{i}\sqrt{T}+ptT, i=1,\dots, k)\\
&=\sum_{\lambda_{1}(T)=m_{1}^{T}}^{M_{1}^{T}}\cdots\sum_{\lambda_{k}(T)=m_{k}^{T}}^{M_{k}^{T}}\mathbb{P}(L_{1}(\lfloor tT \rfloor)=\lambda_{1}(T),\dots,L_{k}(\lfloor tT \rfloor)=\lambda_{k}(T))\\
&=\sum_{\lambda_{1}(T)=m_{1}^{T}}^{M_{1}^{T}}\dots\sum_{\lambda_{k}(T)=m_{k}^{T}}^{M_{k}^{T}}(\sqrt{T})^{-k}\cdot(\sqrt{T})^{k}\mathbb{P}(L_{1}(\lfloor tT \rfloor)=\lambda_{1}(T),\dots,L_{k}(\lfloor tT \rfloor)=\lambda_{k}(T))
\end{align*}

Find sufficiently large $A$ such that $R\subset[-A,A]^{k}$, for example, $A=1+\max_{1\leqslant i\leqslant k}|a_{i}|+\max_{1\leqslant i\leqslant k}|b_{i}|$. Define $f_{T}(z_{1},\cdots,z_{k})$ as a simple function on $\mathbb{R}^{k}$: When $(z_{1},\cdots,z_{k})\in R$, it takes value $(\sqrt{T})^{k}\cdot\mathbb{P}(L_{1}(\lfloor tT \rfloor)=\lambda_{1}(T),\cdots,L_{k}(\lfloor tT \rfloor)=\lambda_{k}(T)) $ if there exist $\lambda_{1}(T),\cdots,\lambda_{k}(T)$ such that $\lambda_{i}(T)\leqslant z_{i}\sqrt{T}+ptT<\lambda_{i}(T)+1$; It takes value $0$ otherwise, when $(z_{1},\cdots,z_{k})\notin R$.  Since the Lebesgue measure of the set $\{z:\lambda_{i}(T)\leqslant z_{i}\sqrt{T}+ptT<\lambda_{i}(T)+1,i=1,\cdots,k\}$ is $(\sqrt{T})^{-k}$, the above probability can be further written as an integral of simple functions $f_{T}(z_{1},\cdots,z_{k})$:
\begin{align*}
\mathbb{P}((Z_{1}^{T},\cdots,Z_{k}^{T})\in R)&=\int_{[-A,A]^{k}}f_{T}(z_{1},\cdots,z_{k})dz_{1}\cdots dz_{k}
\end{align*}

By Lemma \ref{PointConvLemma}, the function $f_{T}(z_{1},\cdots,z_{k})$ pointwise converges to $\rho(z)$ and is bounded on the compact set $[-A,A]^{k}$. Since the Lebesgue measure of $[-A,A]^{k}$ is finite, by the bounded convergence theorem we have:
\begin{align}{\label{ConvOnRect}}
	\lim_{T\rightarrow\infty}\mathbb{P}((Z_{1}^{T},\cdots,Z_{k}^{T})\in R)=\int_{R}\rho(z_{1},\cdots,z_{k})dz_{1}\cdots dz_{k}
\end{align}
\textbf{Step 2. }In this step, we prove the statement \ref{WeakConv}. Take any open set $U\in \mathbb{W}_{k}^{o}$. Then this set can be written as a countable union of closed rectangles with disjoint interiors: $U=\bigcup_{i=1}^{\infty}R_{i}$, where $R_{i}=[a_{1}^{i},b_{1}^{i}]\times\cdots\times[a_{k}^{i},b_{k}^{i}]$(\cite[Theorem 1.4]{Stein}). Choose sufficiently small $\epsilon>0$, and denote $R_{i}^{\epsilon}=[a_{1}^{i}+\epsilon,b_{1}^{i}-\epsilon]\times\cdots\times[a_{k}^{i}+\epsilon,b_{k}^{i}-\epsilon]$, then $R_{i}^{\epsilon}$ are disjoint. Therefore,
\begin{align*}
	&\liminf_{T\rightarrow\infty}\mathbb{P}((Z_{1}^{T},\cdots,Z_{k}^{T})\in U)\geqslant\liminf_{T\rightarrow\infty}\mathbb{P}((Z_{1}^{T},\cdots,Z_{k}^{T})\in \bigcup_{i=1}^{n}R_{i}^{\epsilon})\\
	&=\liminf_{T\rightarrow\infty}\sum_{i=1}^{n}\mathbb{P}((Z_{1}^{T},\cdots,Z_{k}^{T})\in R_{i}^{\epsilon})=\sum_{i=1}^{n}\int_{R_{i}^{\epsilon}}\rho(z_1,\cdots,z_{k})dz_{1}\cdots dz_{k}\\
	&=\int_{\bigcup_{i=1}^{n}R_{i}^{\epsilon}}\rho(z_1,\cdots,z_{k})dz_{1}\cdots dz_{k} \xrightarrow{\epsilon\downarrow 0} \int_{U}\rho(z_1,\cdots,z_{k})dz_{1}\cdots dz_{k}
\end{align*}
The last line uses the monotone convergence theorem. Thus, we have proved the inequality \ref{WeakConv}.\\
\textbf{Step 3. }In this step, we prove that $\rho(z)$ is actually a density. First, it is nonnegative because it is the limit of a sequence of probabilities. Next, we prove it integrates to $1$ over $\mathbb{R}^{k}$. Let the open set $U$ in Step 2 be $\mathbb{W}_{k}^{o}$ to get: $$1=\liminf_{T\rightarrow\infty}\mathbb{P}(Z^{T}\in \mathbb{W}_{k}^{o})\geqslant \int_{\mathbb{W}_{k}^{o}}\rho(z)dz$$
On the other hand, write the open set $\mathbb{W}_{k}^{o}$ as a countable union of almost disjoint closed rectangles: $\mathbb{W}_{k}^{o}=\bigcup_{i=1}^{\infty}R_{i}$. Then, for any $n\in\mathbb{Z}^{+}$ we can find $N_{n}\in \mathbb{Z}^{+}$ such that $\mathbb{P}(Z^{T}\in\mathbb{W}_{k}^{o})-\mathbb{P}(Z^{T}\in\bigcup_{i=1}^{N_{n}}R_{i})\leq 1/n$, and $N_{n}\rightarrow\infty$ as $n\rightarrow\infty$. 
\begin{align*}
	1&=\mathbb{P}((Z_{1}^{T},\cdots,Z_{k}^{T})\in \mathbb{W}_{k}^{o})\leq \mathbb{P}(Z^{T}\in\bigcup_{i=1}^{N_{n}}R_{i})+1/n \leq \sum_{i=1}^{N_{n}}\mathbb{P}(Z^{T}\in R_{i})+1/n
\end{align*}
Let $T\rightarrow\infty$. Then we obtain
$$1\leq \sum_{i=1}^{N_{n}}\int_{R_{i}}\rho(z)dz+1/n=\int_{\bigcup_{i=1}^{N_{n}}R_{i}}\rho(z)dz+1/n$$
due to (\ref{ConvOnRect}) and the fact that $\rho(z)$ integrates to $0$ over the boundary of rectangles. Finally, let $n\rightarrow\infty$. We then have $\int_{\mathbb{W}_{k}^{o}}\rho(z)dz\geq 1$ by the monotone convergence theorem and conclude that $\rho(z)$ is a density.
\end{proof}

\subsection{Multi-indices and multivariate functions}\label{multivar} In this step, we first introduce some notations and results about multivariate functions.

Suppose $\sigma = (\sigma_{1},\cdots,\sigma_{n})$ is a multi-index of \emph{length} $n$. In our context, we require $\sigma_{1},\cdots,\sigma_{n}$ be all non-negative integers(some of them might be equal). We define $|\sigma|=\sum_{i=1}^{n}\sigma_{i}$ as the \emph{order} of $\sigma$. Suppose $\tau=(\tau_{1},\cdots,\tau_{n})$ is another multi-index of length $n$. We say $\tau\leqslant \sigma$ if $\tau_{i}\leqslant \sigma_{i}$ for $i=1,\cdots,n$. We say $\tau<\sigma$ if $\tau\leqslant \sigma$ and there exists at least one index $i$ such that $\tau_{i}<\sigma_{i}$. Then, define the partial derivative with respect to the multi-index $\sigma$:
$$D^{\sigma}f(x_{1},\cdots,x_{n})=\frac{\partial^{|\sigma|}f(x_{1},\cdots,x_{n})}{\partial x_{1}^{\sigma_{1}}\partial x_{2}^{\sigma_{2}}\cdots \partial x_{n}^{\sigma_{n}}}$$ We have the general Leibniz rule:
\begin{align*}
	D^{\sigma}(fg)=\sum_{\tau\leqslant\sigma}\binom{\sigma}{\tau}D^{\tau}f\cdot D^{\sigma-\tau}g
\end{align*}
where $\binom{\sigma}{\tau}=\frac{\sigma_{1}!\cdots\sigma_{n}!}{\tau_{1}!\cdots\tau_{n}!(\sigma_{1}-\tau_{1})!\cdots(\sigma_{n}-\tau_{n})!}$.\\
We also have the Taylor expansion for multi-variable functions:
$$f(x_{1},\cdots,x_{n})=\sum_{|\sigma|\leqslant r}\frac{1}{\sigma!}D^{\sigma}f(\vec{x}_{0})(\vec{x}-\vec{x}_{0})^{\sigma}+R_{r+1}(\vec{x},\vec{x}_{0})$$ 
In the equation, $\sigma!=\sigma_{1}!\sigma_{2}!\cdots\sigma_{n}!$ is the factorial with respect to the multi-index $\sigma$, $\vec{x}_{0}=(x_{1}^{0},\cdots,x_{n}^{0})$ is a constant vector at which we expands the function $f$, $(\vec{x}-\vec{x}_{0})^{\sigma}$ stands for $(x_{1}-x_{1}^{0})^{\sigma_{1}}\cdots(x_{n}-x_{n}^{0})^{\sigma_{n}}$, and $$R_{r+1}(\vec{x},\vec{x}_{0})=\sum_{\sigma:|\sigma|=r+1}\frac{1}{\sigma!}D^{\sigma}f(\vec{x}_{0}+\theta(\vec{x}-\vec{x}_{0}))(\vec{x}-\vec{x}_{0})^{\sigma}$$ is the remainder, where $\theta\in (0,1)$(\cite[Theorem 3.18 \& Corollary 3.19]{CJ}).

We also need some knowledge about \emph{permutation}. Suppose $s_{n}$ is a permutation of $n$ non-negative integers, for example $\{1,\cdots,n\}$, and $s_{n}(i)$ represents the $i$-$th$ element in the permutation $s_{n}$. We define \emph{the number of inversions} of $s_{n}$ by $I(s_{n})=\sum_{i=1}^{n-1}\sum_{j=i+1}^{n}\mathbf{1}_{\{s_{n}(i)>s_{n}(j)\}}$. For example, the permutation $s_{n}=(1,\cdots,n)$ has $0$ number of inversions, while the permutation $s_{5}=(3,2,5,1,4)$ has number of inversions $5(2+1+2+0+0)$. Define the sign of permutation $s_{n}$ by $sgn(s_{n})=(-1)^{I(s_{n})}$. For instance, $sgn((1,\cdots,n))=1$ and $sgn(s_{5})=-1$ in the previous example.

\subsection{Proof of Proposition \ref{WeakConvCollide}}{\label{ProofProp2}}
Next we are going to prove Proposition \ref{WeakConvCollide}. Before that, 
Then, we introduce some notations associated with Proposition \ref{WeakConvCollide} in order to better discuss the problem. Recall that
\begin{align*}
	&\vec{a}_{0}=(\underbrace{\alpha_{1},\cdots,\alpha_{1}}_{m_{1}},\cdots,\underbrace{\alpha_{p},\cdots,\alpha_{p}}_{m_{p}})\\
	&\vec{b}_{0}=(\underbrace{\beta_{1},\cdots,\beta_{1}}_{n_{1}},\cdots,\underbrace{\beta_{q},\cdots,\beta_{q}}_{n_{q}})
\end{align*}
where $\alpha_{1}>\alpha_{2}>\cdots>\alpha_{p}$, $\beta_{1}>\beta_{2}>\cdots>\beta_{q}$ and $\sum_{i=1}^{p}\alpha_{i}=\sum_{i=1}^{q}\beta_{i}=k$. Denote $\vec{a}=(a_{1},\cdots,a_{k})$, $\vec{b}=(b_{1},\cdots,b_{k})$. Also denote $\vec{a}^{(1)}=(a_{1},\cdots,a_{m_1})$, $\vec{a}^{(2)}=(a_{m_{1}+1},\cdots,a_{m_1+m_2})$, $\cdots$, $\vec{a}^{(p)}=(a_{m_1+\cdots+m_{p-1}+1},\cdots, a_{m_1+\cdots+m_{p}})$ and $\vec{a}=(\vec{a}^{(1)},\cdots,\vec{a}^{(p)})$. That is, we divide the vector $\vec{a}$ into $p$ blocks according to the shape of $\vec{a_{0}}$. Similarly, we write $\vec{b}=(b^{(1)},\cdots,b^{(q)})$ according to the shape of $\vec{b}_{0}$. We will keep using similar notations in the following discussion, when we need to divide the vector according to the shape of $\vec{a}_{0}$ and $\vec{b}_{0}$. Next, denote
\begin{align*}
	f(a_{1},\cdots,a_{k})\equiv f(\vec{a})&=\det[e^{c_1(t,p)a_{i}z_{j}}]_{i,j=1}^{k},\quad g(b_{1},\cdots,b_{k})\equiv g(\vec{b})=\det[e^{c_2(t,p)b_{i}z_{j}}]_{i,j=1}^{k}
\end{align*} 
and it's not difficult to see that they are all smooth multi-variable functions with respect to corresponding vectors. In addition, $\lim\limits_{\vec{a}\rightarrow\vec{a}_{0}}f(\vec{a})=0$ and $\lim\limits_{\vec{b}\rightarrow\vec{b}_{0}}g(\vec{b})=0$. 

The last thing before we formally prove Proposition \ref{WeakConvCollide} is to introduce the following lemmas about the non-vanishing of the determinants.

\begin{remark}{\label{DefLambda}}
Denote the set $\Lambda_{a}=\{\sigma_{a}=(\sigma_{a}^{(1)},\cdots,\sigma_{a}^{(p)}):\sigma_{a}^{(i)}\in S_{m_{i}}, i=1,\cdots,p\}$, and we have if $\sigma_{a}\in \Lambda_{a}$, then $D^{\sigma_{a}}f(\vec{a}_{0})$ is non-zero. Similarly, if $\sigma^{(j)}_{b}\in S_{n_j}$ for $j=1,\cdots,q$, then $D^{\sigma_{b}}g(\vec{b})$ is non-zero, and define $\Lambda_{b}=\{\sigma_{b}=(\sigma_{b}^{(1)},\cdots,\sigma_{b}^{(q)}):\sigma_{b}^{(j)}\in S_{n_{j}}, j=1,\cdots,q\}$.
\end{remark}

\begin{lemma}{\label{MinOrder}}
The smallest order of $\sigma_{a}$ that makes the partial derivative $D^{\sigma_{a}}f(\vec{a}_{0})$ non-zero is $u=\sum_{i=1}^{p}\sum_{j=0}^{m_{i}-1}j=\sum_{i=1}^{p}\frac{m_{i}(m_{i}-1)}{2}$. Similarly, $v=\sum_{j=1}^{q}\frac{n_{j}(n_{j}-1)}{2}$ is the smallest order of $\sigma_{b}$ that makes $D^{\sigma_{b}}f(\vec{b}_{0})$ non-zero.
\end{lemma}
\begin{proof}
	If the order of derivative is less than $u$, then there exists a $i\in\{1,\cdots,p\}$ such that $\sigma_{a}^{(i)}$ contains two equal elements, and the determinant $D^{\sigma_{a}}f(\vec{a}_{0})$ would have two equal rows, thus equal to zero. If the order of derivative is $u$, then when $\sigma_{a}\in\Lambda_{a}$, $D^{\sigma_{a}}f(\vec{a}_{0})$ is non-zero by Lemma \ref{NonVanish}. Thus, Lemma \ref{MinOrder} holds.
\end{proof}

Finally, we give the proof for Proposition \ref{WeakConvCollide}. 
\begin{proof}[Proof of Proposition \ref{WeakConvCollide}] For clarity, the proof will be split into $3$ steps. In Step 1, we use multi-variate Taylor expansion to find the speed of convergence of $f(\vec{a})$ and $g(\vec{b})$ to zero. In Step 2, we construct a new density function based on Step 1, and we will prove that $Z^{T}$ weakly converges to the this newly constructed density in Step 3. In Step 3, we use monotone coupling lemma to prove the weak convergence.\\
\textbf{Step 1. }In this step, we estimate the converging speed of $f(\vec{a})$. Take $\epsilon\in (0,k^{-1}\min\limits_{1\leqslant i\leqslant p-1}(\alpha_{i}-\alpha_{i+1}))$ and construct the following vectors:
\begin{align*}
	\vec{A}_{\epsilon,+}&=(\alpha_{1}+m_{1}\epsilon, \alpha_{1}+(m_{1}-1)\epsilon,\cdots,\alpha_{1}+\epsilon,\cdots,\alpha_{p}+m_{p}\epsilon,\cdots,\alpha_{p}+\epsilon)\\
	\vec{A}_{\epsilon,-}&=(\alpha_{1}-\epsilon, \alpha_{1}-2\epsilon,\cdots,\alpha_{1}-m_{1}\epsilon,\cdots,\alpha_{p}-\epsilon,\cdots,\alpha_{p}-m_{p}\epsilon)
\end{align*}
That is, the vector $\vec{A}_{\epsilon,+}(\text{resp. }\vec{A}_{\epsilon,-})$ upwardly(resp. downwardly) spreads out the vector $\vec{a}_{0}$ such that $\vec{A}_{\epsilon,+}(\text{resp. }\vec{A}_{\epsilon,-})$ has distinct elements. In addition, when $\epsilon\downarrow 0$, we have $\vec{A}_{\epsilon,\pm}$ converges to $\vec{a}_{0}$. The main result of this step is the following:
\begin{align}{\label{ConvSpeed}}
	\lim_{\epsilon\downarrow 0}\epsilon^{-u}f(\vec{A}_{\epsilon,\pm})=\varphi(\vec{a}_{0},\vec{z},\vec{m})
\end{align}
where $u=\sum_{i=1}^{p}\frac{m_{i}(m_{i}-1)}{2}$ in Lemma \ref{MinOrder}, $\vec{m}=(m_{1},\cdots,m_{p})$, and $\varphi(\vec{a}_{0},\vec{z},\vec{m})$ is a non-zero function associated with $\vec{a}_{0}$ and $\vec{m}$.

To prove this result, we first expand the function $f(\vec{a})$ to the order of $u$ at $\vec{a}_{0}$:
\begin{align*}
	f(\vec{a})&=\sum_{|\sigma_{a}|\leqslant u}\frac{D^{\sigma_{a}}f(\vec{a}_{0})}{\sigma_{a}!}(\vec{a}-\vec{a}_{0})^{\sigma_{a}}+R_{u+1}(\vec{a},\vec{a}_{0})\\
	&= \sum_{\sigma_{a}\in \Lambda_{a}}\frac{D^{\sigma_{a}}f(\vec{a}_{0})}{\sigma_{a}!}(\vec{a}-\vec{a}_{0})^{\sigma_{a}}+R_{u+1}(\vec{a},\vec{a}_{0})
\end{align*} 
where the $R_{u+1}(\vec{a},\vec{a}_{0})=\sum_{\sigma_{a}:|\sigma_{a}|=u+1}\frac{1}{\sigma_{a}!}D^{\sigma_{a}}f(\vec{a}_{0}+\theta(\vec{a}-\vec{a}_{0}))(\vec{a}-\vec{a}_{0})^{\sigma_{a}}$, $\theta\in(0,1)$ is the remainder. The second equality results from Lemma \ref{MinOrder}, since it indicates that all the terms of order less than $u$ are zero, and for the terms of order $u$, they are non-zero only when $\sigma_{a}\in\Lambda_{a}$.

Consider the first term. Denote $sgn(\sigma_{a}^{(i)})$ as the sign of the permutation $\sigma_{a}^{(i)}\in S_{m_i}$, and define the sign of $\sigma_{a}$ by: $sgn(\sigma_{a})=\prod_{i=1}^{p}sgn(\sigma_{a}^{(i)})$. Denote $\sigma_{a}^{\star}=(\sigma_{a}^{(1)\star},\cdots,\sigma_{a}^{(p)\star})$, where $\sigma_{a}^{(i)\star}=(0,1,\cdots,m_i-1)$. Thus, $\sigma_{a}^{\star}$ is a special element in $\Lambda_{a}$ and $sgn(\sigma_{a}^{\star})=1$ because $\sigma_{a}^{(1)\star},\cdots,\sigma_{a}^{(p)\star}$ all have $0$ number of inversions. Notice that for any $\sigma_{a}\in\Lambda_{a}$, we have $D^{\sigma_{a}}f(\vec{a}_{0})=sgn(\sigma_{a})\cdot D^{\sigma_{a}^{\star}}f(\vec{a}_{0})$ by the property of determinant. Then we obtain:
\begin{align*}
	\sum_{\sigma_{a}\in\Lambda_{a}}\frac{1}{\sigma_{a}!}D^{\sigma_{a}}f(\vec{a}_{0})(\vec{a}-\vec{a}_{0})^{\sigma_{a}}=\frac{D^{\sigma_{a}^{\star}}f(\vec{a}_{0})}{\prod_{i=1}^{p}(m_{i}-1)!}\sum_{\sigma_{a}\in\Lambda_{a}}(\vec{a}-\vec{a}_{0})^{\sigma_{a}}\cdot sgn(\sigma_{a})
\end{align*}
Notice that 
\begin{align*}
	\sum_{\sigma_{a}\in\Lambda_{a}}(\vec{a}-\vec{a}_{0})^{\sigma_{a}}\cdot sgn(\sigma_{a})&=\prod_{i=1}^{p}\Big[\sum_{\sigma_{a}^{(i)}\in S_{m_i}}(\vec{a}^{(i)}-\vec{a}_{0}^{(i)})^{\sigma_{a}^{(i)}}\cdot sgn(\sigma_{a}^{(i)})\Big]\\
	&=\prod_{i=1}^{p}\Delta_{m_i}(a_{1}^{(i)}-\alpha_{i},a_{2}^{(i)}-\alpha_{i},\cdots,a_{m_i}^{(i)}-\alpha_{i})\equiv\prod_{i=1}^{p}\Delta_{m_i}^{a}
\end{align*}
where $\Delta_{n}(x_{1},x_{2},\cdots,x_{n})$ is the Vandermonde Determinant, $a_{j}^{(i)}=a_{m_{1}+\cdots+m_{i-1}+j}$ is the $j$-$th$ element of $\vec{a}^{(i)}$, and the last equality holds by the definition of determinant and Vandermonde Determinant. Now replace $\vec{a}$ with $\vec{A}_{\epsilon,+}$, we get the Vandermonde determinant $\Delta_{m_{i}}^{a}$ is actually $(m_{i}-1)!\cdot\epsilon^{\frac{1}{2}m_{i}(m_{i}-1)}$. Therefore, we have: $$\sum_{\sigma_{a}\in\Lambda_{a}}\frac{1}{\sigma_{a}!}D^{\sigma_{a}}f(\vec{a}_{0})(\vec{a}-\vec{a}_{0})^{\sigma_{a}}=D^{\sigma_{a}^{\star}}f(\vec{a}_{0})\cdot\epsilon^{u}$$

Now we consider the remainder $R_{u+1}(\vec{A}_{\epsilon,+},\vec{a}_{0})$. Since $D^{\sigma_{a}}f(\vec{a})$ is a continuous function of vector $\vec{a}$, we have that the quantity $D^{\sigma_{a}^{\star}}f(\vec{a}_{0}+\theta(\vec{a}-\vec{a}_{0}))$ can be bounded by a constant $M(\vec{a},\vec{a}_{0})$. In addition, $\sigma_{a}!$ only have finitely many possible outcomes when its order is $u+1$, thus $\frac{1}{\sigma_{a}!}$ can be bounded by a constant $N(u)$. Also, $|(\vec{A}_{\epsilon,+}-\vec{a}_{0})|^{\sigma_{a}}\leqslant (\max_{1\leqslant i\leqslant p}m_{i}\cdot \epsilon)^{u+1}$. Therefore,
\begin{align*}
	|R_{u+1}(\vec{A}_{\epsilon,+},\vec{a}_{0})|\leqslant N\cdot M \cdot (\max_{1\leqslant i\leqslant p}m_{i}\cdot \epsilon)^{u+1}
\end{align*}
and this indicates that $R_{u+1}(\vec{A}_{\epsilon,+},\vec{a}_{0})$ is $O(\epsilon^{u+1})$, where the constant in Big $O$ notation only depends on $\vec{a}_{0}$, $\vec{a}$, $\vec{m}$ and $u$ and does not depend on $\epsilon$. Therefore, we conclude that 
$$\lim_{\epsilon\downarrow 0}\epsilon^{-u}f(\vec{A}_{\epsilon,+})=D^{\sigma_{a}^{\star}}f(\vec{a}_{0})$$
By Lemma \ref{NonVanish}, $D^{\sigma_{a}^{\star}}f(\vec{a}_{0})$ is non-zero. Thus, we find the limit function $\varphi(\vec{a}_{0},\vec{z},\vec{m})=D^{\sigma_{a}^{\star}}f(\vec{a}_{0})$, and its expression can be found in Lemma \ref{NonVanish}. Following similar procedure we can prove $\lim_{\epsilon\downarrow 0}\epsilon^{-u}f(\vec{A}_{\epsilon,-})=D^{\sigma_{a}^{\star}}f(\vec{a}_{0})$ also holds, and we established the equation (\ref{ConvSpeed}).

We can construct vectors $\vec{B}_{\epsilon,\pm}$ analogously, which spread out from vector $\vec{b}_{0}$ upward and downward, and get similar results for $g(\vec{B}_{\epsilon,\pm})$ and then we have:
$$\lim_{\epsilon\downarrow 0}\epsilon^{-v}f(\vec{B}_{\epsilon,\pm})=D^{\sigma_{b}^{\star}}g(\vec{b}_{0})\equiv\psi(\vec{b}_{0},\vec{z},\vec{n})$$
where $v=\sum_{i=1}^{q}\frac{n_{i}(n_{i}-1)}{2}$ in Lemma \ref{MinOrder}, $\vec{n}=(n_{1},\cdots,n_{q})$ and the expression of non-zero function $\psi(\vec{b}_{0},\vec{z},\vec{n})$ can be found by Lemma \ref{NonVanish}.\\
\textbf{Step 2. }In this step, we mainly prove the following result:

The function of $\vec{z}$ $$H(\vec{z})=\varphi(\vec{a}_{0},\vec{z},\vec{m})\psi(\vec{b}_{0},\vec{z},\vec{n})\prod_{i=1}^{k}e^{-c_{3}(t,p)z_{i}^{2}}$$ is integrable over $\mathbb{R}^{k}$.

For simplicity, we ignore the constants $c_{1}$, $c_{2}$, $c_{3}$ temporarily and prove the function $H(\vec{z})$ without those constants is integrable. It's not difficult to see $H(\vec{z})$ is still integrable when adding those constants. Notice that $\varphi(\vec{a}_{0},\vec{z},\vec{m})$ is a determinant whose expression is given in Lemma \ref{NonVanish}. Suppose $z_{j_1}^{i_1-1}e^{a_{i_1}z_{j_1}}$ is the entry that has the largest absolute value. Then 
\begin{align*}
	|\varphi(\vec{a}_{0},\vec{z},\vec{m})|\leqslant k!|z_{j_{1}}^{i_{1}-1}e^{a_{i_1}z_{j_1}}|^{k}=k!|z_{j_1}^{k(i_1-1)}|e^{ka_{i_1}z_{j_1}}
\end{align*}
Similarly, we can find index $i_2$ and $j_{2}$ such that $|\varphi(\vec{b}_{0},\vec{z},\vec{n})|\leqslant k!|z_{j_2}^{k(i_2-1)}|e^{kb_{i_2}z_{j_2}}$. Then, we get 
\begin{align*}
	|H(\vec{z})|\leqslant (k!)^{2}\big[\prod_{j\neq j_1,j_2}e^{-z_{j}^{2}}|z_{j_1}^{k(i_1-1)}z_{j_2}^{k(i_2-1)}|\big]e^{ka_{i_1}z_{j_1}-z_{j_1}^2}e^{kb_{i_2}z_{j_2}-z_{j_2}^{2}}
\end{align*}
The right hand side is integrable over $\mathbb{R}^{k}$ because the exponential terms have power of some quadratic functions with negative quadratic coefficients. Thus, $H(\vec{z})$ is integrable.

Since $H(\vec{z})$ is integrable, we can define the constant $Z_{\vec{a}_{0},\vec{b}_{0}}=\int_{\mathbb{R}^{k}}H(\vec{z})\mathbf{1}_{\{z_1>z_2>\cdots>z_{k}\}}dz<\infty$ and the function \begin{align}{\label{NewDensity}}
	\rho_{\vec{a}_{0},\vec{b}_{0}}(z_{1},\cdots,z_{k})=Z_{\vec{a}_{0},\vec{b}_{0}}^{-1}\varphi(\vec{a}_{0},\vec{z},\vec{m})\psi(\vec{b}_{0},\vec{z},\vec{n})\prod_{i=1}^{k}e^{-c_{3}(t,p)z_{i}^{2}}\mathbf{1}_{\{z_1>z_2>\cdots>z_{k}\}}
\end{align}
is a density because it's non-negative and integrates to $1$ over $\mathbb{R}^{k}$.\\
\textbf{Step 3. }Denote $Z^{T}_{\vec{a}_{0},\vec{b}_{0}}$ as the random vector $Z^{T}$ associated with vectors $\vec{a}_{0}$ and $b_{0}$, and in this step we prove it weakly converges to the continuous distribution with density $\rho_{\vec{a}_{0},\vec{b}_{0}}(z)$ we just constructed in (\ref{NewDensity}). Suppose $\mathfrak{L}_{+}^{T}$ is an avoiding Bernoulli line ensemble starting with $\vec{x}^{T}_{+}$ and ending with $\vec{y}^{T}_{+}$ and follows the distribution $\mathbb{P}_{Avoid,Ber}^{0,T,\vec{x}^{T}_{+},\vec{y}^{T}_{+}}$. The vectors $\vec{x}^{T}_{+}$ and $\vec{y}^{T}_{+}$ are two signatures of length $k$ that satisfies the following:\\
(\romannumeral 1)$$\lim_{T\rightarrow\infty}\frac{\vec{x}^{T}_{+}}{\sqrt{T}}=\vec{A}_{\epsilon,+},\quad \lim_{T\rightarrow\infty}\frac{\vec{y}^{T}_{+}-pT1_{k}}{\sqrt{T}}=\vec{B}_{\epsilon,+}$$
(\romannumeral 2) $\vec{x}^{T}_{+}\geqslant \vec{x}^{T}$, $\vec{y}^{T}_{+}\geqslant \vec{y}^{T}$, which means the endpoints of the newly constructed line ensembles dominate the original ones. This can be achieved due to the limiting behavior of $\vec{x}^{T}_{+}$ and $\vec{y}^{T}_{+}$ and the construction of $\vec{A}_{\epsilon,+}$ and $\vec{B}_{\epsilon,+}$.
Analogously, we construct another avoiding Bernoulli line ensemble $\mathfrak{L}_{-}^{T}$ with endpoints $\vec{x}^{T}_{-}$ and $\vec{y}^{T}_{-}$ and distribution $\mathbb{P}_{Avoid,Ber}^{0,T,\vec{x}^{T}_{-},\vec{y}^{T}_{-}}$ such that $\lim_{T\rightarrow\infty}\frac{\vec{x}^{T}_{-}}{\sqrt{T}}=\vec{A}_{\epsilon,-},\quad \lim_{T\rightarrow\infty}\frac{\vec{y}^{T}_{-}-pT1_{k}}{\sqrt{T}}=\vec{B}_{\epsilon,-}$, and $\vec{x}^{T}_{-}\leqslant \vec{x}^{T}$, $\vec{y}^{T}_{-}\leqslant \vec{y}^{T}$.\\
Since now $\vec{A}_{\epsilon,+}$, $\vec{A}_{\epsilon,-}$, $\vec{B}_{\epsilon, +}$, $\vec{B}_{\epsilon,-}$ have distinct elements, we can apply the results in Proposition \ref{WeakConvDistinct} and conclude the weak convergence:
$$Z^{T}_{\vec{A}_{\epsilon,+}, \vec{B}_{\epsilon, +}}\Rightarrow \rho_{\epsilon,+}(z),\quad Z^{T}_{\vec{A}_{\epsilon,-}, \vec{B}_{\epsilon, -}}\Rightarrow \rho_{\epsilon,-}(z)$$
where $Z^{T}_{\vec{A}_{\epsilon,+},\vec{B}_{\epsilon,+}}$ and $Z^{T}_{\vec{A}_{\epsilon,-},\vec{B}_{\epsilon,-}}$ are obtained by scaling the line ensembles $\mathfrak{L}_{+}^{T}$, and $\mathfrak{L}_{-}^{T}$, $\rho_{\epsilon,+}(z)$ and $\rho_{\epsilon,-}(z)$ are densities which are obtained by plugging $\vec{A}_{\epsilon,+}$, $\vec{B}_{\epsilon,+}$ and $\vec{A}_{\epsilon,-}$, $\vec{B}_{\epsilon,-}$ into the formula of $\rho(z)$ in Proposition \ref{WeakConvDistinct}.

In order to prove the weak convergence of $Z^{T}_{\vec{a}_{0},\vec{b}_{0}}$, it is sufficient to prove for any $R=(-\infty,u_{1}]\times(-\infty,u_{2}]\times\cdots\times(-\infty,u_{k}]$, where $u_{i}\in\mathbb{R}$, we have $$\lim_{T\rightarrow\infty}(Z^{T}_{\vec{a}_{0},\vec{b}_{0}}\in R)=\int_{R}\rho_{\vec{a}_{0},\vec{b}_{0}}(z)dz$$
Actually, by Lemma \ref{MCLxy}, we can construct a sequence of probability spaces $(\Omega_{T},\mathcal{F}_{T},\mathbb{P}_{T})_{T\geqslant 1}$ such that for each $T\in\mathbb{Z}^{+}$, we have random variables $\mathfrak{L}_{+}^{T}$ and $\mathfrak{L}^{T}$ have law $\mathbb{P}_{Avoid,Ber}^{0,T,\vec{x}^{T}_{+},\vec{y}^{T}_{+}}$, and $\mathbb{P}_{Avoid,Ber}^{0,T,\vec{x}^{T},\vec{y}^{T}}$ under measure $\mathbb{P}_{T}$, respectively. Also, we have $\mathfrak{L}_{+}^{T}(i,r)\geqslant \mathfrak{L}^{T}(i,r)$ with probability $1$, where $\mathfrak{L}_{+}^{T}(i,r)$(resp., $\mathfrak{L}^{T}(i,r)$) is the value of the $i$-$th$ up-right path of $\mathfrak{L}_{+}^{T}$(resp., $\mathfrak{L}^{T}$) at $r\in\llbracket 0,T\rrbracket$. Similarly, we can construct another sequence of probability spaces $(\Omega_{T}^{\prime},\mathcal{F}_{T}^{\prime},\mathbb{Q}_{T})_{T\geqslant 1}$ such that for each $T\in\mathbb{Z}^{+}$, we have random variables $\mathfrak{L}_{-}^{T}$ and $\mathfrak{L}^{T}$ have law $\mathbb{P}_{avoid,Ber}^{0,T,\vec{x}^{T}_{-},\vec{y}^{T}_{-}}$, and $\mathbb{P}_{avoid,Ber}^{0,T,\vec{x}^{T},\vec{y}^{T}}$ under measure $\mathbb{Q}_{T}$, respectively, along with $\mathbb{Q}_{T}(\mathfrak{L}_{-}^{T}(i,r)\leqslant \mathfrak{L}^{T}(i,r), i=1,\cdots, k, r\in\llbracket 0,T\rrbracket)=1$.

Therefore, we have that under measure $\mathbb{P}_{T}$ and $\mathbb{Q}_{T}$:
$$\mathbb{P}_{T}(Z^{T}_{\vec{A}_{\epsilon,+},\vec{B}_{\epsilon,+}}\in R)\leqslant \mathbb{P}_{T}(Z^{T}_{\vec{a}_{0},\vec{b}_{0}}\in R),\quad \mathbb{Q}_{T}(Z^{T}_{\vec{A}_{\epsilon,-},\vec{B}_{\epsilon,-}}\in R)\geqslant \mathbb{Q}_{T}(Z^{T}_{\vec{a}_{0},\vec{b}_{0}}\in R)$$
Take $liminf$ and $limsup$ on both side of the first and second inequality respectively, we get
\begin{align}{\label{TwoIneq}}
\int_{R}\rho_{\epsilon,+}(z)dz\leqslant \liminf_{T\rightarrow\infty}\mathbb{P}_{T}(Z^{T}_{\vec{a}_{0},\vec{b}_{0}}\in R),\quad \int_{R}\rho_{\epsilon,-}(z)dz\geqslant \limsup_{T\rightarrow\infty}\mathbb{Q}_{T}(Z^{T}_{\vec{a}_{0},\vec{b}_{0}}\in R)
\end{align}
because of the weak convergence of $Z^{T}_{\vec{A}_{\epsilon,+},\vec{B}_{\epsilon,+}}$ and $Z^{T}_{\vec{A}_{\epsilon,-},\vec{B}_{\epsilon,-}}$. Since the distribution of $Z^{T}_{\vec{a}_{0},\vec{b}_{0}}$ under measure $\mathbb{P}_{T}$ and $\mathbb{Q}_{T}$ are the same, we can combine the above two inequalities (\ref{TwoIneq}) and get
\begin{align}{\label{Squeezing}}
\int_{R}\rho_{\epsilon,+}(z)dz\leqslant \liminf_{T\rightarrow\infty}\mathbb{P}_{T}(Z^{T}_{\vec{a}_{0},\vec{b}_{0}}\in R)\leqslant\limsup_{T\rightarrow\infty}\mathbb{P}_{T}(Z^{T}_{\vec{a}_{0},\vec{b}_{0}}\in R)\leqslant\int_{R}\rho_{\epsilon,-}(z)dz	
\end{align}

The rest of the proof establishes the following statement:
\begin{align}{\label{WeakConvConclude}}
\lim_{\epsilon\downarrow 0}\int_{R}\rho_{\epsilon,+}(z)dz=\lim_{\epsilon\downarrow 0}\int_{R}\rho_{\epsilon,-}(z)dz=\int_{R}\rho_{\vec{a}_{0},\vec{b}_{0}}(z)dz
\end{align}
and thereby concluding $$\lim_{T\rightarrow\infty}\mathbb{P}_{T}(Z^{T}_{\vec{a}_{0},\vec{b}_{0}}\in R)=\int_{R}\rho_{\vec{a}_{0},\vec{b}_{0}}(z)dz$$ by letting $\epsilon\downarrow 0$ in the inequality (\ref{Squeezing}), and we prove the weak convergence of $Z^{T}_{\vec{a}_{0},\vec{b}_{0}}$.

To prove the statement (\ref{WeakConvConclude}), first notice that
\begin{align*}
	Z_{\vec{A}_{\epsilon,+},\vec{B}_{\epsilon,+}}&=\int_{\mathbb{R}^{k}}f(\vec{a},\vec{z})g(\vec{b},\vec{z})\prod_{i=1}^{k}e^{-c_{3}(t,p)z^{2}_{i}}dz\\
	&=\int_{\mathbb{R}^{k}}\big[\epsilon^{u+v}\varphi(\vec{a}_{0},\vec{z},\vec{m})\psi(\vec{b}_{0},\vec{z},\vec{n})+o(\epsilon^{u+v})\big]\prod_{i=1}^{k}e^{-c_{3}(t,p)z^{2}_{i}}dz\\
	&=\epsilon^{u+v}\int_{\mathbb{R}^{k}}\big[\varphi(\vec{a}_{0},\vec{z},\vec{m})\psi(\vec{b}_{0},\vec{z},\vec{n})+o(1)\big]\prod_{i=1}^{k}e^{-c_{3}(t,p)z^{2}_{i}}dz\\
\end{align*}
Then, we get
$$\lim_{\epsilon\downarrow 0} \epsilon^{-(u+v)} Z_{\vec{A}_{\epsilon,+},\vec{B}_{\epsilon,+}}=\lim_{\epsilon\downarrow 0}\int_{\mathbb{R}^{k}}\big[\varphi(\vec{a}_{0},\vec{z},\vec{m})\psi(\vec{b}_{0},\vec{z},\vec{n})+o(1)\big]\prod_{i=1}^{k}e^{-c_{3}(t,p)z^{2}_{i}}dz=Z_{\vec{a}_{0},\vec{b}_{0}}$$
by definition of the constant $Z_{\vec{a}_{0},\vec{b}_{0}}$.
Therefore, we conclude
\begin{align*}
	\lim_{\epsilon\downarrow 0}\rho_{\epsilon,+}(z)=\lim_{\epsilon\downarrow 0}(\epsilon^{-(u+v)}Z_{\vec{A}_{\epsilon,+},\vec{B}_{\epsilon,+}})(\epsilon^{u}f(\vec{a},\vec{z}))(\epsilon^{v}g(\vec{b},\vec{z}))=Z_{\vec{a},\vec{b}_{0}}\varphi(\vec{a}_{0},\vec{z},\vec{m})\psi(\vec{b}_{0},\vec{z},\vec{n})=\rho_{\vec{a}_{0},\vec{b}_{0}}(z)
\end{align*}
Since $\rho_{\epsilon,+}(z)\mathbbm{1}_{R}dz\leqslant\rho_{\epsilon,+}(z)dz$ is bounded by an integrable function, by Dominated Convergence Theorem we have: $$\lim_{\epsilon\downarrow 0}\int_{R}\rho_{\epsilon,+}(z)dz=\int_{R}\rho_{\vec{a}_{0},\vec{b}_{0}}(z)dz$$ Analogously, we can get $\lim_{\epsilon\downarrow 0}\int_{R}\rho_{\epsilon,-}(z)dz=\int_{R}\rho_{\vec{a}_{0},\vec{b}_{0}}(z)dz$ and we proved the statement (\ref{WeakConvConclude}), which completes the proof.
\end{proof}

\subsection{Proof of Lemma \ref{LemmaDensity} and Lemma \ref{Integrable}}{\label{Pf2Lemma}}
