%-------------------------------------------------------------------------------------------------------------------------------------------------------------------------------------------------
%    Section 6
%
%-------------------------------------------------------------------------------------------------------------------------------------------------------------------------------------------------
\section{Lower bounds on the acceptance probability}\label{Section6}

\subsection{Proof of Lemma \ref{LemmaAP1}}\label{sect61} Throughout this section we assume the same notation as in Lemma \ref{LemmaAP1}, i.e., we assume that we have fixed $k \in \mathbb{N}$, $p \in (0,1)$, $M_1, M_2 > 0$, $\ell_{bot}: \llbracket -t_3, t_3 \rrbracket \rightarrow \mathbb{R} \cup \{ - \infty \}$, and $\vec{x}, \vec{y} \in \mathfrak{W}_{k-1}$ such that $|\Omega_{avoid}(-t_3, t_3, \vec{x}, \vec{y}, \infty, \ell_{bot})| \geq 1$. We also assume that
\begin{enumerate}
	\item $\sup_{s \in [- t_3,t_3]}\big[\ell_{bot}(s)  - ps \big]  \leq M_2 (2t_3)^{1/2}$,
	\item  $-pt_3 + M_1 (2t_3)^{1/2} \geq  x_1 \geq  x_{k-1} \geq \max\left(\ell_{bot}(-t_3), -pt_3 - M_1 (2t_3)^{1/2}\right),$
	\item $pt_3 + M_1 (2t_3)^{1/2} \geq y_1 \geq y_{k-1} \geq  \max \left( \ell_{bot}(t_3),  p t_3 - M_1(2t_3)^{1/2} \right).$
\end{enumerate}

\begin{definition}\label{TildeDef}
	We write $S = \llbracket -t_3,-t_1\rrbracket\cup \llbracket t_1,t_3\rrbracket$, and we let $\tilde{\ell}_{bot}$ denote the path which is equal to $\ell_{bot}$ on $S$ and equal to $-\infty$ on $\llbracket -t_1 + 1, t_1 - 1\rrbracket$. Additionally, define the subset $\Omega_{a,S}(\cdot)\subseteq \Omega(\cdot)$ by 
	\begin{equation}\label{OmegaASDef}\Omega_{a,S}(\cdot):=\left\{\mathfrak{B} \in \Omega(\cdot)\mid  \forall (i,s)\in\llbracket 1,k-2\rrbracket\times S, B_{i}(s)\geq B_{i+1}(s)\right\}
	\end{equation} We also denote by $\mathfrak{Q} = (Q_1,\dots,Q_{k-1})$ and $\tilde{\mathfrak{Q}} = (\tilde{Q}_1, \dots, \tilde{Q}_{k-1})$ the $\llbracket 1, k-1 \rrbracket$-indexed line ensembles which are uniformly distributed on $\Omega_{avoid}(-t_3,t_3,\vec{x},\vec{y},\ell_{bot})$ and $\Omega_{a,S}(-t_3, t_3, \vec{x}, \vec{y}, \tilde{\ell}_{bot})$ respectively, and we let $\mathbb{P}_{\mathfrak{Q}}$ and $\mathbb{P}_{\tilde{\mathfrak{Q}}}$ denote these uniform measures.
\end{definition}
In other words, $\tilde{\mathfrak{Q}}$ has the law of $k-1$ independent Bernoulli bridges that have been conditioned on not-crossing each other on the set $S$ and also staying above the graph of $\ell_{bot}$ but only on the intervals $\llbracket-t_3, -t_1\rrbracket$ and $\llbracket t_1, t_3\rrbracket$. The latter restriction means that the lines are allowed to cross on $\llbracket -t_1+1,t_1-1\rrbracket$, and  $\tilde{Q}_{k-1}$ is allowed to dip below $\ell_{bot}$ on $\llbracket -t_1+1,t_1-1\rrbracket$ as well. Essentially, the line ensemble is free on $\llbracket -t_1+1,t_1-1\rrbracket$, and avoiding on $S$.

\begin{lemma}\label{LemmaAP2} There exists $N_5 \in \mathbb{N}$ such that for $N \geq N_5$,
	\begin{equation}\label{eqn57}
		\mathbb{P}_{\tilde{\mathfrak{Q}}} \left( Z\big(  -t_1, t_1, \tilde{\mathfrak{Q}}(-t_1) , \tilde{\mathfrak{Q}}(t_1), \ell_{bot}\llbracket -t_1, t_1\rrbracket\big)\geq g    \right) \geq h,
	\end{equation}
	where the constants $g$ and $h$ are as in Lemma \ref{LemmaAP1}.
\end{lemma}
We will prove Lemma \ref{LemmaAP2} in Section \ref{sect62}. In the remainder of this section, we give the proof of Lemma \ref{LemmaAP1}. The proof begins by evaluating the Radon-Nidokum derivative between $\pr_{\mathfrak{Q}'}$ and $\pr_{\tilde{\mathfrak{Q}}'}$. We then use this Radon-Nikodym derivative to transition between $\tilde{\mathfrak{Q}}$ in Lemma \ref{LemmaAP2} which ignores $\ell_{bot}$ on $\llbracket -(t_1-1),t_1-1\rrbracket$ and $\mathfrak{Q}$ in Lemma \ref{LemmaAP1} which avoids $\ell_{bot}$ everywhere. Then we perform some calculations to achieve the desired statement in Equation (\ref{eqn60}).
\begin{proof}[Lemma \ref{LemmaAP1}]
	
	Let us denote by $\pr_{\mathfrak{Q}'}$ and $\pr_{\tilde{\mathfrak{Q}}'}$ the measures on $\llbracket 1, k-1\rrbracket$-indexed Bernoulli line ensembles $\mathfrak{Q}'$, $\tilde{\mathfrak{Q}}'$ on the set $S$ in Definition \ref{TildeDef} induced by the restrictions of the measures $\mathbb{P}_{\mathfrak{Q}}$, $\mathbb{P}_{\tilde{\mathfrak{Q}}}$ to $S$. Also let us write $\Omega_a(\cdot)$ for $\Omega_{avoid}(\cdot)$ for simplicity, and denote by $\Omega_a(S)$ the set of elements of $\Omega_{avoid}(-t_3,t_3,\tilde{\mathfrak{Q}}(-t_3),\tilde{\mathfrak{Q}}(t_3))$ restricted to $S$. We claim that the Radon-Nikodym derivative between these two restricted measures is given on elements $\mathfrak B$ of $\Omega_a(S)$ by \begin{equation}\label{propRadon}
		\frac{d\pr_{\mathfrak{Q}'}}{d\pr_{\tilde{\mathfrak{Q}}'}}\left(\mathfrak B\right) = \frac{\mathbb{P}_{\mathfrak{Q}'}(\mathfrak{B})}{\mathbb{P}_{\tilde{\mathfrak{Q}}'}(\mathfrak{B})} = (Z')^{-1} Z\left(-t_1,t_1,\mathfrak{B}\left(-t_1\right),\mathfrak{B}\left(t_1\right),\ell_{bot}\llbracket -t_1,t_1\rrbracket\right),
	\end{equation}
	with $Z' = \ex_{\tilde{\mathfrak{Q}}'}\left[Z\left(-t_1, t_1, \mathfrak B(-t_1), \mathfrak B(t_1), \ell_{bot}\llbracket -t_1, t_1\rrbracket\right)\right]$. The first equality holds simply because the measures are discrete. To prove the second equality, observe that
	\begin{equation}
		\begin{split}
			\pr_{\mathfrak{Q}'}(\mathfrak{B}) &= \frac{|\Omega_a(-t_1,t_1,\mathfrak{B}(-t_1),\mathfrak{B}(t_1),\ell_{bot}\llbracket -t_1,t_1\rrbracket)|}{|\Omega_a(-t_3,t_3,{\mathfrak{Q}}(-t_3),{\mathfrak{Q}}(t_3),\ell_{bot})|}, 
			\\
			\pr_{\tilde{\mathfrak{Q}}'}(\mathfrak{B}) &= \frac{\prod_{i = 1}^{k-1}|\Omega(-t_1,t_1,B_i(-t_1),B_i(t_1))|}{|\Omega_{a,S}(-t_3,t_3,\tilde{\mathfrak{Q}}(-t_3),\tilde{\mathfrak{Q}}(t_3),\tilde\ell_{bot})|}
		\end{split}
	\end{equation}
	These identities follow from the restriction, and the fact that the measures are uniform. Then, from Definition \ref{DefAP}, 
	\[
	Z(-t_1,t_1,\mathfrak{B}(-t_1),\mathfrak{B}(t_1),\ell_{bot}) = \frac{|\Omega_a(-t_1,t_1,\mathfrak{B}(-t_1),\mathfrak{B}(t_1),\ell_{bot}\llbracket-t_1,t_1\rrbracket)|}{\prod_{i = 1}^{k-1}|\Omega(-t_1,t_1, B_i(-t_1),B_i(t_1))|}
	\]
	and hence
	\begin{equation*}
		\begin{split}
			Z' =&\sum_{\mathfrak{B}\in \Omega_a(S)}\frac{\prod_{i = 1}^{k-1}|\Omega(-t_1,t_1,B_i(-t_1),B_i(t_1))|}{|\Omega_{a,S}(-t_3,t_3,\tilde{\mathfrak{Q}}(-t_3),\tilde{\mathfrak{Q}}(t_3),\tilde\ell_{bot})|}\cdot\frac{|\Omega_a(-t_1,t_1,\mathfrak B(-t_1),\mathfrak{B}(t_1),\ell_{bot})|}{\prod_{i = 1}^{k-1}|\Omega(-t_1,t_1, B_i(-t_1),B_i(t_1))|}=\\ 
			&\frac{\sum_{\mathfrak{B}\in\Omega_a(S)}|\Omega_a(-t_1,t_1,\mathfrak B(-t_1),\mathfrak B(t_1),\ell_{bot})|}{|\Omega_{a,S}(-t_3,t_3,\tilde{\mathfrak{Q}}(-t_3),\tilde{\mathfrak{Q}}(t_3),\tilde{\ell}_{bot})|} = \frac{|\Omega_a(-t_3,t_3,{\mathfrak{Q}}(-t_3),{\mathfrak{Q}}(t_3),\ell_{bot})|}{|\Omega_{a,S}(-t_3,t_3,\tilde{\mathfrak{Q}}(-t_3),\tilde{\mathfrak{Q}}(t_3),\tilde{\ell}_{bot})|}.
		\end{split}
	\end{equation*}
	Comparing the above identities proves the second equality in \eqref{propRadon}.
	
	Now note that $Z\left(-t_1, t_1, \mathfrak B(-t_1), \mathfrak B(t_1), \ell_{bot}\llbracket -t_1, t_1\rrbracket\right)$ is a deterministic function of $\left((\mathfrak B(-t_1), \mathfrak B(t_1)\right)$. In fact, the law of $\left((\mathfrak B(-t_1), \mathfrak B(t_1)\right)$ under $\pr_{\tilde{\mathfrak{Q}}'}$ is the same as that of $\big(\tilde{\mathfrak{Q}}(-t_1), \tilde{\mathfrak{Q}}(t_1)\big)$ by way of the restriction. It follows from Lemma \ref{LemmaAP2} that
	\begin{align*}
		Z' &= \ex_{\tilde{\mathfrak{Q}}'}\left[Z\left(-t_1, t_1, \mathfrak B(-t_1), \mathfrak B(t_1), \ell_{bot}\llbracket -t_1, t_1\rrbracket\right)\right]\\
		&= \ex_{\tilde{\mathfrak{Q}}}\left[Z\left(-t_1, t_1, \mathfrak{Q}(-t_1), \mathfrak{Q}(t_1), \ell_{bot}\llbracket -t_1, t_1\rrbracket\right)\right]\geq gh,
	\end{align*}	
	which gives us 
	\begin{equation}
		\label{Zineq} (Z')^{-1}\leq \frac{1}{gh}.
	\end{equation}
	Similarly,  the law of $\left(\mathfrak B(-t_1), \mathfrak{B}(t_1)\right)$ under $\pr_{\mathfrak{Q}'}$ is the same as that of $\left(\mathfrak{Q}(-t_1), \mathfrak{Q}(t_1)\right)$ under $\pr_{\mathfrak{Q}}$. Hence
	\begin{equation}\label{LBswap}
		\begin{split}
			&\pr_{\mathfrak{Q}}\Big(Z(-t_1, t_1, \mathfrak{Q}(-t_1), \mathfrak{Q}(t_1), \ell_{bot}\llbracket -t_1, t_1\rrbracket)\leq gh\tilde \epsilon\Big)=\\
			&\qquad\pr_{\mathfrak{Q}'}\Big(Z\left(-t_1, t_1, \mathfrak B(-t_1), \mathfrak B(t_1), \ell_{bot}\llbracket -t_1, t_1\rrbracket\right)\leq gh\tilde \epsilon\Big).
		\end{split}
	\end{equation}
	Now let us write $E=\{Z\left(-t_1, t_1, \mathfrak B(-t_1), \mathfrak B(t_1), \ell_{bot}\llbracket -t_1, t_1\rrbracket\right)\leq gh\tilde\epsilon\}\subset \Omega_a(S)$. Then according to \eqref{propRadon}, we have
	\[
	\pr_{\mathfrak{Q}'}(E)=\int_{\Omega_a(S)} \indic_E\, d\pr_{\mathfrak{Q}'} = (Z')^{-1}\int_{\Omega_a(S)}\indic_E \cdot\, Z\left(-t_1, t_1, \mathfrak B(-t_1), \mathfrak B(t_1), \ell_{bot}\llbracket -t_1, t_1\rrbracket\right)\, d\pr_{\tilde{\mathfrak{Q}}'}(\mathfrak{B}).
	\]
	From the definition of $E$, the inequality \eqref{Zineq}, and the fact that $\mathbf{1}_E \leq 1$, it follows that
	\[
	\pr_{\mathfrak{Q}'}(E)\leq (Z')^{-1}\int_{\Omega_a(S)} \indic_E\cdot\, gh\tilde{\epsilon}\, d\pr_{\tilde{\mathfrak{Q}}'} \leq \frac{1}{gh}\int_{\Omega_a(S)} gh\tilde\epsilon\, d\pr_{\tilde{\mathfrak{Q}}'}\leq \tilde{\epsilon}.
	\]
	In combination with \eqref{LBswap}, this proves \eqref{eqn60}.
	
\end{proof}
\subsection{Proof of Lemma \ref{LemmaAP2}} \label{sect62} In this section, we prove Lemma \ref{LemmaAP2}. We first state and prove two auxiliary lemmas necessary for the proof. The first lemma establishes a set of conditions under which we have the desired lower bound on the acceptance probability. 

\begin{lemma}\label{LemmaBP1} Let $\epsilon > 0$ and $V^{top} > 0$ be given such that $V^{top} > M_2 + 6 (k-1) \epsilon$. Suppose further that $\vec{a}, \vec{b} \in \mathfrak{W}_{k-1}$are such that 
	\begin{enumerate}
		\item $V^{top} (2t_2)^{1/2} \geq a_1 + p t_1 \geq a_{k-1} + pt_1 \geq (M_2 + 2 \epsilon) (2t_2)^{1/2}$;
		\item $V^{top} (2t_2)^{1/2} \geq b_1 - p t_1 \geq b_{k-1} - pt_1 \geq (M_2 + 2 \epsilon) (2t_2)^{1/2}$; 
		\item $a_i -a_{i+1} \geq 3\epsilon (2t_2)^{1/2}$ and $b_{i} - b_{i+1} \geq 3 \epsilon (2t_2)^{1/2}$ for $i = 1, \dots, k-2$.
	\end{enumerate}
	Then we can find $g = g(\epsilon, V^{top}, M_2) > 0$ and $N_6 \in \mathbb{N}$ such that for all $N \geq N_6$ we have 
	\begin{equation}\label{eqnRT}
	Z\big(  -t_1, t_1, \vec{a} ,\vec{b}, \ell_{bot}\llbracket -t_1, t_1\rrbracket\big) \geq g.
	\end{equation}
\end{lemma}

\begin{proof}
	
	Observe by the rightmost inequalities in conditions (1) and (2) in the hypothesis, as well as condition (1) in Lemma \ref{LemmaAP1}, that $\ell_{bot}$ lies a distance of at least $2\epsilon(2t_2)^{1/2}$ uniformly below the line segment connecting $a_{k-1}$ and $b_{k-1}$. Also note that (1) and (2) imply $|b_i-a_i-2pt_1| \leq (V^{top} - M_2)(2pt_1)^{1/2}$ for each $i$. Lastly noting (3), we see that the conditions of Lemma \ref{CurveSeparation} are satisfied with $C = 2\epsilon$. This implies \eqref{eqnRT}, with
	\[
	g = \left( 1 - 3\sum_{n=1}^\infty e^{-\epsilon^2n^2/2p(1-p)}\right)^{k-1}.
	\]
	
\end{proof}

\noindent The next lemma helps us derive the lower bound $h$ in \eqref{eqn57}.

\begin{lemma}\label{LemmaBP2} For any $R > 0$ we can find $V_1^t, V_1^b \geq M_2 + R$, $h_1 > 0$ and $N_7 \in \mathbb{N}$ (depending on $R$) such that if $N \geq N_7$ we have
	\begin{equation}\label{eqnRT2}
	\mathbb{P}_{\tilde{\mathfrak{Q}}} \left(  (2t_3)^{1/2} V_1^t \geq \tilde{Q}_1(\pm t_2) \mp p t_2 \geq \tilde{Q}_{k-1}(\pm t_2) \mp p t_2 \geq (2t_3)^{1/2} V_1^b  \right) \geq h_1.
	\end{equation}
	
\end{lemma}

\begin{proof}
	
	Let the constant $C$ be as in \eqref{21Cdef}, and put
	\begin{equation}\label{5.10Vb}
	V_1^b = M_1 + Ck + M_2 + R, \quad K_1 = (4r+10)V_1^b,
	\end{equation}
	\begin{equation}\label{5.10h1}
	h_1 =  \frac{2^{k/2-5}\big(1-2e^{-4/p(1-p)}\big)^{2k}}{(\pi p(1-p))^{k/2}}\,\exp\left(-\frac{2k(K_1+M_1+6)^2}{p(1-p)}\right).
	\end{equation}
	Note in particular that $V_1^b > M_2 + R$. We will choose $V_1^t > V_1^b$ in the below depending on $h_1$. We claim that for these choices of  $V_1^b, V_1^t, h_1$, and for large enough $N$, we have 
	\begin{align}
	\mathbb{P}_{\tilde{\mathfrak{Q}}}\left(\tilde{Q}_{k-1}(\pm t_2) \mp pt_2 \geq (2t_3)^{1/2}V_1^b\right) &\geq 2h_1, \label{5.10bound1}\\
	\mathbb{P}_{\tilde{\mathfrak{Q}}}\left(\tilde{Q}_1(\pm t_2) \mp pt_2 > (2t_3)^{1/2}V_1^t\right) &\leq h_1. \label{5.10bound2}
	\end{align}
	Assuming the validity of the claim, we then observe that the probability in \eqref{eqnRT2} is bounded below by $2h_1 - h_1 = h_1$, proving the lemma. We will prove \eqref{5.10bound1} and \eqref{5.10bound2} in three steps.\\
	
	\noindent\textbf{Step 1.} In this step we prove \eqref{5.10bound1} assuming results from Step 2 below. We condition on the value of $\tilde{\mathfrak{L}}$ at 0 and use the Schur Gibbs property to divide $\tilde{\mathfrak{Q}}$ into two independent line ensembles on $[-t_3,0]$ and $[0,t_3]$. Observe by Lemma \ref{MCLfg} that
	\begin{equation}\label{5.10MC}
	\mathbb{P}_{\tilde{\mathfrak{Q}}}\left(\tilde{Q}_{k-1}(\pm t_2) \mp pt_2 \geq (2t_3)^{1/2}V_1^b\right) \geq \mathbb{P}^{-t_3,t_3,\vec{x},\vec{y}}_{avoid, Ber; S}\left(\tilde{Q}_{k-1}(\pm t_2) \mp pt_2 \geq (2t_3)^{1/2}V_1^b\right).
	\end{equation}
	With $K_1$ as in \eqref{5.10Vb}, we define events
	\[
	E_{\vec{z}} = \left\{\big(\tilde{Q}_1(0),\dots,\tilde{Q}_{k-1}(0)\big) = \vec{z}\right\}, \quad X = \left\{ \vec{z}\in\mathfrak{W}_{k-1} : z_{k-1} \geq K_1(2t_3)^{1/2} \mbox { and } \mathbb{P}^{-t_3,t_3,\vec{x},\vec{y}}_{avoid,Ber; S}(E_{\vec{z}}) > 0\right\},
	\]
	and $E = \bigsqcup_{\vec{z} \in X} E_{\vec{z}}$. Note that $X$ is non-empty if $N$ is sufficiently large depending on $M_1,C,k,M_2,R$. By Lemma \ref{prob19}, we have
	\begin{equation}\label{5.10Ebound}
	\mathbb{P}^{-t_3,t_3,\vec{x},\vec{y}}_{avoid, Ber; S}(E) \geq \mathbb{P}^{-t_3,t_3,\vec{x},\vec{y}}_{avoid,Ber; S}\left(\tilde{Q}_{k-1}(0) \geq K_1(2t_3)^{1/2}\right) \geq A\exp\left(-\frac{2k(K_1+M_1+6)^2}{p(1-p)}\right)
	\end{equation}
	for sufficiently large $N$, where $A = A(p,k)$ is a constant given explicitly in \eqref{19ineq}.
	
	Now let $\tilde{Q}_i^1$ and $\tilde{Q}_i^2$ denote the restrictions of $\tilde{Q}_i$ to $[-t_3,0]$ and $[0,t_3]$ respectively for $1\leq i\leq k-1$, and write $S_1 = S\cap\llbracket -t_3,0\rrbracket$, $S_2 = S\cap\llbracket 0, t_3\rrbracket$. We observe that if $\vec{z}\in X$, then
	\begin{equation}\label{5.10split}
	\mathbb{P}^{-t_3,t_3,\vec{x},\vec{y}}_{avoid,Ber;S}\left(\tilde{Q}^1_{k-1} = \ell_1, \tilde{Q}^2_{k-1} = \ell_2 \, |\, E_{\vec{z}}\right) = \mathbb{P}^{-t_3,0,\vec{x},\vec{z}}_{avoid,Ber;S_1}(\ell_1)\cdot\mathbb{P}^{0,t_3,\vec{z},\vec{y}}_{avoid,Ber;S_2}(\ell_2).
	\end{equation}
	In Step 2, we will prove that for large $N$,
	\begin{equation}\label{5.10fourth}
	\begin{split}
	&\mathbb{P}^{-t_3,0,\vec{x},\vec{z}}_{avoid,Ber;S_1}\left(\tilde{Q}^1_{k-1}(-t_2) + pt_2 \geq (2t_3)^{1/2}V_1^b\right) \geq \frac{1}{4},\\
	&\mathbb{P}^{0,t_3,\vec{x},\vec{z}}_{avoid,Ber;S_2}\left(\tilde{Q}^2_{k-1}(t_2) - pt_{12} \geq (2t_3)^{1/2}V_1^b\right) \geq \frac{1}{4}.
	\end{split}
	\end{equation}
	Using \eqref{5.10Ebound}, \eqref{5.10split}, and \eqref{5.10fourth}, we conclude that
	\[
	\mathbb{P}^{-t_2,t_2,\vec{x},\vec{y}}_{avoid, Ber}\left(\tilde{L}_{k-1}(\pm t_{12}) \mp pt_{12} \geq (2t_2)^{1/2}V_1^b\right) \geq \frac{A}{16}\exp\left(-\frac{2k(K_1+M_1+6)^2}{p(1-p)}\right).
	\]
	In combination with \eqref{5.10MC}, this proves \eqref{5.10bound1} with $h_1$ as in \eqref{5.10h1}.\\
	
	\noindent\textbf{Step 2.} In this step, we prove the inequalities \eqref{5.10fourth} from Step 1, using Lemma \ref{LemmaHalfS4}. Let us define vectors $\vec{x}\,', \vec{z}\,', \vec{y}\,'$ by
	\begin{align*}
	x_i' &= \lfloor -pt_3 - M_1(2t_3)^{1/2}\rfloor - (i-1)\lceil C(2t_3)^{1/2}\rceil,\\
	z_i' &= \lfloor K_1(2t_3)^{1/2}\rfloor - (i-1)\lceil C(2t_3)^{1/2}\rceil,\\
	y_i' &= \lfloor pt_3 - M_1(2t_3)^{1/2}\rfloor - (i-1)\lceil C(2t_3)^{1/2}\rceil.
	\end{align*}
	Note that $x_i' \leq x_{k-1} \leq x_i$ and $x_i' - x_{i+1}' \geq C(2t_3)^{1/2}$ for $1\leq i\leq k - 1$, and likewise for $z_i',y_i'$. By Lemma \ref{MCLxy} we have
	\begin{equation}\label{5.10separate}
	\begin{split}
	&\mathbb{P}^{-t_3,0,\vec{x},\vec{z}}_{avoid,Ber;S_1}\left(\tilde{Q}^1_{k-1}(-t_2) + pt_2 \geq (2t_3)^{1/2}V_1^b\right) \geq \mathbb{P}^{-t_3,0,\vec{x}\,',\vec{z}\,'}_{avoid,Ber;S_1}\left(\tilde{Q}^1_{k-1}(-t_2) + pt_2 \geq (2t_3)^{1/2}V_1^b\right) \geq \\ 
	& \mathbb{P}^{-t_3,0,x_{k-1}',z_{k-1}'}_{Ber}\left(\ell_1(-t_2) + pt_2 \geq (2t_3)^{1/2}V_1^b\right) - \left( 1 - \mathbb{P}^{-t_3,t_3,\vec{x}\,',\vec{z}\,'}_{Ber}\left(\tilde{Q}^1_1 \geq \cdots \geq \tilde{Q}_{k-1}^1\right)\right).
	\end{split}
	\end{equation} 
	To bound the first term on the second line, first note that $x_{k-1}' \geq -pt_3 - (M_1+C(k-1))(2t_3)^{1/2}$ and $z_{k-1}' \geq K_1(2t_3)^{1/2} - C(k-1)(2t_3)^{1/2}$ for sufficiently large $N$. Let us write $\tilde{x},\tilde{z}$ for these two lower bounds. Then by Lemma \ref{LemmaHalfS4}, we have
	\begin{equation}\label{5.10third1}
	\mathbb{P}^{-t_3,0,x_{k-1}',z_{k-1}'}_{Ber}\left(\ell_1(-t_2) \geq \frac{t_2}{t_3}\,\tilde{x} + \frac{t_3-t_2}{t_3}\,\tilde{z} - (2t_3)^{1/4}\right) \geq \frac{1}{3}.
	\end{equation}  
	Moreover, as long as $N^\alpha > 2$, we have 
	\begin{equation}\label{2r+5}
	\frac{t_3-t_2}{t_3} \geq 1 - \frac{(r+2)N^\alpha}{(r+3)N^\alpha - 1} > 1-\frac{r+2}{r+5/2} = \frac{1}{2r+5}.
	\end{equation}
	It follows from our choice of $V_1^b$ and $K_1 = 2(2r+5)V_1^b$ in \eqref{5.10Vb}, as well as \eqref{2r+5}, that 
	\begin{align*}
	\frac{t_2}{t_3}\,\tilde{x} + \frac{t_3-t_2}{t_3}\,\tilde{z} - (2t_3)^{1/4} &= -pt_2 - C(k-1)(2t_3)^{1/2} - \frac{t_2}{t_3}\,M_1(2t_3)^{1/2} + \frac{t_3-t_2}{t_3}\,K_1(2t_3)^{1/2} - (2t_3)^{1/4}\\ 
	&\geq -pt_2 - Ck(2t_3)^{1/2} - M_1(2t_3)^{1/2} + \frac{1}{2r+5}\,K_1(2t_3)^{1/2}\\
	&= -pt_2 + (M_1 + Ck + 2(M_2+R))(2t_3)^{1/2}\\
	&> -pt_2 + (2t_3)^{1/2}V_1^b.
	\end{align*}
	For the first inequality, we used the fact that $t_{12}/t_2 < 1$, and we assumed that $N$ is sufficiently large so that $C(k-1)(2t_2)^{1/2} + (2t_2)^{1/4} \leq Ck(2t_2)^{1/2}$. Using \eqref{5.10third1}, we conclude that
	\begin{equation}\label{5.10third2}
	\mathbb{P}^{-t_3,0,x_{k-1}',z_{k-1}'}_{Ber}\left(\ell_1(-t_2) + pt_2 \geq (2t_3)^{1/2}V_1^b\right) \geq \frac{1}{3}.
	\end{equation}
	Since $|z_i'-x_i'-pt_2| \leq (K_1+M_1+1)(2t_2)^{1/2}$, we have by Lemma \ref{CurveSeparation} and our choice of $C$ that the second probability in the second line of \eqref{5.10separate} is bounded below by
	\[
	\left(1-3e^{-C^2/8p(1-p)}\right)^{k-1} \geq 11/12.
	\]
	It follows from \eqref{5.10separate} and \eqref{5.10third2} that
	\begin{equation*}
	\mathbb{P}^{-t_3,0,\vec{x},\vec{z}}_{avoid,Ber;S_1}\left(\tilde{Q}^1_{k-1}(-t_2) + pt_2 \geq (2t_3)^{1/2}V_1^b\right) \geq \frac{1}{3} - \frac{1}{12} = \frac{1}{4},
	\end{equation*}
	proving the first inequality in \eqref{5.10fourth}. The second inequality is proven similarly.
	\\
	
	\noindent\textbf{Step 3.} Here we prove \eqref{5.10bound2}. Let $C$ be as in Step 1, and define vectors $\vec{x}\,'', \vec{y}\,''\in\mathfrak{W}_{k-1}$ by
	\begin{align*}
	x_i'' &= \lceil -pt_3 + M_1(2t_3)^{1/2}\rceil + (k-i)\lceil C(2t_3)^{1/2}\rceil,\\
	y_i'' &= \lceil pt_3 + M_1(2t_3)^{1/2}\rceil + (k-i)\lceil C(2t_3)^{1/2}\rceil.
	\end{align*}
	Note that $x_i'' \geq x_1 \geq x_i$ and $x_i''-x_{i+1}'' \geq C(2t_3)^{1/2}$, and likewise for $y_i''$. Moreover, $\tilde{\ell}_{bot}$ lies a distance of at least $C(2t_3)^{1/2}$ uniformly below the line segment connecting $x_{k-1}''$ and $y_{k-1}''$. By Lemma \ref{MCLxy} and the Schur Gibbs property, we have
	\begin{align*}
	\mathbb{P}_{\tilde{\mathfrak{Q}}}\left(\tilde{Q}_1(\pm t_2) \mp pt_2 > (2t_3)^{1/2}V_1^t\right) &\leq \mathbb{P}^{-t_3,t_3,\vec{x}\,'',\vec{y}\,'',\infty,\tilde{\ell}_{bot}}_{avoid,Ber;S}\left(\sup_{s\in[-t_3,t_3]} \big[\tilde{Q}_1(s)-ps\big] \geq (2t_3)^{1/2}V_1^t\right)\\
	&\leq \frac{\mathbb{P}^{-t_3,t_3,x_1'',y_1''}_{Ber}\left(\sup_{s\in[-t_3,t_3]} \big[\tilde{L}_1(s)-ps\big] \geq (2t_3)^{1/2}V_1^t\right)}{\mathbb{P}^{-t_3,t_3,\vec{x}\,'',\vec{y}\,''}_{Ber}\left(\tilde{Q}_1\geq\cdots\geq\tilde{Q}_{k-1}\geq\tilde{\ell}_{bot}\right)}.
	\end{align*}
	By Lemma \ref{LemmaMinFreeS4}, since $\min(x_1'' + pt_3, \, y_1'' - pt_3) \leq (M_1+C(k-1))(2t_3)^{1/2}$, we can choose $V_1^t > V_1^b$ large enough so that the numerator is bounded above by $h_1/2$. Since $|y_i'' - x_i'' - 2pt_3| \leq 1$, our choice of $C$ and Lemma \ref{CurveSeparation} imply that the denominator is at least $11/12$. This gives an upper bound of $12/11\cdot h_1/2 < h_1/2$ in the above, proving \eqref{5.10bound2}.
	
	
\end{proof}

We are now equipped to prove Lemma \ref{LemmaAP2}.
\begin{proof}
We first introduce some notation to be used in the proof. For $\vec{c}, \vec{d} \in \mathfrak{W}_{k-1}$, let us write $\Omega(\vec{c},\vec{d}) = \Omega_{avoid}(-t_{2}, t_{2}, \vec{c}, \vec{d}, \infty, -\infty)$ and $\tilde{\Omega}(\vec{c},\vec{d}) = \Omega_{a,S}(-t_{2}, t_{2}, \vec{c}, \vec{d}, \infty, \tilde{\ell}_{bot})$, and define events
\begin{equation*}
\begin{split}
E(\vec c,\vec d, V_1^t, V_1^b)=&\left\{\tilde{\mathfrak{B}}\in \tilde{\Omega}(\vec c,\vec d)\mid (2t_3)^{1/2} V_1^t \geq \tilde{B}_1(\pm t_2) \mp p t_2 \geq \tilde{B}_{k-1}(\pm t_2) \mp p t_2 \geq (2t_3)^{1/2} V_1^b\right\}\\
A(\vec c,\vec d, V^t_1,V^b_1)=&\left\{\tilde{\mathfrak{B}}\in \tilde{\Omega}(\vec c,\vec d)\mid \tilde{B}_1\left(\pm t_1\right)\mp pt_1\leq\left(2t_3\right)^{1/2}V_1^t,\tilde{B}_{k-1}(\pm t_1)\pm pt_1 \geq \left(2t_3\right)^{1/2}V_1^b\right\}\\
B(\vec c, \vec d, \epsilon)=&\left\{\tilde{\mathfrak{B}}\in \tilde{\Omega}(\vec c,\vec d)\mid\min_{\substack{i\in\llbracket 1,k-2\rrbracket\\\varsigma\in \{-1,1\}}}\tilde{B}_i(\varsigma t_2)-\tilde{B}_{i+1}(\varsigma t_2)\geq 3\epsilon (2t_2)^{1/2}\right\}\\
C(\vec c,\vec d,V_1^t,V_1^b,\vec \epsilon)=&\bigg\{\tilde{\mathfrak{B}}\in \Omega_{a,S}(\vec c,\vec d)\mid\min_{\substack{i\in\llbracket 1,k-2\rrbracket\\\varsigma\in \{-1,1\}}}\tilde{B}_i(\varsigma t_1)-\tilde{B}_{i+1}(\varsigma t_1)\geq 3\epsilon (2t_2)^{1/2}\bigg\}
\end{split}
\end{equation*}
The event $E$ is the condition from \ref{LemmaBP1} and ensures that after scaling, the line ensemble is in the compact set $\left[V_1^b, V_1^t\right]$ at times $\pm t_2$; $A$ is the same statement for times $\pm t_1$; $B$ is the event that the ensemble is well separated at times $\pm t_2$, $C$ is the event that the Note that 
\begin{equation*}
A(\vec c,\vec d, V^t_1,V^b_1)\cap C(\vec c,\vec d,V_1^t,V_1^b,\epsilon)
\subseteq 
\left\{\mathfrak{B}\in\Omega_{a,S}(\vec c,\vec d)\mid Z\left(-t_1,t_1,\vec c,\vec d, \ell_{bot}\llbracket-t_1,t_1\rrbracket\right)>g(\epsilon,V_1^t,V_1^b)\right\}
\end{equation*}
since the conditions on the set $A\cap C$ are precisely those required by the Lemma \ref{LemmaBP1}. Moreover, by Lemma \ref{LemmaBP2},  we have $\pr_{\tilde{\mathfrak{Q}}}\left(E(\vec c,\vec d, V_1^t, V_1^b)\right)>h_1$. 
Therefore, we know 
\begin{equation*}
\pr_{\tilde{\mathfrak{Q}}}\left(A\cap C\cap E\right)=\pr_{\tilde{\mathfrak{Q}}}(A\cap C\mid E)\cdot \pr_{\tilde{\mathfrak{Q}}}(E)\geq h_1\cdot \pr(A\cap C\mid E).
\end{equation*}
and if we prove that 
$\pr_{\tilde{\mathfrak{Q}}}(A\cap C\mid E)>h_2$
for some $h_2$, then we have $\pr_{\tilde{\mathfrak{Q}}}(A\cap C\cap E)\geq h_1\cdot h_2=h$, then we have proved Lemma \ref{LemmaAP2}. In order to prove this statement, we split the proof into several steps. 
\end{proof}
\begin{proof} We first introduce some notation to be used in the proof. For $\vec{c}, \vec{d} \in \mathfrak{W}_{k-1}$, let us write $\Omega(\vec{c},\vec{d}) = \Omega_{avoid}(-t_{12}, t_{12}, \vec{c}, \vec{d}, \infty, -\infty)$ and $\tilde{\Omega}(\vec{c},\vec{d}) = \Omega_{avoid}(-t_{12}, t_{12}, \vec{c}, \vec{d}, \infty, \tilde{\ell}_{bot})$, and define events
	\begin{equation}
	\begin{split}
	&\tilde A(\vec{c}, \vec{d}) = \{\mathfrak{L} \in \tilde{\Omega}(\vec{c},\vec{d}): L_{k-1}(\pm t_{1}) \mp pt_1 \geq (M_2 + 1) (2t_2)^{1/2} \}, \\
	&\tilde B(\vec{c}, \vec{d}, \epsilon) = \left\{ \mathfrak{L} \in \tilde{\Omega}(\vec{c},\vec{d}): \min_{\substack {1\leq i\leq k-2 \\ \varsigma \in \{-1, 1\}} } \big(L_{i}(\varsigma t_1) - L_{i+1}(\varsigma t_1)\big) \geq 3\epsilon (2t_2)^{1/2} \right\}, \\
	&A(\vec{c}, \vec{d}) = \{ \mathfrak{L} \in \Omega(\vec{c},\vec{d}): L_{k-1}(\pm t_{1}) \mp pt_1 \geq (M_2 + 1) (2t_2)^{1/2} \}, \\
	&B(\vec{c}, \vec{d}, \epsilon) = \left\{ \mathfrak{L} \in \Omega(\vec{c},\vec{d}): \min_{\substack {1\leq i\leq k-2 \\ \varsigma \in \{-1, 1\}} } \big(L_{i}(\varsigma t_1) - L_{i+1}(\varsigma t_1)\big) \geq 3\epsilon (2t_2)^{1/2} \right\},\\
	& \tilde{C}(\vec{c},\vec{d},V^{top}) = \{ \mathfrak{L} \in \tilde{\Omega}(\vec{c},\vec{d}): L_{1}(\pm t_{1}) \mp pt_1 \leq V^{top} (2t_2)^{1/2} \}.
	\end{split}
	\end{equation}
	Here, $\epsilon$ and $V^{top}$ are constants which we will specify later. By Lemma \ref{LemmaBP1}, for $N$ sufficiently large we have $$\tilde A(\vec{c}, \vec{d}) \cap \tilde B(\vec{c}, \vec{d}, \epsilon) \cap \tilde{C}(\vec{c},\vec{d},V^{top}) \subset \big\{Z\big(  -t_1, t_1, \vec{a} ,\vec{b}, \ell_{bot}\llbracket -t_1, t_1\rrbracket\big) > g\big\}$$
	for some $g$ depending on $\epsilon,V^{top},M_2$. Thus we will prove that probability of the event on the left under $\mathbb{P}_{\tilde{\mathfrak L}}$ is bounded below by $h = h_1/2$, with $h_1$ as in \eqref{5.10h1}. We split the proof into several steps.\\
	
	{\bf \raggedleft Step 1.} In this step, we show that there is an $R > 0$ sufficiently large so that if $c_{k-1} + pt_{12} \geq (2t_2)^{1/2} (M_2 + R)$ and $d_{k-1} - pt_{12} \geq (2t_2)^{1/2} (M_2 + R)$, then we have 
	\begin{equation}\label{5.8step1}
	\begin{split}
	\frac{|\tilde A(\vec{c}, \vec{d})|}{|\tilde{\Omega}(\vec{c}, \vec{d})|} \geq \frac{| A(\vec{c}, \vec{d})|}{|{\Omega}(\vec{c}, \vec{d})|} \geq  \frac{9}{10} \quad \mathrm{and} \quad \frac{|\tilde{\Omega}(\vec{c}, \vec{d})|}{|{\Omega}(\vec{c}, \vec{d})|} \geq \frac{99}{100}.
	\end{split}
	\end{equation} 
	The first inequality follows immediately from Lemma \ref{MCLfg}. 
	
	For the second inequality, define the constant
	\begin{equation}\label{5.8C}
	C = \sqrt{8p(1-p)\log\frac{3}{1-(199/200)^{1/(k-1)}}}
	\end{equation}
	and vectors $\vec{c}\,', \vec{d}\,' \in \mathfrak{W}_k$ by
	\begin{align*}
	c_i' &= \lfloor -pt_{12} + (M_2+R)(2t_2)^{1/2}\rfloor - (i-1)\lceil C(2t_{12})^{1/2}\rceil,\\
	d_i' &= \lfloor pt_{12} + (M_2+R)(2t_2)^{1/2}\rfloor - (i-1)\lceil C(2t_{12})^{1/2}\rceil.
	\end{align*}
	Then by Lemma \ref{MCLxy} and the Schur Gibbs property,
	\begin{equation}\label{5.8step1split}
	\begin{split}
	\mathbb{P}^{-t_{12}, t_{12}, \vec{c}, \vec{d}}_{avoid, Ber} (A(\vec{c}, \vec{d})) &\geq \mathbb{P}^{-t_{12}, t_{12}, \vec{c}\,', \vec{d}\,'}_{avoid, Ber}(A(\vec{c}\,', \vec{d}\,'))\\ 
	&\geq \mathbb{P}^{-t_{12}, t_{12}, c_{k-1}', d_{k-1}'}_{Ber}\Big(\ell(\pm t_1) \mp pt_1 \geq (M_2+1)(2t_2)^{1/2}\Big)\\
	&\qquad - \big( 1 - \mathbb{P}^{-t_{12}, t_{12}, \vec{c}\,', \vec{d}\,'}_{Ber}(L_1 \geq \cdots \geq L_{k-1})\big).
	\end{split}
	\end{equation}
	By Lemma \ref{CurveSeparation} and our choice of $C$, $\mathbb{P}^{-t_{12}, t_{12}, \vec{c}\,', \vec{d}\,'}_{Ber}(L_1 \geq \cdots \geq L_{k-1})>199/200>19/20$ for sufficiently large $N$. Writing $z = d_{k-1}' - c_{k-1}'$, the term in the second line of \eqref{5.8step1split} is equal to
	\begin{align*}
	&\mathbb{P}^{-t_{12}, t_{12}, 0, z}_{Ber}\Big(\ell(\pm t_1) \mp pt_1 + c_{k-1}' \geq (M_2 + 1)(2t_2)^{1/2}\Big)\\
	\geq \; & \mathbb{P}^{0, 2t_{12}, 0, z}_{Ber}\Big(\ell(\pm t_1) \mp pt_1 \geq (-R+Ck+1)(2t_2)^{1/2}\Big)\\
	\geq \; & \mathbb{P}^{0, 2t_{12}, 0, z}_{Ber}\Big(\inf_{s\in[0,2t_{12}]}\big(\ell(s) - ps\big) \geq -(R-Ck-1)(2t_{12})^{1/2}\Big).
	\end{align*}
	In the second line, we used the estimate $c_{k-1}' \geq -pt_{12} + (M_2+R-Ck)(2t_2)^{1/2}$. Now by Lemma \ref{LemmaMinFreeS4}, we can choose $R$ large enough depending on $C,k,M_2,p$ so that this probability is greater than $19/20$ for sufficiently large $N$. This gives a lower bound in \eqref{5.8step1split} of $19/20 - 1/20 = 9/10$, proving the second inequality in \eqref{5.8step1}.
	
	We prove the third inequality in (\ref{5.8step1})similarly. Note that since $\ell_{bot}(s) \leq ps + M_2(2t_2)^{1/2}$ on $[-t_2,t_2]$ by assumption, we have
	\begin{equation} \label{sigma}
	\begin{split}
	\frac{|\tilde{\Omega}(\vec{c},\vec{d})|}{|\Omega(\vec{c},\vec{d})|} &\geq \mathbb{P}^{-t_{12}, t_{12}, \vec{c},\vec{d}}_{avoid, Ber}\Big(\inf_{s\in[-t_{12}, t_{12}]} \big(L_{k-1}(s) - ps\big) \geq M_2(2t_2)^{1/2}\Big)\\
	&\geq \mathbb{P}^{-t_{12}, t_{12}, \vec{c}\,',\vec{d}\,'}_{avoid, Ber}\Big(\inf_{s\in[-t_{12}, t_{12}]} \big(L_{k-1}(s) - ps\big) \geq M_2(2t_2)^{1/2}\Big)\\
	&\geq \mathbb{P}^{0, 2t_{12}, 0, z}_{Ber}\Big(\inf_{s\in[0, 2t_{12}]} \big(\ell(s) - ps\big) \geq -(R-Ck)(2t_2)^{1/2}\Big)\\
	&\qquad - \big(1 - \mathbb{P}^{-t_{12},t_{12},\vec{c}\,',\vec{d}\,'}_{Ber}(L_1\geq \cdots \geq L_{k-1})\big).
	\end{split}
	\end{equation}
	We enlarge $R$ if necessary so that the probability in the third line of \eqref{sigma} is $>199/200$ by Lemma \ref{LemmaMinFreeS4}, and \ref{CurveSeparation} implies as above that the expression in the last line of \eqref{sigma} is $>-1/200$. This gives us a lower bound of $199/200 - 1/200 = 99/100$ as desired.\\
	
	{\bf \raggedleft Step 2.} With $R$ fixed from Step 1, let $V_1^t, V_1^b$ and $h_1$ be as in Lemma \ref{LemmaBP2}  for this choice of $R$. Define the event
	$$E = \{ \vec{c}, \vec{d} \in \mathfrak{W}_{k-1}: (2t_2)^{1/2} V_1^t \geq \max(c_1 + p t_{12}, d_1 - pt_{12}) \mbox{ and }\min(c_{k-1} + p t_{12}, d_{k-1} - pt_{12})  \geq (2t_2)^{1/2} V_1^b \}.$$
	We show in this step that there exists $V^{top} \geq M_2 + 6(k-1)$ such that for all $(\vec{c}, \vec{d}) \in E$, we have
	\begin{equation}\label{5.8step2}
	\frac{|\tilde{C}(\vec{c}, \vec{d}, V^{top})|}{|\tilde{\Omega}(\vec{c},\vec{d})|} \geq  \frac{9}{10}.
	\end{equation}
	Let $C$ be as in \eqref{5.8C}, and define $\vec{c}\,'', \vec{d}\,'' \in \mathfrak{W}_{k-1}$ by
	\begin{align*}
	c_i'' &= \lceil -pt_{12} + (2t_2)^{1/2} V_1^t\rceil + (k-1-i)\lceil C(2t_{12})^{1/2}\rceil,\\
	d_i'' &= \lceil pt_{12} + (2t_2)^{1/2} V_1^t\rceil + (k-1-i)\lceil C(2t_{12})^{1/2}\rceil.
	\end{align*}
	Then $c_i'' \geq c_1 \geq c_i$ and $c_i'' - c_{i+1}'' \geq C(2t_2)^{1/2}$ for each $i$, and likewise for $d_i''$. Furthermore, since $V_1^b \geq M_2+R$, we see that $\tilde{\ell}_{bot}$ lies a distance of at least $R(2t_2)^{1/2}$ uniformly below the line segment connecting $c_{k-1}''$ and $d_{k-1}''$. By construction, $R>C$. By Lemma \ref{MCLxy}, we have
	\begin{equation}\label{5.8step3split}
	\begin{split}
	\frac{|\tilde{C}(\vec{c}, \vec{d}, V^{top})|}{|\tilde{\Omega}(\vec{c},\vec{d})|} &\geq \mathbb{P}^{-t_{12},t_{12},\vec{c}\,'', \vec{d}\,'', \infty,\tilde{\ell}_{bot}}_{avoid, Ber}\Big(L_1(\pm t_1) \mp pt_1 \leq V^{top}(2t_2)^{1/2}\Big)\\
	&\geq \mathbb{P}^{-t_{12},t_{12},\vec{c}\,'', \vec{d}\,'', \infty,\tilde{\ell}_{bot}}_{avoid, Ber}\Big(\sup_{s\in[-t_{12},t_{12}]}\big(L_1(s) - ps\big) \leq V^{top}(2t_2)^{1/2}\Big)\\
	&\geq \mathbb{P}^{0,2t_{12},0,z'}_{Ber}\Big(\sup_{s\in[-t_{12},t_{12}]}\big(\ell(s) - ps\big) \leq (V^{top}-V_1^t-Ck)(2t_2)^{1/2}\Big)\\ 
	&\qquad - \big(1 - \mathbb{P}^{-t_{12},t_{12},\vec{c}\,'', \vec{d}\,'', \infty,\tilde{\ell}_{bot}}_{avoid, Ber}\big(L_1\geq\cdots\geq L_{k-1}\geq\tilde{\ell}_{bot}\big)\big).
	\end{split}
	\end{equation}
	In the last line, we have used the Schur Gibbs property and written $z' = d_1'' - c_1''$. We also used the fact that $c_1'' \leq -pt_{12} + (V_1^t + Ck)(2t_2)^{1/2}$. By Lemma \ref{LemmaMinFreeS4}, we can find $V^{top}$ large enough depending on $V_1^t,C,k,p$ so that the probability in the third line of \eqref{5.8step3split} is at least $19/20$ for sufficiently large $N$. On the other hand, the above observations regarding $\vec{c}\,''$, $\vec{d}\,''$, and $\tilde{\ell}_{bot}$, as well as the fact that $|d_1'' - c_1'' - 2pt_{12}| \leq 1$, allow us to conclude from Lemma \ref{CurveSeparation} that the probability in the last line of \eqref{5.8step3split} is at least $19/20$ for sufficiently large $N$. This gives a lower bound of $19/20 - 1/20 = 9/10$ in \eqref{5.8step3split} as desired.\\
	
	{\bf \raggedleft Step 3.} In this step, we show that with $V_1^t$ and $V_1^b$ as in Step 2, there is an $\epsilon > 0$ sufficiently small such that for $(\vec{c}, \vec{d}) \in E$ we have
	\begin{equation}\label{LemmaBP2Step3}
	\frac{|B(\vec{c}, \vec{d}, \epsilon)|}{|\Omega(\vec{c}, \vec{d})|} \geq  \frac{9}{10} \quad \mathrm{and} \quad \frac{|\tilde{B}(\vec{c}, \vec{d}, \epsilon)|}{|\tilde{\Omega}(\vec{c}, \vec{d})|} \geq  \frac{4}{5}.
	\end{equation}
	We prove this inequality using Lemma \ref{prob 20}. In order to apply this result, we approximate $t_1$ in the form $s\cdot 2t_{12}$, for $s\in(0,1)$.
	
	Observe from \eqref{t12def} and the definitions of $t_1,t_2$ that the ratio $\frac{t_1}{2t_{12}}$ depends on $N$ and satisfies the inequality
	\[
	\frac{(r+1)N^\alpha-1}{(2r+3)N^\alpha}\leq \frac{t_1}{2t_{12}} \leq \frac{(r+1)N^\alpha}{(2r+3)N^\alpha-4}.
	\]
	We put $s = \frac{r+1}{2r+3}$, and observe that $s < 1/2$. Also observe that
	\[
	(2r+3)N^\alpha - 4\leq 2t_{12}\leq (2r+3)N^\alpha,
	\]
	and thus
	\[
	(r+1)N^\alpha \geq s\cdot 2t_{12} \geq (r+1)N^\alpha - 4s \geq t_1 - 2.
	\]
	It follows that $t_1 - 2 \leq \floor{s\cdot 2t_{12}} \leq t_1$, from which we conclude that
	\begin{equation}\label{Lt1Set}
	0 \leq L_i(t_1)-L_i(s\cdot 2t_{12}) \leq 2.
	\end{equation}
	Now applying Lemma \ref{prob 20} with $M_1,M_2=\max(V_1^t,V_1^b)$, we obtain $N_0$ and $\delta>0$ such that if $N \geq N_0$, then
	\[
	\pr^{-t_{12},t_{12},\vec c, \vec d}_{avoid,Ber}\Big(\min_{1\leq i\leq k-1} \big(L_i(s\cdot 2t_{12})-L_{i+1}(s\cdot 2t_{12})\big) < \delta(2t_{12})^{1/2}\Big)<\frac{1}{20}.
	\]
	Together with \eqref{Lt1Set} and the fact that $t_2/4 < t_1 < t_{12}$, this implies that
	\begin{equation}\label{step3delta/2}
	\pr^{-t_{12},t_{12},\vec c, \vec d}_{avoid,Ber}\left(\min_{1\leq i\leq k-1} \big(L_i(t_1)-L_{i+1}(t_1)\big)<(\delta/2)(2t_2)^{1/2}-2\right)<\frac{1}{20}
	\end{equation}
	for $N\geq N_0$. Now we observe that if $N \geq \big(\frac{1+1/\delta}{r+2}\big)^{2/\alpha}$, then $(\delta/4)(2t_2)^{1/2} \leq (\delta/2)(2t_2)^{1/2} - 1$. Thus for all sufficiently large $N$, we have
	\[
	\pr^{-t_{12},t_{12},\vec c, \vec d}_{avoid,Ber}\Big(\min_{1\leq i\leq k-1} \big(L_i(t_1)-L_{i+1}(t_1)\big)<(\delta/4)(2t_2)^{1/2}\Big)<\frac{1}{20}.
	\]
	A similar argument gives us a $\tilde{\delta}>0$ such that
	\[
	\pr^{-t_{12},t_{12},\vec c, \vec d}_{avoid,Ber}\Big(\min_{1\leq i\leq k-2} \big(L_i(-t_1)-L_{i+1}(-t_1)\big)<(\tilde{\delta}/4)(2t_2)^{1/2}\Big)<\frac{1}{20}
	\]
	for large enough $N$. Then putting $\epsilon = \min(\delta,\tilde{\delta})/12$, we find
	\[
	\pr^{-t_{12},t_{12},\vec c, \vec d}_{avoid,Ber}\Big(\min_{1\leq i\leq k-2} \big(L_i(\pm t_1)-L_{i+1}(\pm t_1)\big)<3\epsilon(2t_2)^{1/2}\Big) < \frac{1}{20} + \frac{1}{20} = \frac{1}{10},
	\] 
	and the first inequality in \eqref{LemmaBP2Step3} follows.
	
	For the second inequality in \eqref{LemmaBP2Step3}, we observe via the second inequality in \eqref{5.8step1} that
	\begin{align*}
	\frac{|\tilde{B}(\vec{c},\vec{d},\epsilon)|}{|\tilde{\Omega}(\vec{c},\vec{d})|} &\geq \frac{|B(\vec{c},\vec{d},\epsilon)\cap\tilde{\Omega}(\vec{c},\vec{d})|}{|\Omega(\vec{c},\vec{d})|} \geq \frac{|B(\vec{c},\vec{d},\epsilon)|}{|\Omega(\vec{c},\vec{d})|} -\frac{|\tilde{\Omega}^c(\vec{c},\vec{d})|}{|\Omega(\vec{c},\vec{d})|} \geq \frac{9}{10} - \frac{1}{100} > \frac{4}{5}.
	\end{align*} \\
	
	{\bf \raggedleft Step 4.} In this last step, we complete the proof of the lemma. Let $g=g(\epsilon,V^{top},M_2)$ be as in Lemma \ref{LemmaBP1} for the choices of $\epsilon,V^{top}$ in Steps 2 and 3, and let $h = h_1/2$, with $h_1$ as in Step 2. If $(\vec{c},\vec{d})\in E$, then it follows from \eqref{5.8step1}, \eqref{5.8step2}, and \eqref{LemmaBP2Step3} that
	\begin{align*}
	\frac{|\tilde{B}^c(\vec{c}, \vec{d}, \epsilon) \cup \tilde{A}^c(\vec{c}, \vec{d}) \cup \tilde{C}^c(\vec{c}, \vec{d}, V^{top}) |}{|\tilde{\Omega}(\vec{c}, \vec{d})|} &\leq \frac{|\tilde{B}^c(\vec{c}, \vec{d}, \epsilon)|}{|\tilde{\Omega}(\vec{c}, \vec{d})|} + \frac{| \tilde{A}^c(\vec{c}, \vec{d}) |}{|\tilde{\Omega}(\vec{c}, \vec{d})|} + \frac{|\tilde{C}^c(\vec{c}, \vec{d}, V^{top}) |}{|\tilde{\Omega}(\vec{c}, \vec{d})|}\\
	&\leq \frac{1}{5} + \frac{1}{10} + \frac{1}{10} < \frac{1}{2}.
	\end{align*}
	In particular,
	\begin{equation}
	\frac{|\tilde{B}(\vec{c}, \vec{d}, \epsilon) \cap \tilde{A}(\vec{c}, \vec{d}) \cap \tilde{C}(\vec{c}, \vec{d}, V^{top}) |}{|\tilde{\Omega}(\vec{c}, \vec{d})|} \geq  \frac{1}{2},
	\end{equation}
	for all large $N$, independent of $\vec{c},\vec{d}$. That is, we have for all $(\vec{c}, \vec{d}) \in E$ that
	$$\mathbb{P}_{avoid, Ber}^{-t_{12}, t_{12}, \vec{c}, \vec{d}, \infty, \tilde{\ell}_{bot}} ( G) \geq \frac{1}{2},$$
	where $G$ is the event that
	
	\begin{enumerate}
		\item $V^{top} (2t_2)^{1/2} \geq L_1(-t_1) + p t_1 \geq L_{k-1}(-t_1) + pt_1 \geq (M_2 + 1) (2t_2)^{1/2}$;
		\item $V^{top} (2t_2)^{1/2} \geq L_1(t_1) - p t_1 \geq L_{k-1}(t_1) - pt_1 \geq (M_2 + 1) (2t_2)^{1/2}$; 
		\item $L_i(-t_1) -L_{i+1}(-t_1) \geq 3\epsilon (2t_2)^{1/2}$ and $L_i(t_1) -L_{i+1}(t_1)  \geq 3 \epsilon (2t_2)^{1/2}$ for $i = 1, \dots, k-2$.
	\end{enumerate}
	Let $F$ denote the event appearing in \eqref{eqnRT2}. Then we can write $F = \bigsqcup_{(\vec{c},\vec{d})\in E} F(\vec{c},\vec{d})$, where $F(\vec{c},\vec{d})$ is the event that $\tilde{L}(-t_{12}) = \vec{c}$, $\tilde{L}(t_{12}) = \vec{d}$. It follows from Lemma \ref{LemmaBP2} that
	\begin{align*}
	&\mathbb{P}_{\tilde{\mathfrak{L}}}(G) \geq \mathbb{P}_{\tilde{\mathfrak{L}}}(G\cap F) = \sum_{(\vec{c},\vec{d})\in E} \mathbb{P}_{\tilde{\mathfrak{L}}}(F(\vec{c},\vec{d})) \mathbb{P}_{\tilde{\mathfrak{L}}}(G\,|\,F(\vec{c},\vec{d})) =\\
	&\sum_{(\vec{c},\vec{d})\in E} \mathbb{P}_{\tilde{\mathfrak{L}}}(F(\vec{c},\vec{d})) \mathbb{P}^{-t_{12},t_{12},\vec{c},\vec{d},\infty,\tilde{\ell}_{bot}}_{avoid,Ber}(G) \geq \frac{1}{2}\sum_{(\vec{c},\vec{d})\in E} \mathbb{P}_{\tilde{\mathfrak{L}}}(F(\vec{c},\vec{d})) = \frac{1}{2}\mathbb{P}_{\tilde{\mathfrak{L}}}(F) \geq \frac{1}{2}h_1
	\end{align*}
	for large $N$. Now Lemma \ref{LemmaBP1} implies \eqref{eqn57}, completing the proof.
\end{proof}