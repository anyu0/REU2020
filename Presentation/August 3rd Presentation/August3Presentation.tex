\documentclass[9pt,t,dvipsnames]{beamer}
\usetheme{Madrid}
\usepackage{amsthm, hhline}
\usepackage{stmaryrd, mathtools, stmaryrd}
\usepackage{amsmath, amssymb, graphicx,array}
\usepackage{mathtools}
\DeclareMathOperator{\lcm}{lcm}
\usepackage{booktabs,comment,bbm}
\newcommand{\Z}{\mathbb{Z}}
\newcommand{\N}{\mathbb{N}}
\usepackage[english]{babel}
\usepackage[utf8x]{inputenc}
\usepackage{xcolor}
\usepackage[export]{adjustbox}
\setbeamerfont{title in sidebar}{size=\fontsize{2}{4}\selectfont}
\setbeamerfont{author in sidebar}{size=\fontsize{2}{4}\selectfont}
\setbeamerfont{section in sidebar}{size=\fontsize{2}{4}\selectfont}
\setbeamerfont{subsection in sidebar}{size=\fontsize{2}{4}\selectfont}
\newtheorem{proposition}[theorem]{Proposition}
\fontsize{6pt}{7.2}
\DeclareMathOperator{\ex}{\mathbb{E}}
\DeclareMathOperator{\pr}{\mathbb{P}}
\DeclareMathOperator{\indic}{\mathbbm{1}}
\DeclareMathOperator{\cov}{cov}
\DeclareMathOperator{\var}{var}
\DeclareMathOperator{\sgn}{sgn}
\DeclarePairedDelimiter\ceil{\lceil}{\rceil}
\DeclarePairedDelimiter\floor{\lfloor}{\rfloor}
\DeclarePairedDelimiter\abs{\lvert}{\rvert}


\title{Asymptotics of Bernoulli Line Ensembles}

\author[Fang, Fesser, Serio, Teitler, and Wang]{Xiang Fang, Lukas Fesser, Christian Serio, Carson Teitler, and Angela Wang \\
Advisor: Evgeni Dimitrov\\
Graduate Student Assistant: Weitao Zhu}

\institute[Columbia]{Columbia University REU}

\begin{document}
	
	\begin{frame}
		\maketitle
	\end{frame}


\section{Introduction (5-6 min)}

\begin{frame}{The Gaussian universality class}
Let $\{X_i\}$ be a sequence of i.i.d. random variables, s.t. $\mathbb{E}[X_1] = \mu$, $Var(X_1^2) = \sigma^2$. Let $S_n = \sum_{i = 1}^n X_i$:

\bigskip

\begin{itemize}
\item \textbf{Law of Large Numbers:} $\frac{S_n}{n} \rightarrow \mu$ as $n \rightarrow \infty$ almost surely

\bigskip

\item \textbf{Central Limit Theorem:} $\frac{S_n - n \mu}{\sqrt{n}} \rightarrow N(0, \sigma^2)$ as $n \rightarrow \infty$

\bigskip

\item \textbf{Donsker's Theorem:} Let $S(x) = S_k$ if $x = k$ and linearly interpolate for $x \in [0, n]$ Let $\mu = 0$ and $\sigma = 1$. Then $\frac{S(n\cdot)}{\sqrt{n}} \in C([0, 1])$ and $\frac{S(n\cdot)}{\sqrt{n}} \rightarrow B(\cdot)$, where $B$ denotes a standard Brownian Motion.
\end{itemize}
\begin{figure}
\includegraphics[height=0.3\textheight]{graphics/Gaussian.png}
\caption{An example of a Bernoulli random walk and a Brownian Motion}
\end{figure}

\end{frame}

\begin{frame}{Multiple Random Walks}
Consider again Bernoulli random walks and Brownian Motion. We now increase the number of (non-intersecting) walkers:
\begin{figure}
	\includegraphics[height=0.2\textheight]{graphics/MultipleBernoulli.png}
	\caption{Multiple Avoiding Bernoulli Random Walks}
\end{figure}
When dealing with a family of avoiding Brownian Motions, we speak of Dyson Brownian Motion:
\begin{figure}
\includegraphics[height=0.2\textheight]{graphics/DysonBrownian.png}
	\caption{Dyson Brownian Motion}
\end{figure}
\end{frame}

\begin{frame}{Airy Line Ensemble}
As $N\rightarrow \infty$, the rescaled walks converge in distribution, uniformly over compact sets of $\mathbb{N}\times \mathbb{R}$, to the Airy line ensemble, $\mathcal{A}$, and the top curve converges to Airy process, $\mathcal{A}_1$.
\begin{figure}
	\includegraphics[height=0.25\textheight]{graphics/airy.png}
	\caption{Multiple Dyson Brownian walks}
\end{figure}
 Increasing the number of paths pushes us outside of the Gaussian universality class and into Kardar-Parisi-Zhang (KPZ) universality class.
\end{frame}

\begin{frame}{Open Question}
Show that any random walks with generic initial conditions convergence to the Airy line ensemble. 
\end{frame}


\section{Convergence to Airy Line Ensemble (6-7 min)}

\begin{frame}{Convergence to the Airy Line Ensemble}
	Two sufficient conditions:\begin{enumerate}
		\item Finite dimensional distribution convergence
		\item Tightness, or the existence of weak subsequential limits.
	\end{enumerate}
We focused on tightness, which requires a maximum, minimum, and conditions on the Modulus of Continuity
\begin{figure}
	\includegraphics[height=0.55\textheight]{graphics/ModulusCont.jpg}
	\caption{The Modulus of Continuity}
\end{figure}

\end{frame}

\begin{frame}{Our Result}
\begin{theorem}With $L_1^N$ being the top curve in a Bernoulli Line Ensemble $p\in (0,1)$, and $\lambda, \alpha>0$, if for all $n\in \mathbb{Z}$,
\[\lim_{N\to\infty}P(L_1^{N}(nN^{\alpha}) - nN^{\alpha} p + \lambda n^2 N^{\alpha/2} \leq N^{\alpha/2} x) \to F_{TW}(x)\]
then the Line Ensemble is tight.
\end{theorem}
If the one-point marginal probabilities at integer times weakly converge to the Tracy Widom distribution then the Line Ensemble is tight.
\begin{figure}
	\includegraphics[width=0.75\textwidth]{graphics/ConvToTW.jpg}
	\caption{Integer time points of top line}
\end{figure}
\end{frame}

\begin{frame}{Improvements}
[Duavergne, Nica, \& Virag, 2019] - tightness assuming finite dimensional convergence to the Airy Line Ensemble. 

We achieve the same result with much less restrictive assumptions
\newline\newline\noindent
[Unsure of Image Choice]
\end{frame}


\section{Section of Paper (7-9 min)}

\begin{frame} {History of the line ensembles}
	Arguments in this paper are inspired by 
	\begin{enumerate}
		\item \textit{Brownian Gibbs property for Airy line ensembles} and \textit{KPZ line ensemble}[Corwin-Hammond ‘11, ‘13], which address the issues of {\color{red}continuous} line ensembles
		\item \textit{Transversal fluctuations of the ASEP, stochastic six vertex model, and Hall-Littlewood line ensembles} [Corwin-Dimitrov ‘17], which consider similar questions in a {\color{red}discrete} setting 
	\end{enumerate}
\end{frame}
\begin{frame}{Problem Description}
	Recall that to show tightness, we want to control \begin{enumerate}
		\item \begin{center}
			min
		\end{center}
		\item 
		\begin{center}
			max
		\end{center}
		\item 
		\begin{center}
			modulus of continuity of the line ensembles
		\end{center}
	\end{enumerate}
	We claim that for the {\color{blue}\textbf{top}} curve of our line ensemble to have a {\color{red}\textbf{parabolic shift}}, the {\color{blue}\textbf{bottom}} curve cannot dip too low, i.e. for any $r, \epsilon > 0$, there exist $R, M > 0$ such that for $N$ large enough,
	$$P( \max_{[r, R]} L_k(sN^{\alpha}) - psN^{\alpha} \leq -MN^{\alpha} ) < \epsilon$$ 
	(perhaps insert a picture)
\end{frame}
\begin{frame}
Proof  (mention monotone coupling lemmas somewhere ) - say MC with picture
2min
\end{frame}

\begin{frame}
Proof  (mention strong coupling somewhere) - say SC with picture 
L = Bernoulli bridge B is a Brownian bridge with variance. There is a probability space such that $P( sup \abs*{L - B} \geq k (\log N)^2) < \epsilon$. This is a comparison that allows for example to compare the modulus of continuity of the two. [Dimitrov-Wu ‘19]
2 min
\end{frame}

\begin{frame}{Controlling the minimum: pinning the bottom curve}
	
	\begin{lemma}[------]
		For any $r,\epsilon > 0$, there exists $R>r$ and a constant $M>0$ so that for large $N$,
		\[
		\mathbb{P}\Big(\max_{x\in[r,R]} \big(L_k^N(xN^{2/3}) - pxN^{2/3}\big) \leq -MN^{1/3}\Big) < \epsilon.
		\]
		The same is true of the maximum on $[-R,-r]$.
	\end{lemma}
	\[
	\includegraphics[scale=0.1]{graphics/min.png}
	\]
	
	\begin{itemize}
		
		\item Couple with a Brownian bridge: if ``pinned" at two points $>r$ and $-r$, it cannot be low on scale $N^{1/3}$ on $[-r,r]$.
		
	\end{itemize}
	
	
	
\end{frame}

\begin{frame}{Proving the pinning lemma}
	
	\begin{itemize}
		
		\item Recall our assumption:
		\[
		\mathbb{P}\Big(\textcolor{red}{L_1^N}(nN^{2/3}) - pnN^{2/3} + \textcolor{red}{\lambda n^2} N^{1/3} \leq xN^{1/3}\Big) \underset{N\to\infty}\longrightarrow F_{TW}(x).
		\]
		
		\item The top curve looks like a \textcolor{red}{parabola} with an affine shift on large scales.
		\[
		\includegraphics[scale=0.12]{graphics/parabola.png}
		\]
		
		\item Two curves: if $L_2^N$ is low on $[r,R]$, $L_1^N$ looks like a free Brownian bridge.
		
		\[
		\textcolor{ForestGreen}{\lambda\Big(\frac{R^2+r^2}{2}\Big)} - \textcolor{Fuchsia}{\lambda\Big(\frac{R+r}{2}\Big)^2} = \lambda\frac{R^2+r^2}{4} - \frac{\lambda rR}{2} = \textcolor{red}{O(R^2)}.
		\]
		
		\item For large $R$, the top curve would be far from the parabola at the midpoint!
		
	\end{itemize}

	
	
\end{frame}


\end{document}