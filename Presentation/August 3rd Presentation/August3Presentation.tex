\documentclass[9pt,t,dvipsnames]{beamer}
\usetheme{Madrid}
\usepackage{amsthm, hhline}
\usepackage{stmaryrd, mathtools, stmaryrd}
\usepackage{amsmath, amssymb, graphicx,array}
\usepackage{mathtools}
\DeclareMathOperator{\lcm}{lcm}
\usepackage{booktabs,comment,bbm}
\newcommand{\Z}{\mathbb{Z}}
\newcommand{\N}{\mathbb{N}}
\usepackage[english]{babel}
\usepackage[utf8x]{inputenc}
\usepackage{xcolor}
\usepackage[export]{adjustbox}
\setbeamerfont{title in sidebar}{size=\fontsize{2}{4}\selectfont}
\setbeamerfont{author in sidebar}{size=\fontsize{2}{4}\selectfont}
\setbeamerfont{section in sidebar}{size=\fontsize{2}{4}\selectfont}
\setbeamerfont{subsection in sidebar}{size=\fontsize{2}{4}\selectfont}
\newtheorem{proposition}[theorem]{Proposition}
\fontsize{6pt}{7.2}
\DeclareMathOperator{\ex}{\mathbb{E}}
\DeclareMathOperator{\pr}{\mathbb{P}}
\DeclareMathOperator{\indic}{\mathbbm{1}}
\DeclareMathOperator{\cov}{cov}
\DeclareMathOperator{\var}{var}
\DeclareMathOperator{\sgn}{sgn}
\DeclarePairedDelimiter\ceil{\lceil}{\rceil}
\DeclarePairedDelimiter\floor{\lfloor}{\rfloor}
\DeclarePairedDelimiter\abs{\lvert}{\rvert}


\title{Asymptotics of Bernoulli Line Ensembles}

\author[Fang, Fesser, Serio, Teitler, and Wang]{Xiang Fang, Lukas Fesser, Christian Serio, Carson Teitler, and Angela Wang \\
Graduate Student Assistant: Weitao Zhu\\
Advisor: Evgeni Dimitrov}

\institute[Columbia]{Columbia University REU}

\begin{document}
	
	\begin{frame}
		\maketitle
	\end{frame}


\section{Introduction (5-6 min)}

\begin{frame}{The Gaussian universality class}

Let $\{X_i\}$ be a sequence of independent identically distributed random variables with mean $\mu$ and variance $\sigma^2$. Let $S_n = X_1 + \cdots + X_n$.

\bigskip

\begin{itemize}
\item \textbf{Law of Large Numbers:} $\dfrac{S_n}{n} \longrightarrow \mu$ as $n \rightarrow \infty$ almost surely.

\bigskip

\item \textbf{Central Limit Theorem:} $\dfrac{S_n - n\mu}{\sigma\sqrt{n}} \implies \mathcal{N}(0, 1)$ as $n \rightarrow \infty$.

\bigskip

\item \textbf{Donsker's Theorem:} For $t\in[0,1]$, let $W^{(n)}(t) = \dfrac{S_{nt}-nt\mu}{\sigma\sqrt{n}}$ if $nt\in\mathbb{N}$, and linearly interpolate. Then $W^{(n)} \in C([0, 1])$ and $W^{(n)} \implies W$ as $n\to\infty$, a standard Brownian motion on $[0,1]$.
\end{itemize}
\begin{figure}
\includegraphics[height=0.25\textheight]{graphics/Gaussian.png}
\caption{An example of a random walk and a Brownian motion.}
\end{figure}

\end{frame}

\begin{frame}{Multiple random walks}
If $S_{n+1} - S_n \in \{0, 1\}$, then $\{S_n\}_{n=1}^\infty$ is a \textcolor{red}{\textit{Bernoulli random walk}}. An \textcolor{red}{\textit{avoiding Bernoulli line ensemble}} $\mathfrak{L} = (L_1,\dots,L_k)$ consists of $k$ avoiding Bernoulli random walks on an interval $[T_0,T_1]$, such that $L_1(s) \geq L_2(s) \geq \cdots \geq L_k(s)$ for $s\in[T_0,T_1]$.
\[
	\includegraphics[height=0.3\textheight]{graphics/MultipleBernoulli.png}
\]
When dealing with a family of avoiding Brownian Motions, we speak of Dyson Brownian Motion:
\begin{figure}
\includegraphics[height=0.2\textheight]{graphics/DysonBrownian.png}
	\caption{Dyson Brownian Motion}
\end{figure}
\end{frame}

\begin{frame}{Airy Line Ensemble}
As $k \to \infty$, $k$ avoiding random walks are conjectured to converge to the \textcolor{red}{Airy line ensemble}, $\mathcal{A}$, and the top curve to the \textcolor{red}{Airy process}, $\mathcal{A}_1$.
\begin{figure}
	\includegraphics[height=0.25\textheight]{graphics/airy.png}
	\caption{Multiple Dyson Brownian walks}
\end{figure}

\begin{itemize}
	
	\item Increasing the number of paths pushes us outside of the \textcolor{blue}{\textit{Gaussian universality class}} and into \textcolor{blue}{\textit{Kardar-Parisi-Zhang (KPZ) universality class}} .
	
	\item Open problem: Show that ``generic" random walks with ``generic" initial conditions converge to the Airy line ensemble. 
	
	\item We treat this problem for Bernoulli random walks; the proof is only known if all walks start from 0.
	
\end{itemize}
\end{frame}


\section{Convergence to Airy Line Ensemble (6-7 min)}

\begin{frame}{Convergence to the Airy Line Ensemble}
	
	Two sufficient conditions for convergence in distribution:
	\begin{itemize}
		
			\item \textit{Finite dimensional} convergence -- difficult, requires exact algebraic formulas
			
			\item \textit{Tightness} (existence of weak subsequential limits) -- easier, more qualitative/analytic
	\end{itemize}

	We focused on tightness, which we prove by controlling the maximum, the minimum, and the modulus of continuity
\begin{figure}
	\includegraphics[height=0.55\textheight]{graphics/ModulusCont.jpg}
	\caption{The Modulus of Continuity}
\end{figure}

\end{frame}

\begin{frame}{Our Result}
\begin{theorem} Let $\{\mathfrak{L}^N = (L_1^N,\dots,L_k^N)\}_{N=1}^\infty$ be a sequence of avoiding Bernoulli Gibbsian line ensembles. Fix $p\in(0,1)$ and $\lambda > 0$, and suppose that for all $n\in\mathbb{Z}$ we have
\[
\lim_{N\to\infty} \mathbb{P}\big(L_1^{N}(nN^{2/3}) - pnN^{2/3} + \lambda n^2 N^{1/3} \leq N^{1/3} x \big) = F_{TW}(x).
\]
Then $\{\mathfrak{L}^N\}$ is a tight sequence.
\end{theorem}

\begin{itemize}
	\item $F_{TW}$ denotes the \textit{Tracy-Widom distribution} -- a common limiting distribution in the KPZ universality class.
	
	\item {[Dauvergne-Nica-Vir\'{a}g '19]} showed that finite dimensional convergence of all curves implies tightness, hence convergence to the Airy line ensemble.
	
	\item Our result shows that it suffices for the \textcolor{red}{\textit{top curve}} to converge in the f.d. sense.
\end{itemize}

\[
	\includegraphics[scale=0.18]{graphics/ConvToTW.jpg}
\]
\end{frame}


\section{Section of Paper (7-9 min)}

\begin{frame} {History of the line ensembles}
	Arguments in this paper are inspired by 
	\begin{enumerate}
		\item \textit{Brownian Gibbs property for Airy line ensembles} and \textit{KPZ line ensemble}[Corwin-Hammond ‘11, ‘13], which address the issues of {\color{red}continuous} line ensembles
		\item \textit{Transversal fluctuations of the ASEP, stochastic six vertex model, and Hall-Littlewood line ensembles} [Corwin-Dimitrov ‘17], which consider similar questions in a {\color{red}discrete} setting 
	\end{enumerate}
\end{frame}
\begin{frame}{Proving tightness}
	Recall that to show tightness, we want to control:
	\bigskip
	\begin{enumerate}
		\item \textbf{\textcolor{ForestGreen}{Minimum}} of bottom curve $\textcolor{ForestGreen}{L_k^N}$,
		
		\bigskip
		
		\item \textbf{\textcolor{blue}{Maximum}} of top curve $\textcolor{blue}{L_1^N}$,
		
		\bigskip
		
		\item \textbf{\textcolor{Fuchsia}{Modulus of continuity}} of each curve $\textcolor{Fuchsia}{L_i^N}$.
		
		\bigskip
	\end{enumerate}
	We will focus on bounding the \textcolor{ForestGreen}{minimum}:
	
	\begin{lemma}[------]
		Fix $r,\epsilon > 0$. Then there exist constants $R>0$ and $N_0\in\mathbb{N}$ such that for all $N\geq N_0$,
		\[
		\mathbb{P}\Big(\inf_{x\in[-r,r]} \big(L_k^N(xN^{2/3}) - pxN^{2/3}\big) < -RN^{1/3}\Big) < \epsilon.
		\]
	\end{lemma}
\end{frame}

\begin{frame}{Monotone coupling}
Proof  (mention monotone coupling lemmas somewhere ) - say MC with picture
2min
\end{frame}

\begin{frame}{Strong coupling with Brownian bridges}
Proof  (mention strong coupling somewhere) - say SC with picture 
L = Bernoulli bridge B is a Brownian bridge with variance. There is a probability space such that $P( sup \abs*{L - B} \geq k (\log N)^2) < \epsilon$. This is a comparison that allows for example to compare the modulus of continuity of the two. [Dimitrov-Wu ‘19]
2 min
\end{frame}

\begin{frame}{Controlling the minimum: pinning the bottom curve}
	
	\begin{lemma}[------]
		For any $r,\epsilon > 0$, there exists $R>r$ and a constant $M>0$ so that for large $N$,
		\[
		\mathbb{P}\Big(\max_{x\in[r,R]} \big(L_k^N(xN^{2/3}) - pxN^{2/3}\big) < -MN^{1/3}\Big) < \epsilon.
		\]
		The same is true of the maximum on $[-R,-r]$.
	\end{lemma}
	\[
	\includegraphics[scale=0.1]{graphics/min.png}
	\]
	
	\begin{itemize}
		
		\item Couple with a Brownian bridge: if ``pinned" at two points $>r$ and $-r$, it cannot be low on scale $N^{1/3}$ on $[-r,r]$.
		
	\end{itemize}
	
	
	
\end{frame}

\begin{frame}{Proving the pinning lemma}
	
	\begin{itemize}
		
		\item Recall our assumption:
		\[
		\mathbb{P}\Big(\textcolor{red}{L_1^N}(nN^{2/3}) - pnN^{2/3} + \textcolor{red}{\lambda n^2} N^{1/3} \leq xN^{1/3}\Big) \underset{N\to\infty}\longrightarrow F_{TW}(x).
		\]
		
		\item The top curve looks like a \textcolor{red}{parabola} with an affine shift on large scales.
		\[
		\includegraphics[scale=0.12]{graphics/parabola.png}
		\]
		
		\item Two curves: if $L_2^N$ is low on $[r,R]$, $L_1^N$ looks like a free Brownian bridge.
		
		\[
		\textcolor{ForestGreen}{\lambda\Big(\frac{R^2+r^2}{2}\Big)} - \textcolor{Fuchsia}{\lambda\Big(\frac{R+r}{2}\Big)^2} = \lambda\frac{R^2+r^2}{4} - \frac{\lambda rR}{2} = \textcolor{red}{O(R^2)}.
		\]
		
		\item For large $R$, the top curve would be far from the parabola at the midpoint!
		
	\end{itemize}

	
	
\end{frame}


\end{document}